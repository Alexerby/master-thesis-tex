\documentclass[12pt, a4paper]{article}
%------ SETUP OF THE DOCUMENT ------%
%This part of the document sets up the document and makes it easier to read the main.tex file
%Changes in the setup file are recommended if you wish to customize things like colors in links and such.

%------ ********************* ------%
\usepackage[margin = 25mm]{geometry} %
\usepackage{graphicx} % Required for inserting images
\usepackage[toc,page]{appendix}
\usepackage[english=usenglishmax]{hyphsubst} %sets hyphenation to American English. Google is your friend on hyphenation. Babel could also be used
\usepackage[hyphens]{url}
\usepackage{amsmath, amssymb, setspace, float, subcaption, caption, booktabs, pdflscape, dcolumn, titlesec, tocloft, comment, xcolor, longtable, blindtext, rotating, lipsum}
\usepackage{multicol}
\usepackage[flushleft]{threeparttable}
\usepackage[colorlinks=true]{hyperref}
\usepackage{titlesec}
\usepackage{enumitem}
\usepackage{parskip}
\usepackage[utf8]{inputenc}

% \definecolor{citecolor}{RGB}{0, 51, 102}
\usepackage[round, authoryear]{natbib}

\hypersetup{
     colorlinks   = true, %If links to papers should be colored
     citecolor    = blue, %color of citation text
     urlcolor = blue, %if you use an URL, this is how it is colored
    linkcolor = black
}


%%%%%%%%%%%%%%%%%%%%%%%%%%%%%%%% 
% Tikz
%%%%%%%%%%%%%%%%%%%%%%%%%%%%%%%% 

\usepackage{tikz}
\usetikzlibrary{positioning, arrows.meta, shapes.geometric, fit}


\tikzset{
  pipelinebox/.style={
    rectangle,
    draw=black,
    align=center,
    font=\scriptsize,
    inner sep=4pt,
    minimum height=1cm,
    minimum width=2.5cm,
    text width=2.5cm,
    align=center
  },
  arrow/.style={
    thick, ->, >=stealth
  }
}

\tikzset{
  auxbox/.style={
    rectangle,
    draw=black,
    fill=orange!20,
    align=center,
    font=\footnotesize\itshape,
    inner sep=8pt,
    minimum height=1cm,
    minimum width=3.5cm,
    dashed
  }
}

\tikzset{
  sourcebox/.style={
    rectangle,
    draw=black,
    align=center,
    font=\footnotesize\bfseries,
    text transform=uppercase,
    inner sep=8pt,
    minimum height=1cm,
    minimum width=3.5cm
  }
}

\tikzset{
  actionbox/.style={
    diamond,
    draw=black,
    % fill=orange!20,
    align=center,
    font=\footnotesize,
    aspect=2,
    inner sep=2pt,
    minimum height=1.2cm,
    minimum width=3.5cm
  }
}

\tikzset{
  datasetbox/.style={
    rectangle, 
    draw=black, 
    dashed,
    minimum height=1.5em, 
    minimum width=3cm, 
    text centered, 
  },
}



%%%%%%%%%%%%%%%%%%%%%%%%%%%%%%%%%%%%%%%

\graphicspath{{figures/}}



%\bibliographystyle{apalike} 
\bibliographystyle{plainnat}
% \bibliographystyle{plain}

%%%%%%%%%%%%%%%%%%%%%%%%%%%%%%%%%%%
% Table formatting
%%%%%%%%%%%%%%%%%%%%%%%%%%%%%%%%%%%
\usepackage{multirow}
\renewcommand{\arraystretch}{1.0}


%%%%%%%%%%%%%%%%%%%%%%%%%%%%%%%%%%%
% Equation Formatting
%%%%%%%%%%%%%%%%%%%%%%%%%%%%%%%%%%%
\numberwithin{equation}{section}

%%%%%%%%%%%%%%%%%%%%%%%%%%%%%%%%%%%
% List Formatting
%%%%%%%%%%%%%%%%%%%%%%%%%%%%%%%%%%%
\setlist[itemize]{itemsep=0.05em}



\date{Seminar date}
\begin{document}

%%%%%%%%%%%%%%%%%%%%%%%%%%%%%%%%
% TITLE PAGE
%%%%%%%%%%%%%%%%%%%%%%%%%%%%%%%%
% Suppress overfull warning *only* on title page using \sloppy locally
% and immediately restore with \fussy afterward.

% Title Page
\setstretch{1.5}
\sloppy  % Allow slightly stretched lines on this page

%\includegraphics[scale = 0.3]{images/LU RGB.png}
%\includegraphics[scale = 0.3]{images/LU Black.png} 
\includegraphics[scale = 0.15]{images/LUSEM_RGB.png} %I recommend using this one but the others are fine too
%\includegraphics[scale = 0.15]{images/LUSEM_BLACK.png}

\vspace{2cm}
\begin{center}       
    \vspace*{2cm}
    {\LARGE \textbf{Unclaimed Aid in the German Student Aid System} \\ 
    A Microsimulation of BAföG Eligibility and Non-Take-Up} \\
    \vspace{1cm}
    \Large Alexander Eriksson Byström \& María Sól Antonsdóttir \\[0.3cm]
    \normalsize Department of Economics \\ 
    Lund University School of Economics and Management
\end{center}
\vspace{2cm}

\vfill
\noindent 
\textbf{Supervisor: Petra Thiemann} \\ 
NEKN01 - Master Thesis in Economics 15 ECTS \\ 
Seminar date: 2025-06-03
\thispagestyle{empty}

\fussy  % Restore normal spacing behavior

\newpage
\tableofcontents
\thispagestyle{empty}


% Thanks, But No Thanks: A Microsimulation of Non-Take-Up in Germany’s BAföG Student Aid Program


%%%%%%%%%%%%%%%%%%%%%%%%%%%%%%%%
% ABSTRACT
%%%%%%%%%%%%%%%%%%%%%%%%%%%%%%%%
\newpage
\begin{abstract}
\setstretch{1}
\noindent 
While the body of literature on the non-take-up of public aid has grown substantially in recent years, a notable gap remains in the literature of non-take-up rates for student aid programs, where research is still extremely limited.
This paper examines the non-take-up rate of Germany’s federal student aid program BAföG by creating a microsimulation based on data from the German Socio-Economic Panel for the period 2007--2021.
Using the outcome of our microsimulation, we estimate three specifications of binary choice models to examine how individual characteristics relate to take-up decisions. 
Our findings indicate that non-take-up has increased over the past decade, with an average rate of approximately 60\% for our study period. 
Two main factors help explain this phenomenon. 
First, students are less likely to apply if they expect only a small subsidy. 
Second, greater awareness and understanding of the application process appear to be associated with higher take-up rates. 
Additionally, we find a notable difference in take-up rates between East and West Germany, suggesting that attitudes toward government support may influence the take-up rate.
\vspace{1cm} \\
    \noindent \textbf{Keywords:} Non-take-up, 
    Microsimulation, 
    SOEP, 
    Student Aid, 
    Student Loans, 
    BAföG, 
    Education Funding. \\
    \noindent \textbf{JEL codes:} I22, I23, I24, I38, H53
\end{abstract}

\newpage


%%%%%%%%%%%%%%%%%%%%%%%%%%%%%%%%
% ACKNOWLEDGEMENTS
%%%%%%%%%%%%%%%%%%%%%%%%%%%%%%%%
\begin{center}
    \textbf{Acknowledgements}
\end{center}
We would like to thank our supervisor, Petra Thiemann, for her helpful guidance and support throughout the writing of this paper. 
Her feedback was valuable in shaping the direction and clarity of our work.

\newpage
\begingroup
    \setstretch{1.1}
    \hypersetup{hidelinks}
    \tableofcontents
\endgroup
\thispagestyle{empty}

\newpage
\setcounter{page}{1}


%%%%%%%%%%%%%%%%%%%%%%%%%%%%%%%%
% CONTENT
%%%%%%%%%%%%%%%%%%%%%%%%%%%%%%%%
%%%%%%%%%%%%%%%%%%%%%%%%%%%%%%%%%%%%%%%%%%%%%%%%%%%%%
%
%   INTRODUCTION
%
%%%%%%%%%%%%%%%%%%%%%%%%%%%%%%%%%%%%%%%%%%%%%%%%%%%%%


%TODO: Fix references in above paragraph
%TODO: Add our main findings at end of introduction

\section{Introduction} \label{sec:intro}

In recent years, the issue of non-take-up (NTU) of student financial aid, specifically BAföG (Bundesausbildungsförderungsgesetz), has attracted significant attention in Germany. While BAföG remains the primary policy tool for ensuring fair access to higher education, there is increasing skepticism about whether it effectively achieves its core objectives of unlocking educational potential and ensuring equal opportunities. This skepticism stems primarily from the steady decline in the number of eligible students who actually make use of BAföG \citep{gwosc_krisenbewaltigung_2022, meier_bafog_2024}.

Several recent studies highlight significant structural shortcomings of the current BAföG system as central reasons behind the rising NTU rates. Among the main points that critics emphasize is that BAföG allowances have not kept up with the actual living costs, particularly amid rising housing prices and increasing inflation \citep{meier_bafog_2024, meier_zur_2024, staack_von_2017, gwosc_krisenbewaltigung_2022}. As of recent, some policy amendments have been made (such as the 29th BAföG amendment in 2024) in an attempt to incrementally adjust the support rates, but these attempts have to some extent been met with criticism for being inadequate. For example, student expenditures for 2024 are projected to average around 969 EUR per month, significantly more than the maximum BAföG rate of 812 EUR, even after these adjustments \citep{meier_bafog_2024}.

Additionally, the complexity and lack of transparency in BAföG application procedures further discourage students from applying, which in turn increases NTU rates. Many students find the eligibility criteria unclear, which creates uncertainty about whether or not they qualify for financial support. But these barriers aren’t just administrative, they’re also psychological. Feelings of stigma, or discomfort with applying for what is perceived as a welfare type of benefit, play a role in that as well. So, while economic factors matter, the decision not to apply is often also shaped by information deficits and psychological factors \citep{gwosc_krisenbewaltigung_2022, staack_von_2017}.

This growing awareness of BAföG’s limitations has resulted in demands for more thorough impact evaluations and deeper structural reforms. As \cite{meier_bafog_2024} point out, despite annual public expenditure of approximately three billion EUR, there is still a lack of solid empirical assessment of the effectiveness and efficiency of the system. This knowledge gap has a limiting effect on the ability to make evidence based policy decisions, leading instead to reforms shaped by political compromise instead of objective analysis. Some researchers, like \cite{gwosc_krisenbewaltigung_2022}, argue that changing this might even require a shift from the current needs-based model (Fürsorgeprinzip), which requires students to prove their financial hardship, to a more inclusive, universal support model (Versorgungsprinzip). International comparisons indicate that participation rates in countries with universal support systems are substantially higher, which suggests that systemic change could lead to improvements for Germany in this regard.

The urgency to address NTU is becoming ever more apparent, especially in the context of recent economic challenges, such as rising housing costs and inflation, factors that disproportionately impact students. Critics of BAföG argue that the issues extend beyond just low support rates. They point to structural issues, particularly the mix of grants and loans. For many students, especially those from economically disadvantaged families, the prospect of going into debt is enough to deter them from applying at all \citep{staack_von_2017, gwosc_krisenbewaltigung_2022}.

In summary, the increasing NTU of BAföG has become a central issue in the wider policy discussion on accessibility of education. Addressing these problems will require targeted policy reforms, clearer and more transparent application processes and greater reliance on evidence based analysis. Implementing changes like that is crucial to ensure that public funds for student aid are used in an equitable and effective manner, and thus helping to maximize educational opportunities for students from all backgrounds.

\textcolor{red}{Add text where we say something like: "the layout of the paper is as follows..."}

\textcolor{red}{"Clearly state your research questions/objectives explicitly at the end of the introduction"}

\section{Literature review}
\label{section:literature_review}

As a form of social benefit, federal student aid such as BAföG faces similar challenges as other public support programs. 
One of the main challenges is ensuring that eligible individuals actually claim the assistance available to them. 
When those who qualify for support do not apply, the effectiveness of the policy is reduced. 
This can have broader consequences, since the overall goals of social programmes, such as reducing poverty or acting as automatic stabilisers during economic downturns, depend on reaching those in need \citep{goedeme_concept_2020}. 
Furthermore, if individuals who would benefit the most are not reached, both the efficiency and equity of social policy may be compromised.

This phenomenon, known as non-take-up, refers to situations where individuals meet the legal eligibility requirements but do not receive the benefit, often because they do not apply. 
This is different from “non-enrolment”, which also includes individuals who do not meet the eligibility criteria to begin with. Non-take-up of social benefits can be understood as resulting from factors at three main levels: individual circumstances (such as awareness, perceived stigma, or attitudes toward the benefit), administrative practices, and the broader design or structure of benefit schemes \citep{vanoorschot_failing_2002}. 

While non-take-up can be shaped by factors at multiple levels, much of the economic literature places particular emphasis on the client or individual perspective. In the economic context, the decision to claim aid is generally understood as a cost-benefit trade-off: individuals weigh the expected monetary gain against the costs of claiming, which are typically grouped into three categories: informational, procedural, and social or psychological \citep{booij_role_2012}. 
These costs help explain why eligible individuals might choose not to apply, even when they stand to benefit financially. 
Each type of cost affects decision-making in a distinct way:

\begin{itemize}
  \item \textbf{Information costs} refer to the time and effort required to learn about available benefits, understand the eligibility rules, navigate the application process, and assess the possible consequences of claiming support. This may for example include searching for reliable information or clarifying confusing requirements.
  \item \textbf{Process costs}, on the other hand, involve the resources spent during the actual application process. These can include filling out forms, providing documentation, traveling to relevant offices, waiting in lines, or facing other administrative hurdles.
  \item Finally, \textbf{social and psychological costs} capture the emotional and interpersonal challenges associated with claiming benefits. A key factor here is stigma, which can for example manifest as personal discomfort with claiming support or concern over how others may perceive the claimant. The extent to which stigma is felt can be shaped both by the way benefit programs are designed and by broader social attitudes.
\end{itemize}

Among these different types of costs, recent research suggests that information costs may play a particularly important role in explaining non-take-up. \cite{bolland_information_nodate} find that information costs are a stronger predictor of non-take-up than either process complexity or stigma. Their results show that individuals who are unaware of available benefits are much less likely to claim them, and that higher perceived information costs are consistently associated with lower take-up rates. 

%TODO: cite here some of the sources saying that stigma is not particurlary relevant here (especially maybe in the context of student aid etc)

It follows from the above that these costs of claiming tend to increase when procedures are complex or lack clear explanation. In line with this, standard economic theory predicts that individuals are more likely to claim benefits when the expected payout is large or long-lasting, and less likely when the application process is complicated or socially stigmatised \citep{janssens_totake_2022, booij_role_2012}. This relationship between complexity, costs, and take-up is also highlighted by \cite{akerlof_tagging_1978}, who argues that while targeted welfare programs (“tagging”) are theoretically efficient in resource allocation, the way such programs are structured in themselves often leads to the kinds of complexities and unintended incentives discussed above.

While this view reflects traditional economic thinking, recent research suggests that non-take-up may persist even when financial and administrative barriers are minimal. 
For example, \cite{bhargava_psychological_2015} show that even when procedural barriers are low, cognitive and behavioural factors alone can still lead to high levels of non-take-up. 
Specifically, they find that even relatively minor psychological obstacles, such as uncertainty about eligibility, confusing application steps, or unclear instructions, can be enough to discourage people from claiming benefits. 
This holds true even when the actual effort or time involved is not substantial. 
These findings support broader behavioural models that recognize limits to attention, self-control, and cognitive resources. 
People may procrastinate or avoid tasks that seem uncertain or complex, even if they intend to complete them later, and may be especially influenced by how information is presented or by small logistical hurdles.

%These findings show that non-take-up is shaped by both structural and behavioural factors. While economic models help explain some of this, recent work suggests that small barriers in procedures, communication, or perception can also play a large role.

\begin{table}[htbp]
\footnotesize
\centering
\begin{tabular}{llrlll}
\toprule
\textbf{Author(s)} & \textbf{Year} & \textbf{NTU (\%)} & \textbf{Year of Data} & \textbf{Data Source} & \textbf{Program Type} \\
\midrule
\citeauthor{frick_claim_2007}              & 2007          & 67             & 2002                  & SOEP             & SA               \\
\citeauthor{herber_non-take-up_2019}       & 2016          & 36--40         & 2002--2013            & SOEP             & BAföG            \\
\citeauthor{RePEc:iab:iabfob:201305}       & 2013          & 34--43         & 2008                  & EVS              & BSS              \\
\citeauthor{bruckmeier_benefit_2018}       & 2018          & 43             & 2013--2014            & SOEP             & SA               \\
\citeauthor{bruckmeier_benefit_2018}       & 2018          & 87             & 2013--2014            & SOEP             & HA               \\
\citeauthor{bruckmeier_benefit_2018}       & 2018          & 63             & 2013--2014            & SOEP             & HA \& SA         \\
\citeauthor{bruckmeier_benefit_2018}       & 2018          & 88             & 2013--2014            & SOEP             & SCA              \\
\citeauthor{bruckmeier_new_2012}           & 2012          & 41--49         & 2005--2007            & SOEP             & SA               \\
\bottomrule
\end{tabular}
\caption{\small{Selected previous estimates of non-take-up (NTU) rates for social benefits in Germany. Program type abbreviations: SA = Social Assistance, BAföG = Federal Student Aid, MTG = Means-Tested General Benefits, BSS = Basic Social Security, HA = Housing Allowance, SCA = Supplementary Child Allowance.}}
% \caption*{\small{Note: This table summarizes selected results on non-take-up rates from prior literature using SOEP and other German datasets. See cited references for full details.}}
\label{table:NTU-studies}
\end{table}


The next section provides background on BAföG and outlines the institutional and policy context relevant to student aid in Germany.


\input{sections/3. Background}
%%%%%%%%%%%%%%%%%%%%%%%%%%%%%%%%%%%%%%%%%%%%%%%%%%%%%%%%%%%
%
% DATA
%
%%%%%%%%%%%%%%%%%%%%%%%%%%%%%%%%%%%%%%%%%%%%%%%%%%%%%%%%%%%


%%%%%%%%%%%%%%%%%%%%%%%%%%%%%%%%%%%%%%%%%%%%%%%%%%%%%%%%%%%


\section{Data}
To estimate non-take-up rates of welfare benefits, one typically relies on one or more of three data sources: administrative records, specially designed surveys, and general purpose household surveys.
Each comes with its own trade-offs. 
Administrative data are accurate for tracking benefit receipt but usually lack information on those who do not apply. 
Special surveys can provide richer detail on eligibility and claiming behaviour, though they are costly and rarely implemented. 
General purpose surveys are more readily available and widely used in empirical research on non-take-up, even if they are not designed with this purpose in mind \citep{mechelen_who_2017}.

In line with much of the existing literature, this study relies on data from the German Socio-Economic Panel (SOEP), which falls into the third category of general-purpose household surveys. 
As one of the longest-standing multidisciplinary household surveys in the world, SOEP has been conducted annually since 1984 by the German Institute for Economic Research \citep{soepcore_v39}. 
It is a nationally representative longitudinal study that collects data from around 30,000 individuals in 22,000 households each year. 
The survey includes respondents aged 17 and older and provides rich individual- and household-level information on income, education, labour market activity, household structure, and demographics. 
This study uses the SOEP-Core sample, the central and most comprehensive module of the dataset. While general purpose surveys like SOEP are not specifically designed to measure non-take-up, they have the advantage of covering both benefit receipt and the before mentioned characteristics needed to estimate eligibility \citep{mechelen_who_2017}.

\subsection{Sample Description}
We restrict our analysis to the period between 2007 and 2021, as this corresponds to the range for which we were able to consistently collect and harmonize the necessary statutory parameters from official BAföG regulations \citep{bafoeg_law}. These parameters include annual updates to base need rates, income allowances, asset thresholds, and other legally defined components relevant to BAföG eligibility and award determination.\footnote{
  See Appendix \ref{appendix:simulation-example} for an example illustrating how these rules are applied.
}
Earlier years were excluded due to inconsistencies or incomplete availability of comparable legal documentation. By focusing on this window, we ensure that the simulation model is fully grounded in verifiable legal norms and accurately reflects the policy environment faced by students during this period.

The SOEP, with its detailed household structure, enables us to link students to their parents, siblings, and, in many cases, partners. Using this data, we construct a dataset combining rich information at both the student and household levels, including a wide range of socio-demographic, economic, and relational characteristics.

Our final dataset comprises \(5{,}889\) student-year observations, each representing a student in a given survey year, drawn from a harmonized sample based on SOEP-Core data spanning 2007 to 2022. The panel is unbalanced, as students are observed for varying numbers of years depending on education duration, dropout behavior, and survey participation.

While some students appear only once, others are followed across multiple years of their educational trajectory. Each observation includes comprehensive information on sociodemographic background, enrolment status, income and assets, housing situation, and reported BAföG receipt. Variables used in the simulation are consistently available throughout the entire period.

To ensure accurate simulation of BAföG eligibility and awards, we restrict the sample to students for whom we can identify income information for both parents. 

Although the SOEP survey is nationally representative, this analytic subsample is conditional on respondents enrolled in education and meeting the inclusion criteria of the simulation pipeline. A descriptive overview of key variables is provided in Appendix~\ref{appendix:variable_dictionary}, Table~\ref{table:variable_dictionary}.

\begin{table}[H]
\footnotesize
\caption*{Descriptive statistics: non-take-up (NTU) and full eligible sample}
% \renewcommand{\arraystretch}{1.25}
\centering
\begin{tabular}{p{8cm}ccc|ccc}
\toprule
\textbf{Variable} & \multicolumn{3}{c|}{\textbf{NTU Sample}} & \multicolumn{3}{c}{\textbf{Full Sample}} \\
& Mean & Min & Max & Mean & Min & Max \\
\midrule
\multicolumn{7}{l}{\textbf{Main explanatory variable}} \\
Simulated BAföG Amount (EUR)       & 400    & 52    & 861   & 123    & 0    & 861   \\
\midrule
\multicolumn{6}{l}{\textbf{Demographics and Socioeconomic}} \\
Age                         & 23     & 18    & 34    & 23     & 18   & 41    \\
Female (\%)                 & 52     & n/a   & n/a   & 51     & n/a  & n/a   \\
Has partner (\%)            & 2      & n/a   & n/a   & 2      & n/a  & n/a   \\
Migration background (\%)   & 31     & n/a   & n/a   & 20     & n/a  & n/a   \\
\midrule
\multicolumn{6}{l}{\textbf{Institutional and Informational}} \\
Lives with parents (\%)          & 43     & n/a   & n/a   & 48     & n/a  & n/a   \\
Sibling claimed BAföG (\%)  & 35     & n/a   & n/a   & 30     & n/a  & n/a   \\
East background (\%)        & 17     & n/a   & n/a   & 21     & n/a  & n/a   \\
Parents highly educated (\%)& 25     & n/a   & n/a   & 43     & n/a  & n/a   \\
\midrule
\multicolumn{6}{l}{\textbf{Behavioural Predictors}} \\
Patience (0--10)            & 6.2    & 0     & 10    & 6.0    & 0    & 10    \\
Impulsiveness (0--10)       & 5.0    & 0     & 10    & 4.9    & 0    & 10    \\
Risk appetite (0--10)       & 5.3    & 0     & 10    & 5.1    & 0    & 10    \\
\bottomrule
\end{tabular}
\caption{\small{Descriptive statistics for two groups: the non-take-up (NTU) subsample, consisting of students classified as eligible but not receiving BAföG support, and the full sample of all theoretically eligible students. Means for binary and categorical variables are expressed as percentages. Min/Max values are not applicable for binary variables.}}
\caption*{\small{\textit{Note:}  The mean simulated BAföG in the full sample is lower because it includes all observations with a simulated amount of zero.}}
\label{tab:descriptive_ntu_all}
\end{table}



%%%%%%%%%%%%%%%%%%%%%%%%%%%%%%%%%%%%%%%%%%%%%%%%%%%%%%%%%%%
%
% METHOD
%
%%%%%%%%%%%%%%%%%%%%%%%%%%%%%%%%%%%%%%%%%%%%%%%%%%%%%%%%%%%

%TODO: Maybe reference table A1 in appendix, conditional per different types.
%TODO: Not mentioned how we dealt with measurement errors
%TODO: Did you compare simulated eligibility rates with official BAföG statistics?
%TODO: We need to do sensitivity analysis 
%TODO: How closely do simulated award amounts match reported distributions? Use appendix distribution
%TODO: Also use timeline in appendix for the comparison of theoretical to reported

%%%%%%%%%%%%%%%%%%%%%%%%%%%%%%%%%%%%%%%%%%%%%%%%%%%%%%%%%%%

\section{Method} 

\textcolor{red}{"Brief introductory paragraph clarifying why microsimulation is uniquely suited for this analysis would strengthen this section."}

This study proceeds in two main steps. 
First, we perform a microsimulation to calculate theoretical BAföG eligibility and award amounts based on statutory rules applied to individual-level survey data. 
This simulation serves to identify who should be entitled to student aid under the legal framework.

Second, we estimate binary choice models (Logit and Probit) to analyze behavioural non-take-up. 
That is, the likelihood that students eligible for BAföG according to the simulation nonetheless do not receive it. 
These models incorporate relevant socioeconomic and demographic factors to explore determinants of non-take-up beyond eligibility alone.

%%%%%%%%%%%%%%%%%%%%%%%%%%%%%%%%%%%%%%%%%%%%%%%%%%%%%%%%%%%
%
% Microsimulation
%
%%%%%%%%%%%%%%%%%%%%%%%%%%%%%%%%%%%%%%%%%%%%%%%%%%%%%%%%%%%
\subsection{Microsimulation of Theoretical BAföG Eligibility} 
To analyze non-take-up of BAföG, it is necessary to identify students who would be eligible for support based on statutory criteria, regardless of whether they actually apply or receive benefits. 
To this end, we implement a microsimulation model that reconstructs individual eligibility and award amounts using detailed SOEP survey data and the relevant legal rules.
The following section outlines the purpose, methodological approach, and main components of this simulation. 
For an overview of the approach see Figure \ref{fig:pipeline-overview}.

\subsubsection{Purpose and Scope}
The microsimulation pipeline is designed to calculate a theoretical BAföG eligibility status and award amount for students in the SOEP-Core sample. 
Its primary purpose is to compare these simulated entitlements with reported BAföG take-up, as reported in SOEP.

To construct the theoretical values, the model replicates the legal rules and means-testing procedures defined in the Bundesausbildungsförderungsgesetz (BAföG) for the years 2007 to 2021. 
These rules are applied to individual-level SOEP data, including detailed information on income, assets, housing costs, and household structure.

This approach enables a systematic assessment of the alignment between statutory entitlements and actual BAföG participation. 
Deviations between the modeled and reported outcomes may arise from reporting errors, exceptional administrative decisions, or incomplete data. 
Full documentation of the simulation logic and input structure is provided in Appendix ~\ref{appendix:microsimulation-pipeline} and ~\ref{appendix:simulation-example}.

\paragraph{Identifying the Non-Take-Up (NTU) Rate and Beta Error.} %TODO: Not referenced correctly
We define non-take-up of BAföG in line with Nelson and Nieuwenhuis (2019), as the circumstance when a person is eligible for welfare, but does not receive it. This is in line with terminology commonly used in literature on welfare take up rates. Non take up rate is thus the number of people who are eligible, but do not receive it, divided by the total number of people eligible. 


Formally, this is expressed as:

\begin{equation}
\Pr(\text{NTU} = 1 \mid M = 1) = \frac{\sum_{i=1}^{N} \mathbf{1}\{R_i = 0 \ \text{and} \ M_i = 1\}}{\sum_{i=1}^{N} \mathbf{1}\{M_i = 1\}}, \quad\text{where} 
\end{equation}

\begin{equation}
  \mathbf{1}\{\cdot\} =
  \begin{cases}
  1 & \text{if individual } i \text{ is eligible but does not take up BAföG}, \\
  0 & \text{otherwise}.
  \end{cases}
\label{eq:indicator-function-ntu}
\end{equation}


\paragraph{Beta Error (Type II Error).}  
It is worth noting, however, that these situations are often more complex. 
In some cases, individuals may receive BAföG even though they are not eligible. 
This can occur due to fraud or simply as a result of administrative errors. 
Such cases introduce the concept of beta errors. 
A beta error refers to the probability that a student receives BAföG despite being classified as ineligible by our model. 
It captures false positives in the eligibility classification; cases where students who should not qualify according to the simulation nonetheless receive financial support.

Formally, it is expressed as:
\begin{equation}
\Pr(\text{TU} = 1 \mid M = 0) = \frac{\sum_{i=1}^{N} \mathbf{1}\{R_i = 1 \ \text{and} \ M_i = 0\}}{\sum_{i=1}^{N} \mathbf{1}\{M_i = 0\}},
\end{equation}
where \( \mathbf{1}\{\cdot\} \) is the indicator function defined as
\[
\mathbf{1}\{\cdot\} =
\begin{cases}
1 & \text{if individual } i \text{ is ineligible but receives BAföG}, \\
0 & \text{otherwise}.
\end{cases}
\]

% Maybe ref all our data sources, the statutory and soep
\subsubsection{Simulation Pipeline}

\paragraph{Constructing the Student Dataset.}
The pipeline begins by assembling a harmonized dataset of student-level observations from SOEP-Core and manually harmonizing variables which are not harmonized already.
This is achieved by filtering for individuals who are enrolled in education, fall within the relevant survey years, and are at least 18 years old. 
To ensure a valid estimation of parental contributions, the dataset is further restricted to cases where income data from both legal parents are observable in the panel in order to reduce bias and ensure validity of estimated parental contributions.

The resulting student-level dataframe integrates sociodemographic variables including sex, age, partnership status, number of siblings, number of children, household composition, and federal state of residence. 
Gross student income is also appended at this stage. 
Net student income is derived from gross values by applying year-specific rules for income tax, solidarity surcharge, church tax (where applicable), and standard deductions (e.g., Werbungskostenpauschale), in accordance with §§\,21–23 BAföG \citep{bafoeg_law}.
This net income will later be used to compute the student’s excess income as part of the BAföG need assessment.


\paragraph{Estimating Parental Contributions.}
In the next step, the simulation pipeline aggregates and evaluates parental income to estimate the expected contribution toward the student’s BAföG entitlement. 
For each student, the incomes of both legal parents—identified within the household and linked through SOEP family structure data—are retrieved and converted into annual net income. 
These values account for deductions such as income tax, solidarity surcharge, and church tax, where applicable.

Net incomes from both parents are combined into a joint parental income measure. 
From this, the model subtracts statutory allowances as defined in §§\,24–25 BAföG \citep{bafoeg_law}, which vary depending on the number of parents, number of dependent children, and year-specific legal thresholds. 
Additional deductions are applied if the student has siblings who might also be eligible for support. 
The result is a measure of excess parental income, which feeds directly into the theoretical award calculation in the next stage.

A complete breakdown of the income transformation, applicable thresholds, and illustrative examples is provided in Appendix~\ref{appendix:simulation-example}.

\paragraph{Asset Test.}
The simulation includes an asset test to assess whether students hold financial resources above the statutory exemption thresholds. 
For each student, information on financial assets, real estate, business holdings, private insurances, vehicles, and other tangible property is combined, and reported debts are subtracted to derive total net assets.

Since asset data in SOEP are only collected every five years, missing observations for non-surveyed years are filled using linear interpolation. 
This approach allows for year-specific asset estimates that remain consistent with observed data and ensures full coverage across the entire simulation period.

Total assets are then compared against exemption thresholds defined in §\,29 BAföG \citep{bafoeg_law}, which vary by age, partnership status, and number of dependent children. 
Any amount exceeding the applicable allowance is classified as excess assets and contributes to reducing the student's calculated need. 


\paragraph{Need calculation and theoretical entitlement.}
In the final stage, the simulation model calculates the student's funding need by summing the statutory base need, housing allowance, and health insurance supplement, as defined in §\,13 BAföG \citep{bafoeg_law}. 
From this total, the model subtracts any excess income attributable to the student, their parents, and their assets. 
The resulting amount determines the theoretical monthly BAföG entitlement.

A positive entitlement does not automatically imply eligibility: the model also applies age-based eligibility criteria. 
Students are only considered theoretically eligible if they meet the age requirements defined in the law, typically under 30 for undergraduate studies and under 35 for graduate-level programs. 
The final output includes both the simulated monthly award and a binary eligibility flag, which are used for comparison against self-reported values in SOEP. 
Detailed examples of this calculation and relevant thresholds are provided in Appendix~\ref{appendix:simulation-example}.



%%%%%%%%%%%%%%%%%%%%%%%%%%%%%%%%%%%%%%%%%%%%%%%%%%%%%%%%%%%
% Binary Choice Model of Non-Take-Up
%%%%%%%%%%%%%%%%%%%%%%%%%%%%%%%%%%%%%%%%%%%%%%%%%%%%%%%%%%%
\subsection{Binary Choice Model}
This section presents results from logit, probit, and linear probability model (LPM) estimations. 
Due to the relatively small sample size (\( n = 5{,}889 \)), resulting from restricting the sample to observations with complete parental income information, all models are estimated using a pooled cross-sectional specification. 
That is, we treat all available years as a single pooled sample and do not exploit the panel structure of the SOEP.

\textcolor{red}{"Consider clarifying rationale for selecting both Probit and Logit models explicitly."}

\textcolor{red}{"Mention clearly the expected differences between Logit and Probit results (usually negligible but worth stating explicitly)."}

\subsubsection{Probit Model}
Formally, we model
\begin{equation}
  \Pr(\mathrm{NTU}_i = 1 \mid \mathbf{X}_i) = \Phi(\mathbf{X}^\top \boldsymbol{\beta})
  , \qquad \text{for all } i \text{ with } T_i = 1,
\end{equation}
where \( \Phi(\cdot) \) denotes the cumulative distribution function of the standard normal distribution. 
\begin{equation}
  \Phi(z) = \int_{-\infty}^{z} \frac{1}{\sqrt{2\pi}} \exp\left( -\frac{t^2}{2} \right) \, dt
\end{equation}

Here, \( T_i = 1 \) indicates the theoretical eligibility outcome of our microsimulation, and \( \mathrm{NTU}_i := \mathbf{1}\{R_i = 0\} \) is a binary indicator for non-take-up, based on the observed receipt in SOEP-Core (with \( R_i = 1 \) indicating receipt of BAföG and \( R_i = 0 \) otherwise).

\subsubsection{Logit Model}
In the same way as the Probit model, we fit a Logit model
\begin{equation}
  \Pr(\mathrm{NTU}_i = 1 \mid \mathbf{X}_i) = \Lambda(\mathbf{X}^\top \boldsymbol{\beta})
  , \qquad \text{for all } i \text{ with } T_i = 1,
\end{equation}
where \( \Lambda(\cdot) \) denotes the logistic cumulative distribution function
\begin{equation}
  \Lambda(z) = \frac{1}{1 + \exp(-z)}.
\end{equation}

\paragraph{Interpretation.} %TODO: Maybe explain latent index more -- don't fully understand atm
Since raw logit and probit coefficients reflect changes in the latent index and are not directly interpretable in terms of outcome probabilities, we report average marginal effects (AMEs) for all covariates. These AMEs quantify the average change in the probability of non-take-up associated with a one-unit change in each covariate, holding other variables at their observed values.


\subsubsection{Linear Probability Model}
Analogously, the linear probability model (LPM) specifies a linear relationship between the covariates and the probability of non-take-up:
\begin{equation}
  \Pr(\mathrm{NTU}_i = 1 \mid \mathbf{X}_i) = \mathbf{X}_i^\top \boldsymbol{\beta}
  , \qquad \text{for all } i \text{ with } T_i = 1,
\end{equation}
where, as before, \( \mathbf{X}_i \) denotes the vector of covariates for individual \( i \), and \( \boldsymbol{\beta} \) is the vector of estimated coefficients.

Estimation is performed via ordinary least squares (OLS), clustering standard errors at the student level. 
Unlike the probit and logit models, the LPM does not constrain predicted probabilities to the interval \([0,1]\), but its coefficients are directly interpretable as approximate marginal effects.

Despite its limitations, the LPM is commonly used in applied work due to its simplicity, transparency, and computational efficiency. In particular, it provides straightforward coefficient interpretation and is easily extended to models with fixed effects or instrumental variables. Moreover, when the primary interest lies in estimating average marginal effects rather than modeling the full distribution of probabilities, the LPM often yields results that are qualitatively similar to those from non-linear models.

For this reason, we include the LPM as a useful benchmark alongside the logit and probit specifications.



\subsubsection{Control Variables}
Our models include a set of control variables to account for observed heterogeneity that may influence the probability of non-take-up. 
These controls include demographic factors (e.g., sex, migration background, partnership status, living situation), socioeconomic characteristics (e.g., parental income, own income, parental education), and family context (e.g., sibling previously claimed BAföG). 
We also control for regional differences using an East/West Germany background indicator, reflecting known structural and cultural variations. 
Finally, to capture behavioral differences that might affect take-up decisions, we include a measure of individual risk appetite. 
These covariates help isolate the association between key predictors and non-take-up by adjusting for potential confounders.

The rationale for including several of these variables is further elaborated below, drawing on prior research and theoretical considerations. In particular, the inclusion of factors like risk attitudes, family experience with the application process, and regional socialization helps account for variation in informational access, institutional trust, and attitudes toward public support.

\paragraph{Risk appetite, impulsiveness and patience.} In this analysis, a variable for students' self-assessed willingness to take risks is included. 
Even though BAföG offers relatively safe and generous conditions, some students might still be hesitant to take on any form of debt if they are generally risk-averse. 
By including this variable, we aim to capture whether differences in individual risk preferences help explain why some eligible students choose not to apply.

\cite{herber_non-take-up_2019} also include a risk preference variable in their study, mainly to control for the possibility that risk attitudes could affect take-up behavior or influence how other factors, like impatience, play a role. 
They do not find a strong effect of risk aversion on BAföG take-up, but they still argue it is useful to control for. 
In a similar way, we include this variable to improve our model and to see whether risk aversion plays any role in students’ decisions to reject BAföG.

Furthermore, we include a control variable for impulsiveness, measured using a scale constructed by SOEP from responses to several relevant survey questions.
Similarly, we control for patience using a scale based on a different set of questions from the SOEP questionnaire.
Including these behavioral traits allows us to account for the possibility that impulsiveness or patience may influence students’ decision-making processes and thus help explain variation in BAföG take-up among eligible students.

\paragraph{East German socialization.}  A variable indicating whether the student lives in East Germany is included to account for potential differences in attitudes toward state support rooted in historical and regional context. \cite{alesina_good-bye_2007} show that individuals from the former GDR tend to have stronger preferences for redistribution and a greater belief in the role of the state in providing social services, and that these differences in preferences can persist for one to two generations after reunification. Current residence in East Germany may reflect continued exposure to these norms and institutions and can serve as a reasonable proxy for this form of socialization. Since the variable is statistically significant at the 5\% level in our model, we interpret it as capturing persistent regional differences in how students view and respond to publicly provided financial support like BAföG.

\paragraph{Sibling prior experience with BAföG.} An indicator for whether the student has an older sibling who previously received BAföG is included to capture potential differences in access to informal support and familiarity with the application process. Students with siblings who have already gone through the steps of applying may be more aware of eligibility rules and practical requirements. \cite{herber_non-take-up_2019} highlight that such sibling experience can help reduce informational and procedural barriers, making it more likely that students follow through with the application. This variable is intended to reflect how previous exposure to the system within the family can shape students’ confidence and ability to navigate what is often perceived as a complex process.

\paragraph{Migration background.} Two variable for indirect and direct migration background is included to explore whether differences in familiarity with the BAföG system may influence take-up. 
Some students may come from households with less exposure to German administrative processes or financial aid structures, which could affect their understanding of eligibility or the application itself. 
In addition, studies show that individuals with a migration background in Germany often have lower financial literacy, which may make it harder to evaluate financial aid options like BAföG \citep{Tsegay_2024}. 
Including this variable helps capture potential structural or informational factors that may contribute to lower take-up rates among eligible students. 

\paragraph{Parental education.}
To assess whether parental education influences BAföG take-up rates, we include a control variable that identifies students whose parents hold at least a bachelor’s degree as having a higher education background. 
This classification allows us to examine the potential effect of parental educational attainment on students’ likelihood of applying for and receiving BAföG support.

\paragraph{Siblings who have claimed BAföG.}
We include a control variable indicating whether any of the respondent’s siblings have previously claimed BAföG. 
The presence of a sibling with prior BAföG experience may provide informational advantages or influence the individual’s own decision-making process regarding BAföG take-up.

\paragraph{Age, Sex, and Partnership Status.}
We include control variables for age, sex (female), and partnership status (has partner) to account for demographic factors that may influence BAföG take-up. 
Age is controlled for, as students’ likelihood of applying for financial aid may change with their stage of study or broader life circumstances. 
For instance, older students may encounter different financial pressures or possess greater familiarity with administrative processes compared to younger students.
Sex is included to capture potential gender differences in educational choices, financial decision-making, or access to information about student aid. 
Partnership status is considered, as having a partner could affect an individual’s household resources, financial planning, or the sharing of information relevant to student aid. 
By controlling for these demographic characteristics, we aim to ensure that our analysis more accurately isolates the effects of other variables of interest.



\subsection{Model limitations}
\label{subsection:model_limitations}

\textcolor{red}{Are we also repeating ourselves here?}

\paragraph{Addressing beta errors in eligibility simulations.}
In simulating benefit non-take-up, beta errors occur when individuals report receiving a benefit but are classified by the model as ineligible. 
These mismatches typically reflect limitations in the input data, particularly income and assets.
Since the data in this study is self-reported, inaccuracies may occur in both income and benefit receipt.
Without administrative records, it is not possible to confirm whether a student was truly eligible or actually received the benefit. 
Some studies suggest that beta errors are more often caused by issues in the income or asset data used for eligibility simulation, rather than incorrect reporting of benefit receipt \citep{frick_claim_2007, janssens_takemod_2022}. 

To address these limitations, several strategies are used in the literature. 
These include conducting sensitivity checks by adjusting income levels and applying post-simulation corrections to reclassify borderline cases \citep{herber_non-take-up_2019}. 
Some studies also emphasize the value of combining different data sources where possible, such as using more detailed survey modules on assets or household composition to improve the accuracy of eligibility simulations \citep{janssens_takemod_2022}. 

Although beta errors cannot be completely avoided, it is important to recognise their potential impact on the results. 
In this thesis, particular attention is paid to identifying where beta errors may occur and considering how they might influence the findings. 
Sensitivity checks are applied where relevant to assess the robustness of the findings and to reduce the risk of misinterpretation. %TODO: No no sensitivity checks are applied so far.

\subsubsection{Measurement Errors in Income Data}
A persistent challenge in empirical research is the presence of measurement errors in income data, which can bias the estimation of policy outcomes. 
In the context of our study, accurate measurement of both parental and student incomes is crucial, as these variables are the primary determinants of the theoretical BAföG entitlement to be calculated.

To assess the impact of potential income misreporting or data imperfections, we conduct a sensitivity analysis by introducing normally distributed noise to the log-transformed income variables. 
Logging is applied to address the right-skewness in the income distributions (see Figure~\ref{fig:income-distributions}). 
By simulating random measurement errors in logged income, we can evaluate how such noise affects the estimated non-take-up rates and the reliability (i.e., beta errors) of the simulation outcomes.

This approach allows us to better understand the robustness of our results to plausible inaccuracies in income reporting, and highlights the importance of data quality in studies of means-tested student aid.

%%%%%%%%%%%%%%%%%%%%%%%%%%%%%%%%%%%%%%%%%%%%%%%%%%%%%%%%%%%
%
% RESULTS
%
%%%%%%%%%%%%%%%%%%%%%%%%%%%%%%%%%%%%%%%%%%%%%%%%%%%%%%%%%%%

\section{Results}

\subsection{Microsimulation: Non-take-up rates}

Our microsimulation results indicate that the non-take-up rate of BAföG, among theoretically eligible students, ranged from approximately 50--68\% across the survey years 2007--2021, with an average of 60\% (Table~\ref{table:microsimulation_ntu}). These estimates are broadly in line with previous findings on non-take-up of social benefits in Germany, which generally fall between around 40--70\%, depending on the program and time period (see Table~\ref{table:NTU-studies} for example). While our estimates are broadly consistent with prior research, they are noticeably higher than the 36--40\% non-take-up rate for BAföG reported by \cite{herber_non-take-up_2019}, who also use SOEP survey data, but for the period 2002--2013.

Several factors may help explain the difference in estimated non-take-up rates. These factors include the specific SOEP variables used to capture income and reported BAföG receipt, as well as differences in the time periods covered (2007 to 2021 in our study versus 2002 to 2013 in \cite{herber_non-take-up_2019}). Other aspects of the microsimulation design and modelling approach may also contribute to the variation. Importantly, the overall accuracy of our model in classifying receipt status is 72\%, as defined by the share of correctly predicted recipients and non-recipients (see equation~\ref{eq:accuracy_microsimulation}). While not perfect, this level of accuracy is consistent with expectations given the complexity of the BAföG system and the limitations of self-reported survey data.


\begin{table}[htbp]
\small
\centering
\begin{tabular}{l@{\hspace{2em}}r@{\hspace{2em}}r@{\hspace{2em}}r}
\toprule
\textbf{Year} & \textbf{Non-Take-Up} & \textbf{Take-Up Rate} & \textbf{Beta Error} \\
              & \(\Pr(\text{NTU} = 1 \mid \text{M} = 1)\) & \(\Pr(\text{TU} = 1 \mid \text{M} = 1)\) & \(\Pr(\text{TU} = 1 \mid \text{M} = 0)\) \\
\midrule
2007 & 58.0 & 42.0 & 15.9 \\
2008 & 60.4 & 39.6 & 19.8 \\
2009 & 58.0 & 42.0 & 19.4 \\
2010 & 53.0 & 47.0 & 21.9 \\
2011 & 51.4 & 48.6 & 17.8 \\
2012 & 48.4 & 51.6 & 20.8 \\
2013 & 51.0 & 49.0 & 11.9 \\
2014 & 52.6 & 47.4 & 14.7 \\
2015 & 72.1 & 27.9 & 12.8 \\
2016 & 58.2 & 41.8 & 15.5 \\
2017 & 64.0 & 36.0 & 7.0 \\
2018 & 68.8 & 31.2 & 11.5 \\
2019 & 69.0 & 31.0 & 11.1 \\
2020 & 69.3 & 30.7 & 13.0 \\
2021 & 69.9 & 30.1 & 17.6 \\
\midrule
\textbf{Average} & \textbf{60.0} & \textbf{40.0} & \textbf{15.8} \\
\bottomrule
\end{tabular}
\caption{\small{Non-Take-Up, Take-Up, and Beta Error Rates by Survey Year (\%). Non-take-up is the share of theoretically eligible students (\(M=1\)) who do not receive BAföG. The take-up rate is simply the complement, i.e., the share of eligible students who do receive BAföG \((1 - \Pr(\text{NTU} = 1 \mid M = 1))\). Beta error is the share of ineligible students (\(M=0\)) who nevertheless receive BAföG.}}
\caption*{\small{Notes: SOEP v39, 2007--2021, weighted with individual weights}}
\label{table:microsimulation-ntu}
\end{table}


While there is some variation in non-take-up across years, it remains consistently quite high throughout the period. 
The rate fluctuates from a low of 50\% in 2013 to a high of approximately 68\% in 2019. 
This pattern is clearly illustrated in Figure~\ref{fig:ntu_bounds_over_years}, which shows a visible decline from 2008 up to 2013, followed by a gradual upward trend from 2016 until 2019. 

\begin{figure}[htbp]
  \centering
  \includegraphics[width=0.95\linewidth]{ntu_bounds.png}
\caption{Development of the probability of non-take-up from 2007 to 2021. 
Error bands represent $\pm 1$ standard error.}

  \label{fig:ntu_bounds_over_years}
\end{figure}

The increase in non-take-up in the years preceding 2021 could potentially reflect behavioural or institutional factors such as changes in awareness, perceived complexity, or attitudes toward debt. 
It could also be partly driven by policy changes. 
Several BAföG reforms were introduced during this period, including increases in grant amounts and adjustments to income thresholds, which may have influenced both eligibility and the perceived attractiveness of the program. 
Since the simulation accounts for these legal changes, the results capture not only behavioural responses, but also how reforms may have affected take-up incentives over time.


The fourth column in Table~\ref{table:microsimulation_ntu} shows the estimated beta error, which is the share of students who are classified as ineligible by the simulation but report receiving BAföG. 
On average, the beta error is approximately 15\% across the full period. 
This degree of misclassification is similar to what has been observed in other studies of non-take-up, where issues such as income reporting errors and timing mismatches are common \citep{frick_claim_2007}.
While this level of beta error is not negligible, the simulation seems to capture eligibility status fairly well overall, even if some noise is inevitable.

Taken together, the results suggest that a large share of eligible students do not take up BAföG, and that this has been the case fairly consistently over time. 
The high average non-take-up rate, of around 60\%, points to persistent barriers such as lack of information or procedural hurdles. 
The financial attractiveness of BAföG may also be a factor. 
Although support amounts were increased at several points, the need-based allowances have consistently failed to keep pace with the actual cost of living for students \citep{staack_von_2017}. 
This could help explain why some students perceive the benefit as not worth the effort of applying. 
%These findings underline the importance of outreach efforts and suggest that further reforms may be needed to make the program more accessible and appealing.


\subsubsection{Stability of Simulated Non-Take-Up under Income Noise}
Table~\ref{tab:conditional_probs_noise} reports conditional probabilities of take-up behaviour under varying levels of artificially introduced measurement errors in income. 
To evaluate the robustness of our non-take-up classification, we add normally distributed noise to the log-transformed income variables before recalculating theoretical BAföG entitlements and eligibility indicators. 
The standard deviation of this noise ranges from 0\% (baseline) to 30\%.

\begin{table}[htbp]
\setlength{\tabcolsep}{10pt}
\small
\centering
\caption*{Non‑Take‑Up and Beta Error Rates by Survey Year and Noise Level}
\begin{tabular}{l|cc|cc|cc|cc}
\toprule
Year 
  & \multicolumn{2}{c|}{0\%} 
  & \multicolumn{2}{c|}{10\%} 
  & \multicolumn{2}{c|}{20\%} 
  & \multicolumn{2}{c}{30\%} \\
     & NTU & $\beta$ 
     & NTU & $\beta$ 
     & NTU & $\beta$ 
     & NTU & $\beta$ \\
\midrule
2007 & 60.6 & 13.6 & 55.74 & 13.13 & 57.38 & 13.43 & 57.81 & 13.25 \\
2008 & 63.5 & 17.1 & 65.12 & 17.45 & 65.22 & 17.12 & 64.84 & 17.07 \\
2009 & 61.0 & 18.6 & 60.49 & 18.57 & 63.86 & 19.08 & 61.73 & 18.63 \\
2010 & 60.9 & 17.7 & 61.36 & 17.78 & 60.92 & 17.72 & 61.36 & 17.78 \\
2011 & 53.8 & 16.1 & 55.07 & 16.12 & 55.88 & 16.73 & 55.72 & 17.14 \\
2012 & 51.5 & 18.9 & 51.49 & 19.13 & 52.24 & 19.50 & 52.67 & 20.07 \\
2013 & 50.0 & 15.9 & 50.69 & 16.04 & 52.78 & 16.98 & 53.06 & 16.83 \\
2014 & 55.1 & 16.1 & 55.15 & 16.14 & 55.15 & 16.14 & 55.97 & 16.73 \\
2015 & 64.0 & 12.6 & 64.80 & 13.01 & 63.20 & 12.20 & 63.03 & 12.75 \\
2016 & 56.5 & 12.4 & 56.14 & 12.34 & 58.33 & 12.66 & 56.90 & 12.45 \\
2017 & 62.6 & 10.1 & 62.42 & 9.76  & 63.09 & 10.16 & 64.05 & 10.33 \\
2018 & 63.9 & 15.3 & 67.11 & 16.92 & 66.67 & 16.60 & 67.10 & 16.53 \\
2019 & 67.5 & 11.7 & 67.24 & 11.41 & 66.67 & 11.07 & 66.39 & 10.77 \\
2020 & 63.7 & 13.6 & 65.22 & 14.07 & 65.52 & 14.12 & 69.17 & 15.50 \\
2021 & 66.7 & 12.3 & 66.07 & 11.87 & 66.96 & 12.39 & 65.46 & 11.82 \\
\midrule
Total
     & 59.7 & 15.0 & 60.13 & 15.10 & 60.73 & 15.25 & 60.94 & 15.37 \\
\bottomrule
\end{tabular}
\caption{\small{Non‑take‑up (NTU) and beta error ($\beta$) rates by survey year and noise level (\%)}}
\label{tab:conditional_probs_noise}
\end{table}


The findings reveal that simulated take-up probabilities remain highly stable despite income misreporting. 
Across all survey years, the probability of non-take-up changes only slightly, even at the highest noise level. 

This robustness indicates that small to moderate errors in reported income do not significantly impact eligibility classification or population-level take-up estimates. 
It also reflects characteristics of the BAföG eligibility formula, where income thresholds, flat regions, and buffers reduce the sensitivity to minor income fluctuations.

In summary, this analysis reinforces the reliability of our microsimulation approach, demonstrating that the classification of non-take-up is not overly sensitive to realistic levels of income measurement errors.



\subsection{Determinants of Non-take-up: Binary Choice Models}
Table \ref{tab:logit_probit_lpm_results} gives an overview of coefficients and average marginal effects (AME's)\footnote{Where applicable.} of our Logit, Probit and Linear Probability Model.


As shown in Table \ref{tab:logit_probit_lpm_results}, all three models consistently indicate that a 100 EUR increase in the simulated BAföG entitlement reduces the probability of non-take-up by approximately two to three percentage points. 
While the Logit and Probit models require average marginal effects for direct probability interpretation due to their nonlinear link functions, the Linear Probability Model coefficients represent approximate percentage point changes directly. 
Differences in magnitude between models are minor and expected given their distinct functional forms.

\begin{table}
\caption{$\Pr(\mathrm{NTU} = 1 \mid \mathbf{X})$}
\renewcommand{\arraystretch}{1.25}
\footnotesize
\centering
\begin{tabular}{lllllllll}
\toprule
 & \multicolumn{4}{c}{Logit} & \multicolumn{4}{c}{Probit} \\
\cmidrule(lr){2-5} \cmidrule(lr){6-9}
 & Coef. & SE & AME & SE & Coef. & SE & AME & SE \\
\midrule
\multicolumn{9}{l}{\textbf{Main explanatory variables}} \\
Simulated BAföG amount$^{\circ}$ & -0.160*** & 0.058 & -0.029*** & 0.010 & -0.095*** & 0.034 & -0.030*** & 0.010 \\
\midrule
\multicolumn{9}{l}{\textbf{Controls: Demographics}} \\
Age & 0.099*** & 0.019 & 0.018*** & 0.003 & 0.058*** & 0.011 & 0.018*** & 0.003 \\
Female & -0.059 & 0.256 & -0.011 & 0.047 & -0.020 & 0.149 & -0.006 & 0.046 \\
Has partner & 1.429* & 0.810 & 0.262* & 0.149 & 0.874** & 0.444 & 0.271** & 0.137 \\
Direct Migration background & -0.700* & 0.378 & -0.128* & 0.068 & -0.419* & 0.219 & -0.130* & 0.067 \\
Indirect Migration background & -0.689** & 0.299 & -0.127** & 0.053 & -0.407** & 0.179 & -0.126** & 0.054 \\
\midrule
\multicolumn{9}{l}{\textbf{Controls: Household and Socioeconomic Background}} \\
Living at parents’ home & -0.019 & 0.270 & -0.004 & 0.049 & -0.008 & 0.160 & -0.002 & 0.050 \\
Sibling claimed BAföG before & -0.554* & 0.285 & -0.102** & 0.051 & -0.321* & 0.171 & -0.100* & 0.052 \\
East background & -1.253*** & 0.313 & -0.230*** & 0.052 & -0.749*** & 0.186 & -0.232*** & 0.054 \\
Parents are highly educated & -0.015 & 0.293 & -0.003 & 0.054 & 0.004 & 0.175 & 0.001 & 0.054 \\
\midrule
\multicolumn{9}{l}{\textbf{Controls: Behaviour}} \\
Patience & 0.030 & 0.065 & 0.006 & 0.012 & 0.015 & 0.040 & 0.005 & 0.012 \\
Impulsiveness & -0.039 & 0.068 & -0.007 & 0.012 & -0.021 & 0.042 & -0.006 & 0.013 \\
Risk Apetite & -0.022 & 0.037 & -0.004 & 0.007 & -0.014 & 0.021 & -0.004 & 0.007 \\
\midrule
McFadden Pseudo $R^2$ & \multicolumn{4}{l}{0.10} & \multicolumn{4}{l}{0.10} \\
Cox and Snell Pseudo $R^2$ & \multicolumn{4}{l}{0.11} & \multicolumn{4}{l}{0.11} \\
Nagelkerke Pseudo $R^2$ & \multicolumn{4}{l}{0.16} & \multicolumn{4}{l}{0.16} \\
Likelihood Ratio Test & \multicolumn{4}{l}{53.33 (p = 0.00)} & \multicolumn{4}{l}{53.20 (p = 0.00)} \\
Observations & \multicolumn{8}{l}{458} \\
\bottomrule
\end{tabular}
\caption*{Logit and Probit Coefficients and Average Marginal Effects}
\label{tab:logit_probit_results}
\caption*{\small{Notes: Significance levels: $^{{*}} p < 0.1$, $^{{**}} p < 0.05$, $^{{***}} p < 0.01$. Robust standard errors clustered at the student level. $^\dagger$ Indicates that the variable has been log-transformed. $\circ$ Indicates per 100 EUR.}}
\end{table}


% Looking at the demographic and socioeconomic predictors, we find that older individuals and those with a registered partner are significantly more likely to forego BAföG despite being eligible. 

The model suggests that age is also a strong predictor of non-take-up in our models: each additional year of age increases the probability of not claiming BAföG by approximately 1.8 percentage points (Logit/Probit AMEs), and 3.7 percentage points in the LPM. 
This positive relationship is consistent with \cite{konijn_quantifying_2023}, who find higher non-take-up with age in the Netherlands. 
In contrast, \cite{herber_non-take-up_2019} report no significant age effect for BAföG in Germany, and \cite{fuchs_austria_2007} find lower non-take-up among older individuals in Austria.\footnote{Fuchs (2007) find a significant negative age effect only in the selection stage of their Heckman model.}
\cite{frick_claim_2007} do not report a consistent or significant effect of age across specifications.

Similarly, the presence of a registered partner corresponds to a substantially higher likelihood of non-take-up, with marginal effects indicating an increase of around 26 (16 in LPM) percentage points. 
These patterns may reflect lower BAföG entitlements among these groups, reducing the perceived benefit of applying. 


Moreover, individuals with direct or indirect migration backgrounds are significantly less likely to refuse BAföG. 
This aligns with \cite{herber_non-take-up_2019} for Germany\footnote{Though not statistically significant.} and \cite{konijn_quantifying_2023} for student aid in the Netherlands. 
By contrast, \cite{fuchs_austria_2007} find no significant effect for social assistance in Austria. 
Similarly, \cite{frick_claim_2007} report mixed results for Germany, with a significant effect only in their Heckman selection model.

Furthermore, all three models suggests that whether an individual lives at home or has moved out from their parents does not, in itself, have a statistically significant effect on the likelihood of BAföG take-up. 
By contrast, informational factors appear to play a more substantial role: having a sibling who has previously received BAföG is associated with a significantly lower probability of non-take-up, by approximately 10 percentage points across all three models. 
This finding highlights the importance of informational spillovers within families in shaping take-up behaviour. 

The models further indicate that individuals with an East German background are significantly more likely to take up BAföG, with non-take-up probabilities reduced by approximately 23 to 25 percentage points across all specifications. 
This finding is consistent with \cite{herber_non-take-up_2019} and \cite{harnisch_nontakeup_2019}, who also report higher take-up rates among individuals from East Germany. 
Detailed descriptive statistics can be found in Appendix~\ref{appendix:tables}, Table~\ref{table:take-up-rates-grouped}.

By contrast, parental education appears to have no meaningful influence on take-up, as the estimated coefficients are statistically insignificant and the AMEs are close to zero across all models, suggesting no substantive effect.

Finally, we do not find any statistically significant effects of the behavioural predictors. 
Specifically, patience, impulsiveness, and self-assessed risk appetite do not have a significant impact on the probability of non-take-up. 
This is in line with \cite{herber_non-take-up_2019}, who also report no significant main effects. 
However, they find statistical significance when interacting impatience and impulsiveness. 
We tested a similar interaction term but found no significant effect. 
Across all models, the estimated coefficients are small and not distinguishable from zero, suggesting that these behavioural traits do not systematically influence BAföG take-up decisions in our sample.


\subsubsection{Restricting to Higher Entitlements}
Our baseline analysis includes all students with any positive simulated BAföG entitlement. 
To check whether our results are driven by those with very small entitlements, who might not consider applying to be worth the effort, we rerun the models restricting the sample to students with simulated monthly entitlements of at least 200 EUR.

As shown in Appendix~\ref{appendix:tables}, Table~\ref{tab:logit_probit_lpm_results_200}, the main findings hold. The negative link between entitlement size and non-take-up remains strong and significant. Similarly, age, partnership status, and East German background continue to show consistent effects. This indicates that the main findings are not just driven by students with minimal entitlements, but also reflect broader patterns in take-up behaviour among those eligible for more substantial benefits.

\section{Discussion}
This paper first implements a microsimulation of BAföG eligibility to estimate non-take-up rates for the period 2007–2021. 
We then use three model specifications, Logit, Probit, and a Linear Probability Model (LPM), to analyse the determinants of benefit take-up and investigate the underlying causes of non-participation among eligible students. 
Our preferred models achieve an accuracy of approximately 72\%, indicating a reliable fit to the observed data.


Our microsimulation shows that, across years 2007–2021 and under plausible measurement error, the probability of non-take-up remains high (around 60\%). 
Our econometric models help explain this pattern by identifying predictors of non-take-up: a higher simulated BAföG amount is associated with significantly lower non-take-up, and factors such as age, partnership status, and migration background emerge as robust correlates. 
In contrast, psychological traits such as impulsiveness, debt aversion, and risk appetite are not statistically significant, suggesting that structural and informational barriers, not behavioural predispositions, are the key drivers.

Our study relies on survey data, which inherently suffers from measurement errors due to the use of proxy variables, missing data, and potential reporting biases. 
These data limitations may introduce estimation errors, including the possibility of \( \beta \)-errors in identifying determinants of non-take-up. 
Nevertheless, the 72\% accuracy of our predictive model suggests a reasonably strong fit to the observed data. 
While we cannot fully rule out that the exact magnitude of non-take-up rates might be somewhat imprecise, the overall message remains clear: the core goal of BAföG, targeting financial support to those most in need, is not yet fully achieved.

Although Germany’s student aid offers favourable terms and uses strict means-testing to target those truly in need, the analysis suggests it is not necessarily the most financially disadvantaged students who benefit. 
Instead, those with higher social capital and a more favourable view of government intervention are more likely to make use of the support.

Importantly, non-take-up is not a uniform phenomenon. 
While it might be tempting to attribute low take-up to personal traits, our results show no statistical significance for impulsiveness, debt aversion, or risk appetite. 
This suggests that structural and informational barriers—such as application complexity, lack of awareness, or uncertainty about eligibility—play the primary role. 
Reducing non-take-up will therefore require a combination of administrative simplification, clearer communication and targeted outreach.

\subsection{Policy Implications}

Addressing non-take-up of social benefits is politically sensitive. Policies promoting increased take-up are often seen as calls for higher public spending and can face resistance—even when aimed at improving fairness within existing budgets.

BAföG’s core goal is to prevent financial constraints from blocking access to higher education. Yet, our findings indicate that the application process places a heavy administrative burden on students, particularly deterring those eligible for smaller amounts. This outcome contradicts the programme’s mission to support less affluent students.

To better target financial need, policymakers should simplify the means-testing process. While income verification is essential for precise targeting, the procedure must not be so complex or time-consuming that it discourages eligible applicants. A system this complicated naturally favours applicants with greater social capital and motivation over those who need support most.

Currently, applying demands detailed disclosure of fourteen income types and sixteen categories of assets and debts. Moreover, parents complete a comprehensive four-page form about income and siblings \citep{fidan_why_2021}. This complexity likely excludes economically disadvantaged families who may lack the resources or motivation to navigate the process.

Simplification, such as increased use of pre-filled tax data and automated eligibility checks, could balance accuracy with accessibility.

Ultimately, non-take-up caused by administrative complexity reflects a design flaw rather than a failure of intent. A more user-friendly BAföG could improve both equity and efficiency, ensuring aid reaches those most in need without increasing overall costs.




%%%%%%%%%%%%%%%%%%%%%%%%%%%%%%%%
% REFERENCES
%%%%%%%%%%%%%%%%%%%%%%%%%%%%%%%%
\newpage
\begin{flushleft}
    \renewcommand{\baselinestretch}{1.1}
    \small
    \titleformat{\section}{\normalfont\huge\bfseries}{\thesection}{1em}{} 
    \addcontentsline{toc}{section}{References} 
    \bibliography{bibliography, bafogandg, estg}
\end{flushleft}

%APPENDICES
\newpage
\appendix
\setcounter{page}{1}        % Restarts page counting to 1
\pagenumbering{roman}       % Differentiates the appendix page numbers from the main article

\titleformat{\section}{\centering\normalfont\normalsize\bfseries}{Appendix \thesection: }{0em}{}

\begin{landscape}
\section{Tables} \label{appendix:tables}
\renewcommand{\thetable}{\thesection \arabic{table}}
\setcounter{table}{0}

\begin{table}[H]
\centering
\label{tab:nontakeup_hgtyp_migback_sex_east_siblings}
\begin{tabular}{lccccccccccccccc}
\toprule
Household Type & 2007 & 2008 & 2009 & 2010 & 2011 & 2012 & 2013 & 2014 & 2015 & 2016 & 2017 & 2018 & 2019 & 2020 & 2021 \\
\midrule
1-Person Household      & 52.38 & 52.38 & 64.71 & 72.22 & 60.00 & 38.71 & 41.18 & 37.93 & 56.00 & 55.56 & 46.67 & 64.29 & 54.55 & 59.38 & 63.33 \\
Couple Without Children & 80.00 & 50.00 & 75.00 & 100.00 & 75.00 & 50.00 & 63.64 & 88.89 & 36.36 & 85.71 & 69.23 & 46.15 & 60.00 & 90.00 & 71.43 \\
Couple With Children    & 66.67 & 70.83 & 56.86 & 56.86 & 51.90 & 56.96 & 51.69 & 59.49 & 71.43 & 53.13 & 66.67 & 66.67 & 73.24 & 67.27 & 77.27 \\
\midrule
Migration Background & 2007 & 2008 & 2009 & 2010 & 2011 & 2012 & 2013 & 2014 & 2015 & 2016 & 2017 & 2018 & 2019 & 2020 & 2021 \\
\midrule
No migration background   & 61.22 & 64.41 & 66.10 & 65.08 & 53.33 & 53.47 & 52.44 & 59.04 & 65.85 & 62.32 & 67.39 & 63.04 & 67.11 & 63.29 & 65.43 \\
With migration background & 58.82 & 61.54 & 47.83 & 50.00 & 54.76 & 45.71 & 46.67 & 49.06 & 60.47 & 47.83 & 54.55 & 65.46 & 68.42 & 64.71 & 70.00 \\
\midrule
Sex & 2007 & 2008 & 2009 & 2010 & 2011 & 2012 & 2013 & 2014 & 2015 & 2016 & 2017 & 2018 & 2019 & 2020 & 2021 \\
\midrule
Male   & 68.97 & 69.44 & 58.54 & 69.23 & 50.00 & 53.73 & 47.89 & 53.52 & 61.91 & 64.00 & 70.31 & 64.79 & 64.29 & 58.33 & 68.00 \\
Female & 54.05 & 59.18 & 63.42 & 54.17 & 56.94 & 49.28 & 52.11 & 56.92 & 66.13 & 50.77 & 56.63 & 63.16 & 70.69 & 67.69 & 65.57 \\
\midrule
Region & 2007 & 2008 & 2009 & 2010 & 2011 & 2012 & 2013 & 2014 & 2015 & 2016 & 2017 & 2018 & 2019 & 2020 & 2021 \\
\midrule
West Germany & 73.17 & 69.70 & 67.21 & 62.50 & 56.70 & 59.14 & 50.00 & 59.63 & 67.00 & 62.22 & 67.80 & 67.23 & 70.97 & 71.11 & 72.62 \\
East Germany & 40.00 & 42.11 & 42.86 & 56.52 & 45.71 & 34.88 & 50.00 & 37.04 & 52.00 & 36.00 & 41.38 & 50.00 & 52.38 & 34.78 & 48.15 \\
\midrule
Sibling BAföG history & 2007 & 2008 & 2009 & 2010 & 2011 & 2012 & 2013 & 2014 & 2015 & 2016 & 2017 & 2018 & 2019 & 2020 & 2021 \\
\midrule
No sibling received BAföG & 65.52 & 63.33 & 53.85 & 60.87 & 64.87 & 61.29 & 56.76 & 73.17 & 58.97 & 72.41 & 70.27 & 52.50 & 75.68 & 66.67 & 63.33 \\
Sibling received BAföG    & 61.54 & 64.71 & 46.15 & 53.33 & 48.48 & 32.14 & 37.50 & 56.00 & 68.18 & 44.44 & 44.74 & 63.64 & 60.00 & 43.75 & 55.56 \\
\bottomrule
\end{tabular}
\caption{Non-take-up rates by household type, migration background, sex, region, number of siblings, and sibling BAföG history.}
\end{table}
\end{landscape}

\begingroup
\setlength{\tabcolsep}{4pt}
\renewcommand{\arraystretch}{1.25}
\begin{table}[H]
  \scriptsize
  \centering
  \begin{tabular}{rcc|rr|rr}
  \toprule
  Year & \multicolumn{2}{c|}{Consumer Price Index} & \multicolumn{2}{c|}{Average Payout (EUR)} & \multicolumn{2}{c}{Financial Expenditure (EUR 1,000)} \\
  \cmidrule(lr){2-3} \cmidrule(lr){4-5} \cmidrule(l){6-7}
  & Index (2020=100) & Price Factor (2023) & Nominal & Real (2023) & Nominal & Real (2023) \\
  \midrule
  1991 & 61 & 1.885 & 290 & 547 & 1,538,590 & 2,900,701 \\
  1992 & 65 & 1.795 & 290 & 521 & 1,539,929 & 2,764,764 \\
  1993 & 67 & 1.719 & 297 & 510 & 1,458,164 & 2,506,152 \\
  1994 & 69 & 1.674 & 295 & 494 & 1,257,002 & 2,104,621 \\
  1995 & 71 & 1.644 & 304 & 500 & 1,133,989 & 1,863,894 \\
  1996 & 72 & 1.621 & 322 & 522 & 1,059,270 & 1,716,900 \\
  1997 & 73 & 1.590 & 319 & 507 & 910,038 & 1,446,886 \\
  1998 & 74 & 1.577 & 316 & 498 & 861,688 & 1,358,905 \\
  1999 & 74 & 1.566 & 321 & 503 & 871,140 & 1,364,591 \\
  2000 & 75 & 1.546 & 326 & 504 & 906,857 & 1,401,724 \\
  2001 & 77 & 1.516 & 365 & 553 & 1,161,922 & 1,760,990 \\
  2002 & 78 & 1.494 & 371 & 554 & 1,350,543 & 2,018,032 \\
  2003 & 78 & 1.479 & 370 & 547 & 1,446,120 & 2,138,937 \\
  2004 & 80 & 1.455 & 371 & 540 & 1,513,641 & 2,202,517 \\
  2005 & 81 & 1.432 & 375 & 537 & 1,554,602 & 2,226,037 \\
  2006 & 82 & 1.409 & 375 & 529 & 1,538,770 & 2,168,773 \\
  2007 & 84 & 1.378 & 375 & 517 & 1,490,718 & 2,053,917 \\
  2008 & 86 & 1.343 & 398 & 534 & 1,590,638 & 2,136,104 \\
  2009 & 87 & 1.338 & 434 & 581 & 1,875,731 & 2,510,295 \\
  2010 & 88 & 1.325 & 436 & 578 & 2,019,078 & 2,674,533 \\
  2011 & 90 & 1.297 & 452 & 586 & 2,269,706 & 2,943,052 \\
  2012 & 91 & 1.273 & 448 & 570 & 2,364,963 & 3,009,718 \\
  2013 & 93 & 1.253 & 446 & 559 & 2,349,400 & 2,944,951 \\
  2014 & 94 & 1.241 & 448 & 556 & 2,280,748 & 2,831,524 \\
  2015 & 94 & 1.235 & 448 & 553 & 2,157,634 & 2,664,506 \\
  2016 & 95 & 1.228 & 464 & 570 & 2,099,110 & 2,578,590 \\
  2017 & 96 & 1.211 & 499 & 604 & 2,181,049 & 2,640,336 \\
  2018 & 98 & 1.190 & 493 & 586 & 2,001,732 & 2,381,265 \\
  2019 & 99 & 1.173 & 514 & 603 & 1,954,449 & 2,292,303 \\
  2020 & 100 & 1.167 & 574 & 670 & 2,210,920 & 2,580,143 \\
  2021 & 103 & 1.132 & 579 & 655 & 2,316,926 & 2,622,553 \\
  2022 & 110 & 1.059 & 611 & 647 & 2,454,392 & 2,599,161 \\
  2023 & 116 & 1.000 & 663 & 663 & 2,863,514 & 2,863,514 \\
  \bottomrule
  \end{tabular}
  \caption{
    Average nominal and inflation-adjusted payout under the Federal Training Assistance Act (BAföG) 
    for student recipients (excluding pupils), based on official data published by Destatis.  
    The table includes the Consumer Price Index (CPI, variable \textbf{PREIS1}, base year 2020 = 100) and 
    a derived price factor (column “Factor (2023)”) calculated using these CPI values to express nominal amounts in 2023 euros.  
    The inflation-adjusted average payouts and total financial expenditures were computed using this deflator and are not 
    reported as such in the original Destatis tables.
  }
  \label{table:payout_over_time}
  \end{table}
\endgroup

\setlength{\tabcolsep}{4pt}
\renewcommand{\arraystretch}{1.25}
\begin{table}
\centering
\begin{tabular}{rrrrrrrr}
\toprule
Year & Students & \multicolumn{3}{|c|}{Number of Supported Students} & \multicolumn{3}{c}{Proportion Supported (\%)} \\
\midrule
 & & Total Supported & Fully Supported & Partially Supported & Total & Fully & Partially \\
\midrule
2023 & 2,868,311 & 501,425 & 245,255 & 256,170 & 17.5 & 8.6 & 8.9 \\
2022 & 2,920,263 & 489,347 & 244,559 & 244,788 & 16.8 & 8.4 & 8.4 \\
2021 & 2,941,915 & 467,595 & 200,369 & 267,226 & 15.9 & 6.8 & 9.1 \\
2020 & 2,944,145 & 465,543 & 205,093 & 260,450 & 15.8 & 7.0 & 8.8 \\
2019 & 2,891,049 & 489,313 & 212,217 & 277,096 & 16.9 & 7.3 & 9.6 \\
2018 & 2,868,222 & 517,675 & 218,427 & 299,248 & 18.0 & 7.6 & 10.4 \\
2017 & 2,844,978 & 556,573 & 229,053 & 327,520 & 19.6 & 8.1 & 11.5 \\
2016 & 2,807,010 & 583,567 & 235,163 & 348,404 & 20.8 & 8.4 & 12.4 \\
2015 & 2,757,799 & 611,377 & 231,477 & 379,900 & 22.2 & 8.4 & 13.8 \\
2014 & 2,698,910 & 646,576 & 246,901 & 399,675 & 24.0 & 9.1 & 14.8 \\
2013 & 2,616,881 & 665,928 & 253,371 & 412,557 & 25.4 & 9.7 & 15.8 \\
2012 & 2,499,409 & 671,042 & 254,769 & 416,273 & 26.8 & 10.2 & 16.7 \\
2011 & 2,380,974 & 643,578 & 246,895 & 396,683 & 27.0 & 10.4 & 16.7 \\
2010 & 2,217,294 & 592,430 & 232,796 & 359,633 & 26.7 & 10.5 & 16.2 \\
2009 & 2,121,178 & 550,369 & 211,881 & 338,488 & 25.9 & 10.0 & 16.0 \\
2008 & 2,025,307 & 510,409 & 217,933 & 292,476 & 25.2 & 10.8 & 14.4 \\
2007 & 1,941,405 & 494,480 & 191,268 & 303,212 & 25.5 & 9.9 & 15.6 \\
2006 & 1,979,043 & 498,565 & 189,022 & 309,543 & 25.2 & 9.6 & 15.6 \\
2005 & 1,985,765 & 506,880 & 193,285 & 313,595 & 25.5 & 9.7 & 15.8 \\
2004 & 1,963,108 & 497,257 & 186,956 & 310,301 & 25.3 & 9.5 & 15.8 \\
2003 & 2,019,465 & 481,594 & 179,755 & 301,839 & 23.8 & 8.9 & 14.9 \\
2002 & 1,938,811 & 451,505 & 168,890 & 282,615 & 23.3 & 8.7 & 14.6 \\
2001 & 1,868,331 & 406,776 & 134,933 & 271,843 & 21.8 & 7.2 & 14.6 \\
2000 & 1,798,863 & 348,799 & 100,913 & 247,886 & 19.4 & 5.6 & 13.8 \\
1999 & 1,770,489 & 338,427 & 103,239 & 235,188 & 19.1 & 5.8 & 13.3 \\
1998 & 1,800,651 & 336,355 & 97,539 & 238,810 & 18.7 & 5.4 & 13.3 \\
\bottomrule
\end{tabular}
\caption{
  Number and percentage of students receiving BAföG support.  
  Columns:  
  \textbf{BIL002} = total number of students;  
  \textbf{PER010} = total supported students;  
  \textbf{PER011} = fully supported students;  
  \textbf{PER012} = partially supported students.
}
\label{table:bafoeg_support_landscape}
\end{table}

% # Variable codes 
% # PER 010 | Supported persons
% # PER 011 | Persons receiving full assistance payments
% # PER 012 | Persons receiving partial assistance payments
% # PER 013 | Supported persons (average monthly stock)
% # PER 014 | Average monthly assistance payment per person




\begin{table}
\renewcommand{\arraystretch}{1.2}
\footnotesize
\caption*{$\Pr(\mathrm{NTU} = 1 \mid \mathbf{X})$}
% \renewcommand{\arraystretch}{1.25}
\centering
\begin{tabular}{lccccc}
\toprule
& \multicolumn{2}{c}{Logit} & \multicolumn{2}{c}{Probit} & LPM \\
& Coef. & AME & Coef. & AME & Coef. \\
\midrule
\multicolumn{6}{l}{\textbf{Main explanatory variables}} \\
Simulated BAföG amount$^{\circ}$ & -0.181** & -0.034** & -0.109** & -0.034** & -0.022 \\
 & (0.077) & (0.014) & (0.045) & (0.014) & (0.014) \\
\midrule
\multicolumn{6}{l}{\textbf{Controls: Demographics}} \\
Age & 0.101*** & 0.019*** & 0.060*** & 0.019*** & 0.037*** \\
 & (0.024) & (0.004) & (0.014) & (0.004) & (0.004) \\
Female & 0.007 & 0.001 & 0.020 & 0.006 & 0.016 \\
 & (0.290) & (0.054) & (0.169) & (0.053) & (0.053) \\
Has partner & 1.480* & 0.277* & 0.906* & 0.286* & 0.190* \\
 & (0.882) & (0.164) & (0.483) & (0.151) & (0.102) \\
Direct Migration background & -0.347 & -0.065 & -0.195 & -0.061 & -0.070 \\
 & (0.450) & (0.084) & (0.262) & (0.082) & (0.075) \\
Indirect Migration background & -0.740** & -0.138** & -0.439** & -0.138** & -0.133** \\
 & (0.324) & (0.058) & (0.194) & (0.059) & (0.064) \\
\midrule
\multicolumn{6}{l}{\textbf{Controls: Household and Socioeconomic Background}} \\
Living at parents’ home & 0.070 & 0.013 & 0.050 & 0.016 & 0.054 \\
 & (0.311) & (0.058) & (0.183) & (0.058) & (0.056) \\
Sibling claimed BAföG before & -0.633* & -0.119** & -0.368* & -0.116* & -0.128* \\
 & (0.331) & (0.060) & (0.198) & (0.061) & (0.066) \\
East background & -1.437*** & -0.269*** & -0.865*** & -0.273*** & -0.300*** \\
 & (0.369) & (0.061) & (0.218) & (0.062) & (0.072) \\
Parents are highly educated & 0.002 & 0.000 & 0.009 & 0.003 & 0.015 \\
 & (0.360) & (0.067) & (0.213) & (0.067) & (0.066) \\
\midrule
\multicolumn{6}{l}{\textbf{Controls: Behaviour}} \\
Patience & 0.042 & 0.008 & 0.021 & 0.007 & 0.006 \\
 & (0.072) & (0.013) & (0.044) & (0.014) & (0.013) \\
Impulsiveness & -0.047 & -0.009 & -0.024 & -0.008 & -0.008 \\
 & (0.075) & (0.014) & (0.046) & (0.015) & (0.013) \\
Risk Apetite & -0.014 & -0.003 & -0.008 & -0.003 & -0.000 \\
 & (0.042) & (0.008) & (0.025) & (0.008) & (0.007) \\
\midrule
McFadden Pseudo $R^2$ & \multicolumn{2}{l}{0.11} & \multicolumn{2}{l}{0.11} & \\
Cox and Snell Pseudo $R^2$ & \multicolumn{2}{l}{0.13} & \multicolumn{2}{l}{0.13} & \\
Nagelkerke Pseudo $R^2$ & \multicolumn{2}{l}{0.18} & \multicolumn{2}{l}{0.18} & \\
Likelihood Ratio Test & \multicolumn{2}{l}{48.46 (p = 0.00)} & \multicolumn{2}{l}{48.39 (p = 0.00)} & \\
Adjusted $R^2$ & & & & & 0.72 \\
F-statistic & & & & & 71.5 (p = 0.00) \\
Observations & \multicolumn{5}{l}{352} \\
\bottomrule
\end{tabular}
\caption{\small{Estimates corresponding to Table \ref{tab:logit_probit_lpm_results}, 
using the same model specifications but classifying students as eligible if their theoretical entitlement exceeds 200 EUR.}}
\label{tab:logit_probit_lpm_results_200}
\caption*{\footnotesize{Notes: Significance levels: $^{{*}} p < 0.1$, $^{{**}} p < 0.05$, $^{{***}} p < 0.01$. Robust standard errors clustered at the student level. $\circ$ Indicates per 100 EUR.}}
\end{table}




\newpage
\section{Figures}
\renewcommand{\thefigure}{\thesection \arabic{figure}}
\setcounter{figure}{0}

\begin{figure}[htbp]
  \centering
  \begin{subfigure}[t]{0.48\linewidth}
    \includegraphics[width=\linewidth]{parental_joint_excess_income_pdf.png}
    \caption{Parental joint excess income}
    \label{fig:parental-excess}
  \end{subfigure}
  \hfill
  \begin{subfigure}[t]{0.48\linewidth}
    \includegraphics[width=\linewidth]{student_excess_income_pdf.png}
    \caption{Student excess income}
    \label{fig:student-excess}
  \end{subfigure}
  \caption{Simulated mean excess income for parents (\subref{fig:parental-excess}) and students (\subref{fig:student-excess}).}
  \label{fig:excess-income}
\end{figure}


\begin{figure}[htbp]
  \centering
  \includegraphics[width=0.95\linewidth]{theo_vs_reported_distribution.png}
  \caption{Comparison of the distribution of reported BAföG receipt in the SOEP-Core sample with the simulated (theoretical) distribution of simulated BAföG entitlements from our model.}
  \label{fig:theo-vs-reported}
\end{figure}


% \begin{figure}[H]
%   \centering
%
%   \begin{minipage}[t]{0.48\textwidth}
%     \centering
%     \includegraphics[width=0.95\linewidth]{fraction_of_enrolled_students_receiving_bafog.png}
%     \caption{
%       Fraction of enrolled students in Germany receiving partial, full, or combined BAföG support (loans and grants). Based on official statistics from Destatis. \textit{Own illustration}.
%     }
%     \label{figure:bafoeg_support}
%   \end{minipage}%
%   \hfill
%   \begin{minipage}[t]{0.48\textwidth}
%     \centering
%     \includegraphics[width=0.95\linewidth]{payout_over_time.png}
%     \caption{
%       Average nominal and real monthly BAföG payout for students (excluding pupils), based on Destatis time series. \textit{Own illustration}.
%     }
%     \label{figure:payout_over_time}
%   \end{minipage}
%
% \end{figure}

%%%%%%%%%%%%%%%%%%%%%%%%%%%%%%%%%%%%%%%%%%%%%%%%%%%%%%%%%%%%%%%%%%%%%%%%%%%%
%  APPENDIX – MICROSIMULATION PIPELINE (Revised Table)
%%%%%%%%%%%%%%%%%%%%%%%%%%%%%%%%%%%%%%%%%%%%%%%%%%%%%%%%%%%%%%%%%%%%%%%%%%%%

\newpage
% ========================================================================
\section{Dictionary of Variables in the Microsimulation Pipeline}
\label{appendix:variable_dictionary}
% ========================================================================

{\footnotesize
\begin{longtable}{lll}
\caption{Variable Dictionary by Dataset (Excluding Data Type)}
\label{table:variable_dictionary} \\
\toprule
Dataset & Variable & Description \\
\midrule
\endfirsthead

\multicolumn{3}{l}{\textit{(continued from previous page)}} \\
\toprule
Dataset & Variable & Description \\
\midrule
\endhead

\bottomrule
\multicolumn{3}{r}{\textit{(continued on next page)}} \\
\endfoot

\bottomrule
\endlastfoot

\multicolumn{3}{l}{\textbf{IDENTIFIERS AND CORE DEMOGRAPHICS}} \\
\texttt{ppathl} & \texttt{pid} & Person identifier \\
\texttt{ppathl} & \texttt{hid} & Household ID \\
\texttt{ppathl} & \texttt{syear} & Survey year \\
\texttt{ppathl} & \texttt{gebjahr} & Year of birth \\
\texttt{ppathl} & \texttt{gebmonat} & Month of birth \\
\texttt{ppathl} & \texttt{sex} & Sex \\
\texttt{ppathl} & \texttt{partner} & Partnership status \\
\texttt{ppathl} & \texttt{migback} & Migration background \\
\texttt{regionl} & \texttt{hid} & Household ID \\
\texttt{regionl} & \texttt{syear} & Survey year \\
\texttt{regionl} & \texttt{bula} & Federal state (Bundesland) \\

\midrule
\multicolumn{3}{l}{\textbf{EDUCATION}} \\
\texttt{pl} & \texttt{pid} & Person identifier \\
\texttt{pl} & \texttt{syear} & Survey year \\
\texttt{pl} & \texttt{plg0012\_h} & Currently in education \\
\texttt{pl} & \texttt{plg0014\_v5} & Education level, 1999--2008 \\
\texttt{pl} & \texttt{plg0014\_v6} & Education level, 2009--2012 \\
\texttt{pl} & \texttt{plg0014\_v7} & Education level, 2013--2021 \\

\midrule
\multicolumn{3}{l}{\textbf{RELIGION AND STUDENT AID}} \\
\texttt{pl} & \texttt{plh0258\_h} & Religion / church membership \\
\texttt{pl} & \texttt{plc0167\_h} & BAföG eligibility \\
\texttt{pl} & \texttt{plc0168\_h} & BAföG / scholarship (gross, monthly) \\

\midrule
\multicolumn{3}{l}{\textbf{EMPLOYMENT AND INCOME}} \\
\texttt{pgen} & \texttt{pid} & Person identifier \\
\texttt{pgen} & \texttt{syear} & Survey year \\
\texttt{pgen} & \texttt{pgemplst} & Employment status \\
\texttt{pgen} & \texttt{pgpartnr} & Partner indicator \\
\texttt{biol} & \texttt{pid} & Person identifier \\
\texttt{biol} & \texttt{syear} & Survey year \\
\texttt{biol} & \texttt{lb0267\_v1} & Employment status \\

\midrule
\multicolumn{3}{l}{\textbf{HOUSING AND RENT}} \\
\texttt{pkal} & \texttt{pid} & Person identifier \\
\texttt{pkal} & \texttt{syear} & Survey year \\
\texttt{pkal} & \texttt{kal2a02} & Monthly rent including utilities \\
\texttt{pkal} & \texttt{kal2a03\_h} & Housing benefit \\

\midrule
\multicolumn{3}{l}{\textbf{WEALTH AND ASSETS}} \\
\texttt{pwealth} & \texttt{pid} & Person identifier \\
\texttt{pwealth} & \texttt{syear} & Survey year \\
\texttt{pwealth} & \texttt{f0100a--f0100e} & Financial assets \\
\texttt{pwealth} & \texttt{e0111a--e0111e} & Real estate (net value shares) \\
\texttt{pwealth} & \texttt{b0100a--b0100e} & Business assets \\
\texttt{pwealth} & \texttt{i0100a--i0100e} & Private insurances \\
\texttt{pwealth} & \texttt{v0100a--v0100e} & Vehicles \\
\texttt{pwealth} & \texttt{t0100a--t0100e} & Tangible assets \\
\texttt{pwealth} & \texttt{w0011a--w0011e} & Liabilities and debts \\

\midrule
\multicolumn{3}{l}{\textbf{FAMILY RELATIONSHIPS}} \\
\texttt{biosib} & \texttt{pid} & Person identifier \\
\texttt{biosib} & \texttt{sibpnr1--sibpnr11} & Sibling person numbers \\
\texttt{bioparen} & \texttt{pid} & Person identifier \\
\texttt{bioparen} & \texttt{fnr} & Father’s person ID \\
\texttt{bioparen} & \texttt{mnr} & Mother’s person ID \\

\midrule
\multicolumn{3}{l}{\textbf{HOUSEHOLD COMPOSITION}} \\
\texttt{hgen} & \texttt{hid} & Household ID \\
\texttt{hgen} & \texttt{syear} & Survey year \\
\texttt{hgen} & \texttt{hgtyp1hh} & Household type \\

\end{longtable}
}

%%%%%%%%%%%%%%%%%%%%%%%%%%%%%%%%%%%%%%%%%%%%%%%%%%%%%%%%%%%%%%%%%%%%%%%%%%%%
%  APPENDIX – EXAMPLE FROM MICROSIMULATION
%%%%%%%%%%%%%%%%%%%%%%%%%%%%%%%%%%%%%%%%%%%%%%%%%%%%%%%%%%%%%%%%%%%%%%%%%%%%

% -------------------------------------------------------------------------
%  Introductory motivation 
% -------------------------------------------------------------------------

\newpage
% ========================================================================
\section[Example Calculation: Theoretical BAföG Eligibility]{Example Calculation: Theoretical BAföG Eligibility\footnote{Based on calculations using the microsimulation pipeline introduced in Appendix~\ref{appendix:microsimulation-pipeline}.}}
\label{appendix:bafoeg-example}
% ========================================================================

This appendix documents the step-by-step calculation of theoretical BAföG eligibility for a selected individual from the SOEP-Core dataset. The example is based on data from survey year 2018 and focuses on a university student identified by \texttt{pid = 20156903}.

The purpose of this example is to illustrate how legal rules governing student financial aid—particularly those defined in the Federal Training Assistance Act (BAföG)—are operationalized within the microsimulation pipeline. Each component of the calculation is made transparent, including the determination of the student's assessed need, applicable supplements, and deductions based on income and assets.

The case selected is representative of a full-time student living independently, with modest student income, limited parental support, and non-negligible declared assets. The final theoretical BAföG award is computed by subtracting excess income and asset contributions from the total assessed need.

A summary of the key outcome variables is presented in Table~\ref{table:bafoeg_example_summary}. Subsequent sections decompose and document the logic and parameters behind each component in detail.





% -------------------------------------------------------------------------
%  Base Need
% -------------------------------------------------------------------------


\subsection{Total Base Need}
\subsubsection{Base Need}
The base need (\texttt{base\_need}) is a flat-rate amount representing the monthly minimum subsistence level for students in higher education. It is specified in §~13(1) Nr.~1 of the Federal Training Assistance Act (BAföG) and does not vary by income, living arrangement, or demographic characteristics.

For all eligible university students during the relevant period, the base need was set at 399 EUR. Since the student in this case study meets the criteria for university-level BAföG support, this full amount is assigned without adjustment.

\begin{table}[H]
\scriptsize
\centering
\begin{tabular}{llr}
\toprule
\textbf{Component} & \textbf{Explanation} & \textbf{Value (EUR)} \\
\midrule
Base Need & Flat-rate monthly amount for university students & 399 \\
\bottomrule
\end{tabular}
\caption{Base need (\texttt{base\_need}) for pid 20156903, in accordance with §~13(1) Nr.~1 BAföG.}
\label{table:bafoeg_base_need}
\end{table}

\subsubsection{Housing Allowance}
The housing allowance (\texttt{housing\_allowance}) compensates students for living expenses incurred while living outside the parental home. According to §~13(1) Nr.~2 BAföG, students who do not reside with their parents are entitled to a fixed monthly supplement to cover rent and related costs.

In this example, the student was classified as living independently. While the statutory maximum at the time was 399 EUR, the simulation applies a standardized flat amount of 250 EUR to align with data quality and institutional thresholds reflected in the SOEP housing variables.

\begin{table}[H]
\scriptsize
\centering
\begin{tabular}{llr}
\toprule
\textbf{Component} & \textbf{Explanation} & \textbf{Value (EUR)} \\
\midrule
Housing Allowance & Standard flat rate applied for non-parental housing & 250 \\
\bottomrule
\end{tabular}
\caption{Housing allowance (\texttt{housing\_allowance}) for pid 20156903, based on §~13(1) Nr.~2 BAföG.}
\label{table:bafoeg_housing}
\end{table}

\subsubsection{Insurance Supplement}
Students with statutory health and long-term care insurance are entitled to receive flat-rate supplements as defined in §~13a(1) BAföG. These rates vary by time period and are adjusted periodically by legal amendment.

For survey year 2018, the applicable values—according to the 2020-08-01 rates still valid at the time—were:
\begin{itemize}
    \item 61 EUR for health insurance (§~13a(1) Nr.~1 BAföG)
    \item 25 EUR for long-term care insurance (§~13a(1) Nr.~2 BAföG)
\end{itemize}

These two components sum to 86 EUR, which is assigned as the total insurance supplement for this individual.

\begin{table}[H]
\scriptsize
\centering
\begin{tabular}{llr}
\toprule
\textbf{Component} & \textbf{Explanation} & \textbf{Value (EUR)} \\
\midrule
Insurance Supplement & Sum of flat-rate statutory insurance allowances & 86 \\
\quad Health insurance & §~13a(1) Nr.~1 BAföG (statutory health insurance) & 61 \\
\quad Care insurance & §~13a(1) Nr.~2 BAföG (statutory long-term care insurance) & 25 \\
\bottomrule
\end{tabular}
\caption{Insurance supplement (\texttt{insurance\_supplement}) for pid 20156903. Rates valid for the 2018 survey year.}
\label{table:bafoeg_insurance}
\end{table}


% -------------------------------------------------------------------------
%  Student Excess Income
% -------------------------------------------------------------------------
\subsection{Student Excess Income}
The student’s excess income (\texttt{excess\_income\_stu}) represents the amount by which their own annual income—after standard deductions—exceeds the personal allowance defined under §~23(1) Nr.~1 BAföG. This component is subtracted from the total assessed need to determine theoretical eligibility.

\paragraph{Step 1: Estimating Gross Annual Income}  
The student’s income is derived from the SOEP variable \texttt{kal2a03\_h}, which reports average gross monthly earnings. This value is multiplied by the number of working months in the previous calendar year (\texttt{kal2a02}) to estimate gross annual income.  
For \texttt{pid = 20156903}:
\begin{itemize}
    \item Gross monthly income: 523 EUR
    \item Months worked: 12
    \item $\Rightarrow$ Gross annual income: $523 \times 12 = 6{,}276$ EUR
\end{itemize}

\paragraph{Step 2: Standard Deductions}  
Two statutory deductions are applied to estimate net taxable income:
\begin{itemize}
    \item \textbf{Werbungskostenpauschale} (fixed deduction for work-related expenses): 290 EUR (2018)
    \item \textbf{Sozialversicherungs-Pauschale} (fixed social insurance deduction): 17.2\% of remaining income, capped at 17,200 EUR
\end{itemize}

\begin{itemize}
    \item Step 1: $6{,}276 - 290 = 5{,}986$ EUR
    \item Step 2: $5{,}986 \times 0.828 = 4{,}152.21$ EUR (after 17.2\% deduction)
\end{itemize}

\paragraph{Step 3: Applying Income Tax}
The BAföG calculator applies German income tax tables to compute statutory income tax liabilities. In this case, the taxable income falls below the basic allowance threshold (9,000 EUR in 2018), so no income tax, church tax, or solidarity surcharge is applied:
\begin{itemize}
    \item Income tax: 0 EUR
    \item Church tax: 0 EUR
    \item Solidarity surcharge: 0 EUR
    \item $\Rightarrow$ Net annual income: 4,152.21 EUR
\end{itemize}

\paragraph{Step 4: Monthly Net Income and Allowance}
The student’s net monthly income is calculated as:
\[
\frac{4{,}152.21}{12} \approx 346.02~\text{EUR}
\]

The personal allowance specified in §~23(1) Nr.~1 BAföG for the year 2018 was 290 EUR per month. Thus, the student’s excess income is:
\[
346.02 - 290 = 56.02~\text{EUR}
\]

\begin{table}[H]
\scriptsize
\centering
\begin{tabular}{llr}
\toprule
\textbf{Component} & \textbf{Explanation} & \textbf{Value (EUR)} \\
\midrule
Gross monthly income & From SOEP variable \texttt{kal2a03\_h} & 523 \\
Working months (previous year) & From SOEP variable \texttt{kal2a02} & 12 \\
Gross annual income & Estimated income before deductions & 6,276 \\
Werbungskostenpauschale & Work-related fixed deduction (§~21(2) BAföG) & 290 \\
Sozialversicherungs-Pauschale & 17.2\% statutory deduction & 1,133.79 \\
Net annual income & Income after deductions & 4,152.21 \\
Net monthly income & Annual net income divided by 12 & 346.02 \\
Personal allowance & §~23(1) Nr.~1 BAföG (2018) & 290 \\
\textbf{Student excess income} & Amount exceeding allowance & \textbf{56.02} \\
\bottomrule
\end{tabular}
\caption{Calculation of student’s excess income (\texttt{excess\_income\_stu}) for pid 20156903.}
\label{table:bafoeg_excess_income_stu}
\end{table}



% -------------------------------------------------------------------------
%  Parental Income Father
% -------------------------------------------------------------------------
\subsection{Parental Income Evaluation: Father (pid = 20156901)}

This section documents the step-by-step derivation of net income for the student's father using variables from the SOEP-Core dataset and applying BAföG-compliant statutory deductions.

\paragraph{Step 1: Gross Income}

The parent reported a gross monthly income of 3,500 EUR and worked 12 months in the prior year, resulting in:

\[
\text{Gross annual income} = 3{,}500 \times 12 = 42{,}000~\text{EUR}
\]

\paragraph{Step 2: Werbungskostenpauschale (§~21 Abs.~2 BAföG)}

A fixed deduction of 1,000 EUR is applied to account for work-related expenses:

\[
\texttt{inc\_w} = 42{,}000 - 1{,}000 = 41{,}000~\text{EUR}
\]

\paragraph{Step 3: Sozialversicherungs-Pauschale (§~21 Abs.~2 BAföG)}

Next, a 21.3\% deduction is applied to the income after Werbungskosten:

\[
\texttt{inc\_si} = 41{,}000 \times (1 - 0.213) = 41{,}000 \times 0.787 = 32{,}267~\text{EUR}
\]

\paragraph{Step 4: Income Tax Calculation (§~32a EStG)}

The parent is assessed as an individual (not jointly filed). Based on the 2018 tax table and a taxable income of 32,267 EUR, the following taxes are applied:

- \textbf{Income tax:} 6,062 EUR (per simulation based on §~32a EStG)
- \textbf{Church tax:} 0 EUR (not church-affiliated in SOEP)
- \textbf{Solidarity surcharge (Soli):} 333 EUR

The solidarity surcharge applies since taxable income exceeds the 2018 exemption threshold of 972 EUR (§~32a Abs.~5 \& 6 EStG, pre-2020 version). The surcharge is 5.5\% of income tax, capped by taper rules.

\paragraph{Step 5: Net Annual and Monthly Income}

\[
\texttt{inc\_net} = 32{,}267 - 6{,}062 - 0 - 333 = 25{,}872~\text{EUR}
\]
\[
\texttt{net\_monthly\_income} = \frac{25{,}872}{12} = 2{,}156~\text{EUR}
\]

\begin{table}[H]
\scriptsize
\centering
\begin{tabular}{llr}
\toprule
\textbf{Component} & \textbf{Explanation} & \textbf{Value (EUR)} \\
\midrule
Gross monthly income & Reported by SOEP & 3,500 \\
Working months & From SOEP (previous year) & 12 \\
Gross annual income & $3{,}500 \times 12$ & 42,000 \\
Werbungskostenpauschale & Fixed work-related deduction (§~21(2)) & 1,000 \\
Post-werbung income (\texttt{inc\_w}) & After deduction & 41,000 \\
Sozialversicherungs-Pauschale & 21.3\% of \texttt{inc\_w} & 8,733 \\
Income after SI (\texttt{inc\_si}) & $41{,}000 \times 0.787$ & 32,267 \\
Income tax & Based on §~32a EStG table & 6,062 \\
Church tax & SOEP indicates no affiliation & 0 \\
Solidarity surcharge & 5.5\% of income tax (capped) & 333 \\
Net annual income (\texttt{inc\_net}) & After all taxes & 25,872 \\
Net monthly income & $25{,}872 \div 12$ & 2,156 \\
\bottomrule
\end{tabular}
\caption{Income derivation for father (pid = 20156901) in 2018.}
\label{table:bafoeg_parent_father}
\end{table}


% -------------------------------------------------------------------------
%  Parental Income Mother
% -------------------------------------------------------------------------
\subsection{Parental Income Evaluation: Mother (pid = 20156902)}

The same procedure is applied to evaluate the income of the student’s mother. This parent reports a lower monthly income, but the same deductions are used to compute a BAföG-compliant net income value.

\paragraph{Step 1: Gross Income}

The mother reported a gross monthly income of 300 EUR and worked 12 months in the previous year:

\[
\text{Gross annual income} = 300 \times 12 = 3{,}600~\text{EUR}
\]

\paragraph{Step 2: Werbungskostenpauschale (§~21 Abs.~2 BAföG)}

A fixed work-related deduction of 1,000 EUR is applied:

\[
\texttt{inc\_w} = 3{,}600 - 1{,}000 = 2{,}600~\text{EUR}
\]

\paragraph{Step 3: Sozialversicherungs-Pauschale (§~21 Abs.~2 BAföG)}

A 21.3\% deduction is then applied:

\[
\texttt{inc\_si} = 2{,}600 \times 0.787 = 2{,}046.20~\text{EUR}
\]

\paragraph{Step 4: Income Tax and Surcharges}

Because the income falls well below the basic exemption threshold, no income tax or surcharges apply:

- Income tax: 0 EUR
- Church tax: 0 EUR
- Solidarity surcharge: 0 EUR

\paragraph{Step 5: Net Annual and Monthly Income}

\[
\texttt{inc\_net} = 2{,}046.20~\text{EUR}
\qquad\quad
\texttt{net\_monthly\_income} = \frac{2{,}046.20}{12} = 170.52~\text{EUR}
\]

\begin{table}[H]
\scriptsize
\centering
\begin{tabular}{llr}
\toprule
\textbf{Component} & \textbf{Explanation} & \textbf{Value (EUR)} \\
\midrule
Gross monthly income & Reported by SOEP & 300 \\
Working months & From SOEP (previous year) & 12 \\
Gross annual income & $300 \times 12$ & 3,600 \\
Werbungskostenpauschale & Fixed deduction (§~21(2)) & 1,000 \\
Post-werbung income (\texttt{inc\_w}) & After deduction & 2,600 \\
Sozialversicherungs-Pauschale & 21.3\% of \texttt{inc\_w} & 553.80 \\
Income after SI (\texttt{inc\_si}) & $2{,}600 \times 0.787$ & 2,046.20 \\
Income tax & Below exemption threshold & 0 \\
Church tax & SOEP indicates no affiliation & 0 \\
Solidarity surcharge & Below threshold & 0 \\
Net annual income (\texttt{inc\_net}) & After all taxes & 2,046.20 \\
Net monthly income & $2{,}046.20 \div 12$ & 170.52 \\
\bottomrule
\end{tabular}
\caption{Income derivation for mother (pid = 20156902) in 2018.}
\label{table:bafoeg_parent_mother}
\end{table}


% -------------------------------------------------------------------------
%  Joint Parental Income
% -------------------------------------------------------------------------
\subsection{Joint Parental Income and Deductions}

After calculating net income for each parent individually, their incomes are combined and assessed jointly, following the rules laid out in §~25 and §~21 of the BAföG Act. This section outlines how the parental income is evaluated as a unit, and how the applicable deductions reduce the contribution relevant for BAföG eligibility.

\paragraph{Step 1: Joint Income}

The net monthly incomes of both parents are summed to form the joint income base:

\[
\texttt{joint\_income} = 2{,}156 + 170.52 = 2{,}326.52~\text{EUR}
\]

\paragraph{Step 2: Parental Allowance (§~25(1) Nr.~1 BAföG)}

Because both parents are financially active, the applicable allowance is the joint parental allowance. According to the BAföG schedule valid from 2015-01-01 (25. BAföGÄndG), the relevant allowance value is:

\[
\texttt{total\_allowance} = 1{,}715~\text{EUR}
\]

The remaining income after allowance is:

\[
\texttt{joint\_income\_less\_ba} = 2{,}326.52 - 1{,}715 = 611.52~\text{EUR}
\]

\paragraph{Step 3: Sibling Deduction (§~25(3) BAföG)}

The student has two siblings who are eligible for sibling-related deductions. According to the 2015 allowance table:

- The sibling deduction per eligible sibling is 260 EUR
- Total deduction: $2 \times 260 = 520$ EUR

\[
\texttt{joint\_income\_less\_ba\_and\_sib} = 611.52 - 520 = 91.52~\text{EUR}
\]

\paragraph{Step 4: Additional Allowance (§~25(4) BAföG)}

In addition, §~25(4) BAföG entitles parents to a percentage-based deduction on the remaining income. According to the allowance rules:

- A base allowance of 50\% of the remainder applies
- Plus 5\% per sibling with a positive deduction

Thus, the applied rate is:

\[
50\% + (2 \times 5\%) = 60\%
\]

\[
\texttt{additional\_allowance} = 91.52 \times 0.60 = 54.91~\text{EUR}
\]

\paragraph{Step 5: Final Excess Parental Income}

The final contribution from parental income is the remaining amount after all deductions:

\[
\texttt{excess\_income} = 91.52 - 54.91 = 36.61~\text{EUR}
\]

\begin{table}[H]
\scriptsize
\centering
\begin{tabular}{llr}
\toprule
\textbf{Component} & \textbf{Explanation} & \textbf{Value (EUR)} \\
\midrule
Joint income & Sum of both parents’ net monthly incomes & 2,326.52 \\
Parental allowance & §~25(1) Nr.~1 BAföG (joint allowance) & 1,715 \\
Remaining after allowance & $2{,}326.52 - 1{,}715$ & 611.52 \\
Sibling deduction & $2 \times 260$ (§~25(3) BAföG) & 520 \\
Remaining after siblings & $611.52 - 520$ & 91.52 \\
Additional allowance & 60\% of remaining income (§~25(4)) & 54.91 \\
\textbf{Excess parental income} & Final contribution to be deducted & \textbf{36.61} \\
\bottomrule
\end{tabular}
\caption{Calculation of joint parental excess income for pid 20156903 (2018).}
\label{table:bafoeg_joint_income}
\end{table}


% -------------------------------------------------------------------------
%  Asset Excess
% -------------------------------------------------------------------------
\subsection{Asset-Based Contribution}

Students whose personal assets exceed a legally defined exemption threshold are required to contribute the excess toward their BAföG need (§~29 BAföG). The following table lists all relevant asset categories reported in the SOEP and their treatment in the eligibility assessment for this individual.

\paragraph{Step 1: Declared Asset Categories}

The student’s asset-related information for the 2018 survey year is as follows:

\begin{table}[H]
\scriptsize
\centering
\begin{tabular}{lr}
\toprule
\textbf{Asset Category} & \textbf{Value (EUR)} \\
\midrule
Financial assets (e.g., savings, stocks) & 0 \\
Real estate (e.g., land, housing property) & 0 \\
Business assets & 0 \\
Private insurance assets & 0 \\
Vehicles (e.g., car ownership) & 7,940 \\
Tangible assets (furniture, equipment) & 0 \\
Eligible debts (offsetting) & 0 \\
\midrule
\textbf{Total assets} & 7,940 \\
\textbf{Debts} & 0 \\
\textbf{Net assets} & 7,940 \\
\bottomrule
\end{tabular}
\caption{Declared asset categories for pid 20156903 in 2018.}
\label{table:bafoeg_declared_assets}
\end{table}

\paragraph{Step 2: Asset Allowance (§~29 BAföG)}

Since the student was 25 years old in 2018 (i.e., under 30), the asset allowance for students under age 30 applied. According to the table valid from 2016-08-01 (25. BAföGÄndG), this exemption was:

\[
\texttt{asset\_allowance} = 7{,}500~\text{EUR}
\]

\paragraph{Step 3: Excess Asset Contribution}

The contribution from assets is computed as the difference between net assets and the legal allowance:

\[
\texttt{excess\_assets} = \max(7{,}940 - 7{,}500, 0) = 440~\text{EUR}
\]

\begin{table}[H]
\scriptsize
\centering
\begin{tabular}{llr}
\toprule
\textbf{Component} & \textbf{Explanation} & \textbf{Value (EUR)} \\
\midrule
Net assets & Total assets minus eligible debts & 7,940 \\
Asset allowance & §~29 BAföG (U30 threshold in 2018) & 7,500 \\
\textbf{Excess asset contribution} & Final deduction from BAföG entitlement & \textbf{440} \\
\bottomrule
\end{tabular}
\caption{Excess asset calculation for pid 20156903 in 2018.}
\label{table:bafoeg_excess_assets}
\end{table}


\subsubsection*{Final Theoretical BAföG Award}

After accounting for all relevant supplements and income-based deductions, the theoretical BAföG award is computed by subtracting the student’s and parents’ contributions—as well as any asset-based contributions—from the total assessed need.

\paragraph{Step 1: Total Assessed Need}

The total monthly need is composed of:
\begin{itemize}
    \item Base need (\texttt{base\_need}): 399 EUR
    \item Housing allowance (\texttt{housing\_allowance}): 250 EUR
    \item Insurance supplement (\texttt{insurance\_supplement}): 86 EUR
\end{itemize}

\[
\texttt{total\_base\_need} = 399 + 250 + 86 = 735~\text{EUR}
\]

\paragraph{Step 2: Total Deductions}

The following deductions apply:
\begin{itemize}
    \item Student excess income: 56.02 EUR
    \item Parental excess income: 36.61 EUR
    \item Excess asset contribution: 440.00 EUR
\end{itemize}

\[
\texttt{total\_deductions} = 56.02 + 36.61 + 440 = 532.63~\text{EUR}
\]

\paragraph{Step 3: Theoretical Award Calculation}

\[
\texttt{theoretical\_bafög} = \max(735 - 532.63,\ 0) = \textbf{202.38~EUR}
\]

\begin{table}[H]
\scriptsize
\centering
\begin{tabular}{llr}
\toprule
\textbf{Component} & \textbf{Explanation} & \textbf{Value (EUR)} \\
\midrule
Base need & §~13(1) Nr.~1 BAföG & 399 \\
Housing allowance & §~13(1) Nr.~2 BAföG & 250 \\
Insurance supplement & §~13a(1) BAföG & 86 \\
\midrule
\textbf{Total base need} & Monthly assessed need & \textbf{735} \\
\midrule
Student excess income & §~23(1) Nr.~1 BAföG & 56.02 \\
Parental excess income & §~25 BAföG + sibling adjustment & 36.61 \\
Excess asset contribution & §~29 BAföG & 440.00 \\
\midrule
\textbf{Total deductions} & Income and asset-based contributions & \textbf{532.63} \\
\midrule
\textbf{Theoretical BAföG award} & \textbf{Maximum eligible amount} & \textbf{202.38} \\
\bottomrule
\end{tabular}
\caption{Final theoretical BAföG award for pid 20156903 in 2018.}
\label{table:bafoeg_final_award}
\end{table}

\paragraph{Note on Eligibility Status}

This student qualifies for BAföG under the legal eligibility criteria defined by income, asset, and need thresholds. While their theoretical eligibility status is coded as \texttt{1} (eligible), they did not receive or report any BAföG support in the SOEP dataset:

\begin{itemize}
    \item \texttt{received\_bafög} = 0 EUR
    \item \texttt{reported\_bafög} = 0 EUR
    \item \texttt{theoretical\_eligibility} = 1 (eligible)
\end{itemize}

\newpage
\section{Main Input Parameters for BAföG Entitlement Calculations} \label{appendix:input_params}

This appendix presents the key legal parameter tables used in the theoretical BAföG entitlement calculations. 
The data has been compiled from various legal sources and amendments \cite{bafoeg_law, bafoeg20, bafoeg21, bafoeg22, bafoeg23, bafoeg24, bafoeg25, bafoeg26, bafoeg27, bafoeg28, bafoeg29}. 
Unless stated otherwise, values are given in euros.

The process of compiling these parameters served as an essential foundation for our analysis and, more broadly, enabled our entry into researching non-take-up of BAföG. 
Given the fragmented nature of BAföG legislation over time, assembling a clean, structured dataset was a prerequisite for this thesis.
We hope that this documentation can serve as a resource for future researchers interested in modelling the German student aid system or conducting policy evaluation in this domain.

For the application of these input parameters, we refer to the codebase \citep[][version~\texttt{v1.0}]{bystrom2025msc} and the example given in Appendix \ref{appendix:simulation-example}.
\vspace{1em}

\begin{table}[H]
\centering
\small
\begin{tabularx}{\textwidth}{lXXXX}
\toprule
\textbf{Valid from} & \textbf{§ 13 (1) 1} & \textbf{§ 13 (1) 2} & \textbf{§ 13 (2) 1} & \textbf{§ 13 (2) 2} \\
\midrule
2024-07-25 & 442 & 475 & 59 & 380 \\
2022-07-22 & 421 & 452 & 59 & 360 \\
2020-08-01 & 398 & 427 & 56 & 325 \\
2019-07-16 & 391 & 419 & 55 & 325 \\
2016-08-01 & 372 & 399 & 52 & 250 \\
2010-10-01 & 348 & 373 & 49 & 224 \\
2008-10-01 & 341 & 366 & 48 & 146 \\
2002-01-01 & 310 & 333 & 44 & 133 \\
\bottomrule
\end{tabularx}
\caption{Monthly standard needs rates under § 13 BAföG for students, by validity date. Amounts vary by accommodation type and insurance status.}
\end{table}

\vspace{1em}

\begin{table}[H]
\centering
\small
\begin{tabularx}{\textwidth}{lXXXXXX}
\toprule
\textbf{Valid from} & \textbf{§ 13a (1) 1} & \textbf{§ 13a (1) 2} & \textbf{§ 13a (2) 1} & \textbf{§ 13a (2) 2} & \textbf{§ 13a (3) 1} & \textbf{§ 13a (3) 2} \\
\midrule
2024-08-01 & 102 & 35 & 185 & 48 & 102 & 35 \\
2022-08-01 & 94  & 28 & 168 & 38 & 94  & 28 \\
2022-07-15 & 84  & 25 & 155 & 34 & 84  & 25 \\
2020-08-01 & 84  & 25 & 155 & 34 & 84  & 25 \\
2016-08-01 & 71  & 15 & 155 & 34 & 84  & 25 \\
2010-10-01 & 62  & 11 & 155 & 34 & 84  & 25 \\
2008-10-01 & 50  & 9  & 155 & 34 & 84  & 25 \\
2002-01-01 & 47  & 8  & 155 & 34 & 84  & 25 \\
\bottomrule
\end{tabularx}
\caption{Monthly allowances under § 13a BAföG for health and long-term care insurance contributions, by validity date. Amounts vary by insurance type and student status.}
\label{tab:bafog_values_13a}
\end{table}

\vspace{1em}

\begin{table}[H]
\centering
\small
\begin{tabularx}{\textwidth}{lX}
\toprule
\textbf{Valid from} & \textbf{§ 21 (2) 1} \\
\midrule
2022 & 0.223 \\
2021 & 0.213 \\
2012 & 0.213 \\
2001 & 0.210 \\
\bottomrule
\end{tabularx}
\caption{Deduction rates under § 21 (2) 1 BAföG for income from employment subject to pension insurance, used to approximate social security contributions in the means test, by year.}
\caption*{\textit{Note:} Table only shows years in which the rate changed. Intermediate years are forward filled.}
\end{table}

\vspace{1em}

\begin{table}[H]
\centering
\small
\begin{tabularx}{\textwidth}{lXXX}
\toprule
\textbf{Valid from} & \textbf{§ 23 (1) 1} & \textbf{§ 23 (1) 2} & \textbf{§ 23 (1) 3} \\
\midrule
2024-07-19 & 353 & 850 & 770 \\
2022-07-16 & 330 & 805 & 730 \\
2021-08-01 & 330 & 665 & 605 \\
2020-08-01 & 330 & 630 & 570 \\
2019-07-09 & 330 & 610 & 555 \\
2015-01-01 & 290 & 570 & 520 \\
2010-10-24 & 290 & 535 & 485 \\
2008-08-01 & 290 & 520 & 470 \\
2007-12-24 & 255 & 520 & 470 \\
2002-01-01 & 255 & 480 & 435 \\
\bottomrule
\end{tabularx}
\caption{Monthly income disregards (Freibeträge) under § 23 (1) BAföG for the student, by validity date. Columns refer to students living alone, with a child, or with a spouse/partner.}
\end{table}

\vspace{1em}

\begin{table}[H]
\centering
\small
\begin{tabularx}{\textwidth}{l *{4}{>{\centering\arraybackslash}X} >{\centering\arraybackslash}X >{\centering\arraybackslash}X}
\toprule
\textbf{Valid from} & \textbf{§ 25 (1) 1} & \textbf{§ 25 (1) 2} & \textbf{§ 25 (3) 1} & \textbf{§ 25 (3) 2} & \textbf{§ 25 (4) 1} & \textbf{§ 25 (4) 2} \\
\midrule
2024-07-19 & 2540 & 1690 & 850 & 770 & 50\% & 5\% \\
2022-07-16 & 2415 & 1605 & 805 & 730 & 50\% & 5\% \\
2021-08-01 & 2000 & 1330 & 665 & 605 & 50\% & 5\% \\
2020-08-01 & 1890 & 1260 & 630 & 570 & 50\% & 5\% \\
2019-07-09 & 1835 & 1225 & 610 & 555 & 50\% & 5\% \\
2015-01-01 & 1715 & 1145 & 570 & 520 & 50\% & 5\% \\
2010-10-24 & 1605 & 1070 & 535 & 485 & 50\% & 5\% \\
2007-12-24 & 1555 & 1040 & 520 & 470 & 50\% & 5\% \\
2002-01-01 & 1440 & 520  & 480 & 435 & 50\% & 5\% \\
\bottomrule
\end{tabularx}
\caption{Income exemptions under § 25 BAföG for parents and spouses or partners, by validity date. Columns show fixed allowances and percentage deductions used in the means test.}
\end{table}

\subsection{Other Relevant Input Parameters}

\begin{table}[H]
\centering
\small
\begin{tabularx}{\textwidth}{l >{\centering\arraybackslash}X >{\centering\arraybackslash}X}
\toprule
\textbf{Valid from} & \textbf{§ 32a Abs. 5 \& 6 (joint)} & \textbf{Otherwise (single)} \\
\midrule
2026 & 40700 & 20350 \\
2023 & 36260 & 18130 \\
2021 & 33912 & 16956 \\
2020 & 1944  & 972 \\
\bottomrule
\end{tabularx}
\caption{Solidarity surcharge (Soli) exemption thresholds under § 32a Abs. 5 \& 6 EStG, by year of entry into force. Joint refers to married couples filing jointly; single to individual taxpayers.}
\end{table}


\begin{table}[H]
\centering
\small
\begin{tabularx}{\textwidth}{l >{\centering\arraybackslash}X}
\toprule
\textbf{Year} & \textbf{Werbungskostenpauschale} \\
\midrule
2024 & 1230 \\
2022 & 1200 \\
2021 & 1000 \\
2010 & 920  \\
2003 & 1044 \\
\bottomrule
\end{tabularx}
\caption{Annual employee deduction for work-related expenses (Werbungskostenpauschale) under § 9a Satz 1 Nr. 1a EStG, by year of change. Intermediate years are forward filled in the microsimulation.}
\end{table}


\newpage
\pagenumbering{arabic}                          % Return to standard page numbering (optional)
\renewcommand{\thesection}{\arabic{section}}    % Reset section numbering format if needed
\setcounter{section}{0}                         % Reset section counter if using \section in ai_statement


\section*{AI Statement}
ChatGPT 3.5--4.5 was employed for language and grammar checks in the paper. The authors reviewed, edited and revised any content generated by ChatGPT to their preferences, therefore having ultimate responsibility for the entire paper.

\end{document}
