\documentclass[10pt,a4paper]{article}
%------ SETUP OF THE DOCUMENT ------%
%This part of the document sets up the document and makes it easier to read the main.tex file
%Changes in the setup file are recommended if you wish to customize things like colors in links and such.

%------ ********************* ------%
\usepackage[margin = 25mm]{geometry} %
\usepackage{graphicx} % Required for inserting images
\usepackage[toc,page]{appendix}
\usepackage[english=usenglishmax]{hyphsubst} %sets hyphenation to American English. Google is your friend on hyphenation. Babel could also be used
\usepackage[hyphens]{url}
\usepackage{amsmath, amssymb, setspace, float, subcaption, caption, booktabs, pdflscape, dcolumn, titlesec, tocloft, comment, xcolor, longtable, blindtext, rotating, lipsum}
\usepackage{multicol}
\usepackage{multibib} 
\usepackage[flushleft]{threeparttable}
\usepackage{enumitem}
\usepackage{parskip}
\usepackage[utf8]{inputenc}

\definecolor{aeaBlue}{HTML}{0072CE}
% \usepackage[round, authoryear]{natbib}
\usepackage[round, authoryear,sort&compress,numbers]{natbib}
\definecolor{refcolor}{RGB}{102,178,255} % light blue

\usepackage[
    colorlinks=true,
    linkcolor=aeaBlue,         % for internal links including ToC
    citecolor=aeaBlue,       % for citations
    urlcolor=aeaBlue         % for URLs
]{hyperref}


%%%%%%%%%%%%%%%%%%%%%%%%%%%%%%%% 
% Tikz
%%%%%%%%%%%%%%%%%%%%%%%%%%%%%%%% 

\usepackage{tikz}
\usetikzlibrary{positioning, arrows.meta, shapes.geometric, fit}


\tikzset{
  pipelinebox/.style={
    rectangle,
    draw=black,
    align=center,
    font=\scriptsize,
    inner sep=4pt,
    minimum height=1cm,
    minimum width=2.5cm,
    text width=2.5cm,
    align=center
  },
  arrow/.style={
    thick, ->, >=stealth
  }
}

\tikzset{
  auxbox/.style={
    rectangle,
    draw=black,
    fill=orange!20,
    align=center,
    font=\footnotesize\itshape,
    inner sep=8pt,
    minimum height=1cm,
    minimum width=3.5cm,
    dashed
  }
}

\tikzset{
  sourcebox/.style={
    rectangle,
    draw=black,
    align=center,
    font=\footnotesize\bfseries,
    text transform=uppercase,
    inner sep=8pt,
    minimum height=1cm,
    minimum width=3.5cm
  }
}

\tikzset{
  actionbox/.style={
    diamond,
    draw=black,
    % fill=orange!20,
    align=center,
    font=\footnotesize,
    aspect=2,
    inner sep=2pt,
    minimum height=1.2cm,
    minimum width=3.5cm
  }
}

\tikzset{
  datasetbox/.style={
    rectangle, 
    draw=black, 
    dashed,
    minimum height=1.5em, 
    minimum width=3cm, 
    text centered, 
  },
}



%%%%%%%%%%%%%%%%%%%%%%%%%%%%%%%%%%%%%%%

\graphicspath{{figures/}}


%%%%%%%%%%%%%%%%%%%%%%%%%%%%%%%%%%%
% Table formatting
%%%%%%%%%%%%%%%%%%%%%%%%%%%%%%%%%%%
\usepackage{multirow}
\renewcommand{\arraystretch}{1.0}

\usepackage{tabularx}
\usepackage{booktabs}
\usepackage{threeparttable}
\usepackage{adjustbox}
\usepackage{array}
\usepackage{rotating}  % for rotated headers

%%%%%%%%%%%%%%%%%%%%%%%%%%%%%%%%%%%
% Equation Formatting
%%%%%%%%%%%%%%%%%%%%%%%%%%%%%%%%%%%
\numberwithin{equation}{section}

%%%%%%%%%%%%%%%%%%%%%%%%%%%%%%%%%%%
% Formatting
%%%%%%%%%%%%%%%%%%%%%%%%%%%%%%%%%%%

% Table of Contents
\usepackage{tocloft}
\setlength{\cftbeforesecskip}{0pt}      % Remove vertical space before sections
\setlength{\cftbeforesubsecskip}{0pt}   % Remove vertical space before subsections
\setlength{\cftaftertoctitleskip}{10pt} % Optional: reduce spacing after "Contents" title

\renewcommand{\cftsecfont}{\small}
\renewcommand{\cftsubsecfont}{\small}
\renewcommand{\cftsecpagefont}{\small}
\renewcommand{\cftsubsecpagefont}{\small}



% Bibliography
\bibliographystyle{plainnat}
\setlength{\bibsep}{0.1cm}

% Titles
\newcommand{\sectionbreak}{\clearpage} % Break page when encountering new section

% \titleformat{\section}
%   {\normalfont\huge\bfseries}   % font formatting
%   {\thesection}{1em}             % section number
%   {}                             % code before the title
%
%
% \titlespacing*{\section}
%   {0pt}    % left margin
%   {0pt}    % space before the title
%   {2em}    % space after the title (increase this to move the text below down)

% Lists
\setlist[itemize]{itemsep=0.05em}
\captionsetup{font=footnotesize}




\date{Seminar date}
\begin{document}

% TITLE PAGE 
\setstretch{1.5}

    %\includegraphics[scale = 0.3]{images/LU RGB.png}
    %\includegraphics[scale = 0.3]{images/LU Black.png} 
    \includegraphics[scale = 0.15]{images/LUSEM_RGB.png} %I recommend using this one but the others are fine too
    %\includegraphics[scale = 0.15]{images/LUSEM_BLACK.png}

\vspace{2cm}
    \begin{center}       
        \vspace*{2cm}
        {\LARGE {\textbf{Still Left Behind?}  \\ 
            Updating the Microsimulation of BAföG Non-Take-Up in Germany
        }} \\
        \vspace{1cm}
        \Large{Alexander Eriksson Byström \& María Sól Antonsdóttir} \normalsize{\\ Department of Economics \\ Lund University School of Economics and Management}
    \end{center}
    \vspace{2cm}

\vfill
\noindent 
\textbf{Supervisor: Petra Thiemann} \\ 
NEKN01 - Master Thesis in Economics 15 ECTS \\ 
Seminar date: Month Day, Year
\thispagestyle{empty}

\newpage
\tableofcontents
\thispagestyle{empty}


%%%%%%%%%%%%%%%%%%%%%%%%%%%%%%%%
% ABSTRACT
%%%%%%%%%%%%%%%%%%%%%%%%%%%%%%%%
\newpage
\begin{abstract}
\setstretch{1}
This paper examines the non-take-up of Germany’s federal student aid program (BAföG) using microsimulation techniques based on data from the German Socio-Economic Panel (SOEP) for the years 2007–2021. With almost a decade since non-take-up of BAföG was last analysed, our study provides updated estimates that capture recent developments. This is of relevance since the share of students that take up BAföG is reported to have decreased even further in recent years.
We simulate BAföG eligibility and compare it to observed take-up behavior to estimate non-take-up rates. 
Our findings indicate that non-take-up has increased over the past decade, with an average rate of approximately 60\% during the study period.
Two main factors help explain this phenomenon. 
First, students are less likely to apply if they expect only a small subsidy. 
Second, greater awareness and understanding of the application process significantly increase the likelihood of take-up. 
Additionally, we find a notable difference in take-up rates between East and West Germany, suggesting that cultural attitudes toward government support may influence participation.



    \noindent \textbf{Keywords:} Non-take-up, Non-take-up of student aid, microsimulation, SOEP, student aid, student loans, BAföG. \\
    \noindent \textbf{JEL codes:} I22, I23, I24, I38, H53
\end{abstract}
\newpage
\setcounter{page}{1}

\hypersetup{linkcolor=blue}

%%%%%%%%%%%%%%%%%%%%%%%%%%%%%%%%
% CONTENT
%%%%%%%%%%%%%%%%%%%%%%%%%%%%%%%%
\input{sections/1 intro}
\section{Related Literature} \label{sec:literature}

- What we provide to the literature (short) 

- What other studies have looked into

In broad terms, research on non take up can be split into studies applying traditional economic theory and studies applying behavioural economic theory. Traditional economic theory assumes rational behaviour and thus that agents optimise the trade off between benefit and cost. The field of behavioural economics, that has more recently emerged, instead emphasises deflections from the traditional assumptions on rational behaviour and highlights cognitive bias and behavioural obstacles \citep{mechelen_who_2017}.

...

Optimal policy design aims to target support to those who need it most, while keeping eligibility criteria clear and easy to understand. However, targeted programmes often involve complicated rules, such as income limits and asset tests. This complexity poses some difficulties, both for administrative workers and applicants. For administrators, it increases the risk of mistakes when assessing eligibility. For applicants, understanding the rules and completing the process can be time consuming and require substantial effort, especially if there are uncertainties about eligibility. This can discourage people from applying. Complexity can also result in tertiary non take up, where certain groups are excluded by design because the rules use rough indicators of need. One way to reduce non take up is to use simpler, categorical criteria based on things like age or household type. Although this may make targeting less precise, it can make the application process easier and more accessible \citep{mechelen_who_2017}.

...

To estimate non take up rates of welfare benefits, researchers typically rely on one or more of three data sources: administrative records, specially designed surveys and general purpose surveys. Each has its own strengths and weaknesses. Administrative data is generally precise for welfare receipt, but often it lacks information on those who do not claim benefits. Special purpose surveys can collect more detailed information on eligibility and take up behaviour but are costly and rarely used. On the other hand, general purpose surveys are more readily available and are widely used in empirical research \citep{mechelen_who_2017}.

In this study, data was collected from a general purpose survey, i.e. the German Socio-Economic Panel (SOEP), which is one of the longest standing multidisciplinary household surveys in the world, gathering data from around 30000 individuals across 22000 households annually \citep{berlin_diw_nodate}.

While such data is not specifically designed to measure non take up, it has the advantage of covering both benefit receipt and the characteristics needed to estimate eligibility, such as income, household composition and demographic variables \citep{mechelen_who_2017}.

However, it comes with some limitations. First of all, there are potential biases due to non response bias and undercoverage. Vulnerable groups, such as those without a permanent address or people living in institutions, are often missing from survey samples. Non response may also be correlated with non take up, which can distort estimates. Second of all, measurement errors can be a concern, especially in regards income, asset reporting, and welfare receipt. Respondents may misreport their income or confuse the benefits they receive, leading to inaccurate estimates of eligibility and take up. Third of all, mismatches in the timing and definition of income used in surveys compared to what administrations use to assess eligibility can result in classification errors. For example, surveys often report annual income, but eligibility is commonly assessed monthly. Lastly, general purpose surveys often lack detailed information about reasons for non take up, making it difficult to distinguish between, for example, lack of awareness and administrative barriers \citep{mechelen_who_2017}.
\section{Data}
This study utilizes data from the German Socio-Economic Panel (SOEP), a nationally representative longitudinal survey conducted annually by the DIW (Deutsches Institut für Wirtschaftsforschung) since 1984. Respondents of the survey must be at least 17 year old. The dataset provides individual and household-level information, including data on income, education, household composition, labor market behavior, and demographics. This study relies exclusively on SOEP-Core, the central and most comprehensive module.

We restrict our analysis to the period 2002–2022, following the introduction of the euro, to ensure consistency in income data. Our sample is limited to individuals who are currently enrolled in education, identified through a harmonized education variable in SOEP. This yields an unbalanced panel of students observed for varying numbers of years.

A key strength of the SOEP is its household structure, which allows us to link students to their parents, siblings and, in many cases, partners.
Using this data, we construct a comprehensive dataset that includes detailed student-level and household-level characteristics. For students, we observe age, gender, federal state (Bundesland), household type, and income (if any). Parental information includes gross and net income, employment status, household structure, tax burdens, and relationship status. Where applicable, we also observe sibling characteristics such as enrollment status, income, and household composition. Finally, for students who report having a partner, we include the partner’s income and household role.

Using this data, we simulate the theoretical BAföG eligibility and award based on statutory rules in place during each year. This involves implementing a detailed microsimulation model that replicates the BAföG means test.

\subsection{Sample Description}
The final dataset contains approximately \textbf{N = [UPDATE THIS WHEN WE KNOW]} student-year observations. Each row represents a student in a particular year.

There is substantial variation in how long individuals remain in the panel. Some students appear in only one wave, while others are observed over multiple years. This reflects differences in educational paths, dropout rates, and survey participation.

\subsection{Limitations}

Although the SOEP provides comprehensive socioeconomic data, certain limitations persist.

\paragraph{Parental income coverage.}
Parental income data are essential for constructing a credible BAföG means test. 
Requiring information for both parents would exclude a substantial share of observations, which would bias simulated eligibility downward. 
Students with income data for at least one parent are therefore retained. 
For single- or split-parent households, this reflects actual circumstances and in two-parent households it may understate available resources. 
However, the gain in sample size and representativeness offsets this downward bias. 
This approach maintains alignment with the target population while minimizing selection bias.

Because the SOEP dataset does not identify BAföG‑eligible respondents directly, we construct eligibility through a microsimulation that mirrors the legal rules of the Bundesausbildungsförderungsgesetz (BAföG) for the years 2002–2022 \citep{bafoeg_law,bafoeg20,bafoeg21,bafoeg22,bafoeg23,bafoeg24,bafoeg25,bafoeg26,bafoeg27,bafoeg28,bafoeg29}.  
The model implements the need calculation in §\,13 and the dynamic allowance schedule of §\,25, updating thresholds in line with each amendment act.  
Although undocumented exceptions cannot be fully accounted for, the model follows the statutory rules and ensures consistent application across survey waves.

\paragraph{Modelling taxes.}
Full tax‑return simulations, as in \cite{herber_non-take-up_2019}, require detailed information (e.\,g.\ deductions, extraordinary expenses) that the SOEP does not always provide.  
We therefore approximate net parental income with the statutory bracket formulas of §\,32a EStG—updated for every year since 2002 \citep{estg_law,estg_2025,estg_2024,estg_2023,estg_2022,estg_2021,estg_2020,estg_2019,estg_2018,estg_2017,estg_2016,estg_2015,estg_2014,estg_2013,estg_2012,estg_2007,estg_2006,estg_lohninfo_2012}.  
For years 2002–2006, where official schedules are unavailable, values are linearly interpolated.  
This approximation entails limited precision loss but enables consistent estimation across years, reproduces the primary tax burden, reflects statutory bracket reforms, and isolates income differences relevant for BAföG eligibility.  
Solidarity surcharge rates (5.5\% until 2020, with phased reductions thereafter) follow \citet{solzg_2018,solzg_2019,solzg_2023}.  
Church tax is applied at 9\% (8\% in Bavaria and Baden-Württemberg), conditional on reported church affiliation.

\paragraph{Deviation from official outcomes.}
Even when closely following the legal rules, the simulation can differ from actual BAföG decisions due to missing household details or unobserved individual circumstances. Still, it offers a consistent and transparent benchmark for analysing take-up over time.

While many SOEP variables approximate administrative data, its still the most suitable dataset for examining the BAföG non-take-up rate. 
The eligibility measure used here reflects the legal framework and is sufficiently accurate for a systematic analysis of non-take-up and its underlying factors.

\section{Method} 

\textcolor{red}{MOVE THIS TEXT TO THE LITERATURE REVIEW SECTION}

\paragraph{Non take up of welfare.}

We define non take up of welfare in line with Nelson and Nieuwenhuis (2019), as the circumstance when a person is eligible for welfare, but does not receive it. This is in line with terminology commonly used in literature on welfare take up rates. Non take up rate is thus the number of people who are eligible, but do not receive it, divided by the total number of people eligible. It is worth noting, however, that these situations often prove to be more complicated. In some cases, some individuals might receive welfare even though they aren’t eligible. This might for example happen due to fraud or errors made on the administrative level. This presents the issue of type two beta errors \citep{herber_non-take-up_2019, nelson_towards_2021}.

\paragraph{Tertiary non take up of welfare.}
In 2017, Van Mechelen and Janssens defined tertiary non-take-up
as a situation in which vulnerable individuals aren’t entitled to social welfare due to eligibility rules, even
if they are in need of support. Tertiary non-take up as defined here can thus be considered a specific
form of non take up. Originally, this concept was defined narrowly to include only those who are not
eligible within a vulnerable group. However, some have argued for a broader definition that includes
everyone in the vulnerable group that does not actually receive the welfare benefit, regardless of whether
they are eligible or not <\citep{mechelen_who_2017, goedeme_concept_2020}.

The concept of non-tertiary take-up relates directly to the concept of targeting efficiency, which can be
divided into vertical and horizontal efficiency. Vertical efficiency refers to how well a welfare system avoids
giving support to individuals who fall outside the intended target group. In most cases, the target group
is defined as those who are not considered economically vulnerable. It is essentially about minimising
incorrect inclusion. One way to express this is through leakage, which is defined as the proportion of
benefit recipients who are not a part of the reference population. The reference population is typically
defined as people with low living standards, low income or other markers of economic vulnerability.
In contrast, horizontal efficiency focuses on whether those within the target group actually receive the
support. If many eligible or vulnerable individuals go without welfare benefits, the system is horizontally
inefficient. This concept aligns closely with the broader definition of tertiary non-take-up, which includes
all vulnerable individuals who are unable to access support, regardless of the reason \citep{mechelen_who_2017, goedeme_concept_2020}.

%%%%%%%%%%%%%%%%%%%%%%%%%%%%%%%%%%%%%%%%%%%%%%%%%%%%%%%%%%%
% Method
%%%%%%%%%%%%%%%%%%%%%%%%%%%%%%%%%%%%%%%%%%%%%%%%%%%%%%%%%%%

%%%%%%%%%%%%%%%%%%%%%%%%%%%%%%%%%%%%%%%%%%%%%%%%%%%%%%%%%%%
% Microsimulation
%%%%%%%%%%%%%%%%%%%%%%%%%%%%%%%%%%%%%%%%%%%%%%%%%%%%%%%%%%%
\subsection{Microsimulation of Theoretical BAföG Eligibility} 
\subsubsection{Purpose and Scope}
The microsimulation pipeline is designed to calculate a theoretical BAföG eligibility status and award amount for students in the SOEP-Core sample. 
Its primary purpose is to compare these simulated entitlements with de facto BAföG take-up, as reported in SOEP.

To construct the theoretical values, the model replicates the legal rules and means-testing procedures defined in the Bundesausbildungsförderungsgesetz (BAföG) for the years 2007 to 2021. 
These rules are applied to individual-level SOEP data, including detailed information on income, assets, housing costs, and household structure.

This approach enables a systematic assessment of the alignment between statutory entitlements and actual BAföG participation. 
Deviations between the modeled and reported outcomes may arise from reporting errors, exceptional administrative decisions, or incomplete data. 
Full documentation of the simulation logic and input structure is provided in Appendix ~\ref{appendix:microsimulation-pipeline} and ~\ref{appendix:simulation-example}.

\paragraph{Identifying the Non-Take-Up (NTU) Rate and Beta Error}
In the conditional probabilities 

Following the microsimulation of theoretical BAföG eligibility, we define the non-take-up (NTU) rate as the conditional probability that a student does not receive BAföG, despite being theoretically eligible according to our model. Formally, this is expressed as:

\begin{equation}
\Pr(\text{NTU} = 1 \mid M = 1) = \frac{\sum_{i=1}^{N} \mathbf{1}\{R_i = 0 \ \text{and} \ M_i = 1\}}{\sum_{i=1}^{N} \mathbf{1}\{M_i = 1\}}, \quad\text{where} 
\end{equation}

\begin{equation}
  \mathbf{1}\{\cdot\} =
  \begin{cases}
  1 & \text{if individual } i \text{ is eligible but does not take up BAföG}, \\
  0 & \text{otherwise}.
  \end{cases}
\label{eq:indicator-function-ntu}
\end{equation}


\paragraph{Beta Error (Type II Error).}  
The beta error measures the probability that a student receives BAföG despite being classified as theoretically ineligible by our model. This error captures false positives in eligibility classification, indicating cases where students who should not qualify according to the simulation do receive financial support. Formally, it is expressed as:

\begin{equation}
\Pr(\text{TU} = 1 \mid M = 0) = \frac{\sum_{i=1}^{N} \mathbf{1}\{R_i = 1 \ \text{and} \ M_i = 0\}}{\sum_{i=1}^{N} \mathbf{1}\{M_i = 0\}},
\end{equation}

where \( \mathbf{1}\{\cdot\} \) is the indicator function defined in equation \eqref{eq:indicator-function-ntu}, but here:

\[
\mathbf{1}\{\cdot\} =
\begin{cases}
1 & \text{if individual } i \text{ is ineligible but receives BAföG}, \\
0 & \text{otherwise}.
\end{cases}
\]

% Maybe ref all our data sources, the statutory and soep
\subsubsection{Simulation Pipeline}

\paragraph{Constructing the Student Dataset.}
The pipeline begins by assembling a harmonized dataset of student-level observations from SOEP-Core. 
This is achieved by filtering for individuals who are enrolled in education, fall within the relevant survey years, and are at least 18 years old. 
To ensure a valid estimation of parental contributions, the dataset is further restricted to cases where income data from both legal parents are observable in the panel.

The resulting student-level dataframe integrates sociodemographic variables including sex, age, partnership status, number of siblings, number of children, household composition, and federal state of residence. 
Gross student income is also appended at this stage. 
Net student income is derived from gross values by applying year-specific rules for income tax, solidarity surcharge, church tax (where applicable), and standard deductions (e.g., Werbungskostenpauschale), in accordance with §§\,21–23 BAföG \citep{bafoeg_law}.
This net income will later be used to compute the student’s excess income as part of the BAföG need assessment.


\paragraph{Estimating Parental Contributions.}
In the next step, the simulation pipeline aggregates and evaluates parental income to estimate the expected contribution toward the student’s BAföG entitlement. 
For each student, the incomes of both legal parents—identified within the household and linked through SOEP family structure data—are retrieved and converted into annual net income. 
These values account for deductions such as income tax, solidarity surcharge, and church tax, where applicable.

Net incomes from both parents are combined into a joint parental income measure. 
From this, the model subtracts statutory allowances as defined in §§\,24–25 BAföG \citep{bafoeg_law}, which vary depending on the number of parents, number of dependent children, and year-specific legal thresholds. 
Additional deductions are applied if the student has siblings who might also be eligible for support. 
The result is a measure of excess parental income, which feeds directly into the theoretical award calculation in the next stage.

A complete breakdown of the income transformation, applicable thresholds, and illustrative examples is provided in Appendix~\ref{appendix:simulation-example}.

\paragraph{Asset Test.}
The simulation includes an asset test to assess whether students hold financial resources above the statutory exemption thresholds. 
For each student, information on financial assets, real estate, business holdings, private insurances, vehicles, and other tangible property is combined, and reported debts are subtracted to derive total net assets.

Since asset data in SOEP are only collected every five years, missing observations for non-surveyed years are filled using linear interpolation. 
This approach allows for year-specific asset estimates that remain consistent with observed data and ensures full coverage across the entire simulation period.

Total assets are then compared against exemption thresholds defined in §\,29 BAföG \citep{bafoeg_law}, which vary by age, partnership status, and number of dependent children. 
Any amount exceeding the applicable allowance is classified as excess assets and contributes to reducing the student's calculated need. 


\paragraph{Need calculation and theoretical entitlement.}
In the final stage, the simulation model calculates the student's funding need by summing the statutory base need, housing allowance, and health insurance supplement, as defined in §\,13 BAföG \citep{bafoeg_law}. 
From this total, the model subtracts any excess income attributable to the student, their parents, and their assets. 
The resulting amount determines the theoretical monthly BAföG entitlement.

A positive entitlement does not automatically imply eligibility: the model also applies age-based eligibility criteria. 
Students are only considered theoretically eligible if they meet the age requirements defined in the law, typically under 30 for undergraduate studies and under 35 for graduate-level programs. 
The final output includes both the simulated monthly award and a binary eligibility flag, which are used for comparison against self-reported values in SOEP. 
Detailed examples of this calculation and relevant thresholds are provided in Appendix~\ref{appendix:simulation-example}.



%%%%%%%%%%%%%%%%%%%%%%%%%%%%%%%%%%%%%%%%%%%%%%%%%%%%%%%%%%%
% Binary Choice Model of Non-Take-Up
%%%%%%%%%%%%%%%%%%%%%%%%%%%%%%%%%%%%%%%%%%%%%%%%%%%%%%%%%%%
\subsection{Binary Choice Model}
After simulating statutory eligibility, we analyse behavioural non-take-up: the probability that a student refrains from taking up BAföG despite being theoretically eligible according to our microsimulation. 
We model both using a Logit and a Probit model.

\subsubsection{Probit Model}
Formally, we model
\begin{equation}
  \Pr(\mathrm{NTU}_i = 1 \mid \mathbf{X}_i) = \Phi(\mathbf{X}^\top \boldsymbol{\beta})
  , \qquad \text{for all } i \text{ with } T_i = 1,
\end{equation}
where \( \Phi(\cdot) \) denotes the cumulative distribution function of the standard normal distribution. 

Here, \( T_i = 1 \) indicates the theoretical eligibility outcome of our microsimulation, and \( \mathrm{NTU}_i := \mathbf{1}\{R_i = 0\} \) is a binary indicator for non-take-up, based on the observed receipt in SOEP-Core (with \( R_i = 1 \) indicating receipt of BAföG and \( R_i = 0 \) otherwise).

\subsubsection{Logit Model}
In the same way as the Probit model, we fit a Logit model
\begin{equation}
  \Pr(\mathrm{NTU}_i = 1 \mid \mathbf{X}_i) = \Lambda(\mathbf{X}^\top \boldsymbol{\beta})
  , \qquad \text{for all } i \text{ with } T_i = 1,
\end{equation}
where \( \Lambda(\cdot) \) denotes the logistic cumulative distribution function
\begin{equation}
  \Lambda(z) = \frac{1}{1 + e^{-z}}.
\end{equation}

\subsubsection{Interpretation.} %TODO: Maybe explain latent index more -- don't fully understand atm
Since raw logit and probit coefficients reflect changes in the latent index and are not directly interpretable in terms of outcome probabilities, we report average marginal effects (AMEs) for all covariates. These AMEs quantify the average change in the probability of non-take-up associated with a one-unit change in each covariate, holding other variables at their observed values.

\paragraph{Control Variables}
Our models include a set of control variables to account for observed heterogeneity that may influence the probability of non-take-up. These controls include demographic factors (e.g., sex, migration background, partnership status, living situation), socioeconomic characteristics (e.g., parental income, own income, parental education), and family context (e.g., sibling previously claimed BAföG). We also control for regional differences using an East/West Germany background indicator, reflecting known structural and cultural variations. Finally, to capture behavioral differences that might affect take-up decisions, we include a measure of individual risk appetite. These covariates help isolate the association between key predictors and non-take-up by adjusting for potential confounders.


\subsection{Model limitations}
\label{subsection:model_limitations}

\paragraph{Addressing beta errors in eligibility simulations.}
In simulating benefit non-take-up, beta errors occur when individuals report receiving a benefit but are classified by the model as ineligible. These mismatches typically reflect limitations in the input data, particularly concerning income, assets, or household structure. Since the data this study utilises on both income and benefit receipt is self-reported, inaccuracies can arise from either source. Without administrative records, it is not possible to confirm whether a student was truly eligible or actually received the benefit. Some studies suggest that beta errors are more often caused by issues in the income or asset data used for eligibility simulation, rather than incorrect reporting of benefit receipt \citep{frick_claim_2007, janssens_takemod_2022}.

A key reason for beta errors is that benefit eligibility often depends on precise income thresholds. Even small errors in reported income can shift households across these cut-offs, especially when only annual income is available, despite eligibility being assessed monthly in practice \citep{herber_non-take-up_2019}. Further complications arise from missing or outdated information on assets and the difficulty of capturing criteria like actual household composition or work availability. Unlike more discretionary benefits such as social assistance, BAföG eligibility is determined by clearly codified national rules, leaving little room for individual judgment by administrators. As a result, administrative discretion is unlikely to be a major source of beta errors in this context. \textcolor{red}{MAYBE REMOVE THE LAST TWO SENTENCES}

To address these limitations, several strategies are used in the literature. These include conducting sensitivity checks by adjusting income levels and applying post-simulation corrections to reclassify borderline cases \citep{herber_non-take-up_2019}. Some studies also emphasize the value of combining different data sources where possible, such as using more detailed survey modules on assets or household composition to improve the accuracy of eligibility simulations \citep{janssens_takemod_2022}. While this approach is not possible in all settings, efforts to reduce data limitations can help lower the risk of beta errors and strengthen the credibility of non-take-up estimates. \textcolor{red}{MAYBE REMOVE THE LAST TWO SENTENCES HERE, ARE WE REALLY DOING THAT?}

Although beta errors cannot be completely avoided, it is important to recognise their potential impact on the results. In this thesis, particular attention is paid to identifying where beta errors may occur and considering how they might influence the findings. Sensitivity checks are applied where relevant to assess the robustness of the findings and to reduce the risk of misinterpretation.

\textcolor{red}{OVERALL MAYBE SHORTEN THIS WHOLE BETA ERROR TEXT}
\section{Results}
RESULTS SECTION NOT STARTED ON YET, JUST PASTING SOME RESULTS DOWN BELOW FOR MEETING 7TH OF MAY.

\paragraph{Note (Draft – 7 May Meeting):}
The following tables report preliminary Probit regression results estimating the probability of BAföG non-take-up among theoretically eligible students (\emph{Pr}(no award $|$ eligibility = 1)). Variables are interpreted as follows:

\begin{itemize}
  \item \texttt{z\_age}: Age (standardised)
  \item \texttt{z\_num\_siblings}: Number of full siblings (standardised)
  \item \texttt{living\_with\_parent}: Lives with at least one parent (1 = yes)
  \item \texttt{east\_background}: Lives in former East German states (1 = yes)
  \item \texttt{sex\_2}: Female (1 = female; reference category: male)
  \item \texttt{empstat\_\{1,2,3,4\}}: Employment status dummies
    \begin{itemize}
      \item \texttt{empstat\_1} = Full-time
      \item \texttt{empstat\_2} = Part-time
      \item \texttt{empstat\_3} = Vocational training
      \item \texttt{empstat\_4} = Marginal/irregular
      \item Reference category: Not employed
    \end{itemize}
  \item \texttt{migback\_2}: Direct migration background
  \item \texttt{migback\_3}: Indirect migration background \\
    Reference category: No migration background
\end{itemize}

All continuous variables are z-standardised; categorical variables use one-hot encoding with the first level as baseline. Standard errors are robust (HC2). These are exploratory results for discussion only.

\begin{center}
\begin{tabular}{lclc}
\toprule
\textbf{Dep. Variable:}       &   non\_takeup    & \textbf{  No. Observations:  } &     3416    \\
\textbf{Model:}               &      Probit      & \textbf{  Df Residuals:      } &     3404    \\
\textbf{Method:}              &       MLE        & \textbf{  Df Model:          } &       11    \\
\textbf{Date:}                & Wed, 07 May 2025 & \textbf{  Pseudo R-squ.:     } &  0.04671    \\
\textbf{Time:}                &     14:07:02     & \textbf{  Log-Likelihood:    } &   -1088.5   \\
\textbf{converged:}           &       True       & \textbf{  LL-Null:           } &   -1141.8   \\
\textbf{Covariance Type:}     &       HC2        & \textbf{  LLR p-value:       } & 8.432e-18   \\
\bottomrule
\end{tabular}
\begin{tabular}{lcccccc}
                              & \textbf{coef} & \textbf{std err} & \textbf{z} & \textbf{P$> |$z$|$} & \textbf{[0.025} & \textbf{0.975]}  \\
\midrule
\textbf{const}                &      -1.2497  &        0.082     &   -15.242  &         0.000        &       -1.410    &       -1.089     \\
\textbf{z\_age}               &      -0.1525  &        0.037     &    -4.158  &         0.000        &       -0.224    &       -0.081     \\
\textbf{z\_num\_siblings}     &       0.0093  &        0.030     &     0.315  &         0.753        &       -0.049    &        0.067     \\
\textbf{living\_with\_parent} &       0.1775  &        0.071     &     2.495  &         0.013        &        0.038    &        0.317     \\
\textbf{east\_background}     &      -0.4373  &        0.070     &    -6.280  &         0.000        &       -0.574    &       -0.301     \\
\textbf{sex\_2}               &      -0.1772  &        0.061     &    -2.918  &         0.004        &       -0.296    &       -0.058     \\
\textbf{empstat\_1}           &       0.3392  &        0.193     &     1.760  &         0.078        &       -0.039    &        0.717     \\
\textbf{empstat\_2}           &       0.1832  &        0.171     &     1.068  &         0.285        &       -0.153    &        0.519     \\
\textbf{empstat\_3}           &       0.4284  &        0.098     &     4.360  &         0.000        &        0.236    &        0.621     \\
\textbf{empstat\_4}           &      -0.0316  &        0.088     &    -0.359  &         0.720        &       -0.204    &        0.141     \\
\textbf{migback\_2}           &       0.1796  &        0.099     &     1.813  &         0.070        &       -0.015    &        0.374     \\
\textbf{migback\_3}           &       0.0647  &        0.080     &     0.806  &         0.420        &       -0.093    &        0.222     \\
\bottomrule
\end{tabular}
%\caption{Probit Regression Results}
\end{center}
\begin{center}
\begin{tabular}{lc}
\toprule
\textbf{Dep. Variable:}       &  non\_takeup    \\
\textbf{Method:}              &      dydx       \\
\textbf{At:}                  &    overall      \\
\bottomrule
\end{tabular}
\begin{tabular}{ccccccc}
          \textbf{}           & \textbf{dy/dx} & \textbf{std err} & \textbf{z} & \textbf{P$> |$z$|$} & \textbf{[0.025} & \textbf{0.975]}  \\
\midrule
\bottomrule
\end{tabular}
\begin{tabular}{lcccccc}
\textbf{z\_age}               &      -0.0264   &        0.006     &    -4.161  &         0.000        &       -0.039    &       -0.014     \\
\textbf{z\_num\_siblings}     &       0.0016   &        0.005     &     0.315  &         0.753        &       -0.008    &        0.012     \\
\textbf{living\_with\_parent} &       0.0307   &        0.012     &     2.497  &         0.013        &        0.007    &        0.055     \\
\textbf{east\_background}     &      -0.0756   &        0.012     &    -6.246  &         0.000        &       -0.099    &       -0.052     \\
\textbf{sex\_2}               &      -0.0306   &        0.010     &    -2.923  &         0.003        &       -0.051    &       -0.010     \\
\textbf{empstat\_1}           &       0.0587   &        0.033     &     1.759  &         0.079        &       -0.007    &        0.124     \\
\textbf{empstat\_2}           &       0.0317   &        0.030     &     1.067  &         0.286        &       -0.027    &        0.090     \\
\textbf{empstat\_3}           &       0.0741   &        0.017     &     4.352  &         0.000        &        0.041    &        0.107     \\
\textbf{empstat\_4}           &      -0.0055   &        0.015     &    -0.359  &         0.720        &       -0.035    &        0.024     \\
\textbf{migback\_2}           &       0.0311   &        0.017     &     1.814  &         0.070        &       -0.003    &        0.065     \\
\textbf{migback\_3}           &       0.0112   &        0.014     &     0.806  &         0.420        &       -0.016    &        0.038     \\
\bottomrule
\end{tabular}
%\caption{Probit Marginal Effects}
\end{center}

\section{Discussion}
%%%%%%%%%%%%%%%%%%%%%%%%%%%%%%%%%%%%%%%%%%%%%%%%%%%%%%
%
% Introduction to what we are investigating
%
%%%%%%%%%%%%%%%%%%%%%%%%%%%%%%%%%%%%%%%%%%%%%%%%%%%%%%
This paper examines the non-take-up (NTU) of Germany’s BAföG student aid program, focusing on why eligible students forgo what is effectively ``free money'', a combination of a 50\% grant and a 50\% interest-free loan.

To investigate the individual-level determinants of BAföG non-take-up, we use microdata from the German Socio-Economic Panel (SOEP), which provides rich information on students' family background, financial situation, and behavioural traits relevant to financial aid decisions. 
Although SOEP has a panel structure, we treat the data as a pooled cross-section and assume the relationship between explanatory variables and non-take-up is stable across years.

We estimate three types of binary response models: a logit model, a probit model, and a linear probability model (LPM). 
These models allow us to assess the relationship between individual characteristics and the likelihood of non-take-up. 
Logit and probit models are used to account for the nonlinear nature of binary outcomes, while the LPM offers a straightforward linear benchmark and facilitates interpretation. 
Standard errors are clustered at the individual level to account for potential intra-respondent correlation in the presence of repeated observations.
Table~\ref{tab:logit_probit_lpm_results} presents the results from these three specifications, including estimated coefficients and average marginal effects where applicable.


%%%%%%%%%%%%%%%%%%%%%%%%%%%%%%%%%%%%%%%%%%%%%%%%%%%%%%
%
% Summary of key findings and interpretation
%
%%%%%%%%%%%%%%%%%%%%%%%%%%%%%%%%%%%%%%%%%%%%%%%%%%%%%%
Our results indicate that the most relevant explanatory variables for BAföG non-take-up are demographic characteristics, such as age, having a partner, and migration background. 
These factors suggest that individual life circumstances and aspects of social capital may influence both awareness of the program and the perceived need for financial aid.

Moreover, the significant role of East German background points to potential regional differences in attitudes toward state support. 
This could reflect divergent historical experiences and institutional trust, where East German students may be more inclined to view public assistance as legitimate, while in West Germany, stigma surrounding the uptake of state-subsidized aid may still persist.

In contrast, we find no evidence that behavioural traits such as patience, impulsiveness, or risk appetite significantly influence non-take-up. 
This suggests that psychological or personality-related factors are not the primary drivers behind the decision to forgo applying for BAföG.



%%%%%%%%%%%%%%%%%%%%%%%%%%%%%%%%%%%%%%%%%%%%%%%%%%%%%%
%
% Policy implications or practical relevance
%
%%%%%%%%%%%%%%%%%%%%%%%%%%%%%%%%%%%%%%%%%%%%%%%%%%%%%%
\subsection{Policy Implications}
Addressing the issue of non-take-up of social benefits is politically sensitive. 
Promoting policies that increase benefit take-up is often perceived as advocating for higher public expenditure, which may face resistance, even if the objective is to improve access and fairness within existing frameworks.

However, the very purpose of BAföG is to ensure that financial constraints do not prevent access to higher education. 
Our findings suggest that the current system places a considerable administrative burden on applicants, which may deter eligible students, especially those entitled to smaller amounts from applying. 
(\textcolor{red}{Maria: Source for the amount of hours it takes to apply!})
This is at odds with the program's intention to support students from less affluent backgrounds.

If the primary goal is to target financial need effectively, policymakers should consider simplifying the means-testing procedure to reduce unnecessary barriers. 
While some form of income assessment remains essential to ensure targeted support, the process should not be so complex or burdensome that it discourages eligible students from applying. 
Simplified procedures, increased use of pre-filled data from tax records, or more automated eligibility checks could help strike a better balance between precision and accessibility.

Ultimately, non-take-up driven by administrative complexity represents a failure of design rather than intent. 
A more user-friendly system could improve both the efficiency and equity of BAföG, ensuring that support reaches those who need it most—without necessarily requiring increased overall expenditure.



%%%%%%%%%%%%%%%%%%%%%%%%%%%%%%%%%%%%%%%%%%%%%%%%%%%%%%
%
% Comparison with previous literature
%
%%%%%%%%%%%%%%%%%%%%%%%%%%%%%%%%%%%%%%%%%%%%%%%%%%%%%%




%%%%%%%%%%%%%%%%%%%%%%%%%%%%%%%%%%%%%%%%%%%%%%%%%%%%%%
%
% Limitations of the study
%
%%%%%%%%%%%%%%%%%%%%%%%%%%%%%%%%%%%%%%%%%%%%%%%%%%%%%%

As previously mentioned, this study only takes to 

%%%%%%%%%%%%%%%%%%%%%%%%%%%%%%%%%%%%%%%%%%%%%%%%%%%%%%
%
% Suggestions for future research
%
%%%%%%%%%%%%%%%%%%%%%%%%%%%%%%%%%%%%%%%%%%%%%%%%%%%%%%




%REFERENCES
\newpage
\begin{flushleft}
  \titleformat{\section}{\normalfont\huge\bfseries}{\thesection}{1em}{} 
  \addcontentsline{toc}{section}{References} 
  \bibliography{bibliography, bafogandg, estg, solzg, econometrics}
\end{flushleft}

%APPENDICES
\newpage
\appendix
\setcounter{page}{1} %Restarts page counting to 1
\pagenumbering{roman} %Differentiates the appendix page numbers from the main article

\titleformat{\section}{\centering\normalfont\normalsize\bfseries}{Appendix \thesection: }{0em}{}
\newpage
\section{Tables}
\renewcommand{\thetable}{\thesection \arabic{table}}
\setcounter{table}{0}

\begingroup
\setlength{\tabcolsep}{4pt}
\renewcommand{\arraystretch}{0.95}
  \begin{table}
  \centering
  \begin{tabular}{rcc|rr|rr}
  \toprule
  Year & \multicolumn{2}{c|}{Consumer Price Index} & \multicolumn{2}{c|}{Average Payout (EUR)} & \multicolumn{2}{c}{Financial Expenditure (EUR 1,000)} \\
  \cmidrule(lr){2-3} \cmidrule(lr){4-5} \cmidrule(l){6-7}
  & Index (2020=100) & Price Factor (2023) & Nominal & Real (2023) & Nominal & Real (2023) \\
  \midrule
  1991 & 61 & 1.885 & 290 & 547 & 1,538,590 & 2,900,701 \\
  1992 & 65 & 1.795 & 290 & 521 & 1,539,929 & 2,764,764 \\
  1993 & 67 & 1.719 & 297 & 510 & 1,458,164 & 2,506,152 \\
  1994 & 69 & 1.674 & 295 & 494 & 1,257,002 & 2,104,621 \\
  1995 & 71 & 1.644 & 304 & 500 & 1,133,989 & 1,863,894 \\
  1996 & 72 & 1.621 & 322 & 522 & 1,059,270 & 1,716,900 \\
  1997 & 73 & 1.590 & 319 & 507 & 910,038 & 1,446,886 \\
  1998 & 74 & 1.577 & 316 & 498 & 861,688 & 1,358,905 \\
  1999 & 74 & 1.566 & 321 & 503 & 871,140 & 1,364,591 \\
  2000 & 75 & 1.546 & 326 & 504 & 906,857 & 1,401,724 \\
  2001 & 77 & 1.516 & 365 & 553 & 1,161,922 & 1,760,990 \\
  2002 & 78 & 1.494 & 371 & 554 & 1,350,543 & 2,018,032 \\
  2003 & 78 & 1.479 & 370 & 547 & 1,446,120 & 2,138,937 \\
  2004 & 80 & 1.455 & 371 & 540 & 1,513,641 & 2,202,517 \\
  2005 & 81 & 1.432 & 375 & 537 & 1,554,602 & 2,226,037 \\
  2006 & 82 & 1.409 & 375 & 529 & 1,538,770 & 2,168,773 \\
  2007 & 84 & 1.378 & 375 & 517 & 1,490,718 & 2,053,917 \\
  2008 & 86 & 1.343 & 398 & 534 & 1,590,638 & 2,136,104 \\
  2009 & 87 & 1.338 & 434 & 581 & 1,875,731 & 2,510,295 \\
  2010 & 88 & 1.325 & 436 & 578 & 2,019,078 & 2,674,533 \\
  2011 & 90 & 1.297 & 452 & 586 & 2,269,706 & 2,943,052 \\
  2012 & 91 & 1.273 & 448 & 570 & 2,364,963 & 3,009,718 \\
  2013 & 93 & 1.253 & 446 & 559 & 2,349,400 & 2,944,951 \\
  2014 & 94 & 1.241 & 448 & 556 & 2,280,748 & 2,831,524 \\
  2015 & 94 & 1.235 & 448 & 553 & 2,157,634 & 2,664,506 \\
  2016 & 95 & 1.228 & 464 & 570 & 2,099,110 & 2,578,590 \\
  2017 & 96 & 1.211 & 499 & 604 & 2,181,049 & 2,640,336 \\
  2018 & 98 & 1.190 & 493 & 586 & 2,001,732 & 2,381,265 \\
  2019 & 99 & 1.173 & 514 & 603 & 1,954,449 & 2,292,303 \\
  2020 & 100 & 1.167 & 574 & 670 & 2,210,920 & 2,580,143 \\
  2021 & 103 & 1.132 & 579 & 655 & 2,316,926 & 2,622,553 \\
  2022 & 110 & 1.059 & 611 & 647 & 2,454,392 & 2,599,161 \\
  2023 & 116 & 1.000 & 663 & 663 & 2,863,514 & 2,863,514 \\
  \bottomrule
  \end{tabular}
  \caption{
    Average nominal and inflation-adjusted payout under the Federal Training Assistance Act (BAföG) 
    for student recipients (excluding pupils), based on official data published by Destatis.  
    The table includes the Consumer Price Index (CPI, variable \textbf{PREIS1}, base year 2020 = 100) and 
    a derived price factor (column “Factor (2023)”) calculated using these CPI values to express nominal amounts in 2023 euros.  
    The inflation-adjusted average payouts and total financial expenditures were computed using this deflator and are not 
    reported as such in the original Destatis tables.
  }
  \label{table:payout_over_time}
  \end{table}
\endgroup

\begin{landscape}
\setlength{\tabcolsep}{4pt} % horizontal spacing between columns
\renewcommand{\arraystretch}{0.95} % vertical spacing between rows
\begin{table}
\centering
\begin{tabular}{rrrrrrrr}
\toprule
Year & Students (BIL002) & \multicolumn{3}{|c|}{Number of Supported Students} & \multicolumn{3}{c}{Proportion Supported (\%)} \\
\midrule
 & & Total Supported & Fully Supported & Partially Supported & Total & Fully & Partially \\
\midrule
2023 & 2,868,311 & 501,425 & 245,255 & 256,170 & 17.5 & 8.6 & 8.9 \\
2022 & 2,920,263 & 489,347 & 244,559 & 244,788 & 16.8 & 8.4 & 8.4 \\
2021 & 2,941,915 & 467,595 & 200,369 & 267,226 & 15.9 & 6.8 & 9.1 \\
2020 & 2,944,145 & 465,543 & 205,093 & 260,450 & 15.8 & 7.0 & 8.8 \\
2019 & 2,891,049 & 489,313 & 212,217 & 277,096 & 16.9 & 7.3 & 9.6 \\
2018 & 2,868,222 & 517,675 & 218,427 & 299,248 & 18.0 & 7.6 & 10.4 \\
2017 & 2,844,978 & 556,573 & 229,053 & 327,520 & 19.6 & 8.1 & 11.5 \\
2016 & 2,807,010 & 583,567 & 235,163 & 348,404 & 20.8 & 8.4 & 12.4 \\
2015 & 2,757,799 & 611,377 & 231,477 & 379,900 & 22.2 & 8.4 & 13.8 \\
2014 & 2,698,910 & 646,576 & 246,901 & 399,675 & 24.0 & 9.1 & 14.8 \\
2013 & 2,616,881 & 665,928 & 253,371 & 412,557 & 25.4 & 9.7 & 15.8 \\
2012 & 2,499,409 & 671,042 & 254,769 & 416,273 & 26.8 & 10.2 & 16.7 \\
2011 & 2,380,974 & 643,578 & 246,895 & 396,683 & 27.0 & 10.4 & 16.7 \\
2010 & 2,217,294 & 592,430 & 232,796 & 359,633 & 26.7 & 10.5 & 16.2 \\
2009 & 2,121,178 & 550,369 & 211,881 & 338,488 & 25.9 & 10.0 & 16.0 \\
2008 & 2,025,307 & 510,409 & 217,933 & 292,476 & 25.2 & 10.8 & 14.4 \\
2007 & 1,941,405 & 494,480 & 191,268 & 303,212 & 25.5 & 9.9 & 15.6 \\
2006 & 1,979,043 & 498,565 & 189,022 & 309,543 & 25.2 & 9.6 & 15.6 \\
2005 & 1,985,765 & 506,880 & 193,285 & 313,595 & 25.5 & 9.7 & 15.8 \\
2004 & 1,963,108 & 497,257 & 186,956 & 310,301 & 25.3 & 9.5 & 15.8 \\
2003 & 2,019,465 & 481,594 & 179,755 & 301,839 & 23.8 & 8.9 & 14.9 \\
2002 & 1,938,811 & 451,505 & 168,890 & 282,615 & 23.3 & 8.7 & 14.6 \\
2001 & 1,868,331 & 406,776 & 134,933 & 271,843 & 21.8 & 7.2 & 14.6 \\
2000 & 1,798,863 & 348,799 & 100,913 & 247,886 & 19.4 & 5.6 & 13.8 \\
1999 & 1,770,489 & 338,427 & 103,239 & 235,188 & 19.1 & 5.8 & 13.3 \\
1998 & 1,800,651 & 336,355 & 97,539 & 238,810 & 18.7 & 5.4 & 13.3 \\
\bottomrule
\end{tabular}
\caption{
  Number and percentage of students receiving BAföG support.  
  Columns:  
  \textbf{BIL002} = total number of students;  
  \textbf{PER010} = total supported students;  
  \textbf{PER011} = fully supported students;  
  \textbf{PER012} = partially supported students.
}
\label{table:bafoeg_support_landscape}
\end{table}
\end{landscape}

% # Variable codes 
% # PER 010 | Supported persons
% # PER 011 | Persons receiving full assistance payments
% # PER 012 | Persons receiving partial assistance payments
% # PER 013 | Supported persons (average monthly stock)
% # PER 014 | Average monthly assistance payment per person



\begin{table}[htbp]
\centering
\begin{tabular}{rrrrrr}
\toprule
\textbf{Year} & \(\bar{T}\) & \(\bar{R}\) & \(TU_T\,(\%)\) & \(TU_R\,(\%)\) & \(\Delta TU\,(\%)\) \\
\midrule
2007 & 273.76 & 295.62 & 17 & 15 & 2 \\
2008 & 328.19 & 283.12 & 22 & 18 & 4 \\
2009 & 345.99 & 312.12 & 21 & 18 & 4 \\
2010 & 347.84 & 299.32 & 22 & 17 & 4 \\
2011 & 408.08 & 358.95 & 32 & 21 & 11 \\
2012 & 429.39 & 393.09 & 33 & 23 & 10 \\
2013 & 384.99 & 356.63 & 31 & 23 & 8 \\
2014 & 404.91 & 383.70 & 32 & 23 & 10 \\
2015 & 407.80 & 364.48 & 34 & 18 & 16 \\
2016 & 407.84 & 354.35 & 33 & 20 & 13 \\
2017 & 425.46 & 406.28 & 37 & 19 & 18 \\
2018 & 462.37 & 385.60 & 36 & 21 & 15 \\
2019 & 450.52 & 397.07 & 30 & 16 & 14 \\
2020 & 528.82 & 454.99 & 30 & 19 & 11 \\
2021 & 529.51 & 418.95 & 34 & 17 & 17 \\
\bottomrule
\end{tabular}
\caption{Average BAföG amounts and take-up rates by survey year. 
\(\bar{T}\): average theoretical BAföG; 
\(\bar{R}\): average reported BAföG; 
\(TU_T\): theoretical take-up rate; 
\(TU_R\): reported take-up rate; 
\(\Delta TU = TU_T - TU_R\); 
\(TU_R / TU_T\): ratio of reported to theoretical take-up.}
\end{table}




\newpage
\section{Figures}
\renewcommand{\thefigure}{\thesection \arabic{figure}}
\setcounter{figure}{0}


\begin{figure}[H]
  \centering

  \begin{minipage}[t]{0.48\textwidth}
    \centering
    \includegraphics[width=0.95\linewidth]{fraction_of_enrolled_students_receiving_bafog.png}
    \caption{
      The figure illustrates the fraction of enrolled students in Germany receiving partial, full, or combined partial and full loans and grants over the same period. \textit{Own illustration}.
    }
  \label{figure:bafoeg_support}
  \end{minipage}%
  \hfill
  \begin{minipage}[t]{0.48\textwidth}
    \centering
    \includegraphics[width=0.95\linewidth]{payout_over_time.png}
    \caption{Average nominal and real payout under the Federal Training Assistance Act (BAföG) for category students (pupils excluded). \textit{Own illustration}.}
  \label{figure:payout_over_time}
  \end{minipage}

\end{figure}

%%%%%%%%%%%%%%%%%%%%%%%%%%%%%%%%%%%%%%%%%%%%%%%%%%%%%%%%%%%%%%%%%%%%%%%%%%%%
%  APPENDIX – MICROSIMULATION PIPELINE
%%%%%%%%%%%%%%%%%%%%%%%%%%%%%%%%%%%%%%%%%%%%%%%%%%%%%%%%%%%%%%%%%%%%%%%%%%%%

% -------------------------------------------------------------------------
%  Introductory motivation 
% -------------------------------------------------------------------------


\newpage
% ========================================================================
\section[Microsimulation Pipeline]{Microsimulation Pipeline\footnote{\href{https://github.com/Alexerby/msc-thesis-code-v4}{Byström, A. E., \& Antonsdottir, M. S.} (2025). \texttt{msc-thesis-code-v4}. GitHub. \url{https://github.com/Alexerby/msc-thesis-code-v4}.}}
\label{appendix:microsimulation-pipeline}
% ========================================================================

This appendix documents the microsimulation pipeline used to construct the analysis dataset from raw SOEP extracts. The goal is to make each step in the process transparent, reproducible, and aligned with the legal and institutional rules governing student aid eligibility in Germany.

The pipeline is organized into five main components: a student module, a sociodemographic enrichment module, a student income module, an assets module, and a parental income module. These components interact as illustrated in Figure~\ref{fig:pipeline-overview}.

\begin{figure}[H]
  \centering
  \begin{tikzpicture}[node distance=1.0cm and 1.0cm]

  % Switched layout
  \node[pipelinebox] (income) at (0,0) {Income module};
  \node[pipelinebox, right=of income] (assets) {Assets module};

  \node[pipelinebox, left=of income] (sibling) {Sibling (long)};
  \node[pipelinebox, below=of sibling] (sibling_wide) {Sibling grouped (wide)};

  \node[pipelinebox, left=of sibling] (parent) {Parent income (long)};
  \node[pipelinebox, below=of parent] (parent_wide) {Parents joint income (wide)};

  % Center reference node
  \path (parent) -- (assets) coordinate[midway] (center);

  % Student module centered above the whole group
  \node[pipelinebox, above=of center] (student) {Student module};

  % Sociodemographic module (to the right of Student module)
  \node[pipelinebox, right=of student] (sociodemog) {Sociodemographic module};

  % BAföG Calculations module below income/parent_wide/assets
  \node[pipelinebox, below=2.2cm of income] (bafoeg) {BAföG Calculations};

  % Arrows from submodules to BAföG Calculations
  \draw[arrow] (income) -- (bafoeg);
  \draw[arrow] (assets) -- (bafoeg);
  \draw[arrow] (parent_wide.south)
    -- ++(0,-0.8)
    coordinate (pstep)
    -- ([yshift=0]bafoeg.west |- pstep)
    -- ++(0.01,0);
  \draw[arrow] (parent) -- (parent_wide);

  % Arrows for sibling flow
  \draw[arrow] (sibling) -- (sibling_wide);
  \draw[arrow] (sibling_wide) -- (parent_wide);

  % Arrows from Student module to all first-row modules
  \draw[arrow] (student) -- (income);
  \draw[arrow] (student) -- (assets);
  \draw[arrow] (student) -- (parent);
  \draw[arrow] (student) -- (sibling);

  \draw[<->, thin] (sociodemog) -- (student);

  \end{tikzpicture}
  \caption{End-to-end pipeline overview with grouped raw SOEP sources}
  \label{fig:pipeline-overview}
\end{figure}

Figure~\ref{fig:pipeline-overview} maps the complete microsimulation pipeline, from raw SOEP inputs through modular components to the final BAföG calculation. The remainder of this appendix explains each component in detail.

% -------------------------------------------------------------------------
\subsection{Sociodemographic Module}
% -------------------------------------------------------------------------

This module constructs basic demographic characteristics needed throughout the pipeline. It provides sex, age, federal state, and household type variables. The federal state is used to derive an East/West classification, relevant for BAföG eligibility rules.

Key data sources include:

\begin{itemize}
  \item \texttt{ppathl}: sex, birth year and month
  \item \texttt{hgen}: household type
  \item \texttt{regionl}: federal state of residence
\end{itemize}

These variables are merged into the student module and used in eligibility filtering, modeling allowances, and regional policy differences.

% -------------------------------------------------------------------------
\subsection{Student Module}
% -------------------------------------------------------------------------

The student module is the core unit of the microsimulation. 
It filters and prepares individuals from SOEP who qualify as students, and populates them with relevant characteristics for BAföG simulation. 
This includes education status, household composition, parental identifiers, employment, and relationship status.

It integrates inputs from multiple datasets:

\begin{itemize}
  \item Education and religion: from \texttt{pl} (\texttt{plg0012\_h}, \texttt{plh0258\_h})
  \item Living with parents: based on household and parent ID matches in \texttt{ppathl}
  \item Employment status: from \texttt{pgen} (\texttt{pgemplst})
  \item Number of children: counted from \texttt{bioparen} where the student appears as parent
  \item Partnership status: inferred from household records
\end{itemize}

The resulting dataset includes all eligible students, ready for downstream processing.

% -------------------------------------------------------------------------
\subsection{Student-Income Module}
% -------------------------------------------------------------------------

This module computes the student's BAföG-relevant income. It begins with gross labour income and applies the following processing steps:

\begin{itemize}
  \item Merges reported income from the relevant assessment year (typically the previous calendar year)
  \item Deduction of Werbungskosten (§~9a EStG)
  \item Social insurance allowance of (§~21 BAföG)
  \item Calculation and subtraction of income tax, church tax, and solidarity surcharge
  \item Comparison with personal exemption thresholds (§~23 BAföG)
\end{itemize}

The output is net income above allowances, which may reduce BAföG entitlements.

% -------------------------------------------------------------------------
\subsection{Assets Module}
% -------------------------------------------------------------------------

The assets module compiles student-owned assets from the \texttt{pwealth} dataset. This includes:

\begin{itemize}
  \item \textbf{Financial assets:} bank accounts, savings, stocks, and bonds (\texttt{f0100a}--\texttt{f0100e})
  \item \textbf{Real estate:} other property ownership and shares in real estate (\texttt{e0111a}--\texttt{e0111e})
  \item \textbf{Business assets:} stakes in private businesses or self-employment (\texttt{b0100a}--\texttt{b0100e})
  \item \textbf{Private insurances:} building loan contracts, life and pension insurance (\texttt{i0100a}--\texttt{i0100e})
  \item \textbf{Vehicles:} cars, motorcycles, and other personal transport (\texttt{v0100a}--\texttt{v0100e})
  \item \textbf{Tangible assets:} valuables such as jewelry, art, or furniture (\texttt{t0100a}--\texttt{t0100e})
  \item \textbf{Liabilities and debts:} total outstanding debt, excluding student loans (\texttt{w0011a}--\texttt{w0011e})
\end{itemize}

All components are aggregated into a student-level asset profile. This value is compared against the legally defined asset allowance to determine whether the individual exceeds the threshold and may be excluded from eligibility.

The legal basis for this assessment is §~29 of the Bundesausbildungsförderungsgesetz (BAföG), which specifies the applicable asset allowances (\emph{Freibeträge vom Vermögen}).

% -------------------------------------------------------------------------
\subsection{Parental-Income Module}
% -------------------------------------------------------------------------

This module estimates the amount that parents are expected to contribute to a student's support. It proceeds as follows:

\begin{itemize}
  \item Match student with parents using \texttt{bioparen}
  \item Extract gross income from \texttt{pgen}
  \item Deduct Werbungskosten and social insurance contributions
  \item Apply parental tax model and §~25 BAföG allowances (base, sibling, relationship)
\end{itemize}

If both parents are observed, their contributions are aggregated. The final value is the BAföG-relevant parental contribution.

% -------------------------------------------------------------------------
\subsection{BAföG Calculation}
% -------------------------------------------------------------------------

The final module brings together all student and parental variables to compute a theoretical BAföG entitlement. The following logic is applied:

\begin{itemize}
  \item Deduct student income from allowances
  \item Subtract parental contribution
  \item Exclude students who exceed asset thresholds
\end{itemize}

This results in an estimated monthly benefit, which can be compared to reported values to analyze take-up behavior and simulate reforms.

% -------------------------------------------------------------------------
\subsection{Variable Dictionary}
% -------------------------------------------------------------------------

\begin{longtable}{llll}
\caption{Variable Dictionary by Dataset} 
 \label{table:variable_dictionary} \\
\toprule
Dataset & Variable & Description & Data Type (semantic) \\
\midrule
\endfirsthead

\multicolumn{4}{l}{\textit{(continued from previous page)}} \\
\toprule
Dataset & Variable & Description & Data Type (semantic) \\
\midrule
\endhead

\bottomrule
\multicolumn{4}{r}{\textit{(continued on next page)}} \\
\endfoot

\bottomrule
\endlastfoot
\texttt{ppathl} & \texttt{pid} & Person identifier & int \\
\texttt{ppathl} & \texttt{hid} & Household ID & int \\
\texttt{ppathl} & \texttt{syear} & Survey year & date \\
\texttt{ppathl} & \texttt{gebjahr} & Year of birth & int \\
\texttt{ppathl} & \texttt{sex} & Sex & Categorical \\
\texttt{ppathl} & \texttt{gebmonat} & Month of birth & int \\
\texttt{ppathl} & \texttt{partner} & Partnership status & Categorical \\
\texttt{ppathl} & \texttt{migback} & Migration background & Categorical \\
\texttt{biosib} & \texttt{pid} & Person identifier & int \\
\texttt{biosib} & \texttt{sibpnr1--sibpnr11} & Sibling person numbers & int \\
\texttt{pl} & \texttt{pid} & Person identifier & int \\
\texttt{pl} & \texttt{syear} & Survey year & date \\
\texttt{pl} & \texttt{plg0012\_h} & Currently in education & Ordinal \\
\texttt{pl} & \texttt{plh0258\_h} & Religion / church membership & Categorical \\
\texttt{pl} & \texttt{plc0167\_h} & BAföG eligibility & Binary \\
\texttt{pl} & \texttt{plc0168\_h} & BAföG / scholarship (gross, monthly) & int \\
\texttt{pl} & \texttt{plg0014\_v5} & Education level, 1999--2008 & Ordinal \\
\texttt{pl} & \texttt{plg0014\_v6} & Education level, 2009-2012 & Ordinal \\
\texttt{pl} & \texttt{plg0014\_v7} & Education level, 2013--2022 & Ordinal \\
\texttt{pgen} & \texttt{pid} & Person identifier & int \\
\texttt{pgen} & \texttt{syear} & Survey year & date \\
\texttt{pgen} & \texttt{pglabgro} & Labour income (gross) & int \\
\texttt{pgen} & \texttt{pgemplst} & Employment status & Categorical \\
\texttt{pgen} & \texttt{pgpartnr} & Partner indicator & int \\
\texttt{pkal} & \texttt{pid} & Person identifier & int \\
\texttt{pkal} & \texttt{syear} & Survey year & date \\
\texttt{pkal} & \texttt{kal2a02} & Monthly rent including utilities & int \\
\texttt{pkal} & \texttt{kal2a03\_h} & Housing benefit & int \\
\texttt{pwealth} & \texttt{pid} & Person identifier & int \\
\texttt{pwealth} & \texttt{syear} & Survey year & date \\
\texttt{pwealth} & \texttt{f0100a--f0100e} & Financial assets & int \\
\texttt{pwealth} & \texttt{e0111a--e0111e} & Real estate (net value shares) & int \\
\texttt{pwealth} & \texttt{b0100a--b0100e} & Business assets & int \\
\texttt{pwealth} & \texttt{i0100a--i0100e} & Private insurances & int \\
\texttt{pwealth} & \texttt{v0100a--v0100e} & Vehicles & int \\
\texttt{pwealth} & \texttt{t0100a--t0100e} & Tangible assets & int \\
\texttt{pwealth} & \texttt{w0011a--w0011e} & Liabilities and debts & int \\
\texttt{bioparen} & \texttt{pid} & Person identifier & int \\
\texttt{bioparen} & \texttt{fnr} & Father’s person ID & int \\
\texttt{bioparen} & \texttt{mnr} & Mother’s person ID & int \\
\texttt{regionl} & \texttt{hid} & Household ID & int \\
\texttt{regionl} & \texttt{bula} & Federal state (Bundesland) & Categorical \\
\texttt{regionl} & \texttt{syear} & Survey year & date \\
\texttt{hgen} & \texttt{hid} & Household ID & int \\
\texttt{hgen} & \texttt{hgtyp1hh} & Household type & Categorical \\
\texttt{hgen} & \texttt{syear} & Survey year & date \\
\texttt{pequiv} & \texttt{pid} & Person identifier & int \\
\texttt{pequiv} & \texttt{istuy} & Student grants received & int \\
\texttt{pequiv} & \texttt{syear} & Survey year & date \\
\texttt{biol} & \texttt{pid} & Person identifier & int \\
\texttt{biol} & \texttt{pid} & Person identifier & int \\
\texttt{biol} & \texttt{syear} & Survey year & date \\
\texttt{biol} & \texttt{lb0267\_v1} & Employment Status & Categorical \\
\texttt{biol} & \texttt{syear} & Survey year & date \\
\texttt{biol} & \texttt{lb0285} & Number of Children & int \\
\end{longtable}

%%%%%%%%%%%%%%%%%%%%%%%%%%%%%%%%%%%%%%%%%%%%%%%%%%%%%%%%%%%%%%%%%%%%%%%%%%%%
%  APPENDIX – EXAMPLE FROM MICROSIMULATION
%%%%%%%%%%%%%%%%%%%%%%%%%%%%%%%%%%%%%%%%%%%%%%%%%%%%%%%%%%%%%%%%%%%%%%%%%%%%

% -------------------------------------------------------------------------
%  Introductory motivation 
% -------------------------------------------------------------------------

\newpage
% ========================================================================
\section[Example Calculation: Theoretical BAföG Eligibility]{Example Calculation: Theoretical BAföG Eligibility\footnote{Based on calculations using the microsimulation pipeline introduced in Appendix~\ref{appendix:microsimulation-pipeline}.}}
\label{appendix:bafoeg-example}
% ========================================================================

This appendix documents the step-by-step calculation of theoretical BAföG eligibility for a selected individual from the SOEP-Core dataset. The example is based on data from survey year 2018 and focuses on a university student identified by \texttt{pid = 20156903}.

The purpose of this example is to illustrate how legal rules governing student financial aid—particularly those defined in the Federal Training Assistance Act (BAföG)—are operationalized within the microsimulation pipeline. Each component of the calculation is made transparent, including the determination of the student's assessed need, applicable supplements, and deductions based on income and assets.

The case selected is representative of a full-time student living independently, with modest student income, limited parental support, and non-negligible declared assets. The final theoretical BAföG award is computed by subtracting excess income and asset contributions from the total assessed need.

A summary of the key outcome variables is presented in Table~\ref{table:bafoeg_final_award}. Subsequent sections decompose and document the logic and parameters behind each component in detail.





% -------------------------------------------------------------------------
%  Base Need
% -------------------------------------------------------------------------


\subsection{Total Base Need}
\subsubsection{Base Need}
The base need (\texttt{base\_need}) is a flat-rate amount representing the monthly minimum subsistence level for students in higher education. It is specified in §~13(1) Nr.~1 of the Federal Training Assistance Act (BAföG) and does not vary by income, living arrangement, or demographic characteristics.

For all eligible university students during the relevant period, the base need was set at 399 EUR. Since the student in this case study meets the criteria for university-level BAföG support, this full amount is assigned without adjustment.

\begin{table}[H]
\scriptsize
\centering
\begin{tabular}{llr}
\toprule
\textbf{Component} & \textbf{Explanation} & \textbf{Value (EUR)} \\
\midrule
Base Need & Flat-rate monthly amount for university students & 399 \\
\bottomrule
\end{tabular}
\caption{Base need (\texttt{base\_need}) for pid 20156903, in accordance with §~13(1) Nr.~1 BAföG.}
\label{table:bafoeg_base_need}
\end{table}

\subsubsection{Housing Allowance}
The housing allowance (\texttt{housing\_allowance}) compensates students for living expenses incurred while living outside the parental home. According to §~13(1) Nr.~2 BAföG, students who do not reside with their parents are entitled to a fixed monthly supplement to cover rent and related costs.

In this example, the student was classified as living independently. While the statutory maximum at the time was 399 EUR, the simulation applies a standardized flat amount of 250 EUR to align with data quality and institutional thresholds reflected in the SOEP housing variables.

\begin{table}[H]
\scriptsize
\centering
\begin{tabular}{llr}
\toprule
\textbf{Component} & \textbf{Explanation} & \textbf{Value (EUR)} \\
\midrule
Housing Allowance & Standard flat rate applied for non-parental housing & 250 \\
\bottomrule
\end{tabular}
\caption{Housing allowance (\texttt{housing\_allowance}) for pid 20156903, based on §~13(1) Nr.~2 BAföG.}
\label{table:bafoeg_housing}
\end{table}

\subsubsection{Insurance Supplement}
Students with statutory health and long-term care insurance are entitled to receive flat-rate supplements as defined in §~13a(1) BAföG. These rates vary by time period and are adjusted periodically by legal amendment.

For survey year 2018, the applicable values—according to the 2020-08-01 rates still valid at the time—were:
\begin{itemize}
    \item 61 EUR for health insurance (§~13a(1) Nr.~1 BAföG)
    \item 25 EUR for long-term care insurance (§~13a(1) Nr.~2 BAföG)
\end{itemize}

These two components sum to 86 EUR, which is assigned as the total insurance supplement for this individual.

\begin{table}[H]
\scriptsize
\centering
\begin{tabular}{llr}
\toprule
\textbf{Component} & \textbf{Explanation} & \textbf{Value (EUR)} \\
\midrule
Insurance Supplement & Sum of flat-rate statutory insurance allowances & 86 \\
\quad Health insurance & §~13a(1) Nr.~1 BAföG (statutory health insurance) & 61 \\
\quad Care insurance & §~13a(1) Nr.~2 BAföG (statutory long-term care insurance) & 25 \\
\bottomrule
\end{tabular}
\caption{Insurance supplement (\texttt{insurance\_supplement}) for pid 20156903. Rates valid for the 2018 survey year.}
\label{table:bafoeg_insurance}
\end{table}


% -------------------------------------------------------------------------
%  Student Excess Income
% -------------------------------------------------------------------------
\subsection{Student Excess Income}
The student’s excess income (\texttt{excess\_income\_stu}) represents the amount by which their own annual income—after standard deductions—exceeds the personal allowance defined under §~23(1) Nr.~1 BAföG. This component is subtracted from the total assessed need to determine theoretical eligibility.

\paragraph{Step 1: Estimating Gross Annual Income}  
The student’s income is derived from the SOEP variable \texttt{kal2a03\_h}, which reports average gross monthly earnings. This value is multiplied by the number of working months in the previous calendar year (\texttt{kal2a02}) to estimate gross annual income.  
For \texttt{pid = 20156903}:
\begin{itemize}
    \item Gross monthly income: 523 EUR
    \item Months worked: 12
    \item $\Rightarrow$ Gross annual income: $523 \times 12 = 6{,}276$ EUR
\end{itemize}

\paragraph{Step 2: Standard Deductions}  
Two statutory deductions are applied to estimate net taxable income:
\begin{itemize}
    \item \textbf{Werbungskostenpauschale} (fixed deduction for work-related expenses): 290 EUR (2018)
    \item \textbf{Sozialversicherungs-Pauschale} (fixed social insurance deduction): 17.2\% of remaining income, capped at 17,200 EUR
\end{itemize}

\begin{itemize}
    \item Step 1: $6{,}276 - 290 = 5{,}986$ EUR
    \item Step 2: $5{,}986 \times 0.828 = 4{,}152.21$ EUR (after 17.2\% deduction)
\end{itemize}

\paragraph{Step 3: Applying Income Tax}
The BAföG calculator applies German income tax tables to compute statutory income tax liabilities. In this case, the taxable income falls below the basic allowance threshold (9,000 EUR in 2018), so no income tax, church tax, or solidarity surcharge is applied:
\begin{itemize}
    \item Income tax: 0 EUR
    \item Church tax: 0 EUR
    \item Solidarity surcharge: 0 EUR
    \item $\Rightarrow$ Net annual income: 4,152.21 EUR
\end{itemize}

\paragraph{Step 4: Monthly Net Income and Allowance}
The student’s net monthly income is calculated as:
\[
\frac{4{,}152.21}{12} \approx 346.02~\text{EUR}
\]

The personal allowance specified in §~23(1) Nr.~1 BAföG for the year 2018 was 290 EUR per month. Thus, the student’s excess income is:
\[
346.02 - 290 = 56.02~\text{EUR}
\]

\begin{table}[H]
\scriptsize
\centering
\begin{tabular}{llr}
\toprule
\textbf{Component} & \textbf{Explanation} & \textbf{Value (EUR)} \\
\midrule
Gross monthly income & From SOEP variable \texttt{kal2a03\_h} & 523 \\
Working months (previous year) & From SOEP variable \texttt{kal2a02} & 12 \\
Gross annual income & Estimated income before deductions & 6,276 \\
Werbungskostenpauschale & Work-related fixed deduction (§~21(2) BAföG) & 290 \\
Sozialversicherungs-Pauschale & 17.2\% statutory deduction & 1,133.79 \\
Net annual income & Income after deductions & 4,152.21 \\
Net monthly income & Annual net income divided by 12 & 346.02 \\
Personal allowance & §~23(1) Nr.~1 BAföG (2018) & 290 \\
\textbf{Student excess income} & Amount exceeding allowance & \textbf{56.02} \\
\bottomrule
\end{tabular}
\caption{Calculation of student’s excess income (\texttt{excess\_income\_stu}) for pid 20156903.}
\label{table:bafoeg_excess_income_stu}
\end{table}



% -------------------------------------------------------------------------
%  Parental Income Father
% -------------------------------------------------------------------------
\subsection{Parental Income Evaluation: Father (pid = 20156901)}

This section documents the step-by-step derivation of net income for the student's father using variables from the SOEP-Core dataset and applying BAföG-compliant statutory deductions.

\paragraph{Step 1: Gross Income}

The parent reported a gross monthly income of 3,500 EUR and worked 12 months in the prior year, resulting in:

\[
\text{Gross annual income} = 3{,}500 \times 12 = 42{,}000~\text{EUR}
\]

\paragraph{Step 2: Werbungskostenpauschale (§~21 Abs.~2 BAföG)}

A fixed deduction of 1,000 EUR is applied to account for work-related expenses:

\[
\texttt{inc\_w} = 42{,}000 - 1{,}000 = 41{,}000~\text{EUR}
\]

\paragraph{Step 3: Sozialversicherungs-Pauschale (§~21 Abs.~2 BAföG)}

Next, a 21.3\% deduction is applied to the income after Werbungskosten:

\[
\texttt{inc\_si} = 41{,}000 \times (1 - 0.213) = 41{,}000 \times 0.787 = 32{,}267~\text{EUR}
\]

\paragraph{Step 4: Income Tax Calculation (§~32a EStG)}

The parent is assessed as an individual (not jointly filed). Based on the 2018 tax table and a taxable income of 32,267 EUR, the following taxes are applied:

- \textbf{Income tax:} 6,062 EUR (per simulation based on §~32a EStG)
- \textbf{Church tax:} 0 EUR (not church-affiliated in SOEP)
- \textbf{Solidarity surcharge (Soli):} 333 EUR

The solidarity surcharge applies since taxable income exceeds the 2018 exemption threshold of 972 EUR (§~32a Abs.~5 \& 6 EStG, pre-2020 version). The surcharge is 5.5\% of income tax, capped by taper rules.

\paragraph{Step 5: Net Annual and Monthly Income}

\[
\texttt{inc\_net} = 32{,}267 - 6{,}062 - 0 - 333 = 25{,}872~\text{EUR}
\]
\[
\texttt{net\_monthly\_income} = \frac{25{,}872}{12} = 2{,}156~\text{EUR}
\]

\begin{table}[H]
\scriptsize
\centering
\begin{tabular}{llr}
\toprule
\textbf{Component} & \textbf{Explanation} & \textbf{Value (EUR)} \\
\midrule
Gross monthly income & Reported by SOEP & 3,500 \\
Working months & From SOEP (previous year) & 12 \\
Gross annual income & $3{,}500 \times 12$ & 42,000 \\
Werbungskostenpauschale & Fixed work-related deduction (§~21(2)) & 1,000 \\
Post-werbung income (\texttt{inc\_w}) & After deduction & 41,000 \\
Sozialversicherungs-Pauschale & 21.3\% of \texttt{inc\_w} & 8,733 \\
Income after SI (\texttt{inc\_si}) & $41{,}000 \times 0.787$ & 32,267 \\
Income tax & Based on §~32a EStG table & 6,062 \\
Church tax & SOEP indicates no affiliation & 0 \\
Solidarity surcharge & 5.5\% of income tax (capped) & 333 \\
Net annual income (\texttt{inc\_net}) & After all taxes & 25,872 \\
Net monthly income & $25{,}872 \div 12$ & 2,156 \\
\bottomrule
\end{tabular}
\caption{Income derivation for father (pid = 20156901) in 2018.}
\label{table:bafoeg_parent_father}
\end{table}


% -------------------------------------------------------------------------
%  Parental Income Mother
% -------------------------------------------------------------------------
\subsection{Parental Income Evaluation: Mother (pid = 20156902)}

The same procedure is applied to evaluate the income of the student’s mother. This parent reports a lower monthly income, but the same deductions are used to compute a BAföG-compliant net income value.

\paragraph{Step 1: Gross Income}

The mother reported a gross monthly income of 300 EUR and worked 12 months in the previous year:

\[
\text{Gross annual income} = 300 \times 12 = 3{,}600~\text{EUR}
\]

\paragraph{Step 2: Werbungskostenpauschale (§~21 Abs.~2 BAföG)}

A fixed work-related deduction of 1,000 EUR is applied:

\[
\texttt{inc\_w} = 3{,}600 - 1{,}000 = 2{,}600~\text{EUR}
\]

\paragraph{Step 3: Sozialversicherungs-Pauschale (§~21 Abs.~2 BAföG)}

A 21.3\% deduction is then applied:

\[
\texttt{inc\_si} = 2{,}600 \times 0.787 = 2{,}046.20~\text{EUR}
\]

\paragraph{Step 4: Income Tax and Surcharges}

Because the income falls well below the basic exemption threshold, no income tax or surcharges apply:

- Income tax: 0 EUR
- Church tax: 0 EUR
- Solidarity surcharge: 0 EUR

\paragraph{Step 5: Net Annual and Monthly Income}

\[
\texttt{inc\_net} = 2{,}046.20~\text{EUR}
\qquad\quad
\texttt{net\_monthly\_income} = \frac{2{,}046.20}{12} = 170.52~\text{EUR}
\]

\begin{table}[H]
\scriptsize
\centering
\begin{tabular}{llr}
\toprule
\textbf{Component} & \textbf{Explanation} & \textbf{Value (EUR)} \\
\midrule
Gross monthly income & Reported by SOEP & 300 \\
Working months & From SOEP (previous year) & 12 \\
Gross annual income & $300 \times 12$ & 3,600 \\
Werbungskostenpauschale & Fixed deduction (§~21(2)) & 1,000 \\
Post-werbung income (\texttt{inc\_w}) & After deduction & 2,600 \\
Sozialversicherungs-Pauschale & 21.3\% of \texttt{inc\_w} & 553.80 \\
Income after SI (\texttt{inc\_si}) & $2{,}600 \times 0.787$ & 2,046.20 \\
Income tax & Below exemption threshold & 0 \\
Church tax & SOEP indicates no affiliation & 0 \\
Solidarity surcharge & Below threshold & 0 \\
Net annual income (\texttt{inc\_net}) & After all taxes & 2,046.20 \\
Net monthly income & $2{,}046.20 \div 12$ & 170.52 \\
\bottomrule
\end{tabular}
\caption{Income derivation for mother (pid = 20156902) in 2018.}
\label{table:bafoeg_parent_mother}
\end{table}


% -------------------------------------------------------------------------
%  Joint Parental Income
% -------------------------------------------------------------------------
\subsection{Joint Parental Income and Deductions}

After calculating net income for each parent individually, their incomes are combined and assessed jointly, following the rules laid out in §~25 and §~21 of the BAföG Act. This section outlines how the parental income is evaluated as a unit, and how the applicable deductions reduce the contribution relevant for BAföG eligibility.

\paragraph{Step 1: Joint Income}

The net monthly incomes of both parents are summed to form the joint income base:

\[
\texttt{joint\_income} = 2{,}156 + 170.52 = 2{,}326.52~\text{EUR}
\]

\paragraph{Step 2: Parental Allowance (§~25(1) Nr.~1 BAföG)}

Because both parents are financially active, the applicable allowance is the joint parental allowance. According to the BAföG schedule valid from 2015-01-01 (25. BAföGÄndG), the relevant allowance value is:

\[
\texttt{total\_allowance} = 1{,}715~\text{EUR}
\]

The remaining income after allowance is:

\[
\texttt{joint\_income\_less\_ba} = 2{,}326.52 - 1{,}715 = 611.52~\text{EUR}
\]

\paragraph{Step 3: Sibling Deduction (§~25(3) BAföG)}

The student has two siblings who are eligible for sibling-related deductions. According to the 2015 allowance table:

- The sibling deduction per eligible sibling is 260 EUR
- Total deduction: $2 \times 260 = 520$ EUR

\[
\texttt{joint\_income\_less\_ba\_and\_sib} = 611.52 - 520 = 91.52~\text{EUR}
\]

\paragraph{Step 4: Additional Allowance (§~25(4) BAföG)}

In addition, §~25(4) BAföG entitles parents to a percentage-based deduction on the remaining income. According to the allowance rules:

- A base allowance of 50\% of the remainder applies
- Plus 5\% per sibling with a positive deduction

Thus, the applied rate is:

\[
50\% + (2 \times 5\%) = 60\%
\]

\[
\texttt{additional\_allowance} = 91.52 \times 0.60 = 54.91~\text{EUR}
\]

\paragraph{Step 5: Final Excess Parental Income}

The final contribution from parental income is the remaining amount after all deductions:

\[
\texttt{excess\_income} = 91.52 - 54.91 = 36.61~\text{EUR}
\]

\begin{table}[H]
\scriptsize
\centering
\begin{tabular}{llr}
\toprule
\textbf{Component} & \textbf{Explanation} & \textbf{Value (EUR)} \\
\midrule
Joint income & Sum of both parents’ net monthly incomes & 2,326.52 \\
Parental allowance & §~25(1) Nr.~1 BAföG (joint allowance) & 1,715 \\
Remaining after allowance & $2{,}326.52 - 1{,}715$ & 611.52 \\
Sibling deduction & $2 \times 260$ (§~25(3) BAföG) & 520 \\
Remaining after siblings & $611.52 - 520$ & 91.52 \\
Additional allowance & 60\% of remaining income (§~25(4)) & 54.91 \\
\textbf{Excess parental income} & Final contribution to be deducted & \textbf{36.61} \\
\bottomrule
\end{tabular}
\caption{Calculation of joint parental excess income for pid 20156903 (2018).}
\label{table:bafoeg_joint_income}
\end{table}


% -------------------------------------------------------------------------
%  Asset Excess
% -------------------------------------------------------------------------
\subsection{Asset-Based Contribution}

Students whose personal assets exceed a legally defined exemption threshold are required to contribute the excess toward their BAföG need (§~29 BAföG). The following table lists all relevant asset categories reported in the SOEP and their treatment in the eligibility assessment for this individual.

\paragraph{Step 1: Declared Asset Categories}

The student’s asset-related information for the 2018 survey year is as follows:

\begin{table}[H]
\scriptsize
\centering
\begin{tabular}{lr}
\toprule
\textbf{Asset Category} & \textbf{Value (EUR)} \\
\midrule
Financial assets (e.g., savings, stocks) & 0 \\
Real estate (e.g., land, housing property) & 0 \\
Business assets & 0 \\
Private insurance assets & 0 \\
Vehicles (e.g., car ownership) & 7,940 \\
Tangible assets (furniture, equipment) & 0 \\
Eligible debts (offsetting) & 0 \\
\midrule
\textbf{Total assets} & 7,940 \\
\textbf{Debts} & 0 \\
\textbf{Net assets} & 7,940 \\
\bottomrule
\end{tabular}
\caption{Declared asset categories for pid 20156903 in 2018.}
\label{table:bafoeg_declared_assets}
\end{table}

\paragraph{Step 2: Asset Allowance (§~29 BAföG)}

Since the student was 25 years old in 2018 (i.e., under 30), the asset allowance for students under age 30 applied. According to the table valid from 2016-08-01 (25. BAföGÄndG), this exemption was:

\[
\texttt{asset\_allowance} = 7{,}500~\text{EUR}
\]

\paragraph{Step 3: Excess Asset Contribution}

The contribution from assets is computed as the difference between net assets and the legal allowance:

\[
\texttt{excess\_assets} = \max(7{,}940 - 7{,}500, 0) = 440~\text{EUR}
\]

\begin{table}[H]
\scriptsize
\centering
\begin{tabular}{llr}
\toprule
\textbf{Component} & \textbf{Explanation} & \textbf{Value (EUR)} \\
\midrule
Net assets & Total assets minus eligible debts & 7,940 \\
Asset allowance & §~29 BAföG (U30 threshold in 2018) & 7,500 \\
\textbf{Excess asset contribution} & Final deduction from BAföG entitlement & \textbf{440} \\
\bottomrule
\end{tabular}
\caption{Excess asset calculation for pid 20156903 in 2018.}
\label{table:bafoeg_excess_assets}
\end{table}


\subsection{Final Theoretical BAföG Award}

After accounting for all relevant supplements and income-based deductions, the theoretical BAföG award is computed by subtracting the student’s and parents’ contributions—as well as any asset-based contributions—from the total assessed need.

\paragraph{Step 1: Total Assessed Need}

The total monthly need is composed of:
\begin{itemize}
    \item Base need (\texttt{base\_need}): 399 EUR
    \item Housing allowance (\texttt{housing\_allowance}): 250 EUR
    \item Insurance supplement (\texttt{insurance\_supplement}): 86 EUR
\end{itemize}

\[
\texttt{total\_base\_need} = 399 + 250 + 86 = 735~\text{EUR}
\]

\paragraph{Step 2: Total Deductions}

The following deductions apply:
\begin{itemize}
    \item Student excess income: 56.02 EUR
    \item Parental excess income: 36.61 EUR
    \item Excess asset contribution: 440.00 EUR
\end{itemize}

\[
\texttt{total\_deductions} = 56.02 + 36.61 + 440 = 532.63~\text{EUR}
\]

\paragraph{Step 3: Theoretical Award Calculation}

\[
\texttt{theoretical\_bafög} = \max(735 - 532.63,\ 0) = \textbf{202.38~EUR}
\]

\begin{table}[H]
\scriptsize
\centering
\begin{tabular}{llr}
\toprule
\textbf{Component} & \textbf{Explanation} & \textbf{Value (EUR)} \\
\midrule
Base need & §~13(1) Nr.~1 BAföG & 399 \\
Housing allowance & §~13(1) Nr.~2 BAföG & 250 \\
Insurance supplement & §~13a(1) BAföG & 86 \\
\midrule
\textbf{Total base need} & Monthly assessed need & \textbf{735} \\
\midrule
Student excess income & §~23(1) Nr.~1 BAföG & 56.02 \\
Parental excess income & §~25 BAföG + sibling adjustment & 36.61 \\
Excess asset contribution & §~29 BAföG & 440.00 \\
\midrule
\textbf{Total deductions} & Income and asset-based contributions & \textbf{532.63} \\
\midrule
\textbf{Theoretical BAföG award} & \textbf{Maximum eligible amount} & \textbf{202.38} \\
\bottomrule
\end{tabular}
\caption{Final theoretical BAföG award for pid 20156903 in 2018.}
\label{table:bafoeg_final_award}
\end{table}

\paragraph{Note on Eligibility Status}

This student qualifies for BAföG under the legal eligibility criteria defined by income, asset, and need thresholds. While their theoretical eligibility status is coded as \texttt{1} (eligible), they did not receive or report any BAföG support in the SOEP dataset:

\begin{itemize}
    \item \texttt{received\_bafög} = 0 EUR
    \item \texttt{reported\_bafög} = 0 EUR
    \item \texttt{theoretical\_eligibility} = 1 (eligible)
\end{itemize}

\end{document}
