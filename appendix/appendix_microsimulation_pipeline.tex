%%%%%%%%%%%%%%%%%%%%%%%%%%%%%%%%%%%%%%%%%%%%%%%%%%%%%%%%%%%%%%%%%%%%%%%%%%%%
%  APPENDIX – MICROSIMULATION PIPELINE
%%%%%%%%%%%%%%%%%%%%%%%%%%%%%%%%%%%%%%%%%%%%%%%%%%%%%%%%%%%%%%%%%%%%%%%%%%%%

\newpage
% ========================================================================
\section[Microsimulation Appendix]{Microsimulation Pipeline (Technical Appendix)\footnote{
This section documents the simulation code and data processing steps. Code is on GitHub \cite{bystrom2025msc}. § denotes legal paragraphs; EStG = Einkommensteuergesetz; BAföG = Bundesausbildungsförderungsgesetz.}
}
\label{appendix:microsimulation-pipeline}
% ========================================================================

This appendix documents the microsimulation pipeline used to construct the analysis dataset from raw SOEP extracts. The aim is to ensure transparency, reproducibility, and alignment with the legal and institutional framework governing student aid eligibility in Germany.

The pipeline comprises five primary components: the student module, sociodemographic enrichment module, student income module, assets module, and parental income module.
Each module is detailed below.

% -------------------------------------------------------------------------
\subsection{Sociodemographic Module}
% -------------------------------------------------------------------------

This module constructs core demographic characteristics required throughout the pipeline. It derives sex, age, federal state, and household type. The federal state is used to classify students as residing in East or West Germany, which influences BAföG eligibility.

Key data sources:

\begin{itemize}
  \item \texttt{ppathl}: sex, birth year and month
  \item \texttt{hgen}: household type
  \item \texttt{regionl}: federal state of residence
\end{itemize}

These variables are merged into the student module and used for eligibility filtering, modeling allowances, and accounting for regional policy differences.

% -------------------------------------------------------------------------
\subsection{Student Module}
% -------------------------------------------------------------------------

The student module is the core of the microsimulation. It identifies individuals in SOEP who qualify as students and enriches their profiles with relevant characteristics for BAföG simulation. These include education status, household composition, parental identifiers, employment, and relationship status.

It integrates data from the following sources:

\begin{itemize}
  \item Education and religion: \texttt{pl} (\texttt{plg0012\_h}, \texttt{plh0258\_h})
  \item Living with parents: inferred from household and parental IDs in \texttt{ppathl}
  \item Employment status: \texttt{pgen} (\texttt{pgemplst})
  \item Number of children: derived from \texttt{bioparen} (where the student is listed as a parent)
  \item Partnership status: derived from \texttt{partner} and household records
  \item Number of siblings: from \texttt{biosib}
\end{itemize}

The output is a filtered and annotated dataset of eligible students, ready for further processing.

% -------------------------------------------------------------------------
\subsection{Student-Income Module}
% -------------------------------------------------------------------------

This module calculates the student's BAföG-relevant income. Starting with gross labour income, it applies the following steps:

\begin{itemize}
  \item Merge reported income from the assessment year (typically the prior calendar year)
  \item Deduct Werbungskosten as per §~9a EStG
  \item Apply the social insurance allowance per §~21 BAföG
  \item Calculate and deduct income tax, church tax, and solidarity surcharge
  \item Compare resulting income against exemption thresholds per §~23 BAföG
\end{itemize}

The result is net income above allowances, which may reduce BAföG entitlements.

% -------------------------------------------------------------------------
\subsection{Assets Module}
% -------------------------------------------------------------------------

The assets module compiles asset information from the \texttt{pwealth} dataset. Asset categories include:

\begin{itemize}
  \item \textbf{Financial assets:} accounts, savings, stocks, bonds (\texttt{f0100a}–\texttt{f0100e})
  \item \textbf{Real estate:} owned property or partial stakes (\texttt{e0111a}–\texttt{e0111e})
  \item \textbf{Business assets:} private businesses or self-employment stakes (\texttt{b0100a}–\texttt{b0100e})
  \item \textbf{Private insurance:} life, pension, and building loan contracts (\texttt{i0100a}–\texttt{i0100e})
  \item \textbf{Vehicles:} personal transport (\texttt{v0100a}–\texttt{v0100e})
  \item \textbf{Tangible assets:} valuables (e.g., jewelry, art, furniture) (\texttt{t0100a}–\texttt{t0100e})
  \item \textbf{Liabilities:} total outstanding debt, excluding student loans (\texttt{w0011a}–\texttt{w0011e})
\end{itemize}

All asset types are aggregated into a single student-level asset value, which is then compared to the allowance threshold defined in §~29 BAföG (\emph{Freibeträge vom Vermögen}). Students exceeding the threshold may be deemed ineligible.

% -------------------------------------------------------------------------
\subsection{Parental-Income Module}
% -------------------------------------------------------------------------

This module estimates the parental contribution to student support. It proceeds through:

\begin{itemize}
  \item Matching students to parents via \texttt{bioparen}
  \item Extracting gross income from \texttt{pgen}
  \item Deducting Werbungskosten and social insurance contributions
  \item Applying allowances per §~25 BAföG (base allowance, sibling allowance, and partnership status)
\end{itemize}

If data for both parents is available, their contributions are summed to determine the total parental support obligation.

% -------------------------------------------------------------------------
\subsection{BAföG Calculation}
% -------------------------------------------------------------------------

The final module integrates all variables to compute theoretical BAföG entitlements. The calculation follows this logic:

\begin{itemize}
  \item Subtract allowable income from the student’s income
  \item Subtract estimated parental contribution
  \item Exclude students who exceed asset limits
\end{itemize}

The output is an estimated monthly BAföG benefit, which can be used to assess take-up behavior and simulate policy reforms.


% -------------------------------------------------------------------------
\subsection{Variable Dictionary for Microsimulation Datasets}
% -------------------------------------------------------------------------
{\footnotesize
\begin{longtable}{llll}
\caption{Variable Dictionary by Dataset} 
\label{table:variable_dictionary} \\
\toprule
Dataset & Variable & Description & Data Type / Scale \\
\midrule
\endfirsthead

\multicolumn{4}{l}{\textit{(continued from previous page)}} \\
\toprule
Dataset & Variable & Description & Data Type / Scale \\
\midrule
\endhead

\bottomrule
\multicolumn{4}{r}{\textit{(continued on next page)}} \\
\endfoot

\bottomrule
\endlastfoot

\multicolumn{4}{l}{\textbf{IDENTIFIERS AND CORE DEMOGRAPHICS}} \\
\texttt{ppathl} & \texttt{pid} & Person identifier & int \\
\texttt{ppathl} & \texttt{hid} & Household ID & int \\
\texttt{ppathl} & \texttt{syear} & Survey year & date \\
\texttt{ppathl} & \texttt{gebjahr} & Year of birth & int \\
\texttt{ppathl} & \texttt{gebmonat} & Month of birth & int \\
\texttt{ppathl} & \texttt{sex} & Sex & Categorical \\
\texttt{ppathl} & \texttt{partner} & Partnership status & Categorical \\
\texttt{ppathl} & \texttt{migback} & Migration background & Categorical \\
\texttt{regionl} & \texttt{hid} & Household ID & int \\
\texttt{regionl} & \texttt{syear} & Survey year & date \\
\texttt{regionl} & \texttt{bula} & Federal state (Bundesland) & Categorical \\
 
\midrule
\multicolumn{4}{l}{\textbf{EDUCATION}} \\
\texttt{pl} & \texttt{pid} & Person identifier & int \\
\texttt{pl} & \texttt{syear} & Survey year & date \\
\texttt{pl} & \texttt{plg0012\_h} & Currently in education & Ordinal \\
\texttt{pl} & \texttt{plg0014\_v5} & Education level, 1999--2008 & Ordinal \\
\texttt{pl} & \texttt{plg0014\_v6} & Education level, 2009--2012 & Ordinal \\
\texttt{pl} & \texttt{plg0014\_v7} & Education level, 2013--2021 & Ordinal \\

\midrule
\multicolumn{4}{l}{\textbf{RELIGION AND STUDENT AID}} \\
\texttt{pl} & \texttt{plh0258\_h} & Religion / church membership & Categorical \\
\texttt{pl} & \texttt{plc0167\_h} & BAföG eligibility & Binary \\
\texttt{pl} & \texttt{plc0168\_h} & BAföG / scholarship (gross, monthly) & int \\
\texttt{pequiv} & \texttt{pid} & Person identifier & int \\
\texttt{pequiv} & \texttt{syear} & Survey year & date \\
\texttt{pequiv} & \texttt{istuy} & Student grants received & int \\

\midrule
\multicolumn{4}{l}{\textbf{EMPLOYMENT AND INCOME}} \\
\texttt{pgen} & \texttt{pid} & Person identifier & int \\
\texttt{pgen} & \texttt{syear} & Survey year & date \\
\texttt{pgen} & \texttt{pglabgro} & Labour income (gross) & int \\
\texttt{pgen} & \texttt{pgemplst} & Employment status & Categorical \\
\texttt{pgen} & \texttt{pgpartnr} & Partner indicator & int \\
\texttt{biol} & \texttt{pid} & Person identifier & int \\
\texttt{biol} & \texttt{syear} & Survey year & date \\
\texttt{biol} & \texttt{lb0267\_v1} & Employment status & Categorical \\

\midrule
\multicolumn{4}{l}{\textbf{HOUSING AND RENT}} \\
\texttt{pkal} & \texttt{pid} & Person identifier & int \\
\texttt{pkal} & \texttt{syear} & Survey year & date \\
\texttt{pkal} & \texttt{kal2a02} & Monthly rent including utilities & int \\
\texttt{pkal} & \texttt{kal2a03\_h} & Housing benefit & int \\

\midrule
\multicolumn{4}{l}{\textbf{WEALTH AND ASSETS}} \\
\texttt{pwealth} & \texttt{pid} & Person identifier & int \\
\texttt{pwealth} & \texttt{syear} & Survey year & date \\
\texttt{pwealth} & \texttt{f0100a--f0100e} & Financial assets & int \\
\texttt{pwealth} & \texttt{e0111a--e0111e} & Real estate (net value shares) & int \\
\texttt{pwealth} & \texttt{b0100a--b0100e} & Business assets & int \\
\texttt{pwealth} & \texttt{i0100a--i0100e} & Private insurances & int \\
\texttt{pwealth} & \texttt{v0100a--v0100e} & Vehicles & int \\
\texttt{pwealth} & \texttt{t0100a--t0100e} & Tangible assets & int \\
\texttt{pwealth} & \texttt{w0011a--w0011e} & Liabilities and debts & int \\

\midrule
\multicolumn{4}{l}{\textbf{FAMILY RELATIONSHIPS}} \\
\texttt{biosib} & \texttt{pid} & Person identifier & int \\
\texttt{biosib} & \texttt{sibpnr1--sibpnr11} & Sibling person numbers & int \\
\texttt{bioparen} & \texttt{pid} & Person identifier & int \\
\texttt{bioparen} & \texttt{fnr} & Father’s person ID & int \\
\texttt{bioparen} & \texttt{mnr} & Mother’s person ID & int \\

\midrule
\multicolumn{4}{l}{\textbf{HOUSEHOLD COMPOSITION}} \\
\texttt{hgen} & \texttt{hid} & Household ID & int \\
\texttt{hgen} & \texttt{syear} & Survey year & date \\
\texttt{hgen} & \texttt{hgtyp1hh} & Household type & Categorical \\

\end{longtable}
}
