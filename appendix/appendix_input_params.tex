\newpage
\section{Main Input Parameters for Theoretical BAföG Entitlement Calculations}

This appendix presents the key legal parameter tables used in the theoretical BAföG entitlement calculations. 
The data has been compiled from various legal sources and amendments \cite{bafoeg_law, bafoeg20, bafoeg21, bafoeg22, bafoeg23, bafoeg24, bafoeg25, bafoeg26, bafoeg27, bafoeg28}. 
Unless stated otherwise, values are given in euros and reflect monthly amounts. Where necessary, values have been forward filled in the microsimulation to maintain continuity over time.

The process of compiling these parameters served as an essential foundation for our analysis and, more broadly, enabled our entry into this research area. 
Given the fragmented nature of BAföG legislation over time, assembling a clean, structured dataset was a prerequisite for meaningful empirical work. 
We hope that this documentation can serve as a resource for future researchers interested in modeling the German student aid system or conducting policy evaluation in this domain.

\vspace{1em}

\begin{table}[H]
\centering
\small
\begin{tabularx}{\textwidth}{lXXXX}
\toprule
\textbf{Valid from} & \textbf{§ 13 (1) 1} & \textbf{§ 13 (1) 2} & \textbf{§ 13 (2) 1} & \textbf{§ 13 (2) 2} \\
\midrule
2024-07-25 & 442 & 475 & 59 & 380 \\
2022-07-22 & 421 & 452 & 59 & 360 \\
2020-08-01 & 398 & 427 & 56 & 325 \\
2019-07-16 & 391 & 419 & 55 & 325 \\
2016-08-01 & 372 & 399 & 52 & 250 \\
2010-10-01 & 348 & 373 & 49 & 224 \\
2008-10-01 & 341 & 366 & 48 & 146 \\
2002-01-01 & 310 & 333 & 44 & 133 \\
\bottomrule
\end{tabularx}
\caption{Monthly standard needs rates under § 13 BAföG for students, by validity date. Amounts vary by accommodation type and insurance status.}
\label{tab:bafog_values_13}
\end{table}

\vspace{1em}

\begin{table}[H]
\centering
\small
\begin{tabularx}{\textwidth}{lXXXXXX}
\toprule
\textbf{Valid from} & \textbf{§ 13a (1) 1} & \textbf{§ 13a (1) 2} & \textbf{§ 13a (2) 1} & \textbf{§ 13a (2) 2} & \textbf{§ 13a (3) 1} & \textbf{§ 13a (3) 2} \\
\midrule
2024-08-01 & 102 & 35 & 185 & 48 & 102 & 35 \\
2022-08-01 & 94  & 28 & 168 & 38 & 94  & 28 \\
2022-07-15 & 84  & 25 & 155 & 34 & 84  & 25 \\
2020-08-01 & 84  & 25 & 155 & 34 & 84  & 25 \\
2016-08-01 & 71  & 15 & 155 & 34 & 84  & 25 \\
2010-10-01 & 62  & 11 & 155 & 34 & 84  & 25 \\
2008-10-01 & 50  & 9  & 155 & 34 & 84  & 25 \\
2002-01-01 & 47  & 8  & 155 & 34 & 84  & 25 \\
\bottomrule
\end{tabularx}
\caption{Monthly allowances under § 13a BAföG for health and long-term care insurance contributions, by validity date. Amounts vary by insurance type and student status.}
\label{tab:bafog_values_13a}
\end{table}

\vspace{1em}

\begin{table}[H]
\centering
\small
\begin{tabularx}{\textwidth}{lX}
\toprule
\textbf{Year} & \textbf{§ 21 (2) 1} \\
\midrule
2022 & 0.223 \\
2021 & 0.213 \\
2012 & 0.213 \\
2001 & 0.210 \\
\bottomrule
\end{tabularx}
\caption{Deduction rates under § 21 (2) 1 BAföG for income from employment subject to pension insurance, used to approximate social security contributions in the means test, by year.}
\caption*{\textit{Note:} Table only shows years in which the rate changed. Intermediate years are forward filled.}
\label{tab:bafog_rates}
\end{table}

\vspace{1em}

\begin{table}[H]
\centering
\small
\begin{tabularx}{\textwidth}{lXXX}
\toprule
\textbf{Valid from} & \textbf{§ 23 (1) 1} & \textbf{§ 23 (1) 2} & \textbf{§ 23 (1) 3} \\
\midrule
2024-07-19 & 353 & 850 & 770 \\
2022-07-16 & 330 & 805 & 730 \\
2021-08-01 & 330 & 665 & 605 \\
2020-08-01 & 330 & 630 & 570 \\
2019-07-09 & 330 & 610 & 555 \\
2015-01-01 & 290 & 570 & 520 \\
2010-10-24 & 290 & 535 & 485 \\
2008-08-01 & 290 & 520 & 470 \\
2007-12-24 & 255 & 520 & 470 \\
2002-01-01 & 255 & 480 & 435 \\
\bottomrule
\end{tabularx}
\caption{Monthly income disregards (Freibeträge) under § 23 (1) BAföG for the student, by validity date. Columns refer to students living alone, with a child, or with a spouse/partner.}
\label{tab:bafog_values_23}
\end{table}

\vspace{1em}

\begin{table}[H]
\centering
\small
\begin{tabularx}{\textwidth}{l *{4}{>{\centering\arraybackslash}X} >{\centering\arraybackslash}X >{\centering\arraybackslash}X}
\toprule
\textbf{Valid from} & \textbf{§ 25 (1) 1} & \textbf{§ 25 (1) 2} & \textbf{§ 25 (3) 1} & \textbf{§ 25 (3) 2} & \textbf{§ 25 (4) 1} & \textbf{§ 25 (4) 2} \\
\midrule
2024-07-19 & 2540 & 1690 & 850 & 770 & 50\% & 5\% \\
2022-07-16 & 2415 & 1605 & 805 & 730 & 50\% & 5\% \\
2021-08-01 & 2000 & 1330 & 665 & 605 & 50\% & 5\% \\
2020-08-01 & 1890 & 1260 & 630 & 570 & 50\% & 5\% \\
2019-07-09 & 1835 & 1225 & 610 & 555 & 50\% & 5\% \\
2015-01-01 & 1715 & 1145 & 570 & 520 & 50\% & 5\% \\
2010-10-24 & 1605 & 1070 & 535 & 485 & 50\% & 5\% \\
2007-12-24 & 1555 & 1040 & 520 & 470 & 50\% & 5\% \\
2002-01-01 & 1440 & 520  & 480 & 435 & 50\% & 5\% \\
\bottomrule
\end{tabularx}
\caption{Income exemptions under § 25 BAföG for parents and spouses or partners, by validity date. Columns show fixed allowances and percentage deductions used in the means test.}
\label{tab:bafog_values_25}
\end{table}

\subsection{Other Relevant Input Parameters}

\begin{table}[H]
\centering
\small
\begin{tabularx}{\textwidth}{l >{\centering\arraybackslash}X >{\centering\arraybackslash}X}
\toprule
\textbf{In force} & \textbf{§ 32a Abs. 5 \& 6 (joint)} & \textbf{Otherwise (single)} \\
\midrule
2026 & 40700 & 20350 \\
2023 & 36260 & 18130 \\
2021 & 33912 & 16956 \\
2020 & 1944  & 972 \\
\bottomrule
\end{tabularx}
\caption{Solidarity surcharge (Soli) exemption thresholds under § 32a Abs. 5 \& 6 EStG, by year of entry into force. Joint refers to married couples filing jointly; single to individual taxpayers.}
\label{tab:solz_thresholds}
\end{table}


\begin{table}[H]
\centering
\small
\begin{tabularx}{\textwidth}{l >{\centering\arraybackslash}X}
\toprule
\textbf{Year} & \textbf{Werbungskostenpauschale} \\
\midrule
2024 & 1230 \\
2022 & 1200 \\
2021 & 1000 \\
2010 & 920  \\
2003 & 1044 \\
\bottomrule
\end{tabularx}
\caption{Annual employee deduction for work-related expenses (Werbungskostenpauschale) under § 9a Satz 1 Nr. 1a EStG, by year of change. Intermediate years are forward filled in the microsimulation.}
\label{tab:werbungskostenpauschale}
\end{table}
