\section{Literature review}
\label{section:literature_review}

As a form of social benefit, federal student aid such as BAföG faces similar challenges as other public support programs. One of the main challenges is ensuring that eligible individuals actually claim the assistance available to them. When those who qualify for support do not apply, the effectiveness of the policy is reduced. This can have broader consequences, since the overall goals of social programmes, such as reducing poverty or acting as automatic stabilisers during economic downturns, depend on reaching those in need \citep{goedeme_concept_2020}. Furthermore, if individuals who would benefit the most are not reached, both the efficiency and equity of social policy may be compromised.

This phenomenon, known as non-take-up, refers to situations where individuals meet the legal eligibility requirements but do not receive the benefit, often because they do not apply.
This is different from “non-enrolment”, which also includes individuals that do not meet the eligibility criteria to begin with. Non-take-up of social benefits can be understood as resulting from factors at three main levels: individual circumstances (such as awareness and attitudes), administrative practices, and the broader design or structure of benefit schemes \citep{vanoorschot_failing_2002}. Much of the recent literature most relevant to our analysis concentrates primarily on client-level explanations. 

This focus is not surprising, as most studies rely on general survey data, while information about administrative processes is typically less accessible. It is also important to note that the structural features of benefit schemes generally do not cause non-take-up directly, but rather shape the environment in which both administrators and clients operate \citep{vanoorschot_failing_2002}. In line with this, our study focuses on individual-level explanatory factors. For this approach, we are able to make use of high-quality micro-level data, which allows us to identify patterns and determinants of non-take-up at the client level with a high degree of precision. Moreover, gaining a better understanding of individual-level barriers can offer valuable insights for targeted policy interventions, especially in situations where broader administrative or structural reforms are difficult to implement or take longer to achieve.

\textcolor{red}{Maybe this reasoning about our choice belongs somewhere else}

In the economic context, the decision to claim aid is generally understood as a cost-benefit trade-off: individuals weigh the expected monetary gain against the costs of claiming, which are typically grouped into three categories: informational, procedural, and social or psychological \citep{booij_role_2012}. 

\begin{itemize}
  \item \textbf{Information costs} refer to the time and effort required to learn about available benefits, understand the eligibility rules, navigate the application process, and assess the possible consequences of claiming support. This may for example include searching for reliable information or clarifying confusing requirements.
  \item \textbf{Process costs}, on the other hand, involve the resources spent during the actual application process. These can include filling out forms, providing documentation, traveling to relevant offices, waiting in lines, or facing other administrative hurdles.
  \item Finally, \textbf{social and psychological costs} capture the emotional and interpersonal challenges associated with claiming benefits. A key factor here is stigma, which can for example manifest as personal discomfort with claiming support or concern over how others may perceive the claimant. The extent to which stigma is felt can be shaped both by the way benefit programs are designed and by broader social attitudes.
\end{itemize}

Among these different types of costs, recent research suggests that information costs may play a particularly important role in explaining non-take-up. \cite{bolland_information_nodate} find that information costs are a stronger predictor of non-take-up than either process complexity or stigma. Their results show that individuals who are unaware of available benefits are much less likely to claim them, and that higher perceived information costs are consistently associated with lower take-up rates. 

It follows from the above that these costs of claiming tend to increase when procedures are complex or lack clear explanation. In line with this, standard economic theory predicts that individuals are more likely to claim benefits when the expected payout is large or long-lasting, and less likely when the application process is complicated or socially stigmatised \citep{janssens_totake_2022, booij_role_2012}. This relationship between complexity, costs, and take-up is also highlighted by \cite{akerlof_tagging_1978}, who argues that while targeted welfare programs (“tagging”) are theoretically efficient in resource allocation, the way such programs are structured in themselves often leads to the kinds of complexities and unintended incentives discussed above.

While this view reflects traditional economic thinking, recent research suggests that non-take-up may persist even when financial and administrative barriers are minimal. For example, \cite{bhargava_psychological_2015} show that even when procedural barriers are low, cognitive and behavioural factors alone can still lead to high levels of non-take-up. Specifically, they find that even relatively minor psychological obstacles, such as uncertainty about eligibility, confusing application steps, or unclear instructions, can be enough to discourage people from claiming benefits. This holds true even when the actual effort or time involved is not substantial. These findings support broader behavioural models that recognize limits to attention, self-control, and cognitive resources. People may procrastinate or avoid tasks that seem uncertain or complex, even if they intend to complete them later, and may be especially influenced by how information is presented or by small logistical hurdles.

\textcolor{red}{Add to the text so that it ends in a natural way and ties naturally into the next section}

