\section{Theoretical and Empirical Context (Related literature)}
\label{section:theoretical_empirical_context}
%TODO: Add similar paper of what other studies have found for non-take up rate www.ssrn.com/abstract=3352378

Means-tested student aid in Germany (BAföG) can amount to around 38000 EUR over the course of a university degree. Roughly half of this support comes as a non-repayable grant, while the other half is provided as an interest-free loan, a portion of which is typically waived after graduation. Despite the generosity of the program, microsimulation studies using SOEP data show that nearly 40\% of students who are formally eligible never submit an application. This level of non-take-up (NTU) poses serious challenges to both the equity goal of making higher education accessible to all and the efficiency goal of directing public support where it can have the greatest impact \citep{herber_non-take-up_2019}.

In the NTU literature, non-take-up refers to the case where individuals meet legal eligibility requirements but still do not receive the benefit. This is different from “non-enrolment,” where someone never enters the pool of potential recipients in the first place. In the economic context, the decision to claim aid is generally understood as a cost–benefit trade-off: students weigh the expected monetary gain against the costs of claiming, which are typically grouped into three categories: informational, procedural, and psychological (including stigma). These costs tend to increase when the program is complex and not clearly explained \citep{Booij_role_2012}. Estimating NTU reliably is not straightforward. It depends on having high-quality data and being able to simulate eligibility rules in detail. Studies on German social assistance, for example, show that even small mistakes in reported income can lead to NTU rates being overstated by up to ten percentage points \citep{frick_claim_2007}.

Standard economic theory predicts that students are more likely to claim benefits when the expected payout is large or long-lasting, and less likely when the application process is complicated or socially stigmatised \citep{booij_role_2012}. However, in the case of BAföG, the fact that part of the support is structured as a loan introduces additional behavioural factors. Students may be discouraged from applying due to debt aversion, present bias, or uncertainty about future earnings. Evidence from SOEP data shows that students who score high on impulsivity or impatience are significantly more likely to not take up BAföG. Conversely, those with an older sibling who has already gone through the application process are much more likely to apply, suggesting that informal networks help lower informational barriers. Research from Belgium supports this picture, finding that information-related barriers are often more important than administrative complexity or stigma in explaining why people do not claim benefits they are entitled to \citep{fidan_why_2021, herber_non-take-up_2019, bolland_information_nodate}.

Across different German studies, the estimated NTU rate for BAföG is remarkably consistent at around 40\% \citep{goedeme_concept_2020, mechelen_who_2017}. Herber and Kalinowski’s microsimulation work shows that a 100 EUR increase in monthly BAföG payments reduces the likelihood of non-application by roughly one percentage point. Their findings also highlight the importance of family context: students who have a sibling who already claimed BAföG are significantly more likely to apply themselves \citep{frick_claim_2007, bruckmeier_new_2012}. They further find that students raised in former East Germany tend to claim more often, suggesting that local norms and social context also play a role.

Germany’s experience is not unique. For example, Dutch administrative data reveal that about 24\% of eligible first-year students do not take up the means-tested supplementary grant. Surprisingly, a third of these non-claimants simultaneously take out student loans, which strongly indicates a lack of awareness or understanding of the program \citep{konijn_quantifying_2023}. Estimates of non-take-up vary widely across countries, programs and data sources, but for means-tested social assistance schemes in OECD countries, NTU levels are often found to be quite high, or between 30 and 70\% \citep{goedeme_concept_2020}.

In getting accurate estimates of NTU, a few key factors need to be considered. Among those are access to detailed linked data, a clear and replicable eligibility simulation, and careful correction for measurement errors. In the German case, even moderate misreporting of household income can shift the results substantially \citep{goedeme_concept_2020, frick_claim_2007}. Recent Belgian research using the TAKE survey shows that directly measuring the types of barriers students face, such as information or application hurdles, can dramatically improve the predictive accuracy of NTU models compared to relying only on background characteristics \citep{bolland_information_nodate}.

Although earlier German studies have produced careful microsimulations of BAföG take-up, they rely on SOEP waves that stop in 2013 and therefore cannot speak to developments in the past decade \citep{herber_non-take-up_2019, bruckmeier_new_2012}. No national update has been published since then, even though student demographics and labour-market conditions have shifted. Using the latest SOEP data available up to 2021, this thesis delivers an up-to-date NTU estimate and a concise breakdown by key background characteristics—parental income, region of upbringing and migration status—providing the first nationwide picture of BAföG non-take-up immediately before the 2024 reform and a clean benchmark for future policy evaluations.


\subsection{Studies on non-take up of welfare}

Over the past several decades, a substantial body of research has examined the phenomenon of non-take-up (NTU) of welfare benefits in Germany. 
As summarized in Table~\ref{table:NTU-studies}, previous studies have produced a wide range of NTU estimates, reflecting differences in target populations, program types, methodological approaches, and data sources. 

Early studies using administrative data and household surveys typically reported NTU rates for social assistance between 40\% and 67\%. 
More recent analyses, including those based on the German Socio-Economic Panel (SOEP) and the German Income and Expenditure Survey (EVS), have investigated non-take-up across a broader range of programs, including basic social security, housing allowance, and supplementary child allowance. 
Despite some variation, a persistent pattern emerges: a significant proportion of eligible individuals and households do not claim benefits to which they are entitled. 
The selected studies listed in Table~\ref{table:NTU-studies} illustrate both the magnitude of this issue and the diversity of empirical approaches used to estimate NTU rates in Germany.

\begin{table}[htbp]
\centering
\begin{tabular}{llrlll}
\toprule
\textbf{Author(s)} & \textbf{Year} & \textbf{NTU (\%)} & \textbf{Year of Data} & \textbf{Data Source} & \textbf{Program Type} \\
\midrule
\citeauthor{frick_claim_2007} & 2007 & 67 & 2002 & SOEP & SA \\
\citeauthor{herber_non-take-up_2019} & 2016 & 36--40 & 2002--2013 & SOEP & BAföG \\
\citeauthor{RePEc:iab:iabfob:201305} & 2013 & 41--49 & 2005--2007 & SOEP & BSS \\
\citeauthor{bruckmeier_benefit_2018} & 2018 & 43.1 & 2008 & EVS & SA \\
\citeauthor{bruckmeier_benefit_2018} & 2018 & 63 & 2008 & EVS & HA/SA \\
\citeauthor{bruckmeier_benefit_2018} & 2018 & 88 & 2008 & EVS & SCA \\
\citeauthor{bruckmeier_new_2012}  & 2012 & 41--49 & 2005--2007 & SOEP & SA \\
\bottomrule
\end{tabular}
\caption{Selected previous estimates of non-take-up (NTU) rates for social benefits in Germany. Program type abbreviations: SA = Social Assistance, BAföG = Federal Student Aid, MTG = Means-Tested General Benefits, BSS = Basic Social Security, HA = Housing Allowance, SCA = Supplementary Child Allowance.}
\caption*{\small{Note: This table summarizes selected results on non-take-up rates from prior literature using SOEP and other German datasets. See cited references for full details.}}
\label{table:NTU-studies}
\end{table}

To estimate non take up rates of welfare benefits, researchers typically rely on one or more of three data sources: administrative records, specially designed surveys and general purpose surveys. Each has its own strengths and weaknesses. Administrative data is generally precise for welfare receipt, but often it lacks information on those who do not claim benefits. Special purpose surveys can collect more detailed information on eligibility and take up behaviour but are costly and rarely used. On the other hand, general purpose surveys are more readily available and are widely used in empirical research \citep{mechelen_who_2017}.

In this study, data was collected from a general purpose survey, i.e. the German Socio-Economic Panel (SOEP), which is one of the longest standing multidisciplinary household surveys in the world, gathering data from around 30000 individuals across 22000 households annually \citep{berlin_diw_nodate}.

While such data is not specifically designed to measure non take up, it has the advantage of covering both benefit receipt and the characteristics needed to estimate eligibility, such as income, household composition and demographic variables \citep{mechelen_who_2017}.

However, it comes with some limitations. First of all, there are potential biases due to non response bias and undercoverage. Vulnerable groups, such as those without a permanent address or people living in institutions, are often missing from survey samples. Non response may also be correlated with non take up, which can distort estimates. Second of all, measurement errors can be a concern, especially in regards income, asset reporting, and welfare receipt. Respondents may misreport their income or confuse the benefits they receive, leading to inaccurate estimates of eligibility and take up. Third of all, mismatches in the timing and definition of income used in surveys compared to what administrations use to assess eligibility can result in classification errors. For example, surveys often report annual income, but eligibility is commonly assessed monthly. Lastly, general purpose surveys often lack detailed information about reasons for non take up, making it difficult to distinguish between, for example, lack of awareness and administrative barriers \citep{mechelen_who_2017}.

