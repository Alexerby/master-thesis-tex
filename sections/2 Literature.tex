\section{Theoretical and Empirical Context (Related literature)}
\label{section:theoretical_empirical_context}
%TODO: Add similar paper of what other studies have found for non-take up rate www.ssrn.com/abstract=3352378

According to an OECD working paper (2004), take up rates of welfare are under researched in most of the OECD countries. It is however worth noting that it’s safe to say that interest in this area has increased in recent years. In the paper, which looks at take up rates of welfare in OECD states, take up rates are frequently found to be low across various states and schemes. This is particularly true for means tested social assistance schemes \citep{hernanz_oecd_2004}.

Among the many factors that influence take up rates (such as information- and administrative costs), empirical research indicates that financial determinants are the most significant in influencing involvement in welfare schemes. Financial determinants adhere both to the level of welfare (i.e. support rates) and its expected extent (i.e. time span) \citep{hernanz_oecd_2004}.

...

Herber and Kalinowski (2016) develop a microsimulation model using SOEP data to replicate official BAföG eligibility rules between 2002 and 2013. By comparing simulated eligibility with self-reported benefit receipt, they estimate that around 40\% of eligible students do not claim aid. Their framework treats take up as a rational cost-benefit decision, where the utility of claiming aid increases with the level and expected duration of support and is offset by various costs, including information costs, administrative burdens, and psychological factors such as debt aversion. They find that take up is only modestly responsive to benefit size, with an estimated elasticity of -0.41, and that students from East German backgrounds are more likely to claim aid, while those scoring high on impatience and low on self-control are less likely to do so. Based on these findings, they recommend simplifying the application process and providing the grant component by default, rather than requiring students to accept the loan as well. As their study provides a comprehensive and policy relevant model for estimating theoretical NTU, this study builds on their approach by extending the analysis to more recent years, updating the simulation in line with policy reforms and providing new estimates under the current BAföG system.

For this analysis the German Socio-Economic Panel (SOEP) is used, following the approach of Herber and Kalinowski (2016) as it provides the detailed individual and household level data that is necessary for conducting the microsimulation. In particular, it contains both information on whether students report receiving BAföG and the variables needed to simulate their eligibility under the official rules, such as income, parental background and household composition. While administrative data may offer more accurate information on actual benefit receipt, it usually doesn’t include the comprehensive income and demographic information needed to estimate whether non-recipients would have qualified. As Herber and Kalinowski (2016, p. 18) point out, SOEP’s combination of these features makes it especially well suited for studying theoretical NTU. This dataset makes it possible to replicate the BAföG eligibility criteria and identify students who theoretically meet eligibility criteria but do not receive support. This aligns with the broader modelling approach of this study, where the focus is on comparing expected benefits with application related costs as opposed to focusing on behavioural explanations that are harder to capture using survey data. 

...

A well known issue with means tested programs\footnote{Means-tested benefits are forms of financial support that are only available to people whose income or assets fall below a certain threshold. This is different from universal benefits, which are available to everyone regardless of their financial situation.} is that the rules designed to target support can end up making the system complex and hard to navigate. As highlighted by Mechelen (2017) and Goedemé and Janssens (2020), complex eligibility criteria and administrative procedures can discourage people from applying, even when they would qualify. In the case of BAföG, the application process involves a detailed means test that takes into account various factors such as parental income, assets, and living arrangements, and the standard application form alone runs over 17 pages. For many students, the time and effort required to understand the rules, gather the necessary documents, and complete the process can become a barrier in itself. Ongoing changes to the rules, such as adjustments to income thresholds or documentation requirements, can add another layer of uncertainty, making it difficult to know whether it’s worth applying in the first place. These kinds of hurdles help explain why eligible students opt out, and they highlight the value of regularly updating simulations to reflect current rules and better understand potential NTU.

\subsection{Studies on non-take up of welfare}

Over the past several decades, a substantial body of research has examined the phenomenon of non-take-up (NTU) of welfare benefits in Germany. 
As summarized in Table~\ref{table:NTU-studies}, previous studies have produced a wide range of NTU estimates, reflecting differences in target populations, program types, methodological approaches, and data sources. 

Early studies using administrative data and household surveys typically reported NTU rates for social assistance between 40\% and 67\%. 
More recent analyses, including those based on the German Socio-Economic Panel (SOEP) and the German Income and Expenditure Survey (EVS), have investigated non-take-up across a broader range of programs, including basic social security, housing allowance, and supplementary child allowance. 
Despite some variation, a persistent pattern emerges: a significant proportion of eligible individuals and households do not claim benefits to which they are entitled. 
The selected studies listed in Table~\ref{table:NTU-studies} illustrate both the magnitude of this issue and the diversity of empirical approaches used to estimate NTU rates in Germany.

\begin{table}[htbp]
\centering
\begin{tabular}{llrlll}
\toprule
\textbf{Author(s)} & \textbf{Year} & \textbf{NTU (\%)} & \textbf{Year of Data} & \textbf{Data Source} & \textbf{Program Type} \\
\midrule
\citeauthor{frick_claim_2007} & 2007 & 67 & 2002 & SOEP & SA \\
\citeauthor{herber_non-take-up_2019} & 2016 & 36--40 & 2002--2013 & SOEP & BAföG \\
\citeauthor{RePEc:iab:iabfob:201305} & 2013 & 41--49 & 2005--2007 & SOEP & BSS \\
\citeauthor{bruckmeier_benefit_2018} & 2018 & 43.1 & 2008 & EVS & SA \\
\citeauthor{bruckmeier_benefit_2018} & 2018 & 63 & 2008 & EVS & HA/SA \\
\citeauthor{bruckmeier_benefit_2018} & 2018 & 88 & 2008 & EVS & SCA \\
\citeauthor{bruckmeier_new_2012}  & 2012 & 41--49 & 2005--2007 & SOEP & SA \\
\bottomrule
\end{tabular}
\caption{Selected previous estimates of non-take-up (NTU) rates for social benefits in Germany. Program type abbreviations: SA = Social Assistance, BAföG = Federal Student Aid, MTG = Means-Tested General Benefits, BSS = Basic Social Security, HA = Housing Allowance, SCA = Supplementary Child Allowance.}
\caption*{\small{Note: This table summarizes selected results on non-take-up rates from prior literature using SOEP and other German datasets. See cited references for full details.}}
\label{table:NTU-studies}
\end{table}

To estimate non take up rates of welfare benefits, researchers typically rely on one or more of three data sources: administrative records, specially designed surveys and general purpose surveys. Each has its own strengths and weaknesses. Administrative data is generally precise for welfare receipt, but often it lacks information on those who do not claim benefits. Special purpose surveys can collect more detailed information on eligibility and take up behaviour but are costly and rarely used. On the other hand, general purpose surveys are more readily available and are widely used in empirical research \citep{mechelen_who_2017}.

In this study, data was collected from a general purpose survey, i.e. the German Socio-Economic Panel (SOEP), which is one of the longest standing multidisciplinary household surveys in the world, gathering data from around 30000 individuals across 22000 households annually \citep{berlin_diw_nodate}.

While such data is not specifically designed to measure non take up, it has the advantage of covering both benefit receipt and the characteristics needed to estimate eligibility, such as income, household composition and demographic variables \citep{mechelen_who_2017}.

However, it comes with some limitations. First of all, there are potential biases due to non response bias and undercoverage. Vulnerable groups, such as those without a permanent address or people living in institutions, are often missing from survey samples. Non response may also be correlated with non take up, which can distort estimates. Second of all, measurement errors can be a concern, especially in regards income, asset reporting, and welfare receipt. Respondents may misreport their income or confuse the benefits they receive, leading to inaccurate estimates of eligibility and take up. Third of all, mismatches in the timing and definition of income used in surveys compared to what administrations use to assess eligibility can result in classification errors. For example, surveys often report annual income, but eligibility is commonly assessed monthly. Lastly, general purpose surveys often lack detailed information about reasons for non take up, making it difficult to distinguish between, for example, lack of awareness and administrative barriers \citep{mechelen_who_2017}.

