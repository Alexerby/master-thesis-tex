\section{Related Literature} \label{sec:literature}

- What we provide to the literature (short) 

- What other studies have looked into

In broad terms, research on non take up can be split into studies applying traditional economic theory and studies applying behavioural economic theory. Traditional economic theory assumes rational behaviour and thus that agents optimise the trade off between benefit and cost. The field of behavioural economics, that has more recently emerged, instead emphasises deflections from the traditional assumptions on rational behaviour and highlights cognitive bias and behavioural obstacles \citep{mechelen_who_2017}.

...

Optimal policy design aims to target support to those who need it most, while keeping eligibility criteria clear and easy to understand. However, targeted programmes often involve complicated rules, such as income limits and asset tests. This complexity poses some difficulties, both for administrative workers and applicants. For administrators, it increases the risk of mistakes when assessing eligibility. For applicants, understanding the rules and completing the process can be time consuming and require substantial effort, especially if there are uncertainties about eligibility. This can discourage people from applying. Complexity can also result in tertiary non take up, where certain groups are excluded by design because the rules use rough indicators of need. One way to reduce non take up is to use simpler, categorical criteria based on things like age or household type. Although this may make targeting less precise, it can make the application process easier and more accessible \citep{mechelen_who_2017}.

...

To estimate non take up rates of welfare benefits, researchers typically rely on one or more of three data sources: administrative records, specially designed surveys and general purpose surveys. Each has its own strengths and weaknesses. Administrative data is generally precise for welfare receipt, but often it lacks information on those who do not claim benefits. Special purpose surveys can collect more detailed information on eligibility and take up behaviour but are costly and rarely used. On the other hand, general purpose surveys are more readily available and are widely used in empirical research \citep{mechelen_who_2017}.

In this study, data was collected from a general purpose survey, i.e. the German Socio-Economic Panel (SOEP), which is one of the longest standing multidisciplinary household surveys in the world, gathering data from around 30000 individuals across 22000 households annually \citep{berlin_diw_nodate}.

While such data is not specifically designed to measure non take up, it has the advantage of covering both benefit receipt and the characteristics needed to estimate eligibility, such as income, household composition and demographic variables \citep{mechelen_who_2017}.

However, it comes with some limitations. First of all, there are potential biases due to non response bias and undercoverage. Vulnerable groups, such as those without a permanent address or people living in institutions, are often missing from survey samples. Non response may also be correlated with non take up, which can distort estimates. Second of all, measurement errors can be a concern, especially in regards income, asset reporting, and welfare receipt. Respondents may misreport their income or confuse the benefits they receive, leading to inaccurate estimates of eligibility and take up. Third of all, mismatches in the timing and definition of income used in surveys compared to what administrations use to assess eligibility can result in classification errors. For example, surveys often report annual income, but eligibility is commonly assessed monthly. Lastly, general purpose surveys often lack detailed information about reasons for non take up, making it difficult to distinguish between, for example, lack of awareness and administrative barriers \citep{mechelen_who_2017}.