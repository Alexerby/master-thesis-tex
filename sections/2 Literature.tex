\section{Literature review}
\label{section:literature_review}

As a form of social benefit, federal student aid such as BAföG faces similar challenges as other public support programs. 
One of the main challenges is ensuring that eligible individuals actually claim the assistance available to them. 
When those who qualify for support do not apply, the effectiveness of the policy is reduced. 
This can have broader consequences, since the overall goals of social programmes, such as reducing poverty or acting as automatic stabilisers during economic downturns, depend on reaching those in need \citep{goedeme_concept_2020}. 
Furthermore, if individuals who would benefit the most are not reached, both the efficiency and equity of social policy may be compromised.

This phenomenon, known as non-take-up, refers to situations where individuals meet the legal eligibility requirements but do not receive the benefit, often because they do not apply. 
This is different from “non-enrolment”, which also includes individuals who do not meet the eligibility criteria to begin with. Non-take-up of social benefits can be understood as resulting from factors at three main levels: individual circumstances (such as awareness, perceived stigma, or attitudes toward the benefit), administrative practices, and the broader design or structure of benefit schemes \citep{vanoorschot_failing_2002}. 

While non-take-up can be shaped by factors at multiple levels, much of the economic literature places particular emphasis on the client or individual perspective. In the economic context, the decision to claim aid is generally understood as a cost-benefit trade-off: individuals weigh the expected monetary gain against the costs of claiming, which are typically grouped into three categories: informational, procedural, and social or psychological \citep{booij_role_2012}. 
These costs help explain why eligible individuals might choose not to apply, even when they stand to benefit financially. 
Each type of cost affects decision-making in a distinct way:

\begin{itemize}
  \item \textbf{Information costs} refer to the time and effort required to learn about available benefits, understand the eligibility rules, navigate the application process, and assess the possible consequences of claiming support. This may for example include searching for reliable information or clarifying confusing requirements.
  \item \textbf{Process costs}, on the other hand, involve the resources spent during the actual application process. These can include filling out forms, providing documentation, traveling to relevant offices, waiting in lines, or facing other administrative hurdles.
  \item Finally, \textbf{social and psychological costs} capture the emotional and interpersonal challenges associated with claiming benefits. A key factor here is stigma, which can for example manifest as personal discomfort with claiming support or concern over how others may perceive the claimant. The extent to which stigma is felt can be shaped both by the way benefit programs are designed and by broader social attitudes.
\end{itemize}

Among these different types of costs, recent research suggests that information costs may play a particularly important role in explaining non-take-up. \cite{bolland_information_nodate} finds that individuals who are unaware of available benefits are much less likely to claim them, and that higher perceived information costs are consistently associated with lower take-up rates. Their research further indicates that information costs are a stronger predictor of non-take-up than either process complexity or stigma. In fact, several studies suggest that stigma is less relevant in the context of student aid than in other forms of public support (see for example, \cite{konijn_quantifying_2023} and \cite{bruckmeier_new_2012}).

It follows from the above that these costs of claiming tend to increase when procedures are complex or lack clear explanation. In line with this, standard economic theory predicts that individuals are more likely to claim benefits when the expected payout is large or long-lasting, and less likely when the application process is complicated or socially stigmatised \citep{janssens_totake_2022, booij_role_2012}. This relationship between complexity, costs, and take-up is also highlighted by \cite{akerlof_tagging_1978}, who argues that while targeted welfare programs (“tagging”) are theoretically efficient in resource allocation, the way such programs are structured in themselves often leads to the kinds of complexities and unintended incentives discussed above.

While this view reflects traditional economic thinking, recent research suggests that non-take-up may persist even when financial and administrative barriers are minimal. 
For example, \cite{bhargava_psychological_2015} show that even when procedural barriers are low, cognitive and behavioural factors alone can still lead to high levels of non-take-up. 
Specifically, they find that even relatively minor psychological obstacles, such as uncertainty about eligibility, confusing application steps, or unclear instructions, can be enough to discourage people from claiming benefits. 
This holds true even when the actual effort or time involved is not substantial. 
These findings support broader behavioural models that recognize limits to attention, self-control, and cognitive resources. 
People may procrastinate or avoid tasks that seem uncertain or complex, even if they intend to complete them later, and may be especially influenced by how information is presented or by small logistical hurdles.

Beyond the overall effects, evidence points to variation in how different groups are affected by these barriers. Complex application procedures have the strongest negative effect on take-up among low-income and otherwise disadvantaged groups, such as first-generation students, and those less familiar with paperwork and official processes. These groups, that are often the main target of aid programs, are thus the most likely to miss out on aid because of these barriers. Additionally, language barriers can further reduce take-up for students from migrant backgrounds \citep{dynarski_cost_2006, bhargava_psychological_2015}.

%These findings show that non-take-up is shaped by both structural and behavioural factors. While economic models help explain some of this, recent work suggests that small barriers in procedures, communication, or perception can also play a large role.

Taken together, the factors gone over in this chapter can affect take-up rates substantially. While research on non-take-up of student aid is still remarkably limited, the literature body on non-take-up of other social benefits has grown substantially in recent years. Non-take-up of means-tested benefits specifically appears to be a widespread and persistent phenomenon across different countries and programmes. Some studies have suggested that non-take-up rates rarely fall below 20\%, even in relatively well-studied contexts \citep{vanoorschot_failing_2002}. In a recent paper by \cite{goedeme_concept_2020}, high non-take-up levels for means-tested social assistance schemes are highlighted, often ranging between 30 and 70\%. To illustrate the issue of non-take-up, we present recent selected findings for a range of social benefits in different countries in Table~\ref{table:NTU-studies}.


\begin{table}[htbp]
\renewcommand{\arraystretch}{1.3}
\footnotesize
\centering
\begin{tabularx}{\textwidth}{@{}l l l l X@{}}
\toprule
\textbf{Author(s)} & \textbf{NTU (\%)} & \textbf{Year of Data} & \textbf{Country} & \textbf{Program Type} \\
\midrule
\cite{herber_non-take-up_2019}    & 36--40   & 2002--2013 & Germany     & BAföG \\
\cite{konijn_quantifying_2023}    & 24       & 2019       & Netherlands & Student Aid \\
\cite{frick_claim_2007}            & 67       & 2002       & Germany     & Social Assistance \\
\cite{fuchs_austria_2007}          & 49--61   & 2003       & Austria     & Social Assistance \\
\cite{goedeme_take_2022}           & 60       & 2019       & Belgium     & Social Assistance \\
\cite{harnisch_nontakeup_2019}    & 56       & 2005--2014 & Germany     & Income support \\
\cite{domingo_nonrecours_2014}     & 68       & 2010       & France      & Earnings supplement \\
\bottomrule
\end{tabularx}
\caption{\small{Selected international estimates of non-take-up (NTU) rates for social benefit programs. Program types are abbreviated or summarized for clarity.}}
\label{table:NTU-studies}
\end{table}


%TODO: Remove the datasource coulmn in the table
%TODO: What if we include more studies from other countries, maybe more studies on student aid even. Or maybe focus on articles on NTU that are from fancy journals. Then we could replace the data source column with a country column and maybe move a few of the more irrelevant lines from the table (for example Bruckmeier and Wiemers results on supplementary child allowance).

The next section provides background on BAföG and outlines the institutional and policy context relevant to student aid in Germany.

