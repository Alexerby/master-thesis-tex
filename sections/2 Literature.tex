\section{Theoretical and Empirical Context (Related literature)}
\label{section:theoretical_empirical_context}
%TODO: Add similar paper of what other studies have found for non-take up rate www.ssrn.com/abstract=3352378

Means-tested student aid in Germany (BAföG) can amount to around 38000 EUR over the course of a university degree. Roughly half of this support comes as a non-repayable grant, while the other half is provided as an interest-free loan, a portion of which is typically waived after graduation. Despite the generosity of the program, microsimulation studies show that nearly 40\% of students who are formally eligible never submit an application. This level of non-take-up (NTU) poses serious challenges to both the equity goal of making higher education accessible to all and the efficiency goal of directing public support where it can have the greatest impact \citep{herber_non-take-up_2019}.

In the NTU literature, non-take-up refers to the case where individuals meet legal eligibility requirements but still do not receive the benefit. 
This is different from “non-enrolment,” where someone never enters the pool of potential recipients in the first place. 
In the economic context, the decision to claim aid is generally understood as a cost–benefit trade-off: students weigh the expected monetary gain against the costs of claiming, which are typically grouped into three categories: informational, procedural, and psychological. 
These costs tend to increase when the procedure is complex and not clearly explained \citep{booij_role_2012}. Estimating NTU reliably is not straightforward. It depends on having high-quality data and being able to simulate eligibility rules in detail. Studies on German social assistance, for example, show that even small mistakes in reported income can lead to NTU rates being overstated by up to ten percentage points \citep{frick_claim_2007}.

Standard economic theory predicts that students are more likely to claim benefits when the expected payout is large or long-lasting, and less likely when the application process is complicated or socially stigmatised \citep{booij_role_2012}. However, in the case of BAföG, the fact that part of the support is structured as a loan introduces additional behavioural factors. Students may be discouraged from applying due to debt aversion, present bias, or uncertainty about future earnings. Evidence from SOEP data shows that students who score high on impulsivity or impatience are significantly more likely to not take up BAföG. Conversely, those with an older sibling who has already gone through the application process are much more likely to apply, suggesting that informal networks help lower informational barriers. Research from Belgium supports this picture, finding that information-related barriers are often more important than administrative complexity or stigma in explaining why people do not claim benefits they are entitled to \citep{fidan_why_2021, herber_non-take-up_2019, bolland_information_nodate}.

\textcolor{red}{This is the first time we mention SOEP data, maybe explain shortly what it is?}

Estimates of BAföG NTU in Germany cover a broad range. Most studies report rates between 40--70\%, depending on the data, simulation rules, and time period examined \citep{goedeme_concept_2020}. Several microsimulation studies based on SOEP data, such as \citet{herber_non-take-up_2019} and \citet{bruckmeier_new_2012}, find rates around roughly 35 to 50\%. However, other analyses point to higher figures, particularly when relying on alternative data sources or more recent samples. This variation reflects both methodological differences and the sensitivity of NTU estimates to income measurement, eligibility modeling, and reporting accuracy.

Herber and Kalinowski’s microsimulation furthermore finds that a 100 EUR increase in monthly BAföG payments reduces the likelihood of non-application by roughly one percentage point. Their findings also highlight the importance of family context, as students who have a sibling who already claimed BAföG are significantly more likely to apply themselves \citep{frick_claim_2007, bruckmeier_new_2012}. They further find that students raised in former East Germany tend to claim more often, suggesting that local norms and social context also play a role.

Although BAföG has its own specific features, such as the combination of grants and loans, the issue of NTU is not limited to student aid. Many other means-tested welfare programs in Germany face similar challenges. Across different studies, NTU rates remain notably high, typically between 40 and 70\% depending on the program and the way eligibility is measured. To put the BAföG case in a broader context, the selected studies listed in Table~\ref{table:NTU-studies} illustrate both the magnitude of this issue and the diversity of empirical approaches used to estimate NTU rates in Germany.


\begin{table}[htbp]
\footnotesize
\centering
\begin{tabular}{llrlll}
\toprule
\textbf{Author(s)} & \textbf{Year} & \textbf{NTU (\%)} & \textbf{Year of Data} & \textbf{Data Source} & \textbf{Program Type} \\
\midrule
\citeauthor{frick_claim_2007}              & 2007          & 67             & 2002                  & SOEP             & SA               \\
\citeauthor{herber_non-take-up_2019}       & 2016          & 36--40         & 2002--2013            & SOEP             & BAföG            \\
\citeauthor{RePEc:iab:iabfob:201305}       & 2013          & 34--43         & 2008                  & EVS              & BSS              \\
\citeauthor{bruckmeier_benefit_2018}       & 2018          & 43             & 2013--2014            & SOEP             & SA               \\
\citeauthor{bruckmeier_benefit_2018}       & 2018          & 87             & 2013--2014            & SOEP             & HA               \\
\citeauthor{bruckmeier_benefit_2018}       & 2018          & 63             & 2013--2014            & SOEP             & HA \& SA         \\
\citeauthor{bruckmeier_benefit_2018}       & 2018          & 88             & 2013--2014            & SOEP             & SCA              \\
\citeauthor{bruckmeier_new_2012}           & 2012          & 41--49         & 2005--2007            & SOEP             & SA               \\
\bottomrule
\end{tabular}
\caption{\small{Selected previous estimates of non-take-up (NTU) rates for social benefits in Germany. Program type abbreviations: SA = Social Assistance, BAföG = Federal Student Aid, MTG = Means-Tested General Benefits, BSS = Basic Social Security, HA = Housing Allowance, SCA = Supplementary Child Allowance.}}
% \caption*{\small{Note: This table summarizes selected results on non-take-up rates from prior literature using SOEP and other German datasets. See cited references for full details.}}
\label{table:NTU-studies}
\end{table}


Germany’s experience is not unique. For example, Dutch administrative data reveal that about 24\% of eligible first-year students do not take up the means-tested supplementary grant. Surprisingly, a third of these non-claimants simultaneously take out student loans, which strongly indicates a lack of awareness or understanding of the program \citep{konijn_quantifying_2023}. Estimates of non-take-up vary widely across countries, programs, and data sources, but for means-tested social assistance schemes in OECD countries, NTU levels are frequently found to be high, often falling between 30 and 70\% \citep{goedeme_concept_2020}, which is broadly in line with the range discussed earlier for the German context.

Estimating non-take-up reliably requires careful attention to data quality and eligibility simulation, and this applies to the German case as well. Even with high-quality survey data like the SOEP, small reporting errors can have a large impact. For instance, \citet{frick_claim_2007} show that modest inaccuracies in household income reporting can shift estimated NTU rates by up to ten percentage points. This illustrates how sensitive NTU estimates can be to measurement issues, even in well-documented and commonly used datasets.

Although earlier German studies have produced careful microsimulations of BAföG take-up, they rely on SOEP waves that stop in 2013 and therefore cannot speak to developments in the past decade \citep{herber_non-take-up_2019, bruckmeier_new_2012}. No national update appears to have been published since then, despite notable changes in student demographics and labour-market conditions. Using the latest SOEP data available up to 2021, this thesis aims to produce an updated estimate of BAföG non-take-up and to present a breakdown by key characteristics such as parental income, region of upbringing, and migration background. The result is a new national-level snapshot just prior to the 2024 BAföG reform, offering a baseline for future policy evaluation.

\textcolor{red}{Maybe explain the 2024 BAföG reform if we are going to be referring to it, or just drop it completely}

\textcolor{red}{"Highlight more explicitly why understanding NTU is critical beyond academic interest (e.g., policy implications)"}

\textcolor{red}{Clarify early on how your study specifically contributes to existing research (update since Herber et al. 2019, using latest SOEP waves)}
