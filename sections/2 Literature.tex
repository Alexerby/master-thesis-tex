\section{Literature review}

As a form of social benefit, federal student aid such as BAföG faces similar challenges as other public support programs. 
One of the main challenges is ensuring that eligible individuals actually claim the assistance available to them. 
When those who qualify for support do not apply, the effectiveness of the policy is reduced. 
This can have broader consequences, since the overall goals of social programmes, such as reducing poverty or acting as automatic stabilisers during economic downturns, depend on reaching those in need \citep{goedeme_concept_2020}. 
Furthermore, if individuals who would benefit the most are not reached, both the efficiency and equity of social policy may be compromised.

This phenomenon, known as non-take-up, refers to situations where individuals meet the legal eligibility requirements but do not receive the benefit, often because they do not apply. 
This is different from “non-enrolment”, which also includes individuals who do not meet the eligibility criteria to begin with. Non-take-up of social benefits can be understood as resulting from factors at three main levels: individual circumstances (such as awareness, perceived stigma, or attitudes toward the benefit), administrative practices, and the broader design or structure of benefit schemes \citep{vanoorschot_failing_2002}. 

While non-take-up can be shaped by factors at multiple levels, much of the economic literature places particular emphasis on the individual perspective. To frame our empirical analysis, we draw on economic models of welfare take-up, particularly those emphasizing the cost-benefit decision-making process of eligible individuals \citep{vanoorschot_failing_2002, booij_role_2012}. In this context, the decision to claim aid is generally understood as a cost-benefit trade-off, where individuals weigh the expected benefits, both monetary and non-monetary, against the costs associated with claiming. These costs are typically grouped into three categories: informational, procedural, and psychological or social. This framework helps explain why eligible individuals might choose not to apply even when a financial benefit is available, since each type of cost can affect the decision in a distinct way:

\begin{itemize}
  \item \textbf{Information costs} refer to the time and effort required to learn about available benefits, understand the eligibility rules, navigate the application process, and assess the possible consequences of claiming support. This may for example include searching for reliable information or clarifying confusing requirements.
  \item \textbf{Process costs}, on the other hand, involve the resources spent during the actual application process. These can include filling out forms, providing documentation, traveling to relevant offices, waiting in lines, or facing other administrative hurdles.
  \item Finally, \textbf{social and psychological costs} capture the emotional and interpersonal challenges associated with claiming benefits. A key factor here is stigma, which can for example manifest as personal discomfort with claiming support or concern over how others may perceive the claimant. The extent to which stigma is felt can be shaped both by the way benefit programs are designed and by broader social attitudes.
\end{itemize}

It follows from the above that these costs of claiming tend to increase when procedures are complex or lack clear explanation. In line with this, standard economic theory predicts that individuals are more likely to claim benefits when the expected payout is large or long-lasting, and less likely when the application process is complicated or socially stigmatised \citep{janssens_totake_2022, booij_role_2012}. This relationship between complexity, costs, and take-up is also highlighted by \cite{akerlof_tagging_1978}, who argues that while targeted welfare programs (“tagging”) are theoretically efficient in resource allocation, the way such programs are structured in themselves often leads to the kinds of complexities and unintended incentives discussed above.

Among these different types of costs, recent research suggests that information costs may play a particularly important role in explaining non-take-up. Individuals who are unaware of available benefits are much less likely to claim them, and higher perceived information costs are consistently associated with lower take-up rates. In fact, studies such as \cite{bolland_information_nodate} and \cite{currie_takeup_2004} suggest that, compared to process complexity or stigma, information costs are the most significant predictor of non-take-up. In line with this, several studies suggest that stigma is generally less relevant in the context of student aid than in other forms of public support (see for example, \cite{konijn_quantifying_2023}, \cite{currie_takeup_2004} and \cite{bruckmeier_new_2012}).

While this view reflects traditional economic thinking, recent research suggests that non-take-up may persist even when financial and administrative barriers are minimal. For example, \cite{bhargava_psychological_2015} show that even when procedural barriers are low, cognitive and behavioural factors can still lead to high levels of non-take-up. One important factor is present-bias, where individuals place greater weight on immediate costs or inconveniences compared to future benefits. As a result, relatively minor psychological obstacles, such as uncertainty about eligibility, confusing application steps, or unclear instructions, can discourage people from applying. Tasks like completing forms or gathering documents may feel disproportionately burdensome, causing individuals to delay or avoid the application process altogether, even when they recognize that receiving the benefit would be worthwhile in the long run \citep{currie_takeup_2004}. These findings support broader behavioural models that recognize limits to attention, self-control, and cognitive resources. 

Considered collectively, the literature suggests that both informational and behavioural barriers may be relevant for explaining non-take-up, although studies vary in their assessment of the relative importance of each. Reflecting this, our analysis includes measures of both informational and behavioural factors, with the aim of providing additional evidence on their roles in the context of BAföG specifically.

Beyond the overall effects, evidence points to variation in how different groups are affected by these barriers. Complex application procedures have the strongest negative impact on take-up among low-income and otherwise disadvantaged groups, such as first-generation students and those less familiar with paperwork and official processes. Ironically, these groups are often the main targets of aid programs due to their greater need, yet they appear especially likely to miss out on support because of such barriers. Additionally, language barriers can further reduce take-up for students from migrant backgrounds \citep{dynarski_cost_2006, bhargava_psychological_2015, currie_takeup_2004}.

Although several studies emphasise that migrant students may be especially hindered by informational or language barriers, the empirical evidence is mixed. \cite{herber_non-take-up_2019} report higher take-up among migrant students in Germany, and \cite{konijn_quantifying_2023} find a similar pattern for the Netherlands. These divergent results suggest that the relationship between migration background and take-up may depend on the specific institutional and socio-economic context.

%These findings show that non-take-up is shaped by both structural and behavioural factors. While economic models help explain some of this, recent work suggests that small barriers in procedures, communication, or perception can also play a large role.

Taken together, the factors discussed in this chapter can affect take-up rates substantially. While research on non-take-up of student aid is still remarkably limited, the literature body on non-take-up of other social benefits has grown substantially in recent years. Non-take-up of means-tested benefits specifically appears to be a widespread and persistent phenomenon across different countries and programs. Some studies have suggested that non-take-up rates rarely fall below 20\%, even in relatively well-studied contexts \citep{vanoorschot_failing_2002}. In a recent paper by \cite{goedeme_concept_2020}, high non-take-up levels for means-tested social assistance schemes are highlighted, often ranging between 30 to 70\%. To illustrate the issue of non-take-up, we present recent selected findings for a range of social benefits in Table~\ref{table:NTU-studies}.

\begin{table}[htbp]
\renewcommand{\arraystretch}{1.3}
\footnotesize
\centering
\begin{tabularx}{\textwidth}{@{}l l l l X@{}}
\toprule
\textbf{Author(s)} & \textbf{NTU (\%)} & \textbf{Year of Data} & \textbf{Country} & \textbf{Program Type} \\
\midrule
\cite{herber_non-take-up_2019}    & 36--40   & 2002--2013 & Germany     & BAföG \\
\cite{konijn_quantifying_2023}    & 24       & 2019       & Netherlands & Student Aid \\
\cite{frick_claim_2007}            & 67       & 2002       & Germany     & Social Assistance \\
\cite{fuchs_austria_2007}          & 49--61   & 2003       & Austria     & Social Assistance \\
\cite{goedeme_take_2022}           & 60       & 2019       & Belgium     & Social Assistance \\
\cite{harnisch_nontakeup_2019}    & 56       & 2005--2014 & Germany     & Income support \\
\cite{domingo_nonrecours_2014}     & 68       & 2010       & France      & Earnings supplement \\
\bottomrule
\end{tabularx}
\caption{\small{Selected international estimates of non-take-up (NTU) rates for social benefit programs. Program types are abbreviated or summarized for clarity.}}
\label{table:NTU-studies}
\end{table}


Notably, for student aid programs, only a small number of studies such as \cite{herber_non-take-up_2019} for Germany and \cite{konijn_quantifying_2023} for the Netherlands provide concrete non-take-up estimates. For many other student aid schemes, published measures of non-take-up rates are limited, and the relative lack of research in this area is noteworthy. Much of the existing literature instead focuses on the factors influencing take-up decisions rather than quantifying its overall prevalence.

In the end, it is a complex interaction of multiple different factors, including scheme structure, administrative practices and individual factors, that ultimately determines non-take-up. The research literature shows that the specific combination of factors shaping non-take-up varies across different benefit schemes, target groups, administrative contexts, and over time. This interconnectedness helps explain why studies focusing on similar issues often arrive at different conclusions. It also emphasizes that responsibility for non-take-up does not rest solely with individuals, but is shared by administrators and policymakers who shape the broader environment in which claiming decisions are made \citep{vanoorschot_failing_2002}. This is why efforts to increase take-up in social benefit programs have often been found to fall short when they focus on removing individual administrative barriers in isolation. Research has shown that meaningful improvements are more likely when multiple obstacles are addressed together \citep{currie_takeup_2004}. This perspective highlights the need for policymakers to consider the entire application process from the applicant’s point of view, rather than relying on isolated changes that may leave other important barriers in place.

Building on these insights, the next section focuses on BAföG, outlining the institutional and policy context relevant to student aid provision in Germany.