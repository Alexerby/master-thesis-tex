\section{Related Literature} \label{sec:literature}

- What we provide to the literature (short) 

- What other studies have looked into

In broad terms, research on non take up can be split into studies applying traditional economic theory and studies applying behavioural economic theory. Traditional economic theory assumes rational behaviour and thus that agents optimise the trade off between benefit and cost. The field of behavioural economics, that has more recently emerged, instead emphasises deflections from the traditional assumptions on rational behaviour and highlights cognitive bias and behavioural obstacles \citep{mechelen_who_2017}.

Optimal policy design aims to target support to those who need it most, while keeping eligibility criteria clear and easy to understand. However, targeted programmes often involve complicated rules, such as income limits and asset tests. This complexity poses some difficulties, both for administrative workers and applicants. For administrators, it increases the risk of mistakes when assessing eligibility. For applicants, understanding the rules and completing the process can be time consuming and require substantial effort, especially if there are uncertainties about eligibility. This can discourage people from applying. Complexity can also result in tertiary non take up, where certain groups are excluded by design because the rules use rough indicators of need. One way to reduce non take up is to use simpler, categorical criteria based on things like age or household type. Although this may make targeting less precise, it can make the application process easier and more accessible \citep{mechelen_who_2017}.