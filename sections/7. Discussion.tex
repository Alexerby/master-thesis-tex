\section{Discussion}
This paper first implements a microsimulation of BAföG eligibility to estimate non-take-up rates for the period 2007–2021. 
We then use three model specifications, Logit, Probit, and a Linear Probability Model (LPM), to analyse the determinants of benefit take-up and investigate the underlying causes of non-participation among eligible students. 
Our preferred models achieve an accuracy of approximately 72\%, indicating a reliable fit to the observed data.


Our microsimulation shows that, across years 2007–2021 and under plausible measurement error, the probability of non-take-up remains high (around 60\%). 
Our econometric models help explain this pattern by identifying predictors of non-take-up: a higher simulated BAföG amount is associated with significantly lower non-take-up, and factors such as age, partnership status, and migration background emerge as robust correlates. 
In contrast, psychological traits such as impulsiveness, debt aversion, and risk appetite are not statistically significant, suggesting that structural and informational barriers, not behavioural predispositions, are the key drivers.

Our study relies on survey data, which inherently suffers from measurement errors due to the use of proxy variables, missing data, and potential reporting biases. 
These data limitations may introduce estimation errors, including the possibility of \( \beta \)-errors in identifying determinants of non-take-up. 
Nevertheless, the 72\% accuracy of our predictive model suggests a reasonably strong fit to the observed data. 
While we cannot fully rule out that the exact magnitude of non-take-up rates might be somewhat imprecise, the overall message remains clear: the core goal of BAföG, targeting financial support to those most in need, is not yet fully achieved.

Although Germany’s student aid offers favourable terms and uses strict means-testing to target those truly in need, the analysis suggests it is not necessarily the most financially disadvantaged students who benefit. 
Instead, those with higher social capital and a more favourable view of government intervention are more likely to make use of the support.

Importantly, non-take-up is not a uniform phenomenon. 
While it might be tempting to attribute low take-up to personal traits, our results show no statistical significance for impulsiveness, debt aversion, or risk appetite. 
This suggests that structural and informational barriers—such as application complexity, lack of awareness, or uncertainty about eligibility—play the primary role. 
Reducing non-take-up will therefore require a combination of administrative simplification, clearer communication and targeted outreach.

\subsection{Policy Implications}

Addressing non-take-up of social benefits is politically sensitive. Policies promoting increased take-up are often seen as calls for higher public spending and can face resistance—even when aimed at improving fairness within existing budgets.

BAföG’s core goal is to prevent financial constraints from blocking access to higher education. Yet, our findings indicate that the application process places a heavy administrative burden on students, particularly deterring those eligible for smaller amounts. This outcome contradicts the programme’s mission to support less affluent students.

To better target financial need, policymakers should simplify the means-testing process. While income verification is essential for precise targeting, the procedure must not be so complex or time-consuming that it discourages eligible applicants. A system this complicated naturally favours applicants with greater social capital and motivation over those who need support most.

Currently, applying demands detailed disclosure of fourteen income types and sixteen categories of assets and debts. Moreover, parents complete a comprehensive four-page form about income and siblings \citep{fidan_why_2021}. This complexity likely excludes economically disadvantaged families who may lack the resources or motivation to navigate the process.

Simplification, such as increased use of pre-filled tax data and automated eligibility checks, could balance accuracy with accessibility.

Ultimately, non-take-up caused by administrative complexity reflects a design flaw rather than a failure of intent. A more user-friendly BAföG could improve both equity and efficiency, ensuring aid reaches those most in need without increasing overall costs.

