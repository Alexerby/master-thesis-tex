\section{Discussion}
%%%%%%%%%%%%%%%%%%%%%%%%%%%%%%%%%%%%%%%%%%%%%%%%%%%%%%
%
% Introduction to what we are investigating
%
%%%%%%%%%%%%%%%%%%%%%%%%%%%%%%%%%%%%%%%%%%%%%%%%%%%%%%
This paper examines the non-take-up (NTU) of Germany’s BAföG student aid program, focusing on why eligible students forgo what is effectively ``free money'', a combination of a 50\% grant and a 50\% interest-free loan.

To investigate the individual-level determinants of BAföG non-take-up, we use microdata from the German Socio-Economic Panel (SOEP), which provides rich information on students' family background, financial situation, and behavioural traits relevant to financial aid decisions. 
Although SOEP has a panel structure, we treat the data as a pooled cross-section and assume the relationship between explanatory variables and non-take-up is stable across years.

We estimate three types of binary response models: a logit model, a probit model, and a linear probability model (LPM). 
These models allow us to assess the relationship between individual characteristics and the likelihood of non-take-up. 
Logit and probit models are used to account for the nonlinear nature of binary outcomes, while the LPM offers a straightforward linear benchmark and facilitates interpretation. 
Standard errors are clustered at the individual level to account for potential intra-respondent correlation in the presence of repeated observations. 
Table~\ref{tab:logit_probit_lpm_results} presents the results from these three specifications, including estimated coefficients and average marginal effects where applicable.


%%%%%%%%%%%%%%%%%%%%%%%%%%%%%%%%%%%%%%%%%%%%%%%%%%%%%%
%
% Summary of key findings
%
%%%%%%%%%%%%%%%%%%%%%%%%%%%%%%%%%%%%%%%%%%%%%%%%%%%%%%

%%%%%%%%%%%%%%%%%%%%%%%%%%%%%%%%%%%%%%%%%%%%%%%%%%%%%%
%
% Interpretation of findings and possible explanations
%
%%%%%%%%%%%%%%%%%%%%%%%%%%%%%%%%%%%%%%%%%%%%%%%%%%%%%%

%%%%%%%%%%%%%%%%%%%%%%%%%%%%%%%%%%%%%%%%%%%%%%%%%%%%%%
%
% Comparison with previous literature
%
%%%%%%%%%%%%%%%%%%%%%%%%%%%%%%%%%%%%%%%%%%%%%%%%%%%%%%

%%%%%%%%%%%%%%%%%%%%%%%%%%%%%%%%%%%%%%%%%%%%%%%%%%%%%%
%
% Policy implications or practical relevance
%
%%%%%%%%%%%%%%%%%%%%%%%%%%%%%%%%%%%%%%%%%%%%%%%%%%%%%%

%%%%%%%%%%%%%%%%%%%%%%%%%%%%%%%%%%%%%%%%%%%%%%%%%%%%%%
%
% Limitations of the study
%
%%%%%%%%%%%%%%%%%%%%%%%%%%%%%%%%%%%%%%%%%%%%%%%%%%%%%%

%%%%%%%%%%%%%%%%%%%%%%%%%%%%%%%%%%%%%%%%%%%%%%%%%%%%%%
%
% Suggestions for future research
%
%%%%%%%%%%%%%%%%%%%%%%%%%%%%%%%%%%%%%%%%%%%%%%%%%%%%%%
