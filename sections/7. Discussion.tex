\section{Discussion}
%%%%%%%%%%%%%%%%%%%%%%%%%%%%%%%%%%%%%%%%%%%%%%%%%%%%%%
%
% Introduction to what we are investigating
%
%%%%%%%%%%%%%%%%%%%%%%%%%%%%%%%%%%%%%%%%%%%%%%%%%%%%%%
This paper examines the non-take-up (NTU) of Germany’s BAföG student aid program, focusing on why eligible students forgo what is effectively ``free money'', a combination of a 50\% grant and a 50\% interest-free loan.

To investigate the individual-level determinants of BAföG non-take-up, we use microdata from the German Socio-Economic Panel (SOEP), which provides rich information on students' family background, financial situation, and behavioural traits relevant to financial aid decisions. 
Although SOEP has a panel structure, we treat the data as a pooled cross-section and assume the relationship between explanatory variables and non-take-up is stable across years.

We estimate three types of binary response models: a logit model, a probit model, and a linear probability model (LPM). 
These models allow us to assess the relationship between individual characteristics and the likelihood of non-take-up. 
Logit and probit models are used to account for the nonlinear nature of binary outcomes, while the LPM offers a straightforward linear benchmark and facilitates interpretation. 
Standard errors are clustered at the individual level to account for potential intra-respondent correlation in the presence of repeated observations.
Table~\ref{tab:logit_probit_lpm_results} presents the results from these three specifications, including estimated coefficients and average marginal effects where applicable.


%%%%%%%%%%%%%%%%%%%%%%%%%%%%%%%%%%%%%%%%%%%%%%%%%%%%%%
%
% Summary of key findings and interpretation
%
%%%%%%%%%%%%%%%%%%%%%%%%%%%%%%%%%%%%%%%%%%%%%%%%%%%%%%
Our results indicate that the most relevant explanatory variables for BAföG non-take-up are demographic characteristics, such as age, having a partner, and migration background. 
These factors suggest that individual life circumstances and aspects of social capital may influence both awareness of the program and the perceived need for financial aid.

Moreover, the significant role of East German background points to potential regional differences in attitudes toward state support. 
This could reflect divergent historical experiences and institutional trust, where East German students may be more inclined to view public assistance as legitimate, while in West Germany, stigma surrounding the uptake of state-subsidized aid may still persist.

In contrast, we find no evidence that behavioural traits such as patience, impulsiveness, or risk appetite significantly influence non-take-up. 
This suggests that psychological or personality-related factors are not the primary drivers behind the decision to forgo applying for BAföG.

%%%%%%%%%%%%%%%%%%%%%%%%%%%%%%%%%%%%%%%%%%%%%%%%%%%%%%
%
% Policy implications or practical relevance
%
%%%%%%%%%%%%%%%%%%%%%%%%%%%%%%%%%%%%%%%%%%%%%%%%%%%%%%
\subsection{Policy Implications}
Addressing the issue of non-take-up of social benefits is politically sensitive. 
Promoting policies that increase benefit take-up is often perceived as advocating for higher public expenditure, which may face resistance, even if the objective is to improve access and fairness within existing frameworks.

However, the very purpose of BAföG is to ensure that financial constraints do not prevent access to higher education. 
Our findings suggest that the current system places a considerable administrative burden on applicants, which may deter eligible students, especially those entitled to smaller amounts from applying. 
(\textcolor{red}{Maria: Source for the amount of hours it takes to apply!})
This is at odds with the program's intention to support students from less affluent backgrounds.

If the primary goal is to target financial need effectively, policymakers should consider simplifying the means-testing procedure to reduce unnecessary barriers. 
While some form of income assessment remains essential to ensure targeted support, the process should not be so complex or burdensome that it discourages eligible students from applying. 
Simplified procedures, increased use of pre-filled data from tax records, or more automated eligibility checks could help strike a better balance between precision and accessibility.

Ultimately, non-take-up driven by administrative complexity represents a failure of design rather than intent. 
A more user-friendly system could improve both the efficiency and equity of BAföG, ensuring that support reaches those who need it most—without necessarily requiring increased overall expenditure.

%%%%%%%%%%%%%%%%%%%%%%%%%%%%%%%%%%%%%%%%%%%%%%%%%%%%%%
%
% Comparison with previous literature
%
%%%%%%%%%%%%%%%%%%%%%%%%%%%%%%%%%%%%%%%%%%%%%%%%%%%%%%




%%%%%%%%%%%%%%%%%%%%%%%%%%%%%%%%%%%%%%%%%%%%%%%%%%%%%%
%
% Limitations of the study
%
%%%%%%%%%%%%%%%%%%%%%%%%%%%%%%%%%%%%%%%%%%%%%%%%%%%%%%

%%%%%%%%%%%%%%%%%%%%%%%%%%%%%%%%%%%%%%%%%%%%%%%%%%%%%%
%
% Suggestions for future research
%
%%%%%%%%%%%%%%%%%%%%%%%%%%%%%%%%%%%%%%%%%%%%%%%%%%%%%%
