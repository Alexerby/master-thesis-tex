\section{Discussion}

%TODO: Don't address the low support rates, need to add that.

%TODO: Feels a bit chatty currently, go over and rephrase were needed

This paper first implements a microsimulation of BAföG eligibility to estimate non-take-up rates for the period between 2007 and 2021. 
We then use three model specifications, Logit, Probit, and a Linear Probability Model (LPM), to analyse the determinants of benefit take-up and investigate the underlying causes of non-take-up among eligible students. 

Our microsimulation indicates that, across the observed period the probability of non-take-up remains high at approximately 60\%.
Our econometric models help explain this pattern by identifying key predictors of non-take-up. 
A higher simulated BAföG amount is associated with a significantly lower non-take-up rate. 
In addition, age, partnership status, and migration background appear as important correlates.
In contrast, behavioural traits such as impulsiveness, debt aversion, and risk appetite are not statistically significant, which may suggest that structural and informational barriers play a more prominent role than behavioural predispositions in explaining non-take-up.

Our study relies on survey data, which inherently suffers from measurement errors due to the use of proxy variables, missing data, and potential reporting biases. 
These data limitations may introduce estimation errors, including the possibility of \( \beta \)-errors in identifying determinants of non-take-up. 
Nevertheless, the accuracy of our microsimulation suggests a reasonably strong fit of 72\%.
While we cannot fully rule out that the exact magnitude of non-take-up rates may be subject to some imprecision, the findings nonetheless point to a persistent gap between the intended goals of BAföG and its actual reach among financially eligible students.

Although Germany’s student aid system offers favourable terms and employs strict means-testing to target those in financial need, our analysis suggests that take-up is not concentrated among the most financially disadvantaged students. 
Instead, students with higher levels of social capital and a more favourable view of government intervention appear more likely to make use of the support.

Although personal traits are sometimes considered relevant for explaining low take-up, our results do not find statistically significant effects for impulsiveness, debt aversion, or risk appetite. 
This suggests that structural and informational barriers, such as application complexity, lack of awareness, or uncertainty about eligibility, may play a more important role. 
Addressing non-take-up may therefore require a combination of administrative simplification, clearer communication, and targeted outreach efforts.

\subsection{Policy Implications}

Addressing non-take-up of social benefits is politically sensitive. 
Policies promoting increased take-up are often seen as calls for higher public spending and can face resistance, even when aimed at improving fairness within existing budgets.

One of the main objectives of BAföG is to prevent financial constraints from limiting access to higher education. 
However, the application process is widely seen as administratively burdensome, which may be especially discouraging for students entitled to lower aid amounts or those with limited experience navigating bureaucratic processes.
This raises concerns about the program’s ability to effectively reach less affluent students.

To address this, policymakers should consider simplifying the means-testing process. 
While income verification is essential for precise targeting, the procedure must not be so complex or time-consuming that it discourages eligible applicants from applying. 
Targeted simplification, such as increased use of pre-filled tax data and automated eligibility checks, could help balance accuracy with accessibility.

Ultimately, non-take-up caused by administrative complexity reflects a design flaw rather than a failure of intent. 
A simpler and more accessible application process could improve both equity and efficiency, ensuring that aid reaches those most in need without increasing overall costs.
