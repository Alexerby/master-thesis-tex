
\section{Background}
\label{section:theoretical_empirical_context}

The federal training assistance act (de. Bundesausbildungsförderungsgesetz, BAföG) is a public student aid system supplied by the German federal ministry of education and research. BAföG is designed to financially support students, with the primary aim to promote equal opportunities in the education system and unlock educational potential \citep{meier_bafog_2024}. The eligibility criteria for the loan is therefore relatively strict to make sure that only students who are genuinely in need of the loan have access to it.

Since the beginning, BAföG has adhered to the principle of subsidiarity in its basic conception, which is in line with traditional welfare policies in Germany. That is the principle that smaller local units perform their own tasks and a central authority only provides help when necessary, i.e. has a subsidiary function. In the context of BAföG, this means that first, in order to finance their studies, students must rely on their own income and assets. The next larger social units to be held accountable are spouses or partners, and next the parents become financially responsible. Only after these social units have been exhausted can students seek support from the state through BAföG. It is important to note that this support is not granted automatically; students must actively apply for it themselves \citep{staack_von_2017}. By contrast, countries such as those in the Nordic region tend to provide more universal benefits, reflecting a broader role for the state in supporting students \citep{gwosc_krisenbewaltigung_2022, schwarz_study_2004}. Thus, differences in student aid systems reflect deeper variations in welfare philosophies across countries.

These contrasting approaches are often described in terms of two main design principles for public student funding. The first is the provision principle, where financial aid is narrowly targeted to specific groups, most often through means-testing. This approach is reflected in the design of Germany’s BAföG, which is primarily targeted to reach students from socio-economically disadvantaged backgrounds. The second approach is the welfare principle, which provides public aid to a broader share of the student population. This model is more common in the Nordic countries, where financial support is generally designed to reach most students \citep{gwosc_krisenbewaltigung_2022, oecd_education_2024}.

Recent research suggests that students in countries with welfare-based student aid systems report fewer serious financial difficulties. These systems may also achieve slightly better representation of different social groups in higher education compared to more narrowly targeted models like BAföG  \citep{gwosc_krisenbewaltigung_2022}.

%TODO: Place this table at a reasonable place in the text
\begin{table}[htbp]
\footnotesize
\centering
\begin{tabular}{llrlll}
\toprule
\textbf{Author(s)} & \textbf{Year} & \textbf{NTU (\%)} & \textbf{Year of Data} & \textbf{Data Source} & \textbf{Program Type} \\
\midrule
\citeauthor{frick_claim_2007}              & 2007          & 67             & 2002                  & SOEP             & SA               \\
\citeauthor{herber_non-take-up_2019}       & 2016          & 36--40         & 2002--2013            & SOEP             & BAföG            \\
\citeauthor{RePEc:iab:iabfob:201305}       & 2013          & 34--43         & 2008                  & EVS              & BSS              \\
\citeauthor{bruckmeier_benefit_2018}       & 2018          & 43             & 2013--2014            & SOEP             & SA               \\
\citeauthor{bruckmeier_benefit_2018}       & 2018          & 87             & 2013--2014            & SOEP             & HA               \\
\citeauthor{bruckmeier_benefit_2018}       & 2018          & 63             & 2013--2014            & SOEP             & HA \& SA         \\
\citeauthor{bruckmeier_benefit_2018}       & 2018          & 88             & 2013--2014            & SOEP             & SCA              \\
\citeauthor{bruckmeier_new_2012}           & 2012          & 41--49         & 2005--2007            & SOEP             & SA               \\
\bottomrule
\end{tabular}
\caption{\small{Selected previous estimates of non-take-up (NTU) rates for social benefits in Germany. Program type abbreviations: SA = Social Assistance, BAföG = Federal Student Aid, MTG = Means-Tested General Benefits, BSS = Basic Social Security, HA = Housing Allowance, SCA = Supplementary Child Allowance.}}
% \caption*{\small{Note: This table summarizes selected results on non-take-up rates from prior literature using SOEP and other German datasets. See cited references for full details.}}
\label{table:NTU-studies}
\end{table}


To understand eligibility for BAföG and the amount of support awarded, it is essential to comprehend the means-testing process. The BAföG system calculates entitlements primarily based on the income and assets of the student and their parents, involving a detailed review of the applicant’s financial situation. This includes the parents’ income after accounting for taxes, social security contributions, and other standard allowances, with any income exceeding a predefined threshold deducted directly from the student’s potential entitlement. The resulting support is structured so that roughly half comes as a non-repayable grant, while the other half is provided as an interest-free loan, part of which is typically canceled if certain repayment conditions are met after graduation \citep{herber_non-take-up_2019}.

A description of the process is visualized in Figure~\ref{fig:pipeline-overview}, which outlines a simplified version of the calculation of income and asset adjustments for both students and parents. For a more detailed simulation example of an individual in our dataset, see Appendix~\ref{appendix:simulation-example}.

The BAföG application process is often regarded as overly complicated and discouraging. Applicants must provide detailed information on fourteen types of income and sixteen categories of assets and debts. Parents are also required to complete a comprehensive four-page form about their income and any siblings. Before 2016, all applications had to be submitted in hardcopy. On average, students spend more than five hours completing the paperwork \citep{fidan_why_2021}.

These administrative hurdles not only make the process time-consuming, but can also discourage students from applying in the first place. The forms are long and complicated, and it is not always clear whether an application will be successful. For students who are already unsure about their eligibility, that uncertainty alone can be enough to put them off \citep{kroher_studierendenbefragung_2023}. This is consistent with findings by \cite{fidan_why_2021}, who show that information gaps and behavioural factors, such as students incorrectly assuming they are ineligible or being confused by the process, play a significant role in explaining non-take-up. This is consistent with findings by Fidan (2021), who shows that information gaps and behavioural factors, such as students incorrectly assuming they are ineligible or being confused by the process, play a significant role in explaining non-take-up.
Overall, a combination of these behavioural barriers and structural factors likely contributes to the low rates of BAföG uptake seen today. Since the introduction of BAföG in 1971, the proportion of students receiving financial aid has fallen from around 50 percent to just 13 percent in 2021 \citep{kroher_studierendenbefragung_2023}. Of those recipients, only about half received full funding \citep{meier_zur_2024}. The decline over time appears to be driven in part by stricter eligibility rules, such as income thresholds that have not always kept pace with inflation or with the actual cost of living for students, which results in a fewer students qualifying now than in earlier decades \citep{meier_zur_2024}.
This gap between support and student needs is illustrated by data from the Sozialerhebung, which show that from at least 2000 to 2017, the maximum BAföG support rate remained below average reported student expenses. It was only with the 2022 reform (the 27th BAföG amendment) that the maximum support rate was increased to exceed average living costs for the first time \citep{meier_bafog_2024, meier_zur_2024}. This increase took place after the end of the period covered in this analysis and is therefore not reflected in the data used.
Both the size of the BAföG award and who ultimately receives support are determined by two core components of the system: the income exemption thresholds and the standard support rate. Exemption thresholds exist for different sources of income and assets, applying separately to income of students, parents, and assets. These thresholds specify the portion of income or assets that is disregarded when calculating eligibility, so only amounts above these limits affect the aid a student can receive. The standard support rate specifies the basic amount of financial assistance that a student is eligible to receive, based on factors such as living situation and educational needs.
These two mechanisms are interconnected, as raising the income exemption threshold both increases the number of students eligible for BAföG and raises the amounts granted to those who previously received only partial support. By law (§35 BAföG), both the support rate and the exemption threshold must be reviewed every two years and adjusted as needed to reflect changes in living costs, economic conditions, and income trends \citep{bafoeg_law, meier_zur_2024}.
When considering trends in BAföG uptake, however, it is important to recognise that a declining share of students receiving funding does not necessarily indicate that fewer students are in need of support. Some of the decline might reflect general improvements in living standards. Income per capita in Germany has increased over the past two decades, and shifts in demographics and household income levels may mean that some students are no longer eligible under the current rules. This can be viewed as a general prosperity effect. Furthermore, the share of students receiving financial aid is also affected by various behavioural factors, including fluctuations in demand for education and the social composition of prospective students. This proportion does thus not accurately reflect how many students are actually in need of financial aid nor how many of them receive such aid \citep{meier_bafog_2024, meier_zur_2024}.

While a drop in financial aid rates might suggest that fewer students are in need of support, this interpretation has its limits. Rising income levels and changing demographics may explain some of the reduced eligibility, but they don't account for why many students who seem to be eligible choose not to apply. Things like uncertainty about eligibility, the complexity of the system, or whether the amount of support seems worth the effort, all influence take up rates. As previous studies have shown, it’s not just about who qualifies on paper, it’s also about how the system is experienced by students themselves \citep{meier_bafog_2024, meier_zur_2024}.

These experiences are not the same for all students. In fact, these patterns raise important questions about how effectively BAföG is reaching the groups it is meant to support. For example, informational and structural barriers may disproportionately affect students whose parents did not attend university or those with a migration background \citep{kroher_studierendenbefragung_2023}.

Survey data helps to illustrate how these barriers manifest in practice. The 22\( ^\text{nd} \) German student survey estimated that just about 80\% of students did not apply for BAföG during the term it was conducted (2021). Data was also collected on reasons students had for not applying for BAföG. The most commonly stated reason was thinking that parental income was too high, with about 74\% of non-applicants claiming that as one of the reasons. The second most common reason stated was the perception that own income or assets were too high, with just under 30\% of non-applicants claiming that. Fear of debt was cited by just over 21\% of non-applicants, making it the third most common reason. Additionally, around 8\% indicated that the expected funding amount would be too low as a reason for not applying \citep{kroher_studierendenbefragung_2023}.

