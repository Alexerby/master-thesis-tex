\section{Discussion}

In this paper, we have investigated the non-take-up of student aid in Germany (BAföG) by estimating theoretical eligibility using a detailed microsimulation model and linking the findings to reported benefit receipt in the SOEP panel survey. Across the years 2007 to 2021, our findings suggest that non-take-up among theoretically eligible students has remained high, averaging at approximately 60\%. While there are fluctuations over time, ranging from a low of 50\% to a high of 70\%, a slight upward trend in non-take-up can be observed in the years leading up to 2021.

%TODO: Confirm that the last sentence is correct, about the upward trend in the years leading up to 2021.

In our econometric analysis, we use three model specifications, Logit, Probit, and a Linear Probability Model (LPM), to shed light on several important predictors of take-up. One of the most consistent findings across all models is the negative association between the simulated entitlement size and non-take-up. This result aligns with economic intuition, suggesting that when the expected subsidy is larger, the perceived benefits of applying are more likely to outweigh the associated costs.

Age and partnership status also appear to be important covariates. 
Older students are more likely to forgo BAföG support, which could reflect a lower perceived relevance of the program at later stages of study, or possibly greater financial independence. 
Similarly, students in a registered partnership are found to have higher non-take-up rates. 
This may be due to higher combined household income or a reduced perceived need for financial aid resulting from shared living expenses.

%TODO:  not sure about the “lower perceived relevance of the program at later stages of study”, maybe an interpretation closer to older students being more likely to assess their own eligibility to be low or assume that they are more unlikely to fulfill the eligibility requirements? Could even probably find some numbers (even from our own data) showing correlation between being older and working more alongside studies to solidify this interpretation? The same might to some extent be true for the partnership thing

Although not entirely expected, the finding that students with direct or indirect migration backgrounds have lower non-take-up rates aligns with some earlier research \citep{herber_non-take-up_2019, konijn_quantifying_2023}. One plausible explanation is tighter budget constraints: if migrant students can rely on fewer familial or social safety nets, the expected benefit of BAföG may outweigh the application’s complexity. Cultural attitudes toward debt could also play a role, as personal borrowing remains comparatively uncommon in Germany \citep{seabrooke_germany_2017}, though we have not examined this directly. 

%TODO: Do we maybe need a source on the “fewer alternative familial or social safety nets compared to native students”?
%TODO: Also text about the migration variable (the above two paragraphs) is a bit long, might need to shorten
%TODO: now that lexi pepsi has gone over other studies like this, this finding is maybe not that surprising. If so, we might want to shorten the above two paragraphs and adjust the language in line with that.

Our findings suggest that behavioural traits such as impulsiveness, patience, and risk appetite do not play a significant role in predicting non-take-up of BAföG. This is similar to the results reported by \cite{herber_non-take-up_2019}, who also found no evidence for an effect of impulsiveness or impatience when considered separately. However, they did find that students who were both highly impulsive and highly impatient, captured by an interaction term between the two traits, were more likely to forgo applying. We tested the same interaction in our data but did not find it to be significant. Overall, these patterns point to mixed evidence on the relevance of behavioural characteristics in understanding non-take-up decisions. In our study, structural and informational variables appear to play a clearer role in explaining non-take-up.

%TODO: MARIA START FROM HERE WHEN BACK FROM ICA

It is important to note that our results rely on self-reported survey data, which inherently suffers from measurement errors due to the use of proxy variables, missing data, and potential reporting biases. 
These data limitations may introduce estimation errors, including the possibility of beta errors, in identifying determinants of non-take-up. 
Nevertheless, the accuracy of our microsimulation, with a reasonably strong fit of 72\%, provides confidence in the reliability of our estimates. 
The central finding remains that a substantial share of financially eligible students do not receive the support they are entitled to, highlighting a persistent gap between the intention and outcome of BAföG.

Although BAföG offers favourable terms compared to many student aid systems, its complex application and strict means-testing can deter students less familiar with bureaucracy or expecting limited benefits.
Students with a sibling who previously received BAföG or from East German backgrounds, where attitudes toward public support differ, are more likely to apply.
This suggests that social capital, through familiarity and trust in public programs, plays an important role in take-up.

From an economic perspective, the decision to apply for BAföG can be viewed as a weighing of costs and benefits. 
Policy makers therefore have two primary levers to encourage higher take-up: reducing the non-monetary costs associated with applying, such as informational and procedural hurdles, or increasing the benefits by raising the support rates. 

Our analysis suggests there is considerable room for improvement on both fronts. 
The strict means-tested design of BAföG creates significant informational and procedural barriers that, according to our findings, unnecessarily hinder take-up.
While greater administrative simplicity may entail some loss of precision in targeting, it could substantially reduce non-take-up. 
At the same time, support rates have for many years failed to keep pace with the actual cost of living for students, thereby diminishing the relative benefit of applying. 
Although some steps have recently been taken to adjust benefit levels, these changes occurred after the period covered by our data. 
Further research that incorporates the recent reforms could not only provide a more complete picture of the program’s effectiveness, but also estimate more precisely how such changes affect take-up rates.

Taken together, our results indicate that increasing take-up requires a dual approach: raising subsidy levels to ensure benefits are meaningful in real terms, and streamlining application procedures to lower the costs of accessing support. In line with the program's objective to promote equal access to higher education, it is especially important that BAföG reaches students with the greatest financial need, as reflected in higher entitlements. While simplifying administration would benefit all eligible students, ensuring that those entitled to the largest amounts are able to access support most directly advances the programme’s goals. However, such efforts should be careful not to introduce additional complexity or restrict eligibility, as these factors may themselves discourage take-up.

 Ultimately, our findings reinforce the view that administrative complexity is a design flaw rather than a failure of intent. Addressing both procedural hurdles and lagging support rates is essential for making BAföG more effective in promoting equal access to higher education.
