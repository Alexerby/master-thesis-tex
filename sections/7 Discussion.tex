\section{Discussion}

In this paper, we have investigated the non-take-up of student aid in Germany (BAföG) by estimating theoretical eligibility using a detailed microsimulation model and linking the findings to reported benefit receipt in the SOEP panel survey. Across the years 2007 to 2021, our findings suggest that non-take-up among theoretically eligible students has remained high, averaging at approximately 60\%. While there are fluctuations over time, ranging from a low of 50\% to a high of 70\%, a slight upward trend in non-take-up can be observed in the years leading up to 2021.

%TODO: Confirm that the last sentence is correct, about the upward trend in the years leading up to 2021.

In our econometric analysis, we use three model specifications, Logit, Probit, and a Linear Probability Model (LPM), to shed light on several important predictors of take-up. One of the most consistent findings across all models is the negative association between the simulated entitlement size and non-take-up. This result aligns with economic intuition, suggesting that when the expected subsidy is larger, the perceived benefits of applying are more likely to outweigh the associated costs.

Age and partnership status also appear to be important correlates. Older students are more likely to forgo BAföG support, which could reflect a lower perceived relevance of the program at later stages of study, or possibly greater financial independence. Similarly, students in a registered partnership are found to have higher non-take-up rates. This may be related to household income considerations, or perhaps to a lower perceived need for aid due to shared resources, although we cannot directly observe partner income in our data.

%TODO: not sure about the “lower perceived relevance of the program at later stages of study”, maybe an interpretation closer to older students being more likely to assess their own eligibility to be low or assume that they are more unlikely to fulfill the eligibility requirements? Could even probably find some numbers (even from our own data) showing correlation between being older and working more alongside studies to solidify this interpretation? The same might to some extent be true for the partnership thing

A somewhat unexpected result is that students with either direct or indirect migration backgrounds appear to have lower non-take-up rates compared to students without a migration background. This finding is contrary to the expectation that migrant students may face greater informational or procedural barriers. One possible explanation is that financial constraints could be more binding for these students, for example due to fewer alternative familial or social safety nets compared to native students, although we do not observe this mechanism directly in our data. This could in turn potentially increase the incentive to navigate the application process despite its complexity. Furthermore, comparative research suggests that loans play a relatively marginal role in German welfare and education policy, reflecting both cultural and institutional preferences for savings and a less widespread reliance on personal debt \citep{seabrooke_germany_2017}. 

This finding could thus also in part reflect differences in attitudes towards debt. However, whether this context has a differentiated impact for students from migrant backgrounds remains an open question. Finally, some researchers have pointed out that migrants tend to be overrepresetned among social assistance recipients, which is primarily due to higher rates of eligibility. While this is true, migrants and foreigners have often been underrepresented or misclassified in major datasets, which has been found to be a persistent issue in German survey research \citep{frick_claim_2007, liebaut_surveying_2016}. Such data limitations could affect the precision of our estimates for students with a migration background, and should be kept in mind when interpreting our results.

%TODO: Do we maybe need a source on the “fewer alternative familial or social safety nets compared to native students”?
%TODO: Also text about the migration variable (the above two paragraphs) is a bit long, might need to shorten
%TODO: now that lexi pepsi has gone over other studies like this, this finding is maybe not that surprising. If so, we might want to shorten the above two paragraphs and adjust the language in line with that.

Our analysis does not find statistically significant effects for behavioral traits such as impulsiveness or patience in predicting non-take-up of BAföG. This contrasts with the findings of \cite{herber_non-take-up_2019}, who analyzed similar behavioral variables and found a statistically significant effect only for the interaction between high impulsivity and high impatience, while the main effects of each variable were not significant. Although we do not show these results in our tables, we conducted the same interaction analysis in our data and did not find a significant effect. Taken together, these results contribute to a mixed evidence base regarding the importance of behavioral characteristics in explaining non-take-up. In our analysis, structural and informational barriers appear to be more prominent determinants than the behavioral variables considered.

It is important to note that our results rely on self-reported survey data, which inherently suffers from measurement errors due to the use of proxy variables, missing data, and potential reporting biases. These data limitations may introduce estimation errors, including the possibility of beta errors in identifying determinants of non-take-up. Nevertheless, the accuracy of our microsimulation, with a reasonably strong fit of 72\%, provides confidence in the reliability of our estimates. The central finding remains that a substantial share of financially eligible students do not receive the support they are entitled to, pointing to a persistent gap between the objectives of the BAföG programme and its actual reach.

Although BAföG offers relatively favourable terms compared to many other student aid systems, the complexity of the application process and the strictness of means-testing may deter students who are less familiar with bureaucratic procedures or who expect only modest entitlements. Our findings suggest that take-up is not randomly distributed among eligible students but is instead associated with factors that reflect differential access to information, prior experience, or general comfort in dealing with administrative systems. For example, students with a sibling who previously received BAföG, or those from East German backgrounds, where historical attitudes toward public support may differ, are significantly more likely to take up the aid. These patterns point to social capital, in the form of familiarity and trust in public programmes, as a key enabling factor.

From an economic perspective, the decision to apply for BAföG can be viewed as a weighing of costs and benefits. Policy makers therefore have two primary levers to encourage higher take-up: reducing the non-monetary costs associated with applying, such as informational and procedural hurdles, or increasing the benefits by raising the support rates. Our analysis suggests there is considerable room for improvement on both fronts. The strictly means-tested design of the BAföG programme introduces significant informational and procedural barriers, which, as our findings indicate, constitute major hurdles to take-up. While greater administrative simplicity may entail some loss of precision in targeting, it could substantially reduce non-take-up. At the same time, support rates have for many years failed to keep pace with the actual cost of living for students, thereby diminishing the relative benefit of applying. Although some steps have recently been taken to adjust benefit levels, these changes occurred after the period covered by our data. Further research that incorporates the recent reforms could not only provide a more complete picture of the program’s effectiveness, but also estimate more precisely how such changes affect take-up rates.

Taken together, our results indicate that increasing take-up requires a dual approach: raising subsidy levels to ensure benefits are meaningful in real terms, and streamlining application procedures to lower the costs of accessing support. In line with the program's objective to promote equal access to higher education, it is especially important that BAföG reaches students with the greatest financial need, as reflected in higher entitlements. While simplifying administration would benefit all eligible students, ensuring that those entitled to the largest amounts are able to access support most directly advances the programme’s goals. However, such efforts should be careful not to introduce additional complexity or restrict eligibility, as these factors may themselves discourage take-up.
Ultimately, our findings reinforce the view that administrative complexity is a design flaw rather than a failure of intent. Addressing both procedural hurdles and lagging support rates is essential for making BAföG more effective in promoting equal access to higher education.
