%%%%%%%%%%%%%%%%%%%%%%%%%%%%%%%%%%%%%%%%%%%%%%%%%%%%%%%%%%%
%
% DATA
%
%%%%%%%%%%%%%%%%%%%%%%%%%%%%%%%%%%%%%%%%%%%%%%%%%%%%%%%%%%%


%%%%%%%%%%%%%%%%%%%%%%%%%%%%%%%%%%%%%%%%%%%%%%%%%%%%%%%%%%%


\section{Data}
To estimate non-take-up rates of welfare benefits, researchers typically rely on one or more of three data sources: administrative records, specially designed surveys, and general purpose household surveys. 
Each comes with its own trade-offs. 
Administrative data are accurate for tracking benefit receipt but usually lack information on those who do not apply. 
Special surveys can provide richer detail on eligibility and claiming behaviour, though they are costly and rarely implemented. 
General purpose surveys are more readily available and widely used in empirical research on non-take-up, even if they are not designed with this purpose in mind \citep{mechelen_who_2017}.

In line with much of the existing literature, this study relies on data from the German Socio-Economic Panel (SOEP), which falls into the third category of general-purpose household surveys. 
As one of the longest-standing multidisciplinary household surveys in the world, the SOEP has been conducted annually since 1984 by the German Institute for Economic Research (DIW Berlin) \citep{soepcore_v39}. 
It is a nationally representative longitudinal study that collects data from around 30,000 individuals in 22,000 households each year. 
The survey includes respondents aged 17 and older and provides rich individual- and household-level information on income, education, labour market activity, household structure, and demographics. 
This study uses the SOEP-Core sample, the central and most comprehensive module of the dataset \citep{berlin_diw_nodate}.

While such general purpose surveys like SOEP are not specifically designed to measure non take up, they have the advantage of covering both benefit receipt and the before mentioned characteristics needed to estimate eligibility \citep{mechelen_who_2017}.

We restrict our analysis to the period between 2007 and 2021, as this is the range for which we were able to consistently collect and harmonize the necessary statutory parameters from official BAföG regulations \cite{bafoeg_law}. 

This includes annual updates to base need rates, income allowances, asset thresholds, and other legally defined components relevant to BAföG eligibility and award determination.\footnote{
  See Appendix \ref{appendix:simulation-example} for an example of how these rules are applied.
}
Earlier years were excluded due to inconsistencies or incomplete availability of comparable legal documentation. 
By focusing on this window, we ensure that the simulation model is fully grounded in verifiable legal norms and reflects the actual policy environment faced by students during this time.

SOEP with its household structure allows us to link students to their parents, siblings and, in many cases, partners.
Using this data, we construct a dataset that includes detailed student-level and household-level characteristics. 
For students, we observe age, gender, federal state (Bundesland), household type, and income (if any). 
Parental information includes gross and net income, employment status, household structure, tax burdens, and relationship status. 

Using this data, we simulate the theoretical BAföG eligibility and award based on statutory rules in place during each year. 
This involves implementing a detailed microsimulation model that replicates the BAföG means test.

\subsection{Sample Description}
The final dataset contains 5,889 student-year observations, where each row represents a student in a given survey year. 
The sample spans the period from 2007 to 2022 and is derived from a harmonized student panel constructed using SOEP-Core data.

The panel is unbalanced due to individual variation in education length, dropout behavior, and survey response. 
While some students are observed for a single year, others are followed across multiple years of their educational trajectory.

Each observation contains detailed information on sociodemographic background, enrollment status, income and assets, housing situation, and reported BAföG receipt. Variables used in the simulation are consistently available for this period.

While the SOEP survey is nationally representative, this analytic subsample is conditional on survey respondents who were enrolled in education and met the inclusion criteria of the simulation pipeline. 
A descriptive overview of key variables is provided in Appendix ~\ref{appendix:microsimulation-pipeline}, Table~\ref{table:variable_dictionary}.

\subsection{Limitations}

Although the SOEP provides comprehensive socioeconomic data, such general purpose surveys come with some limitations. First of all, there are potential biases due to non response bias and undercoverage. Non response may also be correlated with non take up, which can distort estimates. Second of all, measurement errors can be a concern, especially in regards to income, asset reporting, and welfare receipt. Respondents may misreport their income or confuse the benefits they receive, leading to inaccurate estimates of eligibility and take up. Third of all, mismatches in the timing and definition of income used in surveys compared to what administrations use to assess eligibility can result in classification errors. Lastly, general purpose surveys often lack detailed information about reasons for non take up, making it difficult to distinguish between, for example, lack of awareness and administrative barriers \citep{mechelen_who_2017}. In addition to these general issues, there are also several limitations that are specific to the data available in this study and the way eligibility is modeled. These are outlined below.

\paragraph{Parental income coverage.} %TODO: Move to method and do sensitivity analysis 
Accurate parental income information is essential for constructing a credible BAföG means test. 
To ensure consistency in the simulation, the analysis is restricted to students for whom income data from both legal parents are available within the household files. 
This means that cases where one or both parents cannot be identified or linked within the dataset—such as due to absence, non-response, or household separation—are excluded from the simulation sample.

Because the SOEP dataset does not directly indicate BAföG eligibility, we construct theoretical eligibility through a microsimulation that mirrors the statutory rules of the Bundesausbildungsförderungsgesetz (BAföG) from 2002 to 2021 \citep{bafoeg_law,bafoeg20,bafoeg21,bafoeg22,bafoeg23,bafoeg24,bafoeg25,bafoeg26,bafoeg27,bafoeg28,bafoeg29}. 
The model implements the need calculation under §\,13 and the evolving allowance schedule under §\,25, incorporating adjustments from each amendment. 
While undocumented exemptions or special cases cannot be captured, the simulation provides a consistent, rule-based approximation across survey waves.

\paragraph{Modelling taxes.}
Full tax‑return simulations, as in \cite{herber_non-take-up_2019}, require detailed information (e.\,g.\ deductions, extraordinary expenses) that the SOEP does not always provide.  
We therefore approximate net parental income with the statutory bracket formulas of §\,32a EStG—updated for every year since 2002 \citep{estg_law,estg_2025,estg_2024,estg_2023,estg_2022,estg_2021,estg_2020,estg_2019,estg_2018,estg_2017,estg_2016,estg_2015,estg_2014,estg_2013,estg_2012,estg_2007,estg_2006,estg_lohninfo_2012}.  
\paragraph{Deviation from official outcomes.}
Even when closely following the legal rules, the simulation can differ from actual BAföG decisions due to missing household details or unobserved individual circumstances. 
Still, it offers a consistent and transparent benchmark for analysing take-up over time.

While many SOEP variables approximate administrative data, its still the most suitable dataset for examining the BAföG non-take-up rate. 
The eligibility measure used here reflects the legal framework and is sufficiently accurate for a systematic analysis of non-take-up and its underlying factors.


\paragraph{Income misreporting.}
When studying benefit take-up using a probit model, there are two common sources of bias that are important to keep in mind: measurement error in income and incorrect reporting of benefit receipt. Income is a key factor in determining eligibility for means-tested programs like BAföG, but it is often self-reported and can be measured with error. If the income recorded in the data does not reflect individuals' true income, some people may be wrongly classified as eligible or ineligible. This can lead to biased estimates and misleading conclusions about the factors that influence take-up \cite{pudney_impact_2001}.

\paragraph{Take-up misreporting.} Another issue is that benefit receipt itself is sometimes misreported. For example, people might say they did not receive aid when they actually did, or the other way around. This kind of misclassification makes it harder to accurately model the take-up decision. As shown in Pudney (2001), even relatively small errors in either income or benefit receipt can have a big impact on the results. This highlights the importance of being aware of potential measurement problems when interpreting the findings from probit models \cite{pudney_impact_2001}.

\paragraph{Asset data limitations and imputation.}
SOEP collects asset data only every five years, with available data for the years 2007, 2012, 2017, and 2022. 
This results in missing information for the intermediate survey waves. 
To address this, we imputed missing asset values using linear interpolation in both directions. 
While more advanced imputation methods exist, linear interpolation offers a straightforward and reasonable approach to create a continuous asset measure for the microsimulation.
