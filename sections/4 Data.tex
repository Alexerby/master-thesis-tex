%%%%%%%%%%%%%%%%%%%%%%%%%%%%%%%%%%%%%%%%%%%%%%%%%%%%%%%%%%%
%
% DATA
%
%%%%%%%%%%%%%%%%%%%%%%%%%%%%%%%%%%%%%%%%%%%%%%%%%%%%%%%%%%%


%%%%%%%%%%%%%%%%%%%%%%%%%%%%%%%%%%%%%%%%%%%%%%%%%%%%%%%%%%%


\section{Data}
To estimate non-take-up rates of welfare benefits, researchers typically rely on one or more of three data sources: administrative records, specially designed surveys, and general purpose household surveys. 
Each comes with its own trade-offs. 
Administrative data are accurate for tracking benefit receipt but usually lack information on those who do not apply. 
Special surveys can provide richer detail on eligibility and claiming behaviour, though they are costly and rarely implemented. 
General purpose surveys are more readily available and widely used in empirical research on non-take-up, even if they are not designed with this purpose in mind \citep{mechelen_who_2017}.

In line with much of the existing literature, this study relies on data from the German Socio-Economic Panel (SOEP), which falls into the third category of general-purpose household surveys. 
As one of the longest-standing multidisciplinary household surveys in the world, SOEP has been conducted annually since 1984 by the German Institute for Economic Research \citep{soepcore_v39}. 
It is a nationally representative longitudinal study that collects data from around 30,000 individuals in 22,000 households each year. 
The survey includes respondents aged 17 and older and provides rich individual- and household-level information on income, education, labour market activity, household structure, and demographics. 
This study uses the SOEP-Core sample, the central and most comprehensive module of the dataset. While general purpose surveys like SOEP are not specifically designed to measure non take up, they have the advantage of covering both benefit receipt and the before mentioned characteristics needed to estimate eligibility \citep{mechelen_who_2017}.

We restrict our analysis to the period between 2007 and 2021, as this is the range for which we were able to consistently collect and harmonize the necessary statutory parameters from official BAföG regulations \citep{bafoeg_law}. This includes annual updates to base need rates, income allowances, asset thresholds, and other legally defined components relevant to BAföG eligibility and award determination.\footnote{
  See Appendix \ref{appendix:simulation-example} for an example of how these rules are applied.
}
Earlier years were excluded due to inconsistencies or incomplete availability of comparable legal documentation. 
By focusing on this window, we ensure that the simulation model is fully grounded in verifiable legal norms and reflects the actual policy environment faced by students during this time.

SOEP with its household structure allows us to link students to their parents, siblings and, in many cases, partners.
Using this data, we construct a dataset that includes detailed student-level and household-level characteristics. 
For students, we observe age, gender, federal state (Bundesland), household type, and income (if any). 
Parental information includes gross and net income, household structure, tax burdens, and relationship status. 

Using this data, we simulate the theoretical BAföG eligibility and award based on statutory rules in place during each year. 
This involves implementing a detailed microsimulation model that replicates the BAföG means test.

\subsection{Sample Description}
The final dataset contains \( 5,889 \) student-year observations, where each row represents a student in a given survey year. As previously stated, the sample spans the period from 2007 to 2022 and is derived from a harmonized student panel constructed using SOEP-Core data. The panel is unbalanced due to individual variation in education length, dropout behavior, and survey response. 

While some students are observed for a single year, others are followed across multiple years of their educational trajectory. Each observation contains detailed information on sociodemographic background, enrollment status, income and assets, housing situation, and reported BAföG receipt. Variables used in the simulation are consistently available for this period.
%TODO: Elaborate here on the sample and ref the table 


While the SOEP survey is nationally representative, this analytic subsample is conditional on survey respondents who were enrolled in education and met the inclusion criteria of the simulation pipeline. 
A descriptive overview of key variables is provided in Appendix ~\ref{appendix:variable_dictionary}, Table~\ref{table:variable_dictionary}.

\begin{table}[H]
\footnotesize
\caption*{Descriptive statistics: non-take-up (NTU) and full eligible sample}
% \renewcommand{\arraystretch}{1.25}
\centering
\begin{tabular}{p{8cm}ccc|ccc}
\toprule
\textbf{Variable} & \multicolumn{3}{c|}{\textbf{NTU Sample}} & \multicolumn{3}{c}{\textbf{Full Sample}} \\
& Mean & Min & Max & Mean & Min & Max \\
\midrule
\multicolumn{7}{l}{\textbf{Main explanatory variable}} \\
Simulated BAföG Amount (EUR)       & 400    & 52    & 861   & 123    & 0    & 861   \\
\midrule
\multicolumn{6}{l}{\textbf{Demographics and Socioeconomic}} \\
Age                         & 23     & 18    & 34    & 23     & 18   & 41    \\
Female (\%)                 & 52     & n/a   & n/a   & 51     & n/a  & n/a   \\
Has partner (\%)            & 2      & n/a   & n/a   & 2      & n/a  & n/a   \\
Migration background (\%)   & 31     & n/a   & n/a   & 20     & n/a  & n/a   \\
\midrule
\multicolumn{6}{l}{\textbf{Institutional and Informational}} \\
Lives with parents (\%)          & 43     & n/a   & n/a   & 48     & n/a  & n/a   \\
Sibling claimed BAföG (\%)  & 35     & n/a   & n/a   & 30     & n/a  & n/a   \\
East background (\%)        & 17     & n/a   & n/a   & 21     & n/a  & n/a   \\
Parents highly educated (\%)& 25     & n/a   & n/a   & 43     & n/a  & n/a   \\
\midrule
\multicolumn{6}{l}{\textbf{Behavioural Predictors}} \\
Patience (0--10)            & 6.2    & 0     & 10    & 6.0    & 0    & 10    \\
Impulsiveness (0--10)       & 5.0    & 0     & 10    & 4.9    & 0    & 10    \\
Risk appetite (0--10)       & 5.3    & 0     & 10    & 5.1    & 0    & 10    \\
\bottomrule
\end{tabular}
\caption{\small{Descriptive statistics for two groups: the non-take-up (NTU) subsample, consisting of students classified as eligible but not receiving BAföG support, and the full sample of all theoretically eligible students. Means for binary and categorical variables are expressed as percentages. Min/Max values are not applicable for binary variables.}}
\caption*{\small{\textit{Note:}  The mean simulated BAföG in the full sample is lower because it includes all observations with a simulated amount of zero.}}
\label{tab:descriptive_ntu_all}
\end{table}


% \subsection{Limitations}
% Although the SOEP provides comprehensive socioeconomic data, such general purpose surveys come with some limitations. First, there are potential biases due to non response bias and undercoverage. Non response may also be correlated with non take up, which can distort estimates. Second, measurement errors can be a concern, especially in regards to income, asset reporting, and welfare receipt. Respondents may misreport their income or confuse the benefits they receive, leading to inaccurate estimates of eligibility and take up. Third, mismatches in the timing and definition of income used in surveys compared to what administrations use to assess eligibility can result in classification errors. Lastly, general purpose surveys often lack detailed information about reasons for non take up, making it difficult to distinguish between, for example, lack of awareness and administrative barriers \citep{mechelen_who_2017}. In addition to these general issues, there are also several limitations that are specific to the data available in this study and the way eligibility is modeled. These are outlined below.


