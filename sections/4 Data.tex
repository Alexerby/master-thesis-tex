\section{Data}
This study utilizes data from the German Socio-Economic Panel (SOEP), a nationally representative longitudinal survey conducted annually since 1984. The dataset provides individual and household-level information, including data on income, education, household composition, labor market behavior, and demographics. This study relies exclusively on SOEP-Core, the central and most comprehensive module.

We restrict our analysis to the period 2002–2022, following the introduction of the euro, to ensure consistency in income data. Our sample is limited to individuals who are currently enrolled in education, identified through a harmonized education variable in SOEP. This yields an unbalanced panel of students observed for varying numbers of years.

A key strength of the SOEP is its household structure, which allows us to link students to their parents, siblings and, in many cases, partners.
Using this data, we construct a comprehensive dataset that includes detailed student-level and household-level characteristics. For students, we observe age, gender, federal state (Bundesland), household type, and income (if any). Parental information includes gross and net income, employment status, household structure, tax burdens, and relationship status. Where applicable, we also observe sibling characteristics such as enrollment status, income, and household composition. Finally, for students who report having a partner, we include the partner’s income and household role.

Using this data, we simulate the theoretical BAföG eligibility and award based on statutory rules in place during each year. This involves implementing a detailed microsimulation model that replicates the BAföG means test.

\subsection{Sample Description}
The final dataset contains approximately \textbf{N = [UPDATE THIS WHEN WE KNOW]} student-year observations. Each row represents a student in a particular year.

There is substantial variation in how long individuals remain in the panel. Some students appear in only one wave, while others are observed over multiple years. This reflects differences in educational paths, dropout rates, and survey participation.

\subsection{Limitations}

Although the SOEP provides comprehensive socioeconomic data, certain limitations persist.

\paragraph{Parental income coverage.}
Parental income data are essential for constructing a credible BAföG means test. 
Requiring information for both parents would exclude a substantial share of observations, which would bias simulated eligibility downward. 
Students with income data for at least one parent are therefore retained. 
For single- or split-parent households, this reflects actual circumstances and in two-parent households it may understate available resources. 
However, the gain in sample size and representativeness offsets this downward bias. 
This approach maintains alignment with the target population while minimizing selection bias.

Because the SOEP dataset does not identify BAföG‑eligible respondents directly, we construct eligibility through a microsimulation that mirrors the legal rules of the Bundesausbildungsförderungsgesetz (BAföG) for the years 2002–2022 \citep{bafoeg_law,bafoeg20,bafoeg21,bafoeg22,bafoeg23,bafoeg24,bafoeg25,bafoeg26,bafoeg27,bafoeg28,bafoeg29}.  
The model implements the need calculation in §\,13 and the dynamic allowance schedule of §\,25, updating thresholds in line with each amendment act.  
Although undocumented exceptions cannot be fully accounted for, the model follows the statutory rules and ensures consistent application across survey waves.

\paragraph{Modelling taxes.}
Full tax‑return simulations, as in \cite{herber_non-take-up_2019}, require detailed information (e.\,g.\ deductions, extraordinary expenses) that the SOEP does not always provide.  
We therefore approximate net parental income with the statutory bracket formulas of §\,32a EStG—updated for every year since 2002 \citep{estg_law,estg_2025,estg_2024,estg_2023,estg_2022,estg_2021,estg_2020,estg_2019,estg_2018,estg_2017,estg_2016,estg_2015,estg_2014,estg_2013,estg_2012,estg_2007,estg_2006,estg_lohninfo_2012}.  
For years 2002–2006, where official schedules are unavailable, values are linearly interpolated.  
This approximation entails limited precision loss but enables consistent estimation across years, reproduces the primary tax burden, reflects statutory bracket reforms, and isolates income differences relevant for BAföG eligibility.  
Solidarity surcharge rates (5.5\% until 2020, with phased reductions thereafter) follow \citet{solzg_2018,solzg_2019,solzg_2023}.  
Church tax is applied at 9\% (8\% in Bavaria and Baden-Württemberg), conditional on reported church affiliation.

\paragraph{Deviation from official outcomes.}
Even when closely following the legal rules, the simulation can differ from actual BAföG decisions due to missing household details or unobserved individual circumstances. Still, it offers a consistent and transparent benchmark for analysing take-up over time.

While many SOEP variables approximate administrative data, its still the most suitable dataset for examining the BAföG non-take-up rate. 
The eligibility measure used here reflects the legal framework and is sufficiently accurate for a systematic analysis of non-take-up and its underlying factors.
