\section{Results}

\begin{table}[htbp]
\centering
\begin{tabular}{l@{\hspace{2em}}r@{\hspace{2em}}r}
\toprule
\textbf{Year} & \(\Pr(NTU=1\,|\,M=1)\) & \(\Pr(TU=1\,|\,M=0)\) \\
              & \textbf{Non-Take-Up} & \textbf{Beta Error} \\
\midrule
2007 & 60.6 & 13.6 \\
2008 & 63.5 & 17.0 \\
2009 & 60.9 & 18.6 \\
2010 & 60.9 & 17.7 \\
2011 & 53.7 & 16.1 \\
2012 & 51.4 & 18.9 \\
2013 & 50.0 & 15.9 \\
2014 & 55.1 & 16.1 \\
2015 & 64.0 & 12.6 \\
2016 & 56.5 & 12.3 \\
2017 & 62.5 & 10.0 \\
2018 & 63.9 & 15.2 \\
2019 & 67.5 & 11.6 \\
2020 & 63.7 & 13.5 \\
2021 & 66.6 & 12.2 \\
\midrule
\textbf{Average} & \textbf{59.7} & \textbf{14.9} \\
\bottomrule
\end{tabular}
\caption{Non-Take-Up and Beta Error Rates by Survey Year (\%). Non-take-up is the share of theoretically eligible students (\(M=1\)) who do not receive BAföG (\(R=0\)); beta error is the share of theoretically ineligible students (\(M=0\)) who do receive BAföG (\(R=1\)).}
\end{table}

% \begin{table}
%   % say what Error we choose to use, HC2 for example
%   % also say that we used SOEP weights 
% \end{table}

Our analysis finds that the non-take-up rate for BAföG among theoretically eligible students averages approximately around 60\% across survey years. This is 

Our results 

% \cite{frick_claim_2007} (67\%) % on soep data
% Data source: SOEP 
% Outcome: 67% non take up rate
% Social assistance in general 

% Bruckmeister, Kerstin and Wiemers Jurgen
% \cite{bruckmeier_new_2018}
% Data source: SOEP
% https://link.springer.com/article/10.1057/s41294-017-0041-5
% BAföG


% Table 3 
% 
% Social Assistance 43.1 \% 
% Housing Allowance or SA 62.8 \%
% SUpplementary child allowance 88.2\%
% The SOEP study, 36\% - 40\%  \cite{herber_non-take-up_2019} (on soep data)


% https://econpapers.repec.org/paper/iabiabfob/201305.htm
% 41-49% (basic social security benefits)
% \cite{RePEc:iab:iabfob:201305}
