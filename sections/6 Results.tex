\section{Results}

\subsection{Microsimulation Non-take-up}
Our microsimulation results indicate that the non-take-up-rate of BAföG, among theoretically eligible students ranged from approximately 50--70\% across the survey years 2007--2021, with an average of 60\%  (Table \ref{table:microsimulation-ntu}). \textcolor{red}{Is this the type of situation where we need to reference our own work?}

These estimates are broadly in line with previous findings on non-take-up of social benefits in Germany, which generally falls between 40--67\%, depending on the program and time period (see Table \ref{table:NTU-studies}). 
While our estimates are broadly consistent with prior research, they are noticeably higher than the 36--40\% non-take-up rate for BAföG reported by \cite{herber_non-take-up_2019}, who also use SOEP survey data, but for the period 2002--2013.

This discrepancy may be attributable to several factors, including differences in the estimation of theoretical eligibility. These factors include the specific SOEP variables used to capture income and reported BAföG receipt, the time periods under study (with our analysis covering 2007--2021, compared to \cite{herber_non-take-up_2019}, which covers 2002--2013), as well as other differences in the microsimulation design and modeling approach.

...

While there is some variation in non-take-up across years, NTU remains consistently quite high throughout the period. The rate fluctuates between a low of 50\% in 2013 and a high of around 68\% in 2019. Notably, the results show a decline from 2010 to 2013, which is then followed by a gradual upwards trend throughout the rest of the period. The increase in NTU in the years leading up to 2020 could potentially reflect behavioural or institutional factors such as changes in awareness, perceived complexity or attitudes toward debt, but it could also be partly driven by policy changes. Several BAföG reforms were introduced during this period, including increases in grant amounts and adjustments to income thresholds, which may have influenced both eligibility and perceived attractiveness of the program. Since the simulation accounts for these legal changes, the results capture not just behavioural responses but also how reforms may have affected take-up incentives over time.

\textcolor{red}{Maybe further elaborations belong in the discussion part?}

The second column in Table~\ref{tab:NTU_rates} shows the estimated beta error, which is the share of students who are classified as ineligible by the simulation but report receiving BAföG. On average, the beta error is about 15\% across the full period. This level of misclassification is in line with what other studies have found when working with survey data, where income reporting and timing mismatches are common issues \citep{frick_claim_2007}. While this level of beta error is not negligible, the simulation seems to capture eligibility status fairly well overall, even if some noise is unavoidable.

Taken together, the results suggest that a large share of eligible students do not take up BAföG, and that this has been the case fairly consistently over time. The high average non-take-up rate, around 60\%, points to persistent barriers such as lack of information or procedural hurdles. The financial attractiveness of BAföG may also be a factor. Although support amounts were increased at several points during the period, the maximum monthly payments have remained relatively modest compared to typical student living costs, especially in larger cities \textcolor{red}{(REMEMBER TO ADD SOURCE HERE)}. This could help explain why some students perceive the benefit as not worth the effort of applying. These findings underline the importance of outreach efforts and suggest that further reforms may be needed to make the program more accessible and appealing.

%TODO: Refer to the table of the distributions of reported bafög vs theoretical, see appendix for table.

\begin{table}[htbp]
\footnotesize
\centering
\begin{tabular}{l@{\hspace{2em}}r@{\hspace{2em}}r}
\toprule
\textbf{Year} & \textbf{Non-Take-Up}  &  \textbf{Beta Error}  \\
              & \(\Pr(\text{NTU} =1\,|\,\text{M} = 1)\) & \(\Pr(\text{TU} = 1\,|\,\text{M} = 0)\) \\
\midrule
2007 & 60.6 & 13.6 \\
2008 & 63.5 & 17.1 \\
2009 & 61.0 & 18.6 \\
2010 & 60.9 & 17.7 \\
2011 & 53.8 & 16.1 \\
2012 & 51.5 & 18.9 \\
2013 & 50.0 & 15.9 \\
2014 & 55.1 & 16.1 \\
2015 & 64.0 & 12.6 \\
2016 & 56.5 & 12.4 \\
2017 & 62.6 & 10.1 \\
2018 & 63.9 & 15.3 \\
2019 & 67.5 & 11.7 \\
2020 & 63.7 & 13.6 \\
2021 & 66.7 & 12.3 \\
\midrule
\textbf{Average} & \textbf{59.7} & \textbf{15.3} \\
\bottomrule
\end{tabular}
\caption{Non-Take-Up and Beta Error Rates by Survey Year (\%). Non-take-up is the share of theoretically eligible students (\(M=1\)) who do not receive BAföG; beta error is the share of theoretically ineligible students (\(M=0\)) who do receive BAföG.}
\caption*{\small{Notes: SOEP v39, 2007--2021, weighted with individual weights}}
\label{table:microsimulation-ntu}
\end{table}

\begin{figure}[htbp]
  \centering
  \includegraphics[width=0.75\linewidth]{ntu_bounds.png}
  \caption{Development of the probability of non-take-up from 2007--2021.}
  \label{fig:ntu_bounds_over_years}
\end{figure}



\subsection{Determinants of Non-take-up}

\textcolor{red}{Add some text here? Could refer back to the method chapter where we go over the control variables}

\subsubsection{Binary Choice Model}


\begin{table}
\caption{$\Pr(\mathrm{NTU} = 1 \mid \mathbf{X})$}
\renewcommand{\arraystretch}{1.25}
\footnotesize
\centering
\begin{tabular}{lllllllll}
\toprule
 & \multicolumn{4}{c}{Logit} & \multicolumn{4}{c}{Probit} \\
\cmidrule(lr){2-5} \cmidrule(lr){6-9}
 & Coef. & SE & AME & SE & Coef. & SE & AME & SE \\
\midrule
\multicolumn{9}{l}{\textbf{Main explanatory variables}} \\
Simulated BAföG amount$^{\circ}$ & -0.160*** & 0.058 & -0.029*** & 0.010 & -0.095*** & 0.034 & -0.030*** & 0.010 \\
\midrule
\multicolumn{9}{l}{\textbf{Controls: Demographics}} \\
Age & 0.099*** & 0.019 & 0.018*** & 0.003 & 0.058*** & 0.011 & 0.018*** & 0.003 \\
Female & -0.059 & 0.256 & -0.011 & 0.047 & -0.020 & 0.149 & -0.006 & 0.046 \\
Has partner & 1.429* & 0.810 & 0.262* & 0.149 & 0.874** & 0.444 & 0.271** & 0.137 \\
Direct Migration background & -0.700* & 0.378 & -0.128* & 0.068 & -0.419* & 0.219 & -0.130* & 0.067 \\
Indirect Migration background & -0.689** & 0.299 & -0.127** & 0.053 & -0.407** & 0.179 & -0.126** & 0.054 \\
\midrule
\multicolumn{9}{l}{\textbf{Controls: Household and Socioeconomic Background}} \\
Living at parents’ home & -0.019 & 0.270 & -0.004 & 0.049 & -0.008 & 0.160 & -0.002 & 0.050 \\
Sibling claimed BAföG before & -0.554* & 0.285 & -0.102** & 0.051 & -0.321* & 0.171 & -0.100* & 0.052 \\
East background & -1.253*** & 0.313 & -0.230*** & 0.052 & -0.749*** & 0.186 & -0.232*** & 0.054 \\
Parents are highly educated & -0.015 & 0.293 & -0.003 & 0.054 & 0.004 & 0.175 & 0.001 & 0.054 \\
\midrule
\multicolumn{9}{l}{\textbf{Controls: Behaviour}} \\
Patience & 0.030 & 0.065 & 0.006 & 0.012 & 0.015 & 0.040 & 0.005 & 0.012 \\
Impulsiveness & -0.039 & 0.068 & -0.007 & 0.012 & -0.021 & 0.042 & -0.006 & 0.013 \\
Risk Apetite & -0.022 & 0.037 & -0.004 & 0.007 & -0.014 & 0.021 & -0.004 & 0.007 \\
\midrule
McFadden Pseudo $R^2$ & \multicolumn{4}{l}{0.10} & \multicolumn{4}{l}{0.10} \\
Cox and Snell Pseudo $R^2$ & \multicolumn{4}{l}{0.11} & \multicolumn{4}{l}{0.11} \\
Nagelkerke Pseudo $R^2$ & \multicolumn{4}{l}{0.16} & \multicolumn{4}{l}{0.16} \\
Likelihood Ratio Test & \multicolumn{4}{l}{53.33 (p = 0.00)} & \multicolumn{4}{l}{53.20 (p = 0.00)} \\
Observations & \multicolumn{8}{l}{458} \\
\bottomrule
\end{tabular}
\caption*{Logit and Probit Coefficients and Average Marginal Effects}
\label{tab:logit_probit_results}
\caption*{\small{Notes: Significance levels: $^{{*}} p < 0.1$, $^{{**}} p < 0.05$, $^{{***}} p < 0.01$. Robust standard errors clustered at the student level. $^\dagger$ Indicates that the variable has been log-transformed. $\circ$ Indicates per 100 EUR.}}
\end{table}


\begin{table}
\caption{Instrumental Variable Estimation: $\Pr(\mathrm{NTU} = 1 \mid \mathbf{X}, \widehat{\text{Simulated BAföG}})$}
\renewcommand{\arraystretch}{1.25}
\footnotesize
\centering
\begin{tabular}{lllllllll}
\toprule
 & \multicolumn{4}{c}{IV Logit} & \multicolumn{4}{c}{IV Probit} \\
\cmidrule(lr){2-5} \cmidrule(lr){6-9}
 & Coef. & SE & AME & SE & Coef. & SE & AME & SE \\
\midrule
\multicolumn{9}{l}{\textbf{Main explanatory variables}} \\
Parental Income$^\dagger$ & -0.713* & 0.365 & -0.122** & 0.062 & -0.438** & 0.210 & -0.127** & 0.061 \\
Student income$^\dagger$ & 0.391 & 0.321 & 0.067 & 0.055 & 0.228 & 0.181 & 0.066 & 0.052 \\
Simulated BAföG amount (IV)$^{\circ}$ & -0.726** & 0.322 & -0.125** & 0.055 & -0.443** & 0.181 & -0.128** & 0.052 \\
\midrule
\multicolumn{9}{l}{\textbf{Demographics}} \\
Age & 0.389 & 0.504 & 0.067 & 0.087 & 0.252 & 0.282 & 0.073 & 0.082 \\
Age sq & -0.006 & 0.012 & -0.001 & 0.002 & -0.004 & 0.007 & -0.001 & 0.002 \\
Female & 0.624* & 0.379 & 0.107* & 0.065 & 0.364* & 0.219 & 0.105* & 0.063 \\
Has partner & 0.858 & 1.181 & 0.147 & 0.203 & 0.531 & 0.613 & 0.154 & 0.178 \\
Migration background & 0.034 & 0.428 & 0.006 & 0.073 & 0.014 & 0.254 & 0.004 & 0.074 \\
\midrule
\multicolumn{9}{l}{\textbf{Controls}} \\
Living at parents’ home & -0.386 & 0.615 & -0.066 & 0.106 & -0.246 & 0.350 & -0.071 & 0.101 \\
Sibling claimed BAföG before & -0.265 & 0.367 & -0.045 & 0.062 & -0.150 & 0.210 & -0.044 & 0.060 \\
East background & -1.474*** & 0.473 & -0.253*** & 0.073 & -0.895*** & 0.281 & -0.259*** & 0.075 \\
Parents are highly educated & 0.066 & 0.462 & 0.011 & 0.079 & 0.043 & 0.262 & 0.012 & 0.076 \\
\midrule
Pseudo $R^2$ & \multicolumn{4}{l}{0.1693} & \multicolumn{4}{l}{0.1708} \\
Observations & \multicolumn{8}{l}{230} \\
\midrule
First-stage $R^2$ & \multicolumn{8}{l}{0.47} \\
First-stage F-stat & \multicolumn{8}{l}{12.83} \\
\bottomrule
\end{tabular}
\caption*{IV Logit and IV Probit Coefficients and Average Marginal Effects}
\label{tab:iv_logit_probit_results}
\caption*{\small{Notes: Significance levels: $^{*} p < 0.1$, $^{**} p < 0.05$, $^{***} p < 0.01$. Robust standard errors clustered at the student level. $^\dagger$ Indicates log-transformed variables. Durbin-Wu-Hausman test residual coef (Logit) = -0.1668 (p = 0.156), residual coef (Probit) = -0.0930 (p = 0.167). $\circ$ Indicates per 100 EUR.}}
\end{table}



\textcolor{red}{Add some intro text here so we don't just go straight to explaining the results?}

...

\paragraph{Interpretation of Average Marginal Effects from the Probit Model.} All interpretations below are based on the average marginal effects (AMEs) from the Probit model presented in Table 3.

Student age is found to be significantly associated with NTU of BAföG. On average, each additional year of age increases the probability of NTU by 2.8 percentage points, holding all other variables constant. Similarly, student income has a significant effect, as a 100 EUR increase in gross monthly income is associated with a 1.2 percentage point increase in the probability of NTU, suggesting that higher-earning students may be less inclined to rely on BAföG support. Parental income matters as well. A 100 EUR increase in parental gross monthly income is associated with a 0.6 percentage point increase in the probability of NTU. \textcolor{red}{NOTE ALEX: LOOK BETTER INTO THIS. AME POSITIVE FOR STUDENTS BUT NEGATIVE FOR PARENTS, DON’T THINK IT IS INTERPRETED CORRECTLY HERE ABOVE, ALSO WHY IS IT DIFFERENT? DOES THAT MAKE SENSE?}

Other variables that have to do with family background were also found to have an effect. For example, having an older sibling who previously received BAföG reduces the probability of NTU by 9.6 percentage points on average, suggesting that familiarity with the system encourages take-up. Migration background is significant only for students with an indirect migration background (those born in Germany to foreign-born parents). For this group, the probability of NTU is 8 percentage points lower on average compared to those without a migration background.  Gender, partnership status, and household size do not appear to significantly affect NTU.

Students from East Germany are much less likely to forgo BAföG than their West German counterparts. The results show that having an East German background decreases the probability of NTU by about 25.9 percentage points, on average. This substantial difference could reflect regional variation in attitudes towards public support or perceived entitlement.

Furthermore, parental education seems to matter, particularly at the highest levels. Having at least one parent with a level 7 ISCED education (equivalent to a master’s degree \textcolor{red}{IS THIS CORRECT ALEX?}
) increases the likelihood of NTU by 47.8 percentage points. The effects of lower education levels are not statistically significant in the model.

Lastly, the estimated theoretical BAföG amount is negatively associated with NTU. A 100 EUR increase in the theoretical amount corresponds to a 2.3 percentage point decrease in the probability of NTU, suggesting that higher expected benefits increase take-up. \textcolor{red}{PLEASE CONFIRM THAT THIS MAKES SENSE ALEX}


