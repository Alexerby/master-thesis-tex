\section{Method} 

\paragraph{Non take up of welfare.}

We define non take up of welfare in line with Nelson and Nieuwenhuis (2019), as the circumstance when a person is eligible for welfare, but does not receive it. This is in line with terminology commonly used in literature on welfare take up rates. Non take up rate is thus the number of people who are eligible, but do not receive it, divided by the total number of people eligible. It is worth noting, however, that these situations often prove to be more complicated. In some cases, some individuals might receive welfare even though they aren’t eligible. This might for example happen due to fraud or errors made on the administrative level. This presents the issue of type two beta errors \citep{herber_non-take-up_2019, nelson_towards_2021}.

\paragraph{Tertiary non take up of welfare.}
In a 2017, Van Mechelen and Janssens defined tertiary non-take-up as a situation in which vulnerable individuals aren’t entitled to social welfare due to eligibility rules, even if they are in need of support. Tertiary non-take up as defined here can thus be considered a specific form of non take up. Originally, this concept was defined narrowly to include only those who are not eligible within a vulnerable group. However, some have argued for a broader definition that includes everyone in the vulnerable group that does not actually receive the welfare benefit, regardless of whether they are eligible or not \citep{goedeme_concept_2020, mechelen_who_2017}.

The concept of non tertiary take up relates directly to the concept of targeting efficiency, which can be divided into vertical- and horizontal efficiency. Vertical efficiency refers to how well a welfare system avoids giving support to individuals who fall outside the intended target group. In most cases, the target group is defined as those who are not considered economically vulnerable. It is essentially about minimising incorrect inclusion. One way to express this is through leakage, which is defined as the proportion of benefit recipients who are not a part of the reference population. The reference population is typically defined as people with low living standards, low income or other markers of economic vulnerability. In contrast, horizontal efficiency focuses on whether those within the target group actually receive the support. If many eligible or vulnerable individuals go without welfare benefits, the system is horizontally inefficient. This concept aligns closely with the broader definition of tertiary non take up, which includes all vulnerable individuals who are unable to access support, regardless of the reason \citep{goedeme_concept_2020, mechelen_who_2017}.


\subsection{Microsimulation of Theoretical BAföG Calculation}
After simulating statutory eligibility, we analyse the \textit{behavioural} non-take-up.
That is the probability that an eligible student refrains from taking up BAföG, even though they would theoretically, according to our statutory microsimulation be eligible to do so. Formally, we model 
\begin{equation}
  \mathrm{Pr}(A_{i} = 0 \mid T_{i} = 1,\; \mathbf{X}_{i}),
\end{equation}
where \( T_{i} = 1 \) denotes that the individual is \textit{theoretically} eligible according to our microsimulation. \( A_{i} \) is the \textit{actual} take-up indicator observed in the SOEP-Core dataset. 
\( A_{i} \) therefore denotes whether the individual \textit{de facto} did take up BAföG. 

Due to the nature of our dependent variable being binary, we employ a Probit model with standard-normal link
\begin{equation}
  y_{i}^* = \mathbf{X}_{i} + \varepsilon_{i}, 
  \quad 
  \varepsilon \sim \mathcal{N}(0,1), 
  \quad 
  y_{i} = \mathbf{1} \{ y_{i}^* > 0 \}
\end{equation}
where \( y_{i} = 1 \) flags the non-take-up of BAföG. 
% TODO: Explain here why we are not using Logit 
% and maybe we also should do that and compare outcomes?

The condition \( T_{i} = 1 \) restricts the dataset to students for whom the microsimulation yields positive statutory entitlement. Therefore, all the inferences pertains to this eligible sub-population whom are \textit{theoretically} (statutorily) eligible. 

The design matrix (\( \mathbf{X} \)) combines three partitions 
\begin{equation}
  \mathbf{X}_{i} = 
  \begin{bmatrix}
    \mathbf{Z}_{i} \mid \mathbf{B}_{i} \mid \mathbf{D}_{i}
  \end{bmatrix}
\end{equation}
where \( \mathbf{Z}_{i} \) (such as \textit{age, number of siblings}) contains continuous covariates, \( \mathbf{B}_{i} \) containing binary indicators (such as \textit{livings with parents})  and \( \mathbf{D}_{i} \) containing dummy vectors (such as \textit{sex, federal state, employment status}). 

The parameters are obtained by maximum-likelihood (ML) where the inference relies on the well known \textbf{Huber-White (HC0)} robust standard errors to guard against residual heteroskedasticity which often arises in large observational panels such as ours.

% What is meant by latent index?
As raw Probit coefficients represents shifts in the latent index\footnote{
With the latent index we refer to the continuous "score" which drives the true/false decision, but is not itself observed. That is the unseen propensity that the model assumes sits underneath every given binary outcome.
}
and are not directly comparable across covariates. To make these comparable, we report the average marginal effects (AMEs) in the results section. This translates the covariate changes into percentage-point differences in the non-take-up probability of the individuals.


