\section{Method} 

\paragraph{Non take up of welfare.}

We define non take up of welfare in line with Nelson and Nieuwenhuis (2019), as the circumstance when a person is eligible for welfare, but does not receive it. This is in line with terminology commonly used in literature on welfare take up rates. Non take up rate is thus the number of people who are eligible, but do not receive it, divided by the total number of people eligible. It is worth noting, however, that these situations often prove to be more complicated. In some cases, some individuals might receive welfare even though they aren’t eligible. This might for example happen due to fraud or errors made on the administrative level. This presents the issue of type two beta errors \citep{herber_non-take-up_2019, nelson_towards_2021}.

\subsection{Microsimulation of Theoretical BAföG Calculation}

\subsection{Construct Probit Model}
Construct a probit model in order to find the probably of not taking up the BAföG loan. 
With this model we will get a non-take-up rate or bafög loans.

