\section{Method}


\subsection{Individual Students' Monthly Requirement}
The monthly requirement the student is eligible for is contingent on the financial support 
the student is currently receiving from his or her family. 
Firstly, the student receives a constant \textbf{requirement rate} \( (\text{RR})  \),\footnote{Represents what BAföG calls "bedarfssätze" in German.} which is not contingent on the students' financial 
circumstances. To this basic amount the student receives requirements for \textbf{accommodation} (A), 
\textbf{health insurance} (HI), \textbf{long term care insurance} (LTCI) and an additional amount per the 
\textbf{number of children} (C) the student has. 
The requirement received for health insurance, long-term care insurance and accommodation is contingent on 
whether the parents are already providing these benefits. 
The total requirement the student will receive is therefore 

\begin{equation} \label{eq:total-requirement}
  \text{R} = \text{RR}  + \text{A} + \text{HI} + \text{LTCI} + \text{C}
\end{equation}
where
\begin{table}[H]
\small
\centering
  \begin{tabular}{lrr}
  \hline
  Variable & Provided by parents & Not provided by parents \\
  \hline
  A & 59 & 380 \\
  HI & 0 & 102 \\
  LTCI & 0 & 35 \\
  \hline
  \end{tabular}
\caption{Benefits contingent on parental provision (values in EUR).}
\end{table}



\subsection{Deductions from Requirement}
Parental and Student Income.

\begin{equation} \label{eq:parental-reduction}
  PR = \text{Parental Reduction} = 
  \begin{cases}
    0 & \text{if } E_5 \text{ or } (T_3 \text{ and } E_3) \\
    0 & \text{if } \text{Age}_{30} \\
    0.5 \times (\text{Parental Income} - \text{Exemption}) & \text{otherwise}
  \end{cases}
\end{equation}

\begin{itemize}
  \item \( E_5 \): Employed for 5 years after age 18
  \item \( T_3 \): Completed 3 years of vocational training
  \item \( E_3 \): Employed for 3 years after vocational training
  \item \( \text{Age}_{30} \): Older than 30 at the start of training
\end{itemize}

\begin{table}[H]
\small
\centering
  \begin{tabular}{lr}
  \hline
  Household Type & Exemption \\
  \hline
  Parents living together & 2,540 \\
  Parents live separately & 1,690 \\
  Spouse or Cohabiting Partner & 1,690 \\
  \hline
  \end{tabular}
\caption{Tax-free amount contingent on household type (values in EUR).} 
%NOTE: Add the fact that it's the penultimate year before approval period that is taken into consideration.
\end{table}

Let 
\[
SR = \text{Student Reduction} = 
\begin{cases} 
    0.5 \times \left( (\text{Income} - 556) + \max(0, \text{Assets} - 15,000)  \right) & \text{if Age}_{30} \\
    0.5 \times \left( (\text{Income} - 556) + \max(0, \text{Assets} - 45,000) \right) & \text{else}
\end{cases}
\]
\[
\text{BAföG}^\text{final}_{i} = \max(0, R - (PR + SR)) % Received BAföG amount
\]

% \[
% D_i = 
% \begin{cases}
% 1, & \text{if } \text{BAföG}_\text{final} = 0 \quad \text{(Ineligible for full requirement)} \\
% 0, & \text{otherwise}
% \end{cases}
% \]

Define a loss function out of the requirements and the deductions 
\begin{equation}
L(R, PR, SR) = R - (PR + SR)
\end{equation}


\subsection{Construction of Fuzzy RD}


Dummy variable for whether student loses any of his or her BAföG requirement
\begin{equation}
D_{i} = 
\begin{cases}
  1, \quad \text{if } L(R, PR, SR) > 0 \quad \text{(Some BAföG deduction occur)} \\
  0, \quad \text{if } L(R, PR, SR) = 0 \quad \text{(No deductions, full requirement)}
\end{cases}
\end{equation}

REVISE THIS ENTIRELY! Use a logit/probit for the first step then 
use these fitted values as regressor for second stage! 
Look into assumptions of both models and determine according to the 
characteristics of our data. 


First Stage (REVISE! Make into Logit/Probit)
\[ 
\text{BAFÖG}_{i} = \alpha + \beta D_{i} + \gamma X_{i} + \varepsilon_{i}
\]

Second Stage
\[ 
 \text{LabourSupply}_{i} = \delta + \lambda \widehat{\text{BAföG}}_{i} + \mu X_{i} + \nu_{i}
\]

\( \lambda \) coefficient for whether BAföG receipt reduces labour supply





