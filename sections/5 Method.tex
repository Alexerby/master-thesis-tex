\section{Method} 

\subsection{Bedarfsatz}
\begin{align}
  \text{Bedarfsatz} ={} &\ \text{Grundbedarf} + \text{Wohnpauschale} \notag \\
                         &+ \text{Krankenversicherung} + \text{Pflegeversicherung} \notag \\
                         &+ (\text{Kinderzuschlag} \times \text{Anzahl der Kinder})
\end{align}

\begin{equation}
  \text{Theoretical BAföG} = \text{Bedarfsatz} - \text{Anrechenbares Einkommen (student + parents)}
\end{equation}

% Find reductions
\subsection{Anrechenbares Einkommen}
\subsubsection{Parental Contribution (Elternbeitrag)}
\subsubsection{Student's own Income (Eigene Einkünfte)}


\subsection{Construct Probit Model}
Construct a probit model in order to find the probably of not taking up the BAföG loan. 
With this model we will get a non-take-up rate for bafög loans.

