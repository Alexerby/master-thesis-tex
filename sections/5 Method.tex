%%%%%%%%%%%%%%%%%%%%%%%%%%%%%%%%%%%%%%%%%%%%%%%%%%%%%%%%%%%
%
% METHOD
%
%%%%%%%%%%%%%%%%%%%%%%%%%%%%%%%%%%%%%%%%%%%%%%%%%%%%%%%%%%%

%TODO: Maybe reference table A1 in appendix, conditional per different types.
%TODO: Not mentioned how we dealt with measurement errors
%TODO: Did you compare simulated eligibility rates with official BAföG statistics?
%TODO: We need to do sensitivity analysis 
%TODO: How closely do simulated award amounts match reported distributions? Use appendix distribution
%TODO: Also use timeline in appendix for the comparison of theoretical to reported
%TODO: 

%%%%%%%%%%%%%%%%%%%%%%%%%%%%%%%%%%%%%%%%%%%%%%%%%%%%%%%%%%%

\section{Method} 
This study proceeds in two main steps. 
First, we perform a microsimulation to calculate theoretical BAföG eligibility and award amounts based on statutory rules applied to individual-level survey data. 
This simulation serves to identify who is entitled to student aid under the legal framework. 
%Microsimulation is particularly suited here because it accounts for individual differences in backgrounds and family situations, applying complex eligibility rules precisely.

%TODO: maybe not the place to be justifying the method, move the justification down to the microsimulation chapter

Second, we estimate two binary response models along with a linear probability model to analyze behavioural non-take-up.
That is, the likelihood that students eligible for BAföG according to the simulation nonetheless do not receive it. 
These models incorporate relevant socioeconomic and demographic factors to explore determinants of non-take-up beyond eligibility alone.

%%%%%%%%%%%%%%%%%%%%%%%%%%%%%%%%%%%%%%%%%%%%%%%%%%%%%%%%%%%
%
% Microsimulation
%
%%%%%%%%%%%%%%%%%%%%%%%%%%%%%%%%%%%%%%%%%%%%%%%%%%%%%%%%%%%
\subsection{Microsimulation of Theoretical BAföG Eligibility}

Microsimulation is a modeling approach that uses detailed individual-level data to apply policy rules directly to each observation. Instead of relying on representative averages, this method captures how specific characteristics and circumstances influence outcomes across the population \citep{klevmarken_microsimulation_2022}.

Given the complexity of BAföG eligibility rules, the need to simulate tax rules when calculating individual incomes, and the importance of capturing individual backgrounds, microsimulation is well suited to this analysis. Our model reconstructs eligibility and award amounts by applying both the statutory BAföG criteria and relevant tax regulations to detailed SOEP survey data. This approach ensures that net incomes for students and their parents, which are central to means-testing under BAföG, are calculated in accordance with actual legal requirements.

We identify eligible students based on statutory criteria, regardless of whether they actually received aid. Figure~\ref{fig:pipeline-overview} provides an overview of the steps in the microsimulation pipeline.

\begin{landscape}
\begin{figure}[htbp]
  \centering
  \scalebox{1}{
  \begin{tikzpicture}[node distance=2.5cm and 3cm, auto]

    % Existing nodes for student and parent flows
    \node[datasetbox] (soep) {SOEP-Core};
    \node[pipelinebox, right=of soep] (student) {Student selection};
    \node[pipelinebox, right=of student] (parent) {Parent selection};
    \node[pipelinebox, below=1.2cm of student] (grossincome) {Gross Income Student};
    \node[pipelinebox, below=1.5cm of grossincome] (subtract) {Subtract WKP, SI, TAX, CTAX};  % Moved under Gross Income
    \node[pipelinebox, below=1.5cm of subtract] (allowances) {Apply §23 Allowances};
    \node[pipelinebox, below=1.5cm of allowances] (excessincome) {Student excess income};

    % Parent income arrows and nodes
    \node[actionbox, below=1.5cm of parent] (bothparents) {Both parents in SOEP?};
    \node[actionbox, right=3cm of bothparents] (dropstudent) {Drop Student};
    \node[pipelinebox, below=2cm of bothparents] (grossincome2) {Gross Income};
    \node[pipelinebox, right=3cm of grossincome2, font=\footnotesize] (subtract_parent) {Subtract WKP, SI, TAX, CTAX};
    \node[pipelinebox, below=1.5cm of subtract_parent] (allowances_parent) {Apply §25 Allowances};
    \node[pipelinebox, below=1.5cm of allowances_parent] (excessincome_parent) {Parental excess income};

    % New asset branch on left side, separate flow
    \node[actionbox, below=1cm of soep, font=\footnotesize, align=center] (assets_identify) {Identify Student Assets};
    \node[pipelinebox, below=1.5cm of assets_identify] (assets_calc) {Calculate Total Assets};
    \node[pipelinebox, below=1.5cm of assets_calc] (allowances_29) {Apply §29 Allowance};
    \node[pipelinebox, below=1.5cm of allowances_29] (excess_assets) {Student excess assets};

    % Arrows for income flows
    \draw[arrow] (soep) -- (student);
    \draw[arrow] (student) -- (parent);
    \draw[arrow] (student.south) -- (grossincome.north);
    \draw[arrow] (grossincome.south) -- (subtract.north);  % Adjusted for new position
    \draw[arrow] (subtract.south) -- (allowances.north);
    \draw[arrow] (allowances.south) -- (excessincome.north);

    % Parent income arrows and decisions
    \draw[arrow] (parent.south) -- (bothparents.north);
    \draw[arrow] (bothparents.south) -- node[midway, left]{Yes} (grossincome2.north);
    \draw[arrow] (bothparents.east) -- node[midway, above]{No} (dropstudent.west);
    \draw[arrow] (grossincome2.east) -- (subtract_parent.west);
    \draw[arrow] (subtract_parent.south) -- (allowances_parent.north);
    \draw[arrow] (allowances_parent.south) -- (excessincome_parent.north);

    % Arrows for asset flow
    \draw[arrow] (student.south west) -- (assets_identify.north);
    \draw[arrow] (assets_identify.south) -- node[midway, left]{Yes} (assets_calc.north);  % Yes arrow for assets identified
    \draw[arrow] (assets_calc.south) -- (allowances_29.north);
    \draw[arrow] (allowances_29.south) -- (excess_assets.north);

  \end{tikzpicture}
}
  \caption{Flowchart for the calculation of student and parental income and asset adjustments. This process includes the identification of relevant income and asset sources, subtraction of allowable expenses, and the application of specific allowances under sections \textit{§23} and \textit{§25} of the BAföG law.}
  \caption*{
    \small {Notes:  
  \textbf{WKP}: Werbungskostenpauschale (Standard deduction for work-related expenses)  
  \textbf{SI}: Sozialversicherung (Social Security)  
  \textbf{TAX}: Tax (Income Tax)  
  \textbf{CTAX}: Church Tax}}
  \label{fig:pipeline-overview}
\end{figure}
\end{landscape}


The microsimulation model implements the legal rules and means-testing procedures set out in BAföG \citep{bafoeg_law} for the years 2007 to 2021. For each student in the SOEP-Core sample, the model uses information on income, assets, and household structure for the student as well as for others who are expected to help support them, such as parents, spouses, or siblings. The process involves calculating gross and net incomes according to the relevant tax rules, applying asset limits and deductions, and then determining eligibility and award amounts using the statutory formulas. In this way, the model aims to reflect how eligibility decisions are made in practice. The result is a theoretical eligibility and award amount for each student, which can be directly compared to self-reported BAföG take-up.

Unlike earlier studies, we focus solely on higher education students aged 18 and older. This ensures a consistent target group, as BAföG eligibility and structure differ between higher education and vocational training. 
Restricting the sample to adults also enhances data quality, as key variables tend to be reported more reliably by adults, whereas data for minors are often incomplete or reported by proxies \citep{soep_pgen_2025, soep_dtc_2005}.
From a policy perspective, university students are central to current debates on BAföG reform, particularly regarding debt aversion, housing costs, and equitable access. 
Focusing on this group enhances the relevance of our findings for ongoing policy discussions.
This approach enables a systematic assessment of the alignment between statutory entitlements and actual BAföG participation. 
For a detailed example of the full calculation logic applied to an actual observation, see Appendix~\ref{appendix:simulation-example}.

% Although some SOEP variables are approximations rather than exact administrative records, the dataset remains the most comprehensive source for studying BAföG non-take-up. 
% Our eligibility measure, grounded in statutory rules, offers sufficient accuracy to support a systematic analysis of take-up behaviour and its determinants.
%
% Notably, measurement error in reported income is the variable most likely to influence our simulation results. 
% To assess the robustness of our eligibility classification to such errors, we conduct a sensitivity analysis by introducing varying levels of normally distributed noise to the log-transformed income variables prior to recalculating theoretical BAföG entitlements. This approach helps evaluate how income misreporting might impact our findings.

%TODO: This chapter kind of ends in a weird way, where do we present the result from this sensitivity analysis for example?

% Maybe ref all our data sources, the statutory and soep
\subsubsection{Simulation Methodology}
The pipeline begins by assembling a harmonized dataset of student-level observations from SOEP-Core and manually harmonizing variables which are not harmonized already.
This is achieved by filtering for individuals who are enrolled in education, fall within the relevant survey years, and are at least 18 years old. 
To ensure a valid estimation of parental contributions, the dataset is further restricted to cases where income data from both legal parents are observable in the panel in order to reduce bias and ensure validity of estimated parental contributions.

\paragraph{Calculating Own Income.} 
This student-level dataset serves as the core of the simulation pipeline, providing the central structure to which all additional information is appended. 
It integrates sociodemographic variables, including sex, age, partnership status, number of siblings, number of children, household composition, and federal state of residence. 
Gross student income is also appended at this stage. 
Net student income is derived from gross values by applying year-specific rules for income tax, solidarity surcharge, and church tax where applicable, as well as standard deductions such as the Werbungskostenpauschale.\footnote{In accordance with §§~21–23 BAföG; see \citet{bafoeg_law}.} 
This net income will later be used to compute the student’s excess income as part of the BAföG need assessment.

\paragraph{Calculating Parental Contributions.}
Accurate parental income information is essential for constructing a credible BAföG means test. 
To ensure consistency in the simulation, the analysis is restricted to students for whom income data from both legal parents are available within the household files. 
This means that cases where one or both parents cannot be identified or linked within the dataset—such as due to absence, non-response, or household separation—are excluded from the simulation sample.

The simulation pipeline aggregates and evaluates parental income to estimate the expected contribution toward the student’s BAföG entitlement. 
For each student, the incomes of both legal parents—identified within the household and linked through SOEP family structure data—are retrieved and converted into annual net income. 
The net income of parents are retrieved the same way as the students own income (applying income tax, solidarity surcharge, standard deductions and church tax).

Net incomes from both parents are combined into a joint parental income measure. 
From this, the model subtracts statutory allowances,\footnote{Defined in \textit{§§}\,24–25 BAföG; see \citet{bafoeg_law}.} which vary depending on the number of parents, number of dependent children, and year-specific legal thresholds. 
Additional deductions are applied if the student has siblings who might also be eligible for support. 
The result is a measure of excess parental income, which feeds directly into the theoretical award calculation.


\paragraph{Measuring Assets.}
SOEP collects asset data only every five years (2007, 2012, 2017, and 2022), leaving gaps in between. 
To address this, we used linear interpolation to estimate missing values. 
Although asset values may not always change linearly, this straightforward method allows us to create a continuous asset measure suitable for the microsimulation.

The simulation applies an asset test to evaluate whether students possess financial resources exceeding statutory exemption thresholds. 
For each student, net assets are calculated by aggregating financial assets, real estate, business holdings, private insurances, vehicles, and other tangible property, minus reported debts. 
These total net assets are then compared against the exemption limits,\footnote{Set out in §\,29 BAföG; see \citet{bafoeg_law}.} which depend on age, partnership status, and number of dependent children. 
Any amount above the relevant threshold is considered excess and reduces the student's calculated need.

\paragraph{Modelling taxes.}
Full tax-return simulations, such as those in \cite{herber_non-take-up_2019}, require detailed information (e.g., deductions, extraordinary expenses) that the SOEP does not always provide.  
Therefore, we approximate net parental income using the statutory tax bracket formulas,\footnote{As defined in §\,32a EStG and updated annually since 2007; see \citet{estg_law,estg_2025,estg_2024,estg_2023,estg_2022,estg_2021,estg_2020,estg_2019,estg_2018,estg_2017,estg_2016,estg_2015,estg_2014,estg_2013,estg_2012,estg_2007,estg_2006}.} updated annually since 2007.

\paragraph{Need calculation and theoretical entitlement.}
In the final stage, the simulation model determines the student's funding need by combining the statutory base need, housing allowance, and health insurance supplement.\footnote{Specified in §\,13 BAföG; see \citet{bafoeg_law}.} 
From this total amount, any excess income of the student, their parents, and any excess assets are deducted. The remainder represents the student's theoretical monthly BAföG entitlement.

In its simplest form, this can be expressed as:

\begin{equation}
    \text{Entitlement} = \text{Need} - \text{Excess Income} - \text{Excess Assets}
\end{equation}

A positive entitlement does not automatically imply eligibility: the model also applies age-based eligibility criteria. 
Students are only considered theoretically eligible if they meet the age requirements defined in the law, typically under 30 for undergraduate studies and under 35 for graduate-level programs. 
The final output includes both the simulated monthly award and a binary eligibility flag, which are used for comparison against self-reported values in SOEP. 
Detailed examples of this calculation and relevant thresholds are provided in Appendix~\ref{appendix:simulation-example}.

\subsubsection{Measuring Non-Take-Up and Eligibility Classification Errors}

We define non-take-up of BAföG (NTU) in line with \cite{nelson_towards_2021} as the circumstance when a person is eligible for welfare but does not receive it. Conversely, take-up (TU) refers to eligible individuals who do receive BAföG. This terminology is commonly used in the literature on welfare take-up rates. 

The non-take-up rate is thus the number of eligible individuals who do not receive BAföG divided by the total number of eligible individuals. Formally, this is expressed as

\begin{equation}
\Pr(\text{NTU} = 1 \mid M = 1) = \frac{\sum_{i=1}^{N} \mathbf{1}\{R_i = 0 \ \text{and} \ M_i = 1\}}{\sum_{i=1}^{N} \mathbf{1}\{M_i = 1\}},
\end{equation}

where the indicator function is defined as

\begin{equation}
  \mathbf{1}\{\cdot\} =
  \begin{cases}
    1 & \text{if the condition inside the braces is true}, \\
    0 & \text{otherwise},
  \end{cases}
\end{equation}

and the binary variables \(R_i\) and \(M_i\) are defined as

\begin{equation}
\begin{aligned}
R_i &= \begin{cases}
1 & \text{if individual } i \text{ reports receiving BAföG in SOEP}, \\[4pt]
0 & \text{otherwise}
\end{cases} \\
M_i &= \begin{cases}
1 & \text{if individual } i \text{ is classified as eligible in our model}, \\[4pt]
0 & \text{otherwise}
\end{cases}
\end{aligned}
\end{equation}

A key limitation in simulating benefit take-up is the potential for misclassification. Even when the legal framework is closely replicated, differences between simulated and actual BAföG decisions can occur. These differences can arise from factors that are unobserved or not accurately measured in the survey data, as well as from administrative exceptions or complexities that the model does not fully capture \citep{frick_claim_2007, janssens_takemod_2022}. Because the analysis relies on self-reported survey data, both income and benefit receipt are subject to potential inaccuracies. Without access to administrative records, it is difficult to determine whether such cases reflect true model misclassification or reporting error.

Some individuals may receive BAföG despite being classified as ineligible by the model. These cases, known as false positives, or sometimes as beta errors or type II errors, often reflect the same reporting errors, missing information or administrative exceptions described above. Formally, the beta error rate is given by:

\begin{equation}
\Pr(\text{TU} = 1 \mid M = 0) = \frac{\sum_{i=1}^{N} \mathbf{1}\{R_i = 1 \ \text{and} \ M_i = 0\}}{\sum_{i=1}^{N} \mathbf{1}\{M_i = 0\}},
\end{equation}
where \( \mathbf{1}\{\cdot\} \) is the indicator function defined as
\[
\mathbf{1}\{\cdot\} =
\begin{cases}
1 & \text{if individual } i \text{ is ineligible but receives BAföG}, \\
0 & \text{otherwise}.
\end{cases}
\]

To assess the reliability of the simulation, we use two strategies. First, a sensitivity analysis is performed by adding normally distributed noise to the log-transformed income variables, to evaluate how income misreporting could affect non-take-up estimates. Second, we interpret the share of false positives (beta errors) as an indicator of potential misclassification. Overall model accuracy is measured as:

\begin{equation} \label{eq:accuracy_microsimulation}
	\text{Accuracy} = \frac{\text{True Positives} + \text{True Negatives}}{\text{Total number of cases}}
\end{equation}

While these measures cannot fully eliminate uncertainty, they provide a basis for assessing the robustness of the results and identifying areas where eligibility classification may be less reliable. The findings should be interpreted with caution, as prior research has shown that even modest levels of misreporting can substantially affect take-up analyses \citep{pudney_impact_2001}. 

Nonetheless, by systematically applying the legal framework to detailed individual data, the simulation offers a consistent and transparent benchmark for studying eligibility and take-up, helping to inform both academic research and policy discussions despite the inherent limitations of survey-based analysis.

%TODO: should we not present the findings from this accuracy measure here? Then we can better reason in the last paragraph that this is still a good model



%%%%%%%%%%%%%%%%%%%%%%%%%%%%%%%%%%%%%%%%%%%%%%%%%%%%%%%%%%%
%
% Binary Choice Model of Non-Take-Up
%
%%%%%%%%%%%%%%%%%%%%%%%%%%%%%%%%%%%%%%%%%%%%%%%%%%%%%%%%%%%
\subsection{Binary Choice Model}

After having identified theoretically eligible students and measured non-take-up (NTU) of BAföG, the focus now shifts to exploring the factors that drive this outcome. To do so, we estimate a binary choice model of the form

\begin{equation}
  \Pr(\mathrm{NTU} = 1 \mid \mathbf{X}) = F(\mathbf{X}^\top \boldsymbol{\beta}),
\end{equation}
where \( \mathrm{NTU} \) is a binary indicator for non-take-up, \( \mathbf{X} \) is the vector of covariates listed in Table~\ref{tab:descriptive_ntu_all}, and \( F(\cdot) \) is a link function that maps the linear index to a probability.\footnote{
This formulation can be motivated by a latent variable model: \( \mathrm{NTU}^* = \mathbf{X}^\top \boldsymbol{\beta} + \varepsilon \), where \( \mathrm{NTU}^* \) is an unobserved continuous variable and \( \varepsilon \) follows a standard normal distribution (probit) or a logistic distribution (logit). The observed binary outcome is \( \mathrm{NTU} = 1 \) if \( \mathrm{NTU}^* > 0 \), and \( \mathrm{NTU} = 0 \) otherwise.
}

We consider three specifications for \( F(\cdot) \), corresponding to commonly used binary choice models. 
The Probit model uses the cumulative distribution function (CDF) of the standard normal distribution, the Logit model uses the CDF of the standard logistic distribution, and the Linear Probability Model (LPM) uses a linear functional form.

\begin{table}[H]
\footnotesize
\centering
\label{tab:model_specifications}
\begin{tabular}{@{}lll@{}}
\toprule
\textbf{Model} & \textbf{Link Function \( F(\mathbf{X}^\top \boldsymbol{\beta}) \)} & \textbf{Estimation Method} \\ \midrule
Probit (Standard Normal CDF) & \( \Phi(\mathbf{X}^\top \boldsymbol{\beta}) \) & Maximum Likelihood \\
Logit (Standard Logistic CDF) & \( \Lambda(\mathbf{X}^\top \boldsymbol{\beta}) \) & Maximum Likelihood \\
Linear Probability Model & \( \mathbf{X}^\top \boldsymbol{\beta} \) & Ordinary Least Squares (OLS) \\
\bottomrule
\end{tabular}
\caption{Specifications of the Binary Choice Models}
\end{table}

While the SOEP is a panel dataset with information collected from individuals across multiple years, the sample size of eligible students is too small to support reliable longitudinal analysis. For this reason, all models are estimated on a pooled cross-section of theoretically eligible individuals (\( n = 458 \)). The sample is restricted to observations with complete parental income information.

%\paragraph{Interpretation and Comparison.} 
The logit and probit models are nonlinear estimators of the probability of non-take-up, based on an underlying latent index framework. 
Their coefficients represent changes in the unobserved latent variable and are not directly interpretable in terms of changes in the probability of the observed outcome. 
To facilitate interpretation, we report average marginal effects (AMEs), which approximate the average change in the probability of non-take-up associated with a one-unit increase in each covariate.\footnote{The AME for covariate \( X_k \) is computed as the sample analogue of the population moment \( \mathbb{E}[f(\mathbf{X}^\top \boldsymbol{\beta}) \cdot \beta_k] \), where \( f(\cdot) \) is the derivative of the link function \( F(\cdot) \). Specifically,
\[
    \widehat{\text{AME}}_k 
    = \frac{1}{n} \sum_{i=1}^n f(\mathbf{X}_i^\top \hat{\boldsymbol{\beta}}) \cdot \hat{\beta}_k
    = \mathbb{E}_n \left[ f(\mathbf{X}^\top \hat{\boldsymbol{\beta}}) \cdot \hat{\beta}_k \right],
\]
where \( \mathbb{E}_n[\cdot] \) denotes the empirical expectation over the sample.}

In practice, logit and probit models typically yield similar qualitative results, with differences largely driven by the tails of the distribution. We report both for completeness. The LPM then serves as a linear benchmark. While it has some limitations in modeling binary outcomes, such as producing predicted probabilities outside the \([0,1]\) range, it provides a simple and transparent interpretation, as coefficients directly represent the marginal effects.

\subsubsection{Key Predictors of Non-Take-Up}
Our model includes a set of explanatory variables informed by existing literature and the institutional context of BAföG. 
These variables capture a range of factors that may influence students' decisions about applying for financial aid, including demographic characteristics, socioeconomic background, and selected behavioural traits. 

In particular, the model includes variables for risk preferences, migration background, prior family experience with BAföG, and regional socialization. These variables are selected to capture differences in access to information, institutional trust, familiarity with the application process, and attitudes toward public support. While our primary goal is to estimate the association between these factors and non-take-up among eligible students, the inclusion of conceptually relevant variables also helps to account for sources of heterogeneity that might otherwise bias the estimated effects of financial incentives.

The rationale for including several of these variables is further elaborated below, drawing on prior research and theoretical considerations. 

\paragraph{Age, sex, and partnership status.}
First, we control for basic demographics that may influence financial aid decisions. Age is included as it can correlate with students’ life circumstances or progression through higher education, potentially influencing their financial situation or familiarity with administrative procedures. Sex (female) is included to account for possible gender-related patterns in financial decision-making or access to information. Partnership status is also considered, as having a partner might affect household resources, support, or information exchange that could be relevant for BAföG take-up.

\paragraph{Migration background.}
Recent research has found that students from migrant households often have lower levels of financial literacy, which can create additional challenges in understanding eligibility requirements and navigating the BAföG application process \citep{Tsegay_2024}. In the analysis, both direct and indirect migration backgrounds are taken into account to consider whether varying levels of familiarity with the German aid system are related to take-up. Including migration background in this way helps to identify potential structural or informational barriers that could contribute to lower BAföG uptake.

\paragraph{Living with parents.}
Living with parents can lower students’ living expenses and provide additional support, which may reduce both the need for financial aid and the likelihood of applying for BAföG. Including this variable helps account for differences in household circumstances that could influence non-take-up among eligible students.

\paragraph{Sibling prior experience with BAföG.}
Having a sibling who has previously received BAföG may provide students with practical knowledge and guidance during the application process. \cite{herber_non-take-up_2019} suggest that this type of family experience can help reduce informational and procedural barriers, potentially encouraging eligible students to apply. Including this variable allows us to examine whether informal support within families influences take-up decisions.

\paragraph{East German socialisation.}  
A student’s social and regional background may influence their attitudes toward government intervention due to historical and cultural context. Research by \cite{alesina_good-bye_2007} finds that individuals with an East German background tend to be more supportive of redistribution and more trusting of public assistance, even years after reunification. In our analysis, we include a regional indicator for current residence in East Germany as a proxy for these socialisation effects. Its significance in our model suggests that this proxy captures meaningful differences in how students perceive and respond to BAföG.

\paragraph{Parental education.}  
Parental education often shapes students’ educational and financial behaviours. We flag students whose parents hold at least a bachelor’s degree as coming from higher-educated households to test whether this background affects BAföG application patterns.

\paragraph{Risk appetite, impulsiveness and patience.}  

To understand why some eligible students do not apply for BAföG, we consider their willingness to take risks. Even though BAföG offers relatively safe and generous terms, risk-averse students might still avoid taking on any debt. Including this measure allows us to test whether risk preferences can help explain patterns of non-take-up. \cite{herber_non-take-up_2019} also include risk attitudes in their analysis, primarily to assess whether risk aversion affects take-up or interacts with traits such as impatience. Although they do not find a strong effect, they suggest that risk attitudes remain relevant for understanding application behavior. Based on this reasoning, risk attitudes are included as a control in our model as well.

In addition to risk appetite, we account for impulsiveness and patience using scales constructed by SOEP based on several survey questions. These behavioral traits might influence how students make financial decisions, so including them helps capture additional sources of variation in BAföG take-up.

% By controlling for these demographic characteristics, we aim to ensure that our analysis more accurately isolates the effects of other variables of interest.



