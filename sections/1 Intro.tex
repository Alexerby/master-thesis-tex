\section{Introduction} \label{sec:intro}

%TODO: maybe establish somewhere why we chose Germany

%TODO: We need to make it more like a summary of the paper

In Germany, access to higher education is primarily supported through BAföG, a means-tested financial aid program designed to provide equal opportunities for students from low-income families. In recent years, however, the proportion of eligible students receiving support has gradually declined. This trend has received increasing attention in both academic research and policy discussions, with ongoing debate about whether BAföG is achieving its intended goals \citep{gwosc_krisenbewaltigung_2022, meier_bafog_2024}.

Several structural factors contribute to this persistent issue. One key factor is the complexity of the application process, which increases the informational burden on students. These information costs can weigh against the potential benefits of applying, especially since BAföG support rates are widely considered too low and have consistently lagged behind students’ actual cost of living. Research has also identified behavioural barriers, such as limited attention and self-control, which are exacerbated by the complicated application process. These intertwined economic, informational and behavioural factors create a challenge for policymakers seeking to enhance BAföG’s reach and impact \citep{staack_von_2017, bhargava_psychological_2015, bolland_information_nodate}.

This paper investigates the factors influencing non-take-up of BAföG among eligible students in Germany, focusing on economic, informational, and behavioural aspects. The analysis is limited to higher education students aged 18 and above, as their circumstances and relevant policy considerations differ from those of younger students and individuals in vocational training. Existing research often adopts a broader focus and typically covers periods only up to 2013, leaving a gap in understanding recent trends. This is particularly important given the further decline in take-up rates in recent years. By analysing microdata from 2007 to 2021, this study adds to the understanding of non-take-up among eligible students, taking into account more recent years when take-up rates have continued to decline.

%TODO: data beyind 2013? or updated estimates have not been provided in over a decade or something like that

To address this, we develop a microsimulation model based on the German Socio-Economic Panel (SOEP), a large, nationally representative household panel survey providing detailed demographic, educational, income, and attitudinal data for the period 2007 to 2021.
This approach enables accurate estimation of individual BAföG eligibility using current legal regulations applied to income and household composition, overcoming limitations of self-reported aid receipt. 
By comparing eligibility with reported receipt, we measure non-take-up and use three specifications of binary choice models to analyze who takes up the entitlement and not. 
In this model we incorporate covariates for expected subsidy amounts, informational constraints, and proxies for attitudes toward government support. 
This framework allows isolation of the relative importance of various factors and their evolution over time.

%TODO: Too detailed text descibing SOEP to have in the introduction. Also is it necessary to use the abbreviation?
%TODO: "overcoming limitations of self-reported aid receipt" feels like it's phrased in a weird way
%TODO: "By comparing eligibility with reported receipt, we measure non-take-up and use three specifications of binary choice models to analyze who takes up the entitlement and not." is it supposed to be, who's eligible and who's not?

Our findings reveal that despite recent policy efforts to expand support, non-take-up remains substantial. 
Key drivers include the persistent mismatch between aid levels and actual living costs, since need rates, deductions, and other relevant factors are adjusted according to BAföG regulations rather than being indexed to inflation or cost-of-living increases. 
Additionally, significant informational barriers and stigma associated with welfare benefits further exacerbate this gap.
Importantly, non-take-up is unevenly distributed: students from lower socio-economic backgrounds or with weaker access to information are more likely to forgo applications, indicating compounded financial and psychological obstacles.

This study updates estimates of BAföG non-take-up rates using a detailed, longitudinal microsimulation approach. 
It considers economic, informational, and attitudinal factors together to better understand the reasons for non-take-up.
Our results indicate that the non-take-up rate among theoretically eligible students ranged from approximately 50\% to 70\% during the period 2007--2021, with an average rate around 60\%. 
These rates are substantially higher than some previous estimates, highlighting persistent and significant barriers to take-up. 
The findings emphasize the limitations of incremental financial aid adjustments and point to the need for more fundamental reforms. 
Such reforms could include simplifying application procedures and improving information dissemination.
Taken together, these insights support ongoing policy debates aiming for more universal and inclusive support models that enhance educational equity and ensure public funds effectively reach students in need.

%TODO: the beginning of the aboe paragraph feels repetitive
%TODO: feel like we also mmaybe need to mmention some numbers from the logit/probit here? Also we should go over those numbers and double check that they're all correct in the results secttion tomorrow

