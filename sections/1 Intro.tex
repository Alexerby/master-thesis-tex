\section{Introduction} \label{sec:intro}

%TODO: maybe establish somewhere why we chose Germany

%TODO: We need to make it more like a sumary of the paper

Access to higher education in Germany is supported primarily through a means-tested student financial aid, known as BAföG, which aims to promote equal educational opportunities by providing financial aid to students from lower-income backgrounds. 
However, despite its pivotal role in social mobility, there is growing concern in both policy circles and academic research that BAföG is falling short of its objectives. 
In particular, the proportion of students eligible for support who actually receive it, known as the take-up rate, has declined in recent years. 

This non-take-up phenomenon limits the effectiveness of BAföG in reducing educational inequality and raises questions about the adequacy of BAföG \citep{gwosc_krisenbewaltigung_2022, meier_bafog_2024}. 
Several structural factors contribute to this persistent issue. 
Most notably, BAföG support levels have lagged behind the rising cost of living, including steep increases in housing expenses and general inflation.
Furthermore, the complexity of the application process and psychological barriers such as stigma around welfare receipt further discourage eligible students from applying. 
These intertwined economic, informational, and behavioural factors create a challenge for policymakers seeking to enhance BAföG’s reach and impact \citep{meier_bafog_2024, staack_von_2017}.

%TODO: including this part: "steep increases in housing expenses and general inflation" feels a bit too detailed for an introduction.

%TODO: stigma not be significant, see lit review

This paper investigates what drives the non-take-up of BAföG among eligible students in Germany, and how economic, informational, and attitudinal factors influence application behaviour. 
% While previous research has documented take-up rates and identified some barriers qualitatively, comprehensive empirical analyses quantifying determinants of non-take-up at the individual level remain scarce, especially using recent data. 
To our knowledge, no study has examined these determinants with data extending to 2021, leaving a gap in understanding amid recent economic developments and policy changes.

%TODO: data beyind 2013? or updated estimates have not been provided in over a decade or something like that

To address this, we develop a microsimulation model based on the German Socio-Economic Panel (SOEP), a large, nationally representative household panel survey providing detailed demographic, educational, income, and attitudinal data for the period 2007 to 2021.
This approach enables accurate estimation of individual BAföG eligibility using current legal regulations applied to income and household composition, overcoming limitations of self-reported aid receipt. 
By comparing eligibility with reported receipt, we measure non-take-up and use three specifications of binary choice models to analyze who takes up the entitlement and not. 
In this model we incorporate covariates for expected subsidy amounts, informational constraints, and proxies for attitudes toward government support. 
This framework allows isolation of the relative importance of various factors and their evolution over time.

%TODO: Too detailed text descibing SOEP to have in the introduction. Also is it necessary to use the abbreviation?
%TODO: "overcoming limitations of self-reported aid receipt" feels like it's phrased in a weird way
%TODO: "By comparing eligibility with reported receipt, we measure non-take-up and use three specifications of binary choice models to analyze who takes up the entitlement and not." is it supposed to be, who's eligible and who's not?

Our findings reveal that despite recent policy efforts to expand support, non-take-up remains substantial. 
Key drivers include the persistent mismatch between aid levels and actual living costs, since need rates, deductions, and other relevant factors are adjusted according to BAföG regulations rather than being indexed to inflation or cost-of-living increases. 
Additionally, significant informational barriers and stigma associated with welfare benefits further exacerbate this gap.
Importantly, non-take-up is unevenly distributed: students from lower socio-economic backgrounds or with weaker access to information are more likely to forgo applications, indicating compounded financial and psychological obstacles.

This study updates estimates of BAföG non-take-up rates using a detailed, longitudinal microsimulation approach. 
It considers economic, informational, and attitudinal factors together to better understand the reasons for non-take-up.
Our results indicate that the non-take-up rate among theoretically eligible students ranged from approximately 50\% to 70\% during the period 2007--2021, with an average rate around 60\%. 
These rates are substantially higher than some previous estimates, highlighting persistent and significant barriers to take-up. 
The findings emphasize the limitations of incremental financial aid adjustments and point to the need for more fundamental reforms. 
Such reforms could include simplifying application procedures and improving information dissemination.
Taken together, these insights support ongoing policy debates aiming for more universal and inclusive support models that enhance educational equity and ensure public funds effectively reach students in need.

%TODO: the beginning of the aboe paragraph feels repetitive
%TODO: feel like we also mmaybe need to mmention some numbers from the logit/probit here? Also we should go over those numbers and double check that they're all correct in the results secttion tomorrow

