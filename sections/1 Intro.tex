\section{Introduction} \label{sec:intro}

%TODO: maybe establish somewhere why we chose Germany

%TODO: We need to make it more like a summary of the paper

%TODO: Petra said to Jack to reference our main sources in the introduction, should go over the sources we are currently referencing. Also we have not referred to Herber yet (our main source), need to fix that. Also currently only referring to sources in the first two paragraphs, also look into that maybe.

In Germany, access to higher education is primarily supported through BAföG, a means-tested financial aid program designed to provide equal opportunities for students from low-income families. In recent years, however, the proportion of eligible students receiving support has gradually declined. This trend has received increasing attention in both academic research and policy discussions, with ongoing debate about whether BAföG is achieving its intended goals \citep{gwosc_krisenbewaltigung_2022, meier_bafog_2024}.

Several structural factors contribute to this persistent issue. One key factor is the complexity of the application process, which increases the informational burden on students. These information costs can weigh against the potential benefits of applying, especially since BAföG support rates are widely considered too low and have consistently lagged behind students’ actual cost of living. Research has also identified behavioural barriers, such as limited attention and self-control, which are exacerbated by the complicated application process. These intertwined economic, informational and behavioural factors create a challenge for policymakers seeking to enhance BAföG’s reach and impact \citep{staack_von_2017, bhargava_psychological_2015, bolland_information_nodate}.

This paper investigates the factors influencing non-take-up of BAföG among eligible students in Germany, focusing on economic, informational, and behavioural aspects. The analysis is limited to higher education students aged 18 and above, as their circumstances and relevant policy considerations differ from those of younger students and individuals in vocational training. Existing research often adopts a broader focus and typically covers periods only up to 2013, leaving a gap in understanding recent trends. This is particularly important given the further decline in take-up rates in recent years. By analysing microdata from 2007 to 2021, this study adds to the understanding of non-take-up among eligible students, taking into account more recent years when take-up rates have continued to decline.

%TODO: data beyond 2013? or updated estimates have not been provided in over a decade or something like that

To address this, we develop a microsimulation model using data from the German SOEP panel survey. The model collects income and background information for relevant individuals and applies the appropriate tax deductions in effect at each point in time, before assessing BAföG eligibility. By comparing simulated eligibility with reported receipt, we measure non-take-up and use three binary choice model specifications to analyze the factors associated with take-up among eligible students. Our analysis incorporates covariates for expected subsidy amounts, informational constraints, and proxies for attitudes toward government support, among others. This framework allows us to isolate the relative importance of different factors and examine how their influence has changed over time.

Our results indicate that the non-take-up rate among theoretically eligible students ranged from approximately 50\% to 70\% during the period, averaging at around 60\%, which is an increase from what earlier studies have found. In examining individual years, a slight upward trend can be detected in the most recent years. Importantly, non-take-up is unevenly distributed. Students with weaker access to information are found to be less likely to take up the support, which confirms the role of informational barriers in non-take-up. Similarly, students from lower socio-economic backgrounds are also less likely to receive support, further reinforcing the concern that BAföG may not be fully achieving its objective of ensuring equal access to education.

%TODO: feel like we also maybe need to mention some numbers from the logit/probit here? Also we should go over those numbers and double check that they're all correct in the results section tomorrow

The findings emphasize the limitations of incremental financial aid adjustments and point to the need for more fundamental reforms. Given the relevance of informational barriers, such reforms could involve simplifying the application procedures and providing clearer, more widespread information to students. Taken together, these insights contribute to ongoing policy discussions regarding more inclusive and effective student aid models that promote educational equity and better target public funds to students in need.

%TODO: "By comparing eligibility with reported receipt, we measure non-take-up and use three specifications of binary choice models to analyze who takes up the entitlement and not." is it supposed to be, who's eligible and who's not?
