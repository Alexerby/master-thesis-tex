\section{Introduction} \label{sec:intro}

Non-Take-Up of BAföG: Recent Developments and Increasing Awareness

In recent years, the issue of non-take-up (NTU) of student financial aid, specifically BAföG (Bundesausbildungsförderungsgesetz), has attracted significant attention in Germany. While BAföG remains the primary policy tool for ensuring fair access to higher education, there is increasing skepticism about whether it effectively achieves its core objectives of unlocking educational potential and ensuring equal opportunities. This skepticism stems primarily from the steady decline in the number of eligible students who actually make use of BAföG (Gwosć and van der Beek, 2022; Meier, Thomsen and Wolf, 2024).

Several recent studies highlight significant structural shortcomings of the current BAföG system as central reasons behind the rising NTU rates. Among the main points that critics emphasize is that BAföG allowances have not kept up with the actual living costs, particularly amid rising housing prices and increasing inflation (Meier et al., 2024). As of recent, some policy amendments have been made (such as the 29th BAföG amendment in 2024) in an attempt to incrementally adjust the support rates, but these attempts have to some extent been met with criticism for being inadequate. For example, student expenditures for 2024 are projected to average around 969 EUR per month, significantly more than the maximum BAföG rate of 812 EUR, even after these adjustments (Meier et al., 2024).

Additionally, the complexity and lack of transparency in BAföG application procedures further discourage students from applying, which in turn increases NTU rates. Many students find the eligibility criteria unclear, which creates uncertainty about whether or not they qualify for financial support. But these barriers aren’t just administrative, they’re also psychological. Feelings of stigma, or discomfort with applying for what is perceived as a welfare type of benefit, play a role in that as well. So, while economic factors matter, the decision not to apply is often also shaped by information deficits and psychological factors (Gwosć and van der Beek, 2022; Keller, Staack and Tschaut, 2017).

This growing awareness of BAföG’s limitations has resultet in demands for more thorough impact evaluations and deeper structural reforms. As Meier et al. (2024) point out, despite annual public expenditure of approximately 3 billion EUR, there is still a lack of solid empirical assessment of the effectiveness and efficiency of the system. This knowledge gap has a limiting effect on the ability to make evidence based policy decisions, leading instead to reforms shaped by political compromise instead of objective analysis. Some researchers, like Gwosć and van der Beek (2022), argue that changing this might even require a shift from the current needs-based model (Fürsorgeprinzip), which requires students to prove their financial hardship, to a more inclusive, universal support model (Versorgungsprinzip). International comparisons indicate that participation rates in countries with universal support systems are substantially higher, which suggests that systemic change could lead to improvements for Germany in this regard.

The urgency to address NTU is becoming ever more apparent, especially in the context of recent economic challenges, such as rising housing costs and inflation, factors that disproportionately impact students. Critics of BAföG argue that the issues extend beyond just low support rates. They point to structural issues, particularly the mix of grants and loans. For many students, especially those from economically disadvantaged families, the prospect of going into debt is enough to deter them from applying at all (Keller et al., 2017; Gwosć and van der Beek, 2022).
\citep{gwosc_krisenbewaltigung_2022}.

In summary, the increasing NTU of BAföG has become a central issue in the wider policy discussion on accessibility of education. Addressing these problems will require targeted policy reforms, clearer and more transparent application processes and greater reliance on evidence based analysis. Implementing changes like that is crucial to ensure that public funds for student aid are used in an equitable and effective manner, and thus helping to maximize educational opportunities for students from all backgrounds.
