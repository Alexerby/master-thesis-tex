\section{Introduction}

In Germany, access to higher education is primarily supported through BAföG, a means-tested financial aid program designed to provide equal opportunities for students from low-income families. In recent years, however, the proportion of eligible students receiving support has gradually declined. This trend has received increasing attention in both academic research and policy discussions, with ongoing debate about whether BAföG is achieving its intended goals \citep{gwosc_krisenbewaltigung_2022, meier_bafog_2024}.

Several structural factors contribute to the persistent issue of low take-up. 
A central obstacle is the complexity of the application process, which imposes substantial informational burdens on students. 
These burdens can deter applications, particularly given that BAföG support rates are widely viewed as insufficient relative to students’ actual cost of living. 
In addition to these economic and informational barriers, existing research points to behavioural factors that may further discourage take-up, especially in the face of procedural complexity. 
Together, these intertwined factors pose a significant challenge for policymakers aiming to expand BAföG’s reach and effectiveness \citep{staack_von_2017, bhargava_psychological_2015, bolland_information_nodate}.

This paper investigates the factors influencing non-take-up of BAföG among eligible students in Germany, focusing on economic, informational, and behavioural aspects. 
The analysis focuses on higher education students aged 18 and above, whose circumstances and relevant policy considerations differ from those of younger students and individuals in vocational training.
There has been little focused research on the non-take-up of BAföG among eligible students, and recent trends over the past decade have yet to be systematically examined.
This is particularly important given the further decline in take-up rates in recent years. 
By analysing microdata from 2007 to 2021, this study adds to the understanding of non-take-up among eligible students, taking into account more recent years.


To address this, we develop a microsimulation model using data from the German SOEP panel survey \citep{soepcore_v39}. 
The model collects income and background information for relevant individuals and applies the appropriate statutory rules in effect at each point in time, before assessing BAföG eligibility. 
By comparing simulated eligibility with reported receipt, we measure non-take-up and use three binary choice model specifications to analyze the factors associated with take-up among eligible students. 
Our analysis incorporates covariates for expected subsidy amounts, informational constraints, and proxies for attitudes toward government support, among others. 

Our results indicate that the non-take-up rate among theoretically eligible students ranged from approximately 50\% to 70\% during the period, averaging around 60\%, which is higher than estimates reported in earlier studies. A slight upward trend in non-take-up can be observed in recent years. 
Students with weaker social capital are less likely to take up the support, highlighting the role of informational barriers in non-take-up. 
Existing concerns remain that BAföG may not fully achieve its objective of ensuring equal access to education.


Our findings emphasize the limitations of incremental financial aid adjustments and point to the need for more fundamental reforms. In recent years, BAföG support rates have not kept pace with the rising cost of living for students. Future reforms could include ensuring that support levels are regularly adjusted to reflect actual student expenses, alongside simplifying application procedures and improving the accessibility of information. Taken together, these insights contribute to ongoing policy discussions about how to design student aid models that are both adequate and accessible, in order to promote educational equity and ensure that support reaches those who need it most.

