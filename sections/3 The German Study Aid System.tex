\section{Student Aid Systems: International and German Context}
\label{section:student-aid-systems}

\cite{schwarz_study_2004} address a worldwide trend that took place in the 1970s, where higher education expanded from being elite to becoming more accessible for the general population. This trend was also prominent in Europe, and brought about an exceptional increase in student enrolment. With these developments, European nations had to take greater responsibility for supporting students financially. In doing so, different nations adopted different methods and shaped their support systems in different ways. It is thus not possible to claim a common European method of student financial aid.

\subsection{International Student Aid Models} \label{subsection:international-sstudent-aid-models} 

When comparing the proportion of students receiving public financial aid across OECD countries, Germany proves to be one of the nations with the lowest coverage. Other OECD nations with low coverage include Austria, Croatia and Switzerland. What these countries tend to have in common is that public financial aid is narrowly targeted, focusing on students from socio-economically disadvantaged backgrounds. At the other end of the spectrum are countries such as Sweden, Denmark, Australia, and the United States, where a much larger share of students benefit from public financial aid \citep{oecd_education_2024}.

These differences capture the two main design principles when it comes to public student funding, i.e. the welfare principle and the provision principle. The latter one aligns with Germany's BAföG, which is narrowly targeted to specific groups. The welfare principle on the other hand applies to systems where public aid benefits a larger share of the student population (i.e. the Nordic countries) \citep{gwosc_krisenbewaltigung_2022}.

Gwosć and van der Beek (2022) conduct an empirical comparison between two groups of countries in Europe, where one group consists of countries that apply the welfare principle and the other of countries that follow the provision principle. The results indicate that countries that follow the latter have a significantly greater share of students that receive public aid on average, and that public aid accounts for a greater proportion of the receiver’s overall earnings. Moreover, the probability of students in these countries reporting serious financial issues is lower.

The authors also look into what can roughly be translated into participation equity - which refers to proportionate representation of different social groups in higher education. They find that, to a slight extent, the countries that follow the provision principle do worse than the countries that follow the welfare principle \citep{gwosc_krisenbewaltigung_2022}.

The nations also differ when it comes to distribution and administration of financial aid. On one hand, there are the Nordic countries where financial aid is managed by national authorities, and on the other hand there are countries like Germany, where local authorities bear the responsibility \citep{schwarz_study_2004}.

There is however a common trend in that the main form of student aid across Europe is in the form of grants, i.e. monetary public support that is not to be repaid. These grants can cover general cost of living or more particular needs like tuition fees or accommodation. Student loans are the alternative (or even complement), where public monetary support needs too be repaid, typically after ones studies have been completed. Typically, such loans come with low interest rates (although there are exemptions to this), often lower than interest rates on private loans in a given country \citep{schwarz_study_2004}.

\section{The German Study Aid System} 
\label{section:the-german-study-aid-system}

The \textbf{Federal Training Assistance Act} (g. Bundesausbildungsförderungsgesetz, BAföG) is a public student aid system supplied by the \textbf{Federal Ministry of Education and Research} (g. Bundesministerium für Bildung und Forschung). BAföG is designed to financially support students, with the primary aim to promote equal opportunities in the education system and unlock educational potential \citep{meier_bafog_2024}. The eligibility criteria for the loan is therefore relatively strict to make sure that only students who are genuinely in need of the loan have access to it.

BAföG replaced various federal state regulations pertaining to attendance at vocational and general education schools that had already been consolidated in the Education Support Act (AföG) in 1969. It also replaced other regulations that were centred around providing assistance to gifted students and were not associated with any legal entitlement. Thus, this was the first law to establish a "legally enforceable right to educational funding” \citep{staack_von_2017}.

Since the beginning, BAföG has adhered to the principle of subsidiarity in its basic conception, which is in line with traditional welfare policies in Germany. That is the principle that smaller local units perform their own tasks and a central authority only provides help when necessary, i.e. has a subsidiary function. In the context of BAföG, this means that first, in order to finance their studies, students must rely on their own income and assets. The next larger social units to be held accountable are spouses or partners, and next the parents become financially responsible. Only after these social units have been exhausted do the mechanisms of the welfare state intervene through BAföG \citep{staack_von_2017}.

BAföG was introduced in 1971 in the form of a 100 percent non-repayable grant and was generally very successful with almost half (44.6\%) receiving the subsidy---a level never reached again. 
The early success of BAföG came with significant financial burdens for both the federal states and the federal government, prompting a series of reforms—particularly in response to the energy crises of the 1970s. In 1974, a mandatory loan component was introduced, and by 1977, the loan share had increased even further. By the 1980s, BAföG underwent a complete overhaul, the so-called "BAföG Kahlschlag" or BAföG clear-cut, transforming it into a fully subsidised loan program. As a result, the grant portion was eliminated, significantly reducing BAföG's appeal. Due to the rapid decline of students applying for BAföG it was once again overhauled in the 1990s, making it so that half the funding was in form of a grant and half in form of an interest-free loan -- the structure of which is still in force today. Since 2005, the share of the funding that is provided as a grant has been just over 50\% and the share provided as a loan just under 50\%. Further changes were made in 2001 when a repayment cap of 10.000 EUR was introduced  \citep{lost_geschichte_2025, staack_von_2017, meier_bafog_2024}.

BAföG continues to face low interest among students today, with one of its major issues being that students are not utilizing it, as it lacks appeal (see table \ref{table:bafoeg_support} and figure \ref{figure:bafoeg_support}). 

\subsection{Declining uptake and its reasons}
\label{subsection:declining-uptake}

Some sources claim that since the introduction of BAföG in 1971, the proportion of students receiving financial aid has fallen from around 50\% to around 15\% as of 2023. Thereof, around 50\% received full funding \citep{meier_bafog_2024}. According to the German student survey (g. Die Studierendenbefragung in Deutschland) it even went as low as 12.9\% in the summer semester of 2021 (ADD A FOOTNOTE EXPLAINING THE SCHOOL TERMS IN GERMANY) \citep{kroher_studierendenbefragung_2023}.

The survey supports the finding that BAföG uptake has declined over time, and furthermore found that uptake has consistently been lower among students with higher parental education levels, which likely reflects income based ineligibility. Furthermore, the study shows that as parental educational attainment increases, funding rates decrease. At the same time, a general decline in funding rates can be observed over time across all groups \citep{kroher_studierendenbefragung_2023}.
...

WRITE SOMETHING ABOUT THE REASONS BEING A) THAT PEOPLE ARE NOT APPLYING AS MUCH (BECAUSE IT IS DIFFICULT ETC.) AND B) THE RULES HAVE BECOME STRICTER SO FEWER PEOPLE ARE ELIGIBLE (OR SOMETHING LIKE THAT).

...

The 22nd German student survey estimated that just about 80\% of students did not apply for BAföG during the term it was conducted. It also found that around 66\% of students had never applied for BAföG. A further 10\% of students had also never received BAföG, but had submitted an (unsuccessful) application. ADD SOMETHING HERE TO CONNECT THOSE TWO. BAföG funded students can then be further divided into those who received BAföG dependent on parental support (9.3\%) and those who received BAföG independently of parental support (3.4\%) \citep{kroher_studierendenbefragung_2023}.

Data was also collected on reasons students had for not applying for BAföG. The most commonly stated reason was thinking that parental income was too high, but 73.7\% of non-applicants claimed that as one of the reasons. The second most common reason stated was thinking that own income/assets were too high, with 29.7\% of non-applicants claiming that. The third most common reason stated was fear of debt (21.4\%). Also notable is that 7.8\% of non-applicants stated that expected funding amount would be too low as one of the reasons \citep{kroher_studierendenbefragung_2023}.

The study also looked further into the group of former recepients. Out of the students who were not receiving BAföG because their funding had expired, just over 20\% reported that they had exceeded the maximum funding period or standard period of study. The study claims that this is not a surprising result since it is a well documented fact that most students in Germany exceed the standard period of study. Only just over a third of all students complete their studies within the standard period of study, with even almost a quarter exceeding the standard period of study by more than two semesters. Completing a degree within the allocated time is rather the exception than the norm \citep{kroher_studierendenbefragung_2023}.

Other important reasons for the expiration of funding include that a student's own income or assets or those of relatives are too high (23.5\% and 18.6\%, respectively), that a student has changed their field of study (15.6\%), or that the required credit scores could not be provided (12.2\%). However, over 15\% of students whose funding has expired also state that they do not want to incur further debt or that the expected funding amount is too low (14.3\%). The current course of study not being eligible for funding or exceeding age limit play a minor role, accounting for around 7\% and 4\% respectively \citep{kroher_studierendenbefragung_2023}.

...

In this context, it’s worth noting that the average reported student expenses have been higher than the maximum support rate according to reports from the social survey from 2000 to 2017. The 2022 reform (the 27th BAföG amendment) brought the maximum support rate above the average reported student expenses for the first time \citep{meier_bafog_2024}.

...

In this context, it is however worth noting that declining financial aid rates do not inevitably indicate a deterioration of the situation, although that is also a possibility that can’t be ruled out. Over the last two decades, income per capita has increased significantly in Germany. This, as well as demographic trends, could at least partly explain the drop in proportion of students in need of financial aid. This can be viewed as a general prosperity effect. Furthermore, the share of students receiving financial aid is also affected by various behavioural factors, including fluctuations in demand for education and the social composition of prospective students. This proportion does thus not accurately reflect how many students are actually in need of financial aid nor how many of them receive such aid \citep{meier_bafog_2024}.


\subsection{Institutional Design and Policy Instruments} 
\label{section:institutional-design-and-policy-instruments}

\subsubsection{The Income Exemption Threshold and the Support Rate}
\label{subsection:the-income-exemption-threshold-and-the-support-rate}
BAföG uses two main tools in order to achieve its central objectives, the so-called Freibetragsgrenzen, which is the income exemption threshold, and the so-called Bedarfssätze, which is the support rate \citep{meier_bafog_2024}.

\paragraph{Support Rates.}
In order to determine the support rates, three main reference points are used: 1) the development of basic social security benefits (”citizens allowance” or Bürgergeld), 2) the development of consumer prices, which reflects the increase in general costs of living, and 3) the specific living expenses of students, which are surveyed every four to five years in the Sozialerhebung (”the social survey”). Additionally, the financial situation of the federal government is taken into account in order to ensure that increases in support rates, income exemption thresholds and social allowances are fiscally feasible \citep{meier_bafog_2024}.

\paragraph{Income Exemption Threshold.}
In reviewing and determining the income exemption threshold, net income (g. arbeitnehmereinkommen) is primarily used as a reference indicator in BAföG reports. The income exemption threshold is also normatively determined by the legislature, i.e. the decision is not based on a fixed rule or an automatic formula, but on policy choices \citep{meier_bafog_2024}.


\paragraph{Interactions Between the Income Exemption Threshold and the Support Rate.}
These tools are interconnected, as raising the income exemption threshold increases the number of students eligible for BAföG. In addition to that, raising the income exemption threshold makes it so that those who previously received only partial support become eligible for more support, and thus raises the amounts granted to this group of students \citep{meier_bafog_2024}.

According to the law on BAföG the support rate and the income exemption threshold must be reviewed every two years and, if necessary, re-determined accounting for the cost of living, general economic conditions, trends in income levels and wealth development \citep{meier_bafog_2024}.

\subsubsection{Two Loan Repayment Models}
\label{subsection:loan-repayment-plan}
% Comparing the two alternatives: TBRLs vs. ICL

CONSIDE

The two main ways of financing studies in higher education (HE) using a loan is to either use a traditional 
\textbf{time-based repayment loan} (TBRL) which is of the same style as standard "mortgage-loans" 
where the principal is amortized on a fixed reimbursement schedule.

The alternative to the TBRL plans are \textbf{income contingent loans} (ICL), where the 
principal you are allowed to borrow and the rate at which you amortize the principal is contingent on your financial status. The principal you are allowed to borrow and the 
rate at which you amortize the principal is contingent on your earned 
and capital income. In some systems, as in the German one, the household earnings 
and capital gains are also considered when applying for the income contingent BAföG loan.

% Pros with ICL 
An obvious benefit of the ICL loan structure is that it eliminates the likelihood 
of defaulting on your debt, as the reimbursement period 
(and rate of amortisation) is adapted to the individual (or household) income. 
Time based repayments are known to overburden the poorer part of the population which decides to 
educate themselves. For instance, among the 20\% of the poorest graduates in South Korea and United States 
almost all students have a repayment burden exceeding 100\% of their income 
\citep{chapman_income-contingent_2022}. 
Income contingent loans therefore provides an insurance against low income for the debtor and promotes social benefits such as mobility and human capital formation.

% Cons with ICL 
However, there are some important drawbacks to income-contingent loans that policymakers should consider when implementing them. 
One concern is that, as long as the borrower has an outstanding balance, the loan effectively acts as a marginal tax on income above the repayment threshold. 
This can potentially reduce the borrower’s incentive to work more, as higher earnings lead to higher repayments. 
If borrowers respond by working less to avoid steeper repayment rates, the loan will be repaid more slowly, increasing the cost borne by the creditor — in this case, the state.
Whether this is an actual problem is yet to be investigated further, but has been
shown that for instance in the UK's income contingent repayment plan to not 
be an actual problem \citep{britton_income_2020}. 

% Germany and ICL
In the case of BAföG, this issue is less pronounced, as the repayment system is only partially income-contingent. Repayments are capped at 130 EUR per month, and after a maximum of 77 installments (a total of 10,010 EUR), any remaining debt is forgiven \citep{studentenwerk_bafog}.

%TODO: 
% -Income Thresholds and Means Testing
% - Write about the problem of students not applying for bafög




