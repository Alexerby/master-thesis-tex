\section{The German Study Aid System} 
\label{section:the-german-study-aid-system}


- Write about how pupils are funding their education 



\subsection{Federal Training Assistance Act ("Bundesausbildungsförderungsgesetz") } 
\label{subsection:federal-training-assistance-act}
The \textbf{Federal Training Assistance Act} (DE: \textbf{Bundesausbildungsförderungsgesetz}) is a student loan supplied by the \textbf{Federal Ministry of Education and Research} (DE: \textbf{Bundesministerium für Bildung und Forschung}). 

The loan was introduced in 1971 in the form of a 100 percent grant and was generally very successful with almost half (44.6\%) receiving the subsidy---a level never reached again. 
The early success of BAföG came with significant financial burdens for both the federal states and the federal government, prompting a series of reforms—particularly in response to the energy crises of the 1970s. In 1974, a mandatory loan component was introduced, and by 1977, the loan share had increased even further. By the 1980s, BAföG underwent a complete overhaul, transforming it into a fully subsidised loan program. As a result, the grant portion was eliminated, significantly reducing BAföG's appeal.
Due to the rapid decline of students applying for BAföG it had to once again be overhauled in the 1990s and BAföG was now half a grant, and half a loan, where the loan part has zero interest -- the structure of which is still in force today \citep{lost_geschichte_2025}.

BAföG continues to face low interest among students today, with one of its major issues being that students are not utilizing it, as it lacks appeal (see table \ref{table:bafoeg_support} and figure \ref{figure:bafoeg_support}). 

\subsubsection{Two Loan Repayment Models}
\label{subsection:loan-repayment-plan}
% Comparing the two alternatives: TBRLs vs. ICL
The two main ways of financing studies in higher education (HE) is to either use a traditional 
\textbf{time-based repayment loan} (TBRL) which is of the same style as "mortgage-loans" 
where the principal is amortized on a fixed reimbursement schedule.

The alternative to the TBRL plans are \textbf{income contingent loans} (ICL), where the 
principal you are allowed to borrow and the rate at which you amortize the principal is contingent on your financial status. The principal you are allowed to borrow and the 
rate at which you amortize the principal is contingent on your earned 
and capital income. In some systems, as in the German one, the household earnings 
and capital gains are also considered when applying for the income contingent BAföG loan.

% Pros with ICL 
An obvious benefit of the ICL loan structure is that it eliminates the likelihood 
of defaulting on your debt, as the reimbursement period 
(and rate of amortisation) is adapted to the individual (or household) income. 
Time based repayments are known to overburden the poorer part of the population which decides to 
educate themselves. For instance, among the 20\% of the poorest graduates in South Korea and United States 
almost all students have a repayment burden exceeding 100\% of their income 
\citep{chapman_income-contingent_2022}. 
Income contingent loans does therefore provide an insurance against low income for the debtor and promotes social benefits such as mobility and human capital formation.

% Cons with ICL 
However, there are some important drawbacks to income-contingent loans that policymakers should consider when implementing them. 
One concern is that, as long as the borrower has an outstanding balance, the loan effectively acts as a marginal tax on income above the repayment threshold. 
This can potentially reduce the borrower’s incentive to work more, as higher earnings lead to higher repayments. 
If borrowers respond by working less to avoid steeper repayment rates, the loan will be repaid more slowly, increasing the cost borne by the creditor — in this case, the state.
Whether this is an actual problem is yet to be investigated further, but has been
shown that for instance in the UK's income contingent repayment plan to not 
be an actual problem \citep{britton_income_2020}. 

% Germany and ICL
In the case of BAföG, this issue is less pronounced, as the repayment system is only partially income-contingent. Repayments are capped at 130 EUR per month, and after a maximum of 77 installments (a total of 10,010 EUR), any remaining debt is forgiven \citep{studentenwerk_bafog}.


%TODO: 
% -Income Thresholds and Means Testing
% - Write about the problem of students not applying for bafög

\subsubsection{Reforms (any reforms relevant?)} 
\label{subsection:reforms}

\subsection{Training Loans ("Bildungskredit")}

