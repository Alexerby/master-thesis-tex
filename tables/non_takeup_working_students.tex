\begin{table}[htbp]
\centering
\begin{tabular}{l@{\hspace{2em}}r@{\hspace{2em}}r@{\hspace{2em}}r}
\toprule
\textbf{Year} & \textbf{Non-Take-Up} & \textbf{Take-Up Rate} & \textbf{Beta Error} \\
              & \(\Pr(\text{NTU} = 1 \mid \text{M} = 1)\) & \(\Pr(\text{TU} = 1 \mid \text{M} = 1)\) & \(\Pr(\text{TU} = 1 \mid \text{M} = 0)\) \\
\midrule
2007 & 71.4 & 28.6 & 5.1 \\
2008 & 62.5 & 37.5 & 17.9 \\
2009 & 72.2 & 27.8 & 13.2 \\
2010 & 72.7 & 27.3 & 13.4 \\
2011 & 75.0 & 25.0 & 10.1 \\
2012 & 72.7 & 27.3 & 12.8 \\
2013 & 64.5 & 35.5 & 9.5 \\
2014 & 57.1 & 42.9 & 7.8 \\
2015 & 76.5 & 23.5 & 13.6 \\
2016 & 75.0 & 25.0 & 15.2 \\
2017 & 69.2 & 30.8 & 9.6 \\
2018 & 70.6 & 29.4 & 13.3 \\
2019 & 78.1 & 21.9 & 9.8 \\
2020 & 75.6 & 24.4 & 12.2 \\
2021 & 71.4 & 28.6 & 11.7 \\
\midrule
\textbf{Average} & \textbf{71.5} & \textbf{28.5} & \textbf{11.6} \\
\bottomrule
\end{tabular}
\caption{Non-Take-Up, Take-Up, and Beta Error Rates by Survey Year (\%) among Working Students. Non-take-up is the share of theoretically eligible working students (\(M=1\)) who do not receive BAföG. The take-up rate is the share who do receive it. Beta error is the share of ineligible working students (\(M=0\)) who nevertheless receive BAföG.}
\caption*{\small{Notes: SOEP v39, 2007--2021, restricted to students with monthly income > 200 EUR; weighted with individual weights.}}
\label{table:ntu-working-students}
\end{table}
