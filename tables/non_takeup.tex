\begin{table}[htbp]
\small
\centering
\begin{tabular}{l@{\hspace{2em}}r@{\hspace{2em}}r@{\hspace{2em}}r}
\toprule
\textbf{Year} & \textbf{Non-Take-Up} & \textbf{Take-Up Rate} & \textbf{Beta Error} \\
              & \(\Pr(\text{NTU} = 1 \mid \text{M} = 1)\) & \(\Pr(\text{TU} = 1 \mid \text{M} = 1)\) & \(\Pr(\text{TU} = 1 \mid \text{M} = 0)\) \\
\midrule
2007 & 58.0 & 42.0 & 15.9 \\
2008 & 60.4 & 39.6 & 19.8 \\
2009 & 58.0 & 42.0 & 19.4 \\
2010 & 53.0 & 47.0 & 21.9 \\
2011 & 51.4 & 48.6 & 17.8 \\
2012 & 48.4 & 51.6 & 20.8 \\
2013 & 51.0 & 49.0 & 11.9 \\
2014 & 52.6 & 47.4 & 14.7 \\
2015 & 72.1 & 27.9 & 12.8 \\
2016 & 58.2 & 41.8 & 15.5 \\
2017 & 64.0 & 36.0 & 7.0 \\
2018 & 68.8 & 31.2 & 11.5 \\
2019 & 69.0 & 31.0 & 11.1 \\
2020 & 69.3 & 30.7 & 13.0 \\
2021 & 69.9 & 30.1 & 17.6 \\
\midrule
\textbf{Average} & \textbf{60.0} & \textbf{40.0} & \textbf{15.8} \\
\bottomrule
\end{tabular}
\caption{\small{Non-Take-Up, Take-Up, and Beta Error Rates by Survey Year (\%). Non-take-up is the share of theoretically eligible students (\(M=1\)) who do not receive BAföG. The take-up rate is simply the complement, i.e., the share of eligible students who do receive BAföG \((1 - \Pr(\text{NTU} = 1 \mid M = 1))\). Beta error is the share of ineligible students (\(M=0\)) who nevertheless receive BAföG.}}
\caption*{\small{Notes: SOEP v39, 2007--2021, weighted with individual weights}}
\label{table:microsimulation-ntu}
\end{table}
