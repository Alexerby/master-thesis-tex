\begin{table}[htbp]
\small
\centering
\begin{tabular}{l@{\hspace{2em}}r@{\hspace{2em}}r@{\hspace{2em}}r}
\toprule
\textbf{Year} & \textbf{Non-Take-Up} & \textbf{Take-Up Rate} & \textbf{Beta Error} \\
              & \(\Pr(\text{NTU} = 1 \mid \text{M} = 1)\) & \(\Pr(\text{TU} = 1 \mid \text{M} = 1)\) & \(\Pr(\text{TU} = 1 \mid \text{M} = 0)\) \\
\midrule
2007 & 60.6 & 39.4 & 13.6 \\
2008 & 63.5 & 36.5 & 17.1 \\
2009 & 61.0 & 39.0 & 18.6 \\
2010 & 60.9 & 39.1 & 17.7 \\
2011 & 53.8 & 46.2 & 16.1 \\
2012 & 51.5 & 48.5 & 18.9 \\
2013 & 50.0 & 50.0 & 15.9 \\
2014 & 55.1 & 44.9 & 16.1 \\
2015 & 64.0 & 36.0 & 12.6 \\
2016 & 56.5 & 43.5 & 12.4 \\
2017 & 62.6 & 37.4 & 10.1 \\
2018 & 63.9 & 36.1 & 15.3 \\
2019 & 67.5 & 32.5 & 11.7 \\
2020 & 63.7 & 36.3 & 13.6 \\
2021 & 66.7 & 33.3 & 12.3 \\
\midrule
\textbf{Average} & \textbf{59.7} & \textbf{40.3} & \textbf{15.0} \\
\bottomrule
\end{tabular}
\caption{\small{Non-Take-Up, Take-Up, and Beta Error Rates by Survey Year (\%). Non-take-up is the share of theoretically eligible students (\(M=1\)) who do not receive BAföG. The take-up rate is simply the complement, i.e., the share of eligible students who do receive BAföG \((1 - \Pr(\text{NTU} = 1 \mid M = 1))\). Beta error is the share of ineligible students (\(M=0\)) who nevertheless receive BAföG.}}
\label{table:microsimulation_ntu}
\end{table}
