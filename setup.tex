%------ SETUP OF THE DOCUMENT ------%
%This part of the document sets up the document and makes it easier to read the main.tex file
%Changes in the setup file are recommended if you wish to customize things like colors in links and such.

%------ ********************* ------%
\usepackage[margin = 25mm]{geometry} %
\usepackage{graphicx} % Required for inserting images
\usepackage[toc,page]{appendix}
\usepackage[english=usenglishmax]{hyphsubst} %sets hyphenation to American English. Google is your friend on hyphenation. Babel could also be used
\usepackage[hyphens]{url}
\usepackage{amsmath, amssymb, setspace, float, subcaption, caption, booktabs, pdflscape, dcolumn, titlesec, tocloft, comment, xcolor, longtable, blindtext, rotating, lipsum}
\usepackage[flushleft]{threeparttable}
\usepackage[round, authoryear]{natbib}
\usepackage[hidelinks]{hyperref}
\usepackage{titlesec}
\usepackage{enumitem}
\usepackage{parskip}
\usepackage[utf8]{inputenc}



\hypersetup{
     colorlinks   = true, %If links to papers should be colored
     citecolor    = blue, %color of citation text
     urlcolor = blue, %if you use an URL, this is how it is colored
    linkcolor = black
}


%%%%%%%%%%%%%%%%%%%%%%%%%%%%%%%% 
% Tikz
%%%%%%%%%%%%%%%%%%%%%%%%%%%%%%%% 

\usepackage{tikz}
\usetikzlibrary{positioning, arrows.meta, shapes.geometric, fit}


\tikzset{
  pipelinebox/.style={
    rectangle,
    draw=black,
    align=center,
    font=\scriptsize,
    inner sep=4pt,
    minimum height=1cm,
    minimum width=2.5cm,
    text width=2.5cm,
    align=center
  },
  arrow/.style={
    thick, ->, >=stealth
  }
}

\tikzset{
  auxbox/.style={
    rectangle,
    draw=black,
    fill=orange!20,
    align=center,
    font=\footnotesize\itshape,
    inner sep=8pt,
    minimum height=1cm,
    minimum width=3.5cm,
    dashed
  }
}

\tikzset{
  sourcebox/.style={
    rectangle,
    draw=black,
    align=center,
    font=\footnotesize\bfseries,
    text transform=uppercase,
    inner sep=8pt,
    minimum height=1cm,
    minimum width=3.5cm
  }
}

\tikzset{
  actionbox/.style={
    diamond,
    draw=black,
    % fill=orange!20,
    align=center,
    font=\footnotesize,
    aspect=2,
    inner sep=2pt,
    minimum height=1.2cm,
    minimum width=3.5cm
  }
}

\tikzset{
  datasetbox/.style={
    rectangle, 
    draw=black, 
    dashed,
    minimum height=1.5em, 
    minimum width=3cm, 
    text centered, 
  },
}



%%%%%%%%%%%%%%%%%%%%%%%%%%%%%%%%%%%%%%%

\graphicspath{{figures/}}



%\bibliographystyle{apalike} 
\bibliographystyle{plainnat}
% \bibliographystyle{plain}

%%%%%%%%%%%%%%%%%%%%%%%%%%%%%%%%%%%
% Table formatting
%%%%%%%%%%%%%%%%%%%%%%%%%%%%%%%%%%%
\usepackage{multirow}
\renewcommand{\arraystretch}{1.0}


%%%%%%%%%%%%%%%%%%%%%%%%%%%%%%%%%%%
% Equation Formatting
%%%%%%%%%%%%%%%%%%%%%%%%%%%%%%%%%%%
\numberwithin{equation}{section}

%%%%%%%%%%%%%%%%%%%%%%%%%%%%%%%%%%%
% List Formatting
%%%%%%%%%%%%%%%%%%%%%%%%%%%%%%%%%%%
\setlist[itemize]{itemsep=0.05em}

