%------ SETUP OF THE DOCUMENT ------%
%This part of the document sets up the document and makes it easier to read the main.tex file
%Changes in the setup file are recommended if you wish to customize things like colors in links and such.

%------ ********************* ------%
\usepackage[margin = 25mm]{geometry} %
\usepackage{graphicx} % Required for inserting images
\usepackage[toc,page]{appendix}
\usepackage[english=usenglishmax]{hyphsubst} %sets hyphenation to American English. Google is your friend on hyphenation. Babel could also be used
\usepackage[hyphens]{url}
\usepackage{amsmath, amssymb, setspace, float, subcaption, caption, booktabs, pdflscape, dcolumn, titlesec, tocloft, comment, xcolor, longtable, blindtext, rotating, lipsum}
\usepackage[flushleft]{threeparttable}
\usepackage[round, numbers, authoryear]{natbib}
\usepackage[hidelinks]{hyperref}
\usepackage{titlesec}
\usepackage{enumitem}
\usepackage{parskip}
\usepackage[utf8]{inputenc}

\hypersetup{
     colorlinks   = true, %If links to papers should be colored
     citecolor    = blue, %color of citation text
     urlcolor = blue, %if you use an URL, this is how it is colored
    linkcolor = black
}

\graphicspath{{figures/}}

%CHANGE SECTION STYLE

\bibliographystyle{apalike} %closest to LUSEM Harvard referencing guide
% \bibliographystyle{plain} %closest to LUSEM Harvard referencing guide

%%%%%%%%%%%%%%%%%%%%%%%%%%%%%%%%%%%
% Table formatting
%%%%%%%%%%%%%%%%%%%%%%%%%%%%%%%%%%%
\usepackage{multirow}
\renewcommand{\arraystretch}{1.0}


%%%%%%%%%%%%%%%%%%%%%%%%%%%%%%%%%%%
% Equation Formatting
%%%%%%%%%%%%%%%%%%%%%%%%%%%%%%%%%%%
\numberwithin{equation}{section}

%%%%%%%%%%%%%%%%%%%%%%%%%%%%%%%%%%%
% List Formatting
%%%%%%%%%%%%%%%%%%%%%%%%%%%%%%%%%%%
\setlist[itemize]{itemsep=0.05em}
